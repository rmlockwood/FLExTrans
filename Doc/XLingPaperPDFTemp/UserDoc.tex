\documentclass[12pt,twoside]{article}
\setlength{\paperheight}{11in}
\setlength{\paperwidth}{8.5in}
\setlength{\topmargin}{0pt}
\setlength{\voffset}{-21.681000000000004pt}
\setlength{\evensidemargin}{0pt}
\setlength{\oddsidemargin}{36.135000000000005pt}
\setlength{\textwidth}{433.62pt}
\setlength{\textheight}{659.46375pt}
\setlength{\headheight}{14.5pt}
\setlength{\headsep}{4.067499999999999pt}
\setlength{\footskip}{.25in}
\DeclareTextSymbol{\textsquarebracketleft}{EU1}{91}
\DeclareTextSymbol{\textsquarebracketright}{EU1}{93}
\usepackage{xltxtra}
\usepackage{setspace}
\usepackage[normalem]{ulem}
\usepackage{color}
\usepackage{colortbl}
\usepackage{tabularx}
\usepackage{longtable}
\usepackage{multirow}
\usepackage{booktabs}
\usepackage{calc}
\usepackage{fancyhdr}
\usepackage{fontspec}
\usepackage{hyperref}
\hypersetup{colorlinks=true, citecolor=black, filecolor=black, linkcolor=black, urlcolor=blue, bookmarksopen=true, bookmarksdepth=3, pdfauthor={}, pdfcreator={XLingPaper version 3.10.0 (https://software.sil.org/xlingpaper/)}, pdftitle={FLExTrans User Documentation}}
\pagestyle{fancy}
\fancyhf{}
\fancyhead[RO,LE]{\HeaderFooterFontFamily{\small{\textit{\thepage}}}}
\fancyhead[RE]{\HeaderFooterFontFamily{\small{\textit{\leftmark}}}}
\fancyhead[LO]{\HeaderFooterFontFamily{\small{\textit{\rightmark}}}}
\renewcommand{\headrulewidth}{0pt}
\renewcommand{\footrulewidth}{0pt}
\fancypagestyle{plain}{\fancyhf{}
\fancyfoot[C]{\HeaderFooterFontFamily{\small{\textit{\thepage}}}}
\renewcommand{\headrulewidth}{0pt}
\renewcommand{\footrulewidth}{0pt}
}
\setmainfont{Times New Roman}
\font\MainFont="Times New Roman" at 12pt
\newfontfamily{\HeaderFooterFontFamily}{Times New Roman}
\newfontfamily{\TitleFontFamily}{Times New Roman}
\newfontfamily{\SubtitleFontFamily}{Times New Roman}
\newfontfamily{\AuthorFontFamily}{Times New Roman}
\newfontfamily{\AffiliationFontFamily}{Times New Roman}
\newfontfamily{\EmailAddressFontFamily}{Times New Roman}
\newfontfamily{\DateFontFamily}{Times New Roman}
\newfontfamily{\ChapterFontFamily}{Times New Roman}
\newfontfamily{\SectionLevelOneFontFamily}{Times New Roman}
\newfontfamily{\SectionLevelTwoFontFamily}{Times New Roman}
\newfontfamily{\SectionLevelThreeFontFamily}{Times New Roman}
\newfontfamily{\SectionLevelFourFontFamily}{Times New Roman}
\newfontfamily{\SectionLevelFiveFontFamily}{Times New Roman}
\newfontfamily{\SectionLevelSixFontFamily}{Times New Roman}
\newfontfamily{\LanglVernacularFontFamily}{Courier New}
\newfontfamily{\LanglGlossFontFamily}{Times New Roman}
\newfontfamily{\LangenFontFamily}{Times New Roman}
\newfontfamily{\LanghrFontFamily}{Times New Roman}
\newfontfamily{\LangtBoldArialFontFamily}{Arial}
\newfontfamily{\LangtSmallCapsFontFamily}{Charis SIL Small Caps}
\newfontfamily{\LangtCollectionFontFamily}{Arial}
\newfontfamily{\LangtCourierFontFamily}{Courier New}
\newfontfamily{\LangtExportDraftBPtxToolFontFamily}{Courier New}
\newfontfamily{\LangtExportFLExBPtxToolFontFamily}{Courier New}
\newfontfamily{\LangtFoldernameFontFamily}{Tahoma}
\newfontfamily{\LangtluAffixFontFamily}{Courier New}
\newfontfamily{\LangtluGrammCatFontFamily}{Courier New}
\newfontfamily{\LangtluNotFoundFontFamily}{Courier New}
\newfontfamily{\LangtluPunctuationFontFamily}{Courier New}
\newfontfamily{\LangtMenuFontFamily}{Courier New}
\newfontfamily{\LangtModuleFontFamily}{Courier New}
\newfontfamily{\LangtRuleElemInXXEFontFamily}{Arial}
\newfontfamily{\LangtRuleAttribInXXEFontFamily}{Arial}
\newfontfamily{\LangtToolFontFamily}{Courier New}
\setlength{\parindent}{1em}
\catcode`^^^^200b=\active
\def^^^^200b{\hskip0pt}\renewenvironment{quotation}{\list{}{\leftmargin=.125in\rightmargin=.125in}\item[]{}}{\endlist}
\clubpenalty=10000
\widowpenalty=10000
\begin{document}
\baselineskip=\glueexpr\baselineskip + 0pt plus 2pt minus 1pt\relax
\newlength{\leveloneindent}
\newlength{\levelonewidth}
\newlength{\leveltwoindent}
\newlength{\leveltwowidth}
\newlength{\levelthreeindent}
\newlength{\levelthreewidth}
\newlength{\levelfourindent}
\newlength{\levelfourwidth}
\newlength{\levelfiveindent}
\newlength{\levelfivewidth}
\newlength{\levelsixindent}
\newlength{\levelsixwidth}
\newdimen\XLingPapertempdim
\newdimen\XLingPapertempdimletter
\newcommand{\XLingPapertableofcontents}{\immediate\openout8 = \jobname.toc\relax
\immediate\write8{<toc>}}
\newcommand{\XLingPaperaddtocontents}[1]{\write8{<tocline ref="#1" page="\thepage"/>}}
\newcommand{\XLingPaperendtableofcontents}{\immediate\write8{</toc>}\closeout8\relax
}
\newcommand{\XLingPaperdotfill}{\leaders\hbox{$\mathsurround 0pt\mkern 4.5 mu\hbox{.}\mkern 4.5 mu$}\hfill}
\newcommand{\XLingPaperdottedtocline}[4]{
\newdimen\XLingPapertempdim
\vskip0pt plus .2pt{
\leftskip#1\relax% left glue for indent
\rightskip\XLingPapertocrmarg% right glue for for right margin
\parfillskip-\rightskip% so can go into margin if need be???
\parindent#1\relax
\interlinepenalty10000
\leavevmode
\XLingPapertempdim#2\relax% numwidth
\advance\leftskip\XLingPapertempdim\null\nobreak\hskip-\leftskip{#3}\nobreak
\XLingPaperdotfill\nobreak
\hbox to\XLingPaperpnumwidth{\hfil\normalfont\normalcolor#4}
\par}}
\newlength{\XLingPaperpnumwidth}
\newlength{\XLingPapertocrmarg}
\setlength{\XLingPaperpnumwidth}{1.55em}\setlength{\XLingPapertocrmarg}{\XLingPaperpnumwidth+1em}
\newsavebox{\XLingPapertempbox}
\newlength{\XLingPapertemplen}
\newlength{\XLingPaperavailabletablewidth}
\newlength{\XLingPapertableminwidth}
\newlength{\XLingPapertablemaxwidth}
\newlength{\XLingPapertablewidthminustableminwidth}
\newlength{\XLingPapertablemaxwidthminusminwidth}
\newlength{\XLingPapertablewidthratio}
\newlength{\XLingPapermincola}\newlength{\XLingPapermaxcola}\newlength{\XLingPapercolawidth}
\newlength{\XLingPapermincolb}\newlength{\XLingPapermaxcolb}\newlength{\XLingPapercolbwidth}
\newlength{\XLingPapermincolc}\newlength{\XLingPapermaxcolc}\newlength{\XLingPapercolcwidth}
\newlength{\XLingPapermincold}\newlength{\XLingPapermaxcold}\newlength{\XLingPapercoldwidth}
\newlength{\XLingPapermincole}\newlength{\XLingPapermaxcole}\newlength{\XLingPapercolewidth}
\newlength{\XLingPapermincolf}\newlength{\XLingPapermaxcolf}\newlength{\XLingPapercolfwidth}
\newlength{\XLingPapermincolg}\newlength{\XLingPapermaxcolg}\newlength{\XLingPapercolgwidth}
\newlength{\XLingPapermincolh}\newlength{\XLingPapermaxcolh}\newlength{\XLingPapercolhwidth}
\newlength{\XLingPapermincoli}\newlength{\XLingPapermaxcoli}\newlength{\XLingPapercoliwidth}
\newlength{\XLingPapermincolj}\newlength{\XLingPapermaxcolj}\newlength{\XLingPapercoljwidth}
\newlength{\XLingPapermincolk}\newlength{\XLingPapermaxcolk}\newlength{\XLingPapercolkwidth}
\newlength{\XLingPapermincoll}\newlength{\XLingPapermaxcoll}\newlength{\XLingPapercollwidth}
\newlength{\XLingPapermincolm}\newlength{\XLingPapermaxcolm}\newlength{\XLingPapercolmwidth}
\newlength{\XLingPapermincoln}\newlength{\XLingPapermaxcoln}\newlength{\XLingPapercolnwidth}
\newlength{\XLingPapermincolo}\newlength{\XLingPapermaxcolo}\newlength{\XLingPapercolowidth}
\newlength{\XLingPapermincolp}\newlength{\XLingPapermaxcolp}\newlength{\XLingPapercolpwidth}
\newlength{\XLingPapermincolq}\newlength{\XLingPapermaxcolq}\newlength{\XLingPapercolqwidth}
\newlength{\XLingPapermincolr}\newlength{\XLingPapermaxcolr}\newlength{\XLingPapercolrwidth}
\newlength{\XLingPapermincols}\newlength{\XLingPapermaxcols}\newlength{\XLingPapercolswidth}
\newlength{\XLingPapermincolt}\newlength{\XLingPapermaxcolt}\newlength{\XLingPapercoltwidth}
\newlength{\XLingPapermincolu}\newlength{\XLingPapermaxcolu}\newlength{\XLingPapercoluwidth}
\newlength{\XLingPapermincolv}\newlength{\XLingPapermaxcolv}\newlength{\XLingPapercolvwidth}
\newlength{\XLingPapermincolw}\newlength{\XLingPapermaxcolw}\newlength{\XLingPapercolwwidth}
\newlength{\XLingPapermincolx}\newlength{\XLingPapermaxcolx}\newlength{\XLingPapercolxwidth}
\newlength{\XLingPapermincoly}\newlength{\XLingPapermaxcoly}\newlength{\XLingPapercolywidth}
\newlength{\XLingPapermincolz}\newlength{\XLingPapermaxcolz}\newlength{\XLingPapercolzwidth}
\newcommand{\XLingPaperlongestcell}[2]{
\ifdim#1>#2
#2=#1
\fi
}
\newcommand{\XLingPaperminmaxcellincolumn}[5]{
\savebox{\XLingPapertempbox}{#3}
\settowidth{\XLingPapertemplen}{\usebox{\XLingPapertempbox}}
\addtolength{\XLingPapertemplen}{#5}
\XLingPaperlongestcell{\XLingPapertemplen}{#4}
\setlength{\XLingPapertemplen}{\widthof{#1}}
\addtolength{\XLingPapertemplen}{#5}
\ifdim\XLingPapertemplen>#4
\XLingPapertemplen=#4
\fi
\XLingPaperlongestcell{\XLingPapertemplen}{#2}}
\newcommand{\XLingPapersetcolumnwidth}[4]{
\ifdim\XLingPapertableminwidth>\XLingPaperavailabletablewidth
#1=#2
\else
\ifdim\XLingPapertableminwidth=\XLingPaperavailabletablewidth
#1=#2
\else
\ifdim\XLingPapertablemaxwidth<\XLingPaperavailabletablewidth
#1=#3
\else
\setlength{\XLingPapertemplen}{#3-#2}
\divide\XLingPapertemplen by 100
\multiply\XLingPapertemplen by \XLingPapertablewidthratio
#1=#2
\addtolength{#1}{\XLingPapertemplen}
\addtolength{#1}{#4}
\fi
\fi
\fi
}
\newcommand{\XLingPapercalculatetablewidthratio}{
\setlength{\XLingPapertablewidthminustableminwidth}{\XLingPaperavailabletablewidth-\XLingPapertableminwidth}
\setlength{\XLingPapertablemaxwidthminusminwidth}{\XLingPapertablemaxwidth-\XLingPapertableminwidth}
\ifdim\XLingPapertablemaxwidthminusminwidth<0.002pt
\XLingPapertablemaxwidthminusminwidth=10000sp
\fi
\setlength{\XLingPapertablewidthratio}{\XLingPapertablewidthminustableminwidth}
\divide\XLingPapertablemaxwidthminusminwidth by 100
\divide\XLingPapertablewidthratio by \XLingPapertablemaxwidthminusminwidth}
\newcommand{\XLingPaperneedspace}[1]{\penalty-100\begingroup
\newdimen{\XLingPaperspaceneeded}
\newdimen{\XLingPaperspaceavailable}
\setlength{\XLingPaperspaceneeded}{#1}%
\XLingPaperspaceavailable\pagegoal \advance\XLingPaperspaceavailable-\pagetotal
\ifdim \XLingPaperspaceneeded>\XLingPaperspaceavailable
\ifdim \XLingPaperspaceavailable>0pt
\vfil
\fi
\break
\fi\endgroup}

\newlength{\XLingPaperlistinexampleindent}
\newlength{\XLingPaperisocodewidth}\setlength{\XLingPaperlistinexampleindent}{.125in+ 2.75em}
\newlength{\XLingPaperlistitemindent}
\newlength{\XLingPaperbulletlistitemwidth}\settowidth{\XLingPaperbulletlistitemwidth}{•\ }\newlength{\XLingPapersingledigitlistitemwidth}
\settowidth{\XLingPapersingledigitlistitemwidth}{8.\ }\newlength{\XLingPaperdoubledigitlistitemwidth}
\settowidth{\XLingPaperdoubledigitlistitemwidth}{88.\ }\newlength{\XLingPapertripledigitlistitemwidth}
\settowidth{\XLingPapertripledigitlistitemwidth}{888.\ }\newlength{\XLingPapersingleletterlistitemwidth}
\settowidth{\XLingPapersingleletterlistitemwidth}{m.\ }\newlength{\XLingPaperdoubleletterlistitemwidth}
\settowidth{\XLingPaperdoubleletterlistitemwidth}{mm.\ }\newlength{\XLingPapertripleletterlistitemwidth}
\settowidth{\XLingPapertripleletterlistitemwidth}{mmm.\ }\newlength{\XLingPaperromanviilistitemwidth}
\settowidth{\XLingPaperromanviilistitemwidth}{vii.\ }\newlength{\XLingPaperromanviiilistitemwidth}
\settowidth{\XLingPaperromanviiilistitemwidth}{viii.\ }\newlength{\XLingPaperromanxviiilistitemwidth}
\settowidth{\XLingPaperromanxviiilistitemwidth}{xviii.\ }\newlength{\XLingPaperspacewidth}
\settowidth{\XLingPaperspacewidth}{\ }\newcommand{\XLingPaperlistitem}[4]{
\newdimen\XLingPapertempdim
\vskip0pt plus .2pt{
\leftskip#1\relax% left glue for indent
\parindent#1\relax
\interlinepenalty10000
\leavevmode
\XLingPapertempdim#2\relax% label width
\advance\leftskip\XLingPapertempdim\null\nobreak\hskip-\leftskip\hbox to\XLingPapertempdim{\hfil\normalfont\normalcolor#3\ }{#4}\nobreak
\par}}
\newcommand{\XLingPaperexample}[5]{
\newdimen\XLingPapertempdim
\vskip0pt plus .2pt{
\leftskip#1\relax% left glue for indent
\hspace*{#1}\relax
\rightskip#2\relax% right glue for indent
\interlinepenalty10000
\leavevmode
\XLingPapertempdim#3\relax% example number width
\advance\leftskip\XLingPapertempdim\null\nobreak\hskip-\leftskip\hbox to\XLingPapertempdim{\normalfont\normalcolor#4\hfil}{#5}\nobreak
\par}}
\newcommand{\XLingPaperexampleintable}[5]{
\newdimen\XLingPapertempdim
\leftskip#1\relax% left glue for indent
\hspace*{#1}\relax
\rightskip#2\relax% right glue for indent
\interlinepenalty10000
\leavevmode
\XLingPapertempdim#3\relax% example number width
\hbox to\XLingPapertempdim{\normalfont\normalcolor#4\hfil}{
\begin{tabular}
[t]{@{}l@{}}#5\end{tabular}
}\nobreak
}
\newcommand{\XLingPaperfree}[2]{\vskip0pt plus .2pt{
\leftskip#1\relax% left glue for indent
\parindent#1\relax
\interlinepenalty10000
\leavevmode{#2}\nobreak
\par}}
\newcommand{\XLingPaperlistinterlinear}[5]{\vskip0pt plus .2pt{\hspace*{#1}\hspace*{#2}
\XLingPapertempdimletter#3\relax% letter width
\advance\leftskip\XLingPapertempdimletter\null\nobreak\hskip-\leftskip\hspace*{-.3em}\hbox to\XLingPapertempdimletter{\normalfont\normalcolor#4\ \hfil}{#5}\nobreak
\par}}
\newcommand{\XLingPaperlistinterlinearintable}[5]{
\XLingPapertempdimletter#3\relax% letter width
\hspace*{-.3em}\hbox to\XLingPapertempdimletter{\normalfont\normalcolor#4\ \hfil}{
\begin{tabular}
[t]{@{}l@{}}#5\end{tabular}
}\nobreak
}

\newlength{\XLingPaperexamplefreeindent}\setlength{\XLingPaperexamplefreeindent}{-.3 em}\newskip\XLingPaperinterwordskip
\XLingPaperinterwordskip=6.66666pt plus 3.33333pt minus 2.22222pt
\def\XLingPaperintspace{\hskip\XLingPaperinterwordskip}
\def\XLingPaperraggedright{\rightskip=0pt plus1fil\pretolerance=10000}\raggedbottom

\begin{MainFont}
\vspace*{1in}
{\centering{\TitleFontFamily{\LARGE{\textbf{{\LangtToolFontFamily{\textbf{\textcolor[rgb]{0,0,0.5019607843137255}{FLExTrans}}}} User Documentation\\}}}}
}\thispagestyle{plain}
\markboth{{\LangtToolFontFamily{\textbf{\textcolor[rgb]{0,0,0.5019607843137255}{FLExTrans}}}} User Documentation}{}
{\centering{\AuthorFontFamily{\textit{}}}}{
\centering{\EmailAddressFontFamily{\textit{flextrans\_help@sil.org\\}}}}{
\centering{\DateFontFamily{23 July 2025\\}}}{
\centering{Version: 3.14\\}}\XLingPapertableofcontents
\vspace{12pt plus 2pt minus 1pt}
\XLingPaperneedspace{5\baselineskip}

\penalty-3000{{\centering{\raisebox{\baselineskip}[0pt]{\protect\hypertarget{rXLingPapContents}{}}\large{\raisebox{\baselineskip}[0pt]{\pdfbookmark[1]{Contents}{rXLingPapContents}}\textbf{Contents\\}}}
}\markright{Contents}
\XLingPaperaddtocontents{rXLingPapContents}}\par{}\penalty10000
\vspace{12pt plus 2pt minus 1pt}
\hyperlink{rXLingPapAbstract0}{\XLingPaperdottedtocline{0pt}{0pt}{Abstract}{??}
}\settowidth{\leveloneindent}{}\settowidth{\levelonewidth}{1\thinspace\thinspace}\hyperlink{sIntro}{\XLingPaperdottedtocline{\leveloneindent}{\levelonewidth}{1 Introduction}{??}
}\settowidth{\leveloneindent}{}\settowidth{\levelonewidth}{2\thinspace\thinspace}\hyperlink{sGettingStarted}{\XLingPaperdottedtocline{\leveloneindent}{\levelonewidth}{2 Overview and Getting Started}{??}
}\settowidth{\leveltwoindent}{2\ }\settowidth{\leveltwowidth}{2.1\thinspace\thinspace}\hyperlink{sPrepSteps}{\XLingPaperdottedtocline{\leveltwoindent}{\leveltwowidth}{2.1 Preparatory Steps}{??}
}\settowidth{\levelthreeindent}{2\ 2.1\ }\settowidth{\levelthreewidth}{2.1.1\thinspace\thinspace}\hyperlink{sProjectSetup}{\XLingPaperdottedtocline{\levelthreeindent}{\levelthreewidth}{2.1.1 Set up your projects}{??}
}\settowidth{\levelthreeindent}{2\ 2.1\ }\settowidth{\levelthreewidth}{2.1.2\thinspace\thinspace}\hyperlink{sHermitCrabSetup}{\XLingPaperdottedtocline{\levelthreeindent}{\levelthreewidth}{2.1.2 Choose your parser for synthesis, {\LangtToolFontFamily{\textbf{\textcolor[rgb]{0,0,0.5019607843137255}{XAmple}}}} or {\LangtToolFontFamily{\textbf{\textcolor[rgb]{0,0,0.5019607843137255}{HermitCrab}}}}}{??}
}\settowidth{\levelthreeindent}{2\ 2.1\ }\settowidth{\levelthreewidth}{2.1.3\thinspace\thinspace}\hyperlink{sDataPrep}{\XLingPaperdottedtocline{\levelthreeindent}{\levelthreewidth}{2.1.3 Prepare your data}{??}
}\settowidth{\levelthreeindent}{2\ 2.1\ }\settowidth{\levelthreewidth}{2.1.4\thinspace\thinspace}\hyperlink{sCategorySetup}{\XLingPaperdottedtocline{\levelthreeindent}{\levelthreewidth}{2.1.4 Pre-populate the transfer rules file with Categories and Attributes}{??}
}\settowidth{\leveltwoindent}{2\ }\settowidth{\leveltwowidth}{2.2\thinspace\thinspace}\hyperlink{sWalkthrough}{\XLingPaperdottedtocline{\leveltwoindent}{\leveltwowidth}{2.2 Walkthrough of Basic Machine Translation Steps}{??}
}\settowidth{\levelthreeindent}{2\ 2.2\ }\settowidth{\levelthreewidth}{2.2.1\thinspace\thinspace}\hyperlink{sAnalyze}{\XLingPaperdottedtocline{\levelthreeindent}{\levelthreewidth}{2.2.1 Analyze the Source Text}{??}
}\settowidth{\levelthreeindent}{2\ 2.2\ }\settowidth{\levelthreewidth}{2.2.2\thinspace\thinspace}\hyperlink{sLinkSenses}{\XLingPaperdottedtocline{\levelthreeindent}{\levelthreewidth}{2.2.2 Link Senses}{??}
}\settowidth{\levelthreeindent}{2\ 2.2\ }\settowidth{\levelthreewidth}{2.2.3\thinspace\thinspace}\hyperlink{sWriteRules}{\XLingPaperdottedtocline{\levelthreeindent}{\levelthreewidth}{2.2.3 Write Transfer Rules}{??}
}\settowidth{\levelthreeindent}{2\ 2.2\ }\settowidth{\levelthreewidth}{2.2.4\thinspace\thinspace}\hyperlink{sRunModules}{\XLingPaperdottedtocline{\levelthreeindent}{\levelthreewidth}{2.2.4 Run {\LangtToolFontFamily{\textbf{\textcolor[rgb]{0,0,0.5019607843137255}{FLExTrans}}}} Modules}{??}
}\settowidth{\leveloneindent}{}\settowidth{\levelonewidth}{3\thinspace\thinspace}\hyperlink{sInfrastructure}{\XLingPaperdottedtocline{\leveloneindent}{\levelonewidth}{3 {\LangtToolFontFamily{\textbf{\textcolor[rgb]{0,0,0.5019607843137255}{FLExTrans}}}} User Interface Orientation}{??}
}\settowidth{\leveltwoindent}{3\ }\settowidth{\leveltwowidth}{3.1\thinspace\thinspace}\hyperlink{sFTUserInterface}{\XLingPaperdottedtocline{\leveltwoindent}{\leveltwowidth}{3.1 FLExTrans Window}{??}
}\settowidth{\leveltwoindent}{3\ }\settowidth{\leveltwowidth}{3.2\thinspace\thinspace}\hyperlink{sSettings}{\XLingPaperdottedtocline{\leveltwoindent}{\leveltwowidth}{3.2 The {\LangtToolFontFamily{\textbf{\textcolor[rgb]{0,0,0.5019607843137255}{FLExTrans Settings}}}}}{??}
}\settowidth{\leveltwoindent}{3\ }\settowidth{\leveltwowidth}{3.3\thinspace\thinspace}\hyperlink{sCollections}{\XLingPaperdottedtocline{\leveltwoindent}{\leveltwowidth}{3.3 Using Collections}{??}
}\settowidth{\leveloneindent}{}\settowidth{\levelonewidth}{4\thinspace\thinspace}\hyperlink{sTools}{\XLingPaperdottedtocline{\leveloneindent}{\levelonewidth}{4 {\LangtToolFontFamily{\textbf{\textcolor[rgb]{0,0,0.5019607843137255}{FLExTrans}}}} Tools}{??}
}\settowidth{\leveltwoindent}{4\ }\settowidth{\leveltwowidth}{4.1\thinspace\thinspace}\hyperlink{sLinker}{\XLingPaperdottedtocline{\leveltwoindent}{\leveltwowidth}{4.1 The {\LangtToolFontFamily{\textbf{\textcolor[rgb]{0,0,0.5019607843137255}{Sense Linker Tool}}}}}{??}
}\settowidth{\levelthreeindent}{4\ 4.1\ }\settowidth{\levelthreewidth}{4.1.1\thinspace\thinspace}\hyperlink{sColorKey}{\XLingPaperdottedtocline{\levelthreeindent}{\levelthreewidth}{4.1.1 Color Coding Key}{??}
}\settowidth{\levelthreeindent}{4\ 4.1\ }\settowidth{\levelthreewidth}{4.1.2\thinspace\thinspace}\hyperlink{sHowLink}{\XLingPaperdottedtocline{\levelthreeindent}{\levelthreewidth}{4.1.2 Linking}{??}
}\settowidth{\levelthreeindent}{4\ 4.1\ }\settowidth{\levelthreewidth}{4.1.3\thinspace\thinspace}\hyperlink{sSLTChangeSource}{\XLingPaperdottedtocline{\levelthreeindent}{\levelthreewidth}{4.1.3 Source Text}{??}
}\settowidth{\levelthreeindent}{4\ 4.1\ }\settowidth{\levelthreewidth}{4.1.4\thinspace\thinspace}\hyperlink{sSearchList}{\XLingPaperdottedtocline{\levelthreeindent}{\levelthreewidth}{4.1.4 Search the All Target Senses List}{??}
}\settowidth{\levelthreeindent}{4\ 4.1\ }\settowidth{\levelthreewidth}{4.1.5\thinspace\thinspace}\hyperlink{sAddEntry}{\XLingPaperdottedtocline{\levelthreeindent}{\levelthreewidth}{4.1.5 Add Entry}{??}
}\settowidth{\levelthreeindent}{4\ 4.1\ }\settowidth{\levelthreewidth}{4.1.6\thinspace\thinspace}\hyperlink{sShowUnlinked}{\XLingPaperdottedtocline{\levelthreeindent}{\levelthreewidth}{4.1.6 Show Only Unlinked}{??}
}\settowidth{\levelthreeindent}{4\ 4.1\ }\settowidth{\levelthreewidth}{4.1.7\thinspace\thinspace}\hyperlink{sHidePN}{\XLingPaperdottedtocline{\levelthreeindent}{\levelthreewidth}{4.1.7 Hide Proper Nouns}{??}
}\settowidth{\levelthreeindent}{4\ 4.1\ }\settowidth{\levelthreewidth}{4.1.8\thinspace\thinspace}\hyperlink{sFontControl}{\XLingPaperdottedtocline{\levelthreeindent}{\levelthreewidth}{4.1.8 Zoom and Font Button}{??}
}\settowidth{\levelthreeindent}{4\ 4.1\ }\settowidth{\levelthreewidth}{4.1.9\thinspace\thinspace}\hyperlink{sRebuildBiling}{\XLingPaperdottedtocline{\levelthreeindent}{\levelthreewidth}{4.1.9 Rebuild Bilingual Lexicon}{??}
}\settowidth{\levelthreeindent}{4\ 4.1\ }\settowidth{\levelthreewidth}{4.1.10\thinspace\thinspace}\hyperlink{sExportUnlinked}{\XLingPaperdottedtocline{\levelthreeindent}{\levelthreewidth}{4.1.10 Export Unlinked Senses}{??}
}\settowidth{\levelthreeindent}{4\ 4.1\ }\settowidth{\levelthreewidth}{4.1.11\thinspace\thinspace}\hyperlink{sSenses2Link}{\XLingPaperdottedtocline{\levelthreeindent}{\levelthreewidth}{4.1.11 Senses to link}{??}
}\settowidth{\leveltwoindent}{4\ }\settowidth{\leveltwowidth}{4.2\thinspace\thinspace}\hyperlink{sRuleTester}{\XLingPaperdottedtocline{\leveltwoindent}{\leveltwowidth}{4.2 The {\LangtToolFontFamily{\textbf{\textcolor[rgb]{0,0,0.5019607843137255}{Live Rule Tester Tool}}}}}{??}
}\settowidth{\levelthreeindent}{4\ 4.2\ }\settowidth{\levelthreewidth}{4.2.1\thinspace\thinspace}\hyperlink{sTesterQuickGuide}{\XLingPaperdottedtocline{\levelthreeindent}{\levelthreewidth}{4.2.1 Quick Guide}{??}
}\settowidth{\levelthreeindent}{4\ 4.2\ }\settowidth{\levelthreewidth}{4.2.2\thinspace\thinspace}\hyperlink{sLRTChangeSource}{\XLingPaperdottedtocline{\levelthreeindent}{\levelthreewidth}{4.2.2 Change Source Text}{??}
}\settowidth{\levelthreeindent}{4\ 4.2\ }\settowidth{\levelthreewidth}{4.2.3\thinspace\thinspace}\hyperlink{sHover}{\XLingPaperdottedtocline{\levelthreeindent}{\levelthreewidth}{4.2.3 Hover to See Bilingual Lexicon Entry}{??}
}\settowidth{\levelthreeindent}{4\ 4.2\ }\settowidth{\levelthreewidth}{4.2.4\thinspace\thinspace}\hyperlink{sTesterOther}{\XLingPaperdottedtocline{\levelthreeindent}{\levelthreewidth}{4.2.4 Other Parts of the Tool}{??}
}\settowidth{\leveltwoindent}{4\ }\settowidth{\leveltwowidth}{4.3\thinspace\thinspace}\hyperlink{sViewSrcTgt}{\XLingPaperdottedtocline{\leveltwoindent}{\leveltwowidth}{4.3 The {\LangtToolFontFamily{\textbf{\textcolor[rgb]{0,0,0.5019607843137255}{View Source/Target Apertium Text Tool}}}}}{??}
}\settowidth{\leveltwoindent}{4\ }\settowidth{\leveltwowidth}{4.4\thinspace\thinspace}\hyperlink{sParatext}{\XLingPaperdottedtocline{\leveltwoindent}{\leveltwowidth}{4.4 {\LangtToolFontFamily{\textbf{\textcolor[rgb]{0,0,0.5019607843137255}{Paratext}}}} Import and Export Tools}{??}
}\settowidth{\levelthreeindent}{4\ 4.4\ }\settowidth{\levelthreewidth}{4.4.1\thinspace\thinspace}\hyperlink{sPtxImport}{\XLingPaperdottedtocline{\levelthreeindent}{\levelthreewidth}{4.4.1 {\LangtToolFontFamily{\textbf{\textcolor[rgb]{0,0,0.5019607843137255}{Paratext}}}} Import}{??}
}\settowidth{\levelthreeindent}{4\ 4.4\ }\settowidth{\levelthreewidth}{4.4.2\thinspace\thinspace}\hyperlink{sPtxExport}{\XLingPaperdottedtocline{\levelthreeindent}{\levelthreewidth}{4.4.2 {\LangtToolFontFamily{\textbf{\textcolor[rgb]{0,0,0.5019607843137255}{Paratext}}}} Export}{??}
}\settowidth{\levelfourindent}{4\ 4.4\ 4.4.2\ }\settowidth{\levelfourwidth}{4.4.2.1\thinspace\thinspace}\hyperlink{sPtxExportFromSynth}{\XLingPaperdottedtocline{\levelfourindent}{\levelfourwidth}{4.4.2.1 {\LangtExportDraftBPtxToolFontFamily{\textbf{\textcolor[rgb]{0,0,0.5019607843137255}{Export FLExTrans Draft to Paratext}}}}}{??}
}\settowidth{\levelfourindent}{4\ 4.4\ 4.4.2\ }\settowidth{\levelfourwidth}{4.4.2.2\thinspace\thinspace}\hyperlink{sPtxExportFromFLEx}{\XLingPaperdottedtocline{\levelfourindent}{\levelfourwidth}{4.4.2.2 {\LangtExportFLExBPtxToolFontFamily{\textbf{\textcolor[rgb]{0,0,0.5019607843137255}{Export Text from Target FLEx to Paratext}}}}}{??}
}\settowidth{\leveltwoindent}{4\ }\settowidth{\leveltwowidth}{4.5\thinspace\thinspace}\hyperlink{sSetupGramCat}{\XLingPaperdottedtocline{\leveltwoindent}{\leveltwowidth}{4.5 The {\LangtToolFontFamily{\textbf{\textcolor[rgb]{0,0,0.5019607843137255}{Set Up Transfer Rule Categories and Attributes Tool}}}}}{??}
}\settowidth{\levelthreeindent}{4\ 4.5\ }\settowidth{\levelthreewidth}{4.5.1\thinspace\thinspace}\hyperlink{sSrcCats}{\XLingPaperdottedtocline{\levelthreeindent}{\levelthreewidth}{4.5.1 Source Categories}{??}
}\settowidth{\levelthreeindent}{4\ 4.5\ }\settowidth{\levelthreewidth}{4.5.2\thinspace\thinspace}\hyperlink{sGramCat}{\XLingPaperdottedtocline{\levelthreeindent}{\levelthreewidth}{4.5.2 {\LangtCourierFontFamily{{a\_gram\_cat}}} Attribute}{??}
}\settowidth{\levelthreeindent}{4\ 4.5\ }\settowidth{\levelthreewidth}{4.5.3\thinspace\thinspace}\hyperlink{sInflectionFeatures}{\XLingPaperdottedtocline{\levelthreeindent}{\levelthreewidth}{4.5.3 Inflection Features}{??}
}\settowidth{\levelthreeindent}{4\ 4.5\ }\settowidth{\levelthreewidth}{4.5.4\thinspace\thinspace}\hyperlink{sInflectionClasses}{\XLingPaperdottedtocline{\levelthreeindent}{\levelthreewidth}{4.5.4 Inflection Classes}{??}
}\settowidth{\levelthreeindent}{4\ 4.5\ }\settowidth{\levelthreewidth}{4.5.5\thinspace\thinspace}\hyperlink{sTemplateSlots}{\XLingPaperdottedtocline{\levelthreeindent}{\levelthreewidth}{4.5.5 Template Slots}{??}
}\settowidth{\leveltwoindent}{4\ }\settowidth{\leveltwowidth}{4.6\thinspace\thinspace}\hyperlink{sSynthTest}{\XLingPaperdottedtocline{\leveltwoindent}{\leveltwowidth}{4.6 The {\LangtToolFontFamily{\textbf{\textcolor[rgb]{0,0,0.5019607843137255}{Synthesis Test Tool}}}}}{??}
}\settowidth{\levelthreeindent}{4\ 4.6\ }\settowidth{\levelthreewidth}{4.6.1\thinspace\thinspace}\hyperlink{sRunSynthTest}{\XLingPaperdottedtocline{\levelthreeindent}{\levelthreewidth}{4.6.1 Running the Test}{??}
}\settowidth{\levelthreeindent}{4\ 4.6\ }\settowidth{\levelthreewidth}{4.6.2\thinspace\thinspace}\hyperlink{sDebugSynth}{\XLingPaperdottedtocline{\levelthreeindent}{\levelthreewidth}{4.6.2 Debugging Synthesis}{??}
}\settowidth{\leveltwoindent}{4\ }\settowidth{\leveltwowidth}{4.7\thinspace\thinspace}\hyperlink{sRuleAssist}{\XLingPaperdottedtocline{\leveltwoindent}{\leveltwowidth}{4.7 The {\LangtToolFontFamily{\textbf{\textcolor[rgb]{0,0,0.5019607843137255}{Rule Assistant}}}}}{??}
}\settowidth{\leveltwoindent}{4\ }\settowidth{\leveltwowidth}{4.8\thinspace\thinspace}\hyperlink{sTextIn}{\XLingPaperdottedtocline{\leveltwoindent}{\leveltwowidth}{4.8 The {\LangtToolFontFamily{\textbf{\textcolor[rgb]{0,0,0.5019607843137255}{Text In Rules Tool}}}}}{??}
}\settowidth{\leveltwoindent}{4\ }\settowidth{\leveltwowidth}{4.9\thinspace\thinspace}\hyperlink{sTextOut}{\XLingPaperdottedtocline{\leveltwoindent}{\leveltwowidth}{4.9 The {\LangtToolFontFamily{\textbf{\textcolor[rgb]{0,0,0.5019607843137255}{Text Out Rules Tool}}}}}{??}
}\settowidth{\leveltwoindent}{4\ }\settowidth{\leveltwowidth}{4.10\thinspace\thinspace}\hyperlink{sReplEditTool}{\XLingPaperdottedtocline{\leveltwoindent}{\leveltwowidth}{4.10 The {\LangtToolFontFamily{\textbf{\textcolor[rgb]{0,0,0.5019607843137255}{Replacement Dictionary Editor}}}}}{??}
}\settowidth{\leveloneindent}{}\settowidth{\levelonewidth}{5\thinspace\thinspace}\hyperlink{sCluster}{\XLingPaperdottedtocline{\leveloneindent}{\levelonewidth}{5 Cluster Project Support}{??}
}\settowidth{\leveltwoindent}{5\ }\settowidth{\leveltwowidth}{5.1\thinspace\thinspace}\hyperlink{sClustExisting}{\XLingPaperdottedtocline{\leveltwoindent}{\leveltwowidth}{5.1 Cluster Support in Existing Tools}{??}
}\settowidth{\leveltwoindent}{5\ }\settowidth{\leveltwowidth}{5.2\thinspace\thinspace}\hyperlink{sClustNew}{\XLingPaperdottedtocline{\leveltwoindent}{\leveltwowidth}{5.2 Cluster Support in New Tools}{??}
}\settowidth{\levelthreeindent}{5\ 5.2\ }\settowidth{\levelthreewidth}{5.2.1\thinspace\thinspace}\hyperlink{sOpenMultProjs}{\XLingPaperdottedtocline{\levelthreeindent}{\levelthreewidth}{5.2.1 Open Multiple {\LangtToolFontFamily{\textbf{\textcolor[rgb]{0,0,0.5019607843137255}{FLEx}}}} Projects Tool}{??}
}\settowidth{\levelthreeindent}{5\ 5.2\ }\settowidth{\levelthreewidth}{5.2.2\thinspace\thinspace}\hyperlink{sRestoreProjs}{\XLingPaperdottedtocline{\levelthreeindent}{\levelthreewidth}{5.2.2 Restore Multiple {\LangtToolFontFamily{\textbf{\textcolor[rgb]{0,0,0.5019607843137255}{FLEx}}}} Projects Tool}{??}
}\settowidth{\levelthreeindent}{5\ 5.2\ }\settowidth{\levelthreewidth}{5.2.3\thinspace\thinspace}\hyperlink{sFixClusterProjs}{\XLingPaperdottedtocline{\levelthreeindent}{\levelthreewidth}{5.2.3 Fix {\LangtToolFontFamily{\textbf{\textcolor[rgb]{0,0,0.5019607843137255}{FLEx}}}} Projects Tool}{??}
}\settowidth{\levelthreeindent}{5\ 5.2\ }\settowidth{\levelthreewidth}{5.2.4\thinspace\thinspace}\hyperlink{sAddClusterAdHoc}{\XLingPaperdottedtocline{\levelthreeindent}{\levelthreewidth}{5.2.4 Add Ad Hoc Constraint for a Cluster Tool}{??}
}\settowidth{\leveloneindent}{}\settowidth{\levelonewidth}{6\thinspace\thinspace}\hyperlink{sTransferTutorial}{\XLingPaperdottedtocline{\leveloneindent}{\levelonewidth}{6 A Tutorial on Writing Transfer Rules}{??}
}\settowidth{\leveltwoindent}{6\ }\settowidth{\leveltwowidth}{6.1\thinspace\thinspace}\hyperlink{sOverview}{\XLingPaperdottedtocline{\leveltwoindent}{\leveltwowidth}{6.1 Overview}{??}
}\settowidth{\levelthreeindent}{6\ 6.1\ }\settowidth{\levelthreewidth}{6.1.1\thinspace\thinspace}\hyperlink{sFormalities}{\XLingPaperdottedtocline{\levelthreeindent}{\levelthreewidth}{6.1.1 Some Formalities}{??}
}\settowidth{\levelthreeindent}{6\ 6.1\ }\settowidth{\levelthreewidth}{6.1.2\thinspace\thinspace}\hyperlink{sApproach}{\XLingPaperdottedtocline{\levelthreeindent}{\levelthreewidth}{6.1.2 Approaching the Process of Writing Transfer Rules}{??}
}\settowidth{\levelthreeindent}{6\ 6.1\ }\settowidth{\levelthreewidth}{6.1.3\thinspace\thinspace}\hyperlink{sLexicalStruct}{\XLingPaperdottedtocline{\levelthreeindent}{\levelthreewidth}{6.1.3 Lexical Transfer and Structural Transfer}{??}
}\settowidth{\levelfourindent}{6\ 6.1\ 6.1.3\ }\settowidth{\levelfourwidth}{6.1.3.1\thinspace\thinspace}\hyperlink{sLexTransProc}{\XLingPaperdottedtocline{\levelfourindent}{\levelfourwidth}{6.1.3.1 How Lexical Transfer is Processed}{??}
}\settowidth{\levelthreeindent}{6\ 6.1\ }\settowidth{\levelthreewidth}{6.1.4\thinspace\thinspace}\hyperlink{sPrelim}{\XLingPaperdottedtocline{\levelthreeindent}{\levelthreewidth}{6.1.4 Some Preliminaries}{??}
}\settowidth{\levelthreeindent}{6\ 6.1\ }\settowidth{\levelthreewidth}{6.1.5\thinspace\thinspace}\hyperlink{sFileOverview}{\XLingPaperdottedtocline{\levelthreeindent}{\levelthreewidth}{6.1.5 Overview of a Transfer File}{??}
}\settowidth{\levelthreeindent}{6\ 6.1\ }\settowidth{\levelthreewidth}{6.1.6\thinspace\thinspace}\hyperlink{sRulesApplied}{\XLingPaperdottedtocline{\levelthreeindent}{\levelthreewidth}{6.1.6 How Rules are Applied}{??}
}\settowidth{\leveltwoindent}{6\ }\settowidth{\leveltwowidth}{6.2\thinspace\thinspace}\hyperlink{sPracticalEx}{\XLingPaperdottedtocline{\leveltwoindent}{\leveltwowidth}{6.2 Practical Example}{??}
}\settowidth{\levelthreeindent}{6\ 6.2\ }\settowidth{\levelthreewidth}{6.2.1\thinspace\thinspace}\hyperlink{sTutSetup}{\XLingPaperdottedtocline{\levelthreeindent}{\levelthreewidth}{6.2.1 Getting Set Up}{??}
}\settowidth{\levelthreeindent}{6\ 6.2\ }\settowidth{\levelthreewidth}{6.2.2\thinspace\thinspace}\hyperlink{sLexicalTransfer}{\XLingPaperdottedtocline{\levelthreeindent}{\levelthreewidth}{6.2.2 Lexical Transfer}{??}
}\settowidth{\levelthreeindent}{6\ 6.2\ }\settowidth{\levelthreewidth}{6.2.3\thinspace\thinspace}\hyperlink{sThinking}{\XLingPaperdottedtocline{\levelthreeindent}{\levelthreewidth}{6.2.3 Thinking it Through}{??}
}\settowidth{\levelfourindent}{6\ 6.2\ 6.2.3\ }\settowidth{\levelfourwidth}{6.2.3.1\thinspace\thinspace}\hyperlink{sProc}{\XLingPaperdottedtocline{\levelfourindent}{\levelfourwidth}{6.2.3.1 Procedures}{??}
}\settowidth{\levelfourindent}{6\ 6.2\ 6.2.3\ }\settowidth{\levelfourwidth}{6.2.3.2\thinspace\thinspace}\hyperlink{sDecl}{\XLingPaperdottedtocline{\levelfourindent}{\levelfourwidth}{6.2.3.2 Declarations}{??}
}\settowidth{\levelfourindent}{6\ 6.2\ 6.2.3\ }\settowidth{\levelfourwidth}{6.2.3.3\thinspace\thinspace}\hyperlink{sWorkOrder}{\XLingPaperdottedtocline{\levelfourindent}{\levelfourwidth}{6.2.3.3 Work Order}{??}
}\settowidth{\levelfourindent}{6\ 6.2\ 6.2.3\ }\settowidth{\levelfourwidth}{6.2.3.4\thinspace\thinspace}\hyperlink{sCheatsheet}{\XLingPaperdottedtocline{\levelfourindent}{\levelfourwidth}{6.2.3.4 Cheat Sheet}{??}
}\settowidth{\levelthreeindent}{6\ 6.2\ }\settowidth{\levelthreewidth}{6.2.4\thinspace\thinspace}\hyperlink{sImplementation}{\XLingPaperdottedtocline{\levelthreeindent}{\levelthreewidth}{6.2.4 Implementation}{??}
}\settowidth{\levelfourindent}{6\ 6.2\ 6.2.4\ }\settowidth{\levelfourwidth}{6.2.4.1\thinspace\thinspace}\hyperlink{step1}{\XLingPaperdottedtocline{\levelfourindent}{\levelfourwidth}{6.2.4.1 Step 1}{??}
}\settowidth{\levelfourindent}{6\ 6.2\ 6.2.4\ }\settowidth{\levelfourwidth}{6.2.4.2\thinspace\thinspace}\hyperlink{step2}{\XLingPaperdottedtocline{\levelfourindent}{\levelfourwidth}{6.2.4.2 Step 2}{??}
}\settowidth{\levelfourindent}{6\ 6.2\ 6.2.4\ }\settowidth{\levelfourwidth}{6.2.4.3\thinspace\thinspace}\hyperlink{step3}{\XLingPaperdottedtocline{\levelfourindent}{\levelfourwidth}{6.2.4.3 Step 3}{??}
}\settowidth{\levelfourindent}{6\ 6.2\ 6.2.4\ }\settowidth{\levelfourwidth}{6.2.4.4\thinspace\thinspace}\hyperlink{step4}{\XLingPaperdottedtocline{\levelfourindent}{\levelfourwidth}{6.2.4.4 Step 4}{??}
}\settowidth{\levelfourindent}{6\ 6.2\ 6.2.4\ }\settowidth{\levelfourwidth}{6.2.4.5\thinspace\thinspace}\hyperlink{step5}{\XLingPaperdottedtocline{\levelfourindent}{\levelfourwidth}{6.2.4.5 Step 5}{??}
}\settowidth{\levelfourindent}{6\ 6.2\ 6.2.4\ }\settowidth{\levelfourwidth}{6.2.4.6\thinspace\thinspace}\hyperlink{step6}{\XLingPaperdottedtocline{\levelfourindent}{\levelfourwidth}{6.2.4.6 Step 6}{??}
}\settowidth{\leveltwoindent}{6\ }\settowidth{\leveltwowidth}{6.3\thinspace\thinspace}\hyperlink{sBestPractices}{\XLingPaperdottedtocline{\leveltwoindent}{\leveltwowidth}{6.3 Best Practices}{??}
}\settowidth{\leveloneindent}{}\settowidth{\levelonewidth}{7\thinspace\thinspace}\hyperlink{sTestbed}{\XLingPaperdottedtocline{\leveloneindent}{\levelonewidth}{7 The {\LangtToolFontFamily{\textbf{\textcolor[rgb]{0,0,0.5019607843137255}{FLExTrans}}}} {\LangtToolFontFamily{\textbf{\textcolor[rgb]{0,0,0.5019607843137255}{Testbed}}}}}{??}
}\settowidth{\leveltwoindent}{7\ }\settowidth{\leveltwowidth}{7.1\thinspace\thinspace}\hyperlink{sTestbedWorkflow}{\XLingPaperdottedtocline{\leveltwoindent}{\leveltwowidth}{7.1 Workflow with the {\LangtToolFontFamily{\textbf{\textcolor[rgb]{0,0,0.5019607843137255}{Testbed}}}}}{??}
}\settowidth{\leveltwoindent}{7\ }\settowidth{\leveltwowidth}{7.2\thinspace\thinspace}\hyperlink{sTestbedTools}{\XLingPaperdottedtocline{\leveltwoindent}{\leveltwowidth}{7.2 {\LangtToolFontFamily{\textbf{\textcolor[rgb]{0,0,0.5019607843137255}{Testbed}}}} Tools}{??}
}\settowidth{\levelthreeindent}{7\ 7.2\ }\settowidth{\levelthreewidth}{7.2.1\thinspace\thinspace}\hyperlink{sTestbedLogViewer}{\XLingPaperdottedtocline{\levelthreeindent}{\levelthreewidth}{7.2.1 {\LangtToolFontFamily{\textbf{\textcolor[rgb]{0,0,0.5019607843137255}{Testbed Log Viewer}}}}}{??}
}\settowidth{\levelthreeindent}{7\ 7.2\ }\settowidth{\levelthreewidth}{7.2.2\thinspace\thinspace}\hyperlink{sStartTestbed}{\XLingPaperdottedtocline{\levelthreeindent}{\levelthreewidth}{7.2.2 {\LangtToolFontFamily{\textbf{\textcolor[rgb]{0,0,0.5019607843137255}{Start Testbed}}}}}{??}
}\settowidth{\levelthreeindent}{7\ 7.2\ }\settowidth{\levelthreewidth}{7.2.3\thinspace\thinspace}\hyperlink{sEndTestbed}{\XLingPaperdottedtocline{\levelthreeindent}{\levelthreewidth}{7.2.3 {\LangtToolFontFamily{\textbf{\textcolor[rgb]{0,0,0.5019607843137255}{End Testbed}}}}}{??}
}\settowidth{\levelthreeindent}{7\ 7.2\ }\settowidth{\levelthreewidth}{7.2.4\thinspace\thinspace}\hyperlink{sTestbedLRT}{\XLingPaperdottedtocline{\levelthreeindent}{\levelthreewidth}{7.2.4 {\LangtToolFontFamily{\textbf{\textcolor[rgb]{0,0,0.5019607843137255}{Live Rule Tester Tool}}}}}{??}
}\settowidth{\levelthreeindent}{7\ 7.2\ }\settowidth{\levelthreewidth}{7.2.5\thinspace\thinspace}\hyperlink{sTestbedEditor}{\XLingPaperdottedtocline{\levelthreeindent}{\levelthreewidth}{7.2.5 Testbed Editing}{??}
}\settowidth{\leveloneindent}{}\settowidth{\levelonewidth}{8\thinspace\thinspace}\hyperlink{sHowTo}{\XLingPaperdottedtocline{\leveloneindent}{\levelonewidth}{8 {\LangtToolFontFamily{\textbf{\textcolor[rgb]{0,0,0.5019607843137255}{FLExTrans}}}} How To’s}{??}
}\settowidth{\leveltwoindent}{8\ }\settowidth{\leveltwowidth}{8.1\thinspace\thinspace}\hyperlink{sOneVerse}{\XLingPaperdottedtocline{\leveltwoindent}{\leveltwowidth}{8.1 Complete Process for Translating One Verse (video)}{??}
}\settowidth{\leveltwoindent}{8\ }\settowidth{\leveltwowidth}{8.2\thinspace\thinspace}\hyperlink{sTransferRuleHowTos}{\XLingPaperdottedtocline{\leveltwoindent}{\leveltwowidth}{8.2 Transfer Rule How To’s}{??}
}\settowidth{\levelthreeindent}{8\ 8.2\ }\settowidth{\levelthreewidth}{8.2.1\thinspace\thinspace}\hyperlink{sAffixRef}{\XLingPaperdottedtocline{\levelthreeindent}{\levelthreewidth}{8.2.1 How do I refer to affix glosses in transfer rules?}{??}
}\settowidth{\levelthreeindent}{8\ 8.2\ }\settowidth{\levelthreewidth}{8.2.2\thinspace\thinspace}\hyperlink{sCategory}{\XLingPaperdottedtocline{\levelthreeindent}{\levelthreewidth}{8.2.2 How do I use the category element in the transfer rules?}{??}
}\settowidth{\levelthreeindent}{8\ 8.2\ }\settowidth{\levelthreewidth}{8.2.3\thinspace\thinspace}\hyperlink{sAttribute}{\XLingPaperdottedtocline{\levelthreeindent}{\levelthreewidth}{8.2.3 How do I use the attribute element in the transfer rules?}{??}
}\settowidth{\levelthreeindent}{8\ 8.2\ }\settowidth{\levelthreewidth}{8.2.4\thinspace\thinspace}\hyperlink{sPatternMatch}{\XLingPaperdottedtocline{\levelthreeindent}{\levelthreewidth}{8.2.4 How does pattern matching work?}{??}
}\settowidth{\levelthreeindent}{8\ 8.2\ }\settowidth{\levelthreewidth}{8.2.5\thinspace\thinspace}\hyperlink{sDeleteWord}{\XLingPaperdottedtocline{\levelthreeindent}{\levelthreewidth}{8.2.5 How do I delete a word from the target output? (How do I prevent a source word from being transferred to the target text?)}{??}
}\settowidth{\levelthreeindent}{8\ 8.2\ }\settowidth{\levelthreewidth}{8.2.6\thinspace\thinspace}\hyperlink{sInsertWord}{\XLingPaperdottedtocline{\levelthreeindent}{\levelthreewidth}{8.2.6 How do I insert a word into the target output?}{??}
}\settowidth{\levelthreeindent}{8\ 8.2\ }\settowidth{\levelthreewidth}{8.2.7\thinspace\thinspace}\hyperlink{sDeleteAffix}{\XLingPaperdottedtocline{\levelthreeindent}{\levelthreewidth}{8.2.7 How do I delete an affix?}{??}
}\settowidth{\levelthreeindent}{8\ 8.2\ }\settowidth{\levelthreewidth}{8.2.8\thinspace\thinspace}\hyperlink{sCondLogic}{\XLingPaperdottedtocline{\levelthreeindent}{\levelthreewidth}{8.2.8 How do I use conditional logic in a transfer rule? (How do I say if this ... then that?) (video)}{??}
}\settowidth{\levelthreeindent}{8\ 8.2\ }\settowidth{\levelthreewidth}{8.2.9\thinspace\thinspace}\hyperlink{sMacro}{\XLingPaperdottedtocline{\levelthreeindent}{\levelthreewidth}{8.2.9 How do I use a macro? (How can I repeat rule statements in multiple places?) (video)}{??}
}\settowidth{\levelthreeindent}{8\ 8.2\ }\settowidth{\levelthreewidth}{8.2.10\thinspace\thinspace}\hyperlink{sAgreement}{\XLingPaperdottedtocline{\levelthreeindent}{\levelthreewidth}{8.2.10 How do I handle noun agreement? (How do I get a target word or words to agree with a noun?) (video)}{??}
}\settowidth{\levelthreeindent}{8\ 8.2\ }\settowidth{\levelthreewidth}{8.2.11\thinspace\thinspace}\hyperlink{sSampleLogic}{\XLingPaperdottedtocline{\levelthreeindent}{\levelthreewidth}{8.2.11 How do I use the Sample logic that’s in the transfer rules file?}{??}
}\settowidth{\levelthreeindent}{8\ 8.2\ }\settowidth{\levelthreewidth}{8.2.12\thinspace\thinspace}\hyperlink{sSpecialEl}{\XLingPaperdottedtocline{\levelthreeindent}{\levelthreewidth}{8.2.12 How do I use special elements in my rules?}{??}
}\settowidth{\levelfourindent}{8\ 8.2\ 8.2.12\ }\settowidth{\levelfourwidth}{8.2.12.1\thinspace\thinspace}\hyperlink{sAnd}{\XLingPaperdottedtocline{\levelfourindent}{\levelfourwidth}{8.2.12.1 {\LangtRuleElemInXXEFontFamily{{\fontspec[Scale=0.8]{Arial}\textcolor[rgb]{0,0.4,0.2}{\textbf{and}}}}}}{??}
}\settowidth{\levelfourindent}{8\ 8.2\ 8.2.12\ }\settowidth{\levelfourwidth}{8.2.12.2\thinspace\thinspace}\hyperlink{sApend}{\XLingPaperdottedtocline{\levelfourindent}{\levelfourwidth}{8.2.12.2 {\LangtRuleElemInXXEFontFamily{{\fontspec[Scale=0.8]{Arial}\textcolor[rgb]{0,0.4,0.2}{\textbf{append to}}}}}}{??}
}\settowidth{\levelfourindent}{8\ 8.2\ 8.2.12\ }\settowidth{\levelfourwidth}{8.2.12.3\thinspace\thinspace}\hyperlink{sBeginsWith}{\XLingPaperdottedtocline{\levelfourindent}{\levelfourwidth}{8.2.12.3 {\LangtRuleElemInXXEFontFamily{{\fontspec[Scale=0.8]{Arial}\textcolor[rgb]{0,0.4,0.2}{\textbf{begins with}}}}}}{??}
}\settowidth{\levelfourindent}{8\ 8.2\ 8.2.12\ }\settowidth{\levelfourwidth}{8.2.12.4\thinspace\thinspace}\hyperlink{sBeginsWithList}{\XLingPaperdottedtocline{\levelfourindent}{\levelfourwidth}{8.2.12.4 {\LangtRuleElemInXXEFontFamily{{\fontspec[Scale=0.8]{Arial}\textcolor[rgb]{0,0.4,0.2}{\textbf{begins with something in list}}}}}}{??}
}\settowidth{\levelfourindent}{8\ 8.2\ 8.2.12\ }\settowidth{\levelfourwidth}{8.2.12.5\thinspace\thinspace}\hyperlink{sConcat}{\XLingPaperdottedtocline{\levelfourindent}{\levelfourwidth}{8.2.12.5 {\LangtRuleElemInXXEFontFamily{{\fontspec[Scale=0.8]{Arial}\textcolor[rgb]{0,0.4,0.2}{\textbf{concat}}}}}}{??}
}\settowidth{\levelfourindent}{8\ 8.2\ 8.2.12\ }\settowidth{\levelfourwidth}{8.2.12.6\thinspace\thinspace}\hyperlink{sContainsSub}{\XLingPaperdottedtocline{\levelfourindent}{\levelfourwidth}{8.2.12.6 {\LangtRuleElemInXXEFontFamily{{\fontspec[Scale=0.8]{Arial}\textcolor[rgb]{0,0.4,0.2}{\textbf{contains substring}}}}}}{??}
}\settowidth{\levelfourindent}{8\ 8.2\ 8.2.12\ }\settowidth{\levelfourwidth}{8.2.12.7\thinspace\thinspace}\hyperlink{sCaseOf}{\XLingPaperdottedtocline{\levelfourindent}{\levelfourwidth}{8.2.12.7 {\LangtRuleElemInXXEFontFamily{{\fontspec[Scale=0.8]{Arial}\textcolor[rgb]{0,0.4,0.2}{\textbf{case of}}}}}}{??}
}\settowidth{\levelfourindent}{8\ 8.2\ 8.2.12\ }\settowidth{\levelfourwidth}{8.2.12.8\thinspace\thinspace}\hyperlink{sEndsWith}{\XLingPaperdottedtocline{\levelfourindent}{\levelfourwidth}{8.2.12.8 {\LangtRuleElemInXXEFontFamily{{\fontspec[Scale=0.8]{Arial}\textcolor[rgb]{0,0.4,0.2}{\textbf{ends with}}}}}}{??}
}\settowidth{\levelfourindent}{8\ 8.2\ 8.2.12\ }\settowidth{\levelfourwidth}{8.2.12.9\thinspace\thinspace}\hyperlink{sEndsWithList}{\XLingPaperdottedtocline{\levelfourindent}{\levelfourwidth}{8.2.12.9 {\LangtRuleElemInXXEFontFamily{{\fontspec[Scale=0.8]{Arial}\textcolor[rgb]{0,0.4,0.2}{\textbf{ends with something in list}}}}}}{??}
}\settowidth{\levelfourindent}{8\ 8.2\ 8.2.12\ }\settowidth{\levelfourwidth}{8.2.12.10\thinspace\thinspace}\hyperlink{sGetCaseFrom}{\XLingPaperdottedtocline{\levelfourindent}{\levelfourwidth}{8.2.12.10 {\LangtRuleElemInXXEFontFamily{{\fontspec[Scale=0.8]{Arial}\textcolor[rgb]{0,0.4,0.2}{\textbf{get case from item}}}}}}{??}
}\settowidth{\levelfourindent}{8\ 8.2\ 8.2.12\ }\settowidth{\levelfourwidth}{8.2.12.11\thinspace\thinspace}\hyperlink{sIn}{\XLingPaperdottedtocline{\levelfourindent}{\levelfourwidth}{8.2.12.11 {\LangtRuleElemInXXEFontFamily{{\fontspec[Scale=0.8]{Arial}\textcolor[rgb]{0,0.4,0.2}{\textbf{in list}}}}}}{??}
}\settowidth{\levelfourindent}{8\ 8.2\ 8.2.12\ }\settowidth{\levelfourwidth}{8.2.12.12\thinspace\thinspace}\hyperlink{sModifyCase}{\XLingPaperdottedtocline{\levelfourindent}{\levelfourwidth}{8.2.12.12 {\LangtRuleElemInXXEFontFamily{{\fontspec[Scale=0.8]{Arial}\textcolor[rgb]{0,0.4,0.2}{\textbf{modify case}}}}}}{??}
}\settowidth{\levelfourindent}{8\ 8.2\ 8.2.12\ }\settowidth{\levelfourwidth}{8.2.12.13\thinspace\thinspace}\hyperlink{sNot}{\XLingPaperdottedtocline{\levelfourindent}{\levelfourwidth}{8.2.12.13 {\LangtRuleElemInXXEFontFamily{{\fontspec[Scale=0.8]{Arial}\textcolor[rgb]{0,0.4,0.2}{\textbf{not}}}}}}{??}
}\settowidth{\levelfourindent}{8\ 8.2\ 8.2.12\ }\settowidth{\levelfourwidth}{8.2.12.14\thinspace\thinspace}\hyperlink{sOr}{\XLingPaperdottedtocline{\levelfourindent}{\levelfourwidth}{8.2.12.14 {\LangtRuleElemInXXEFontFamily{{\fontspec[Scale=0.8]{Arial}\textcolor[rgb]{0,0.4,0.2}{\textbf{or}}}}}}{??}
}\settowidth{\leveltwoindent}{8\ }\settowidth{\leveltwowidth}{8.3\thinspace\thinspace}\hyperlink{sOther}{\XLingPaperdottedtocline{\leveltwoindent}{\leveltwowidth}{8.3 Other How To’s}{??}
}\settowidth{\levelthreeindent}{8\ 8.3\ }\settowidth{\levelthreewidth}{8.3.1\thinspace\thinspace}\hyperlink{sViewBiling}{\XLingPaperdottedtocline{\levelthreeindent}{\levelthreewidth}{8.3.1 How do I view the bilingual lexicon?}{??}
}\settowidth{\levelthreeindent}{8\ 8.3\ }\settowidth{\levelthreewidth}{8.3.2\thinspace\thinspace}\hyperlink{sSynthesisSelfTest}{\XLingPaperdottedtocline{\levelthreeindent}{\levelthreewidth}{8.3.2 How do I run a synthesis self-test on a text in my Target Project?}{??}
}\settowidth{\levelthreeindent}{8\ 8.3\ }\settowidth{\levelthreewidth}{8.3.3\thinspace\thinspace}\hyperlink{sRepl}{\XLingPaperdottedtocline{\levelthreeindent}{\levelthreewidth}{8.3.3 I have a source word that maps to two different target words depending on the inflection of the source word. How do I handle that?}{??}
}\settowidth{\levelthreeindent}{8\ 8.3\ }\settowidth{\levelthreewidth}{8.3.4\thinspace\thinspace}\hyperlink{sInflVariantTgt}{\XLingPaperdottedtocline{\levelthreeindent}{\levelthreewidth}{8.3.4 How do I handle inflectional variants in the target language? (video)}{??}
}\settowidth{\levelthreeindent}{8\ 8.3\ }\settowidth{\levelthreewidth}{8.3.5\thinspace\thinspace}\hyperlink{sInflVariantSrc}{\XLingPaperdottedtocline{\levelthreeindent}{\levelthreewidth}{8.3.5 How do I handle inflectional variants in the source language? (video)}{??}
}\settowidth{\levelthreeindent}{8\ 8.3\ }\settowidth{\levelthreewidth}{8.3.6\thinspace\thinspace}\hyperlink{sPhrasalVerbs}{\XLingPaperdottedtocline{\levelthreeindent}{\levelthreewidth}{8.3.6 How do I deal with phrasal verbs using {\LangtToolFontFamily{\textbf{\textcolor[rgb]{0,0,0.5019607843137255}{FLExTrans}}}}?}{??}
}\settowidth{\levelfourindent}{8\ 8.3\ 8.3.6\ }\settowidth{\levelfourwidth}{8.3.6.1\thinspace\thinspace}\hyperlink{sPVSrcIsPhrasal}{\XLingPaperdottedtocline{\levelfourindent}{\levelfourwidth}{8.3.6.1 The source uses a phrasal verb, but the target uses a normal verb.}{??}
}\settowidth{\levelfourindent}{8\ 8.3\ 8.3.6\ }\settowidth{\levelfourwidth}{8.3.6.2\thinspace\thinspace}\hyperlink{sPVTgtIsPhrasal}{\XLingPaperdottedtocline{\levelfourindent}{\levelfourwidth}{8.3.6.2 The source uses a normal verb, but the target uses a phrasal verb.}{??}
}\settowidth{\levelfourindent}{8\ 8.3\ 8.3.6\ }\settowidth{\levelfourwidth}{8.3.6.3\thinspace\thinspace}\hyperlink{sPVBothPhrasal}{\XLingPaperdottedtocline{\levelfourindent}{\levelfourwidth}{8.3.6.3 The source uses a phrasal verb and the target also uses a phrasal verb (and the components don't match).}{??}
}\settowidth{\levelthreeindent}{8\ 8.3\ }\settowidth{\levelthreewidth}{8.3.7\thinspace\thinspace}\hyperlink{sLinkDup}{\XLingPaperdottedtocline{\levelthreeindent}{\levelthreewidth}{8.3.7 I just copied my source {\LangtToolFontFamily{\textbf{\textcolor[rgb]{0,0,0.5019607843137255}{FLEx}}}} project to become my target {\LangtToolFontFamily{\textbf{\textcolor[rgb]{0,0,0.5019607843137255}{FLEx}}}} project. Is there a way to link all the senses?}{??}
}\settowidth{\leveloneindent}{}\settowidth{\levelonewidth}{9\thinspace\thinspace}\hyperlink{sTroubleshooting}{\XLingPaperdottedtocline{\leveloneindent}{\levelonewidth}{9 Troubleshooting}{??}
}\settowidth{\leveltwoindent}{9\ }\settowidth{\leveltwowidth}{9.1\thinspace\thinspace}\hyperlink{sSynFail}{\XLingPaperdottedtocline{\leveltwoindent}{\leveltwowidth}{9.1 Synthesis Troubleshooting}{??}
}\settowidth{\leveltwoindent}{9\ }\settowidth{\leveltwowidth}{9.2\thinspace\thinspace}\hyperlink{sErrorCond}{\XLingPaperdottedtocline{\leveltwoindent}{\leveltwowidth}{9.2 Error Conditions}{??}
}\settowidth{\leveltwoindent}{9\ }\settowidth{\leveltwowidth}{9.3\thinspace\thinspace}\hyperlink{sOtherTips}{\XLingPaperdottedtocline{\leveltwoindent}{\leveltwowidth}{9.3 Other Troubleshooting Hints}{??}
}\settowidth{\leveloneindent}{}\settowidth{\levelonewidth}{10\thinspace\thinspace}\hyperlink{sLocalization}{\XLingPaperdottedtocline{\leveloneindent}{\levelonewidth}{10 User Interface Languages}{??}
}\settowidth{\leveltwoindent}{10\ }\settowidth{\leveltwowidth}{10.1\thinspace\thinspace}\hyperlink{sChangingLanguages}{\XLingPaperdottedtocline{\leveltwoindent}{\leveltwowidth}{10.1 Changing The {\LangtToolFontFamily{\textbf{\textcolor[rgb]{0,0,0.5019607843137255}{FLExTrans}}}} User Interface Language}{??}
}\settowidth{\levelthreeindent}{10\ 10.1\ }\settowidth{\levelthreewidth}{10.1.1\thinspace\thinspace}\hyperlink{sChangeFromEnglish}{\XLingPaperdottedtocline{\levelthreeindent}{\levelthreewidth}{10.1.1 Changing From English}{??}
}\settowidth{\levelthreeindent}{10\ 10.1\ }\settowidth{\levelthreewidth}{10.1.2\thinspace\thinspace}\hyperlink{sChangeOtherLanguages}{\XLingPaperdottedtocline{\levelthreeindent}{\levelthreewidth}{10.1.2 Changing From Other Language}{??}
}\settowidth{\leveltwoindent}{10\ }\settowidth{\leveltwowidth}{10.2\thinspace\thinspace}\hyperlink{sChangingXXELanguage}{\XLingPaperdottedtocline{\leveltwoindent}{\leveltwowidth}{10.2 Change User Interface Language of XMLmind XML Editor}{??}
}\settowidth{\leveloneindent}{}\settowidth{\levelonewidth}{11\thinspace\thinspace}\hyperlink{sReferenceDocs}{\XLingPaperdottedtocline{\leveloneindent}{\levelonewidth}{11 Reference documents}{??}
}\hyperlink{aAppend}{\XLingPaperdottedtocline{0pt}{0pt}{A Appendix}{??}
}\settowidth{\leveloneindent}{A\ }\settowidth{\levelonewidth}{A.1\thinspace\thinspace}\hyperlink{sStartFlextools}{\XLingPaperdottedtocline{\leveloneindent}{\levelonewidth}{A.1 Starting {\LangtToolFontFamily{\textbf{\textcolor[rgb]{0,0,0.5019607843137255}{FLExTools}}}}}{??}
}\settowidth{\leveloneindent}{A\ }\settowidth{\levelonewidth}{A.2\thinspace\thinspace}\hyperlink{sDataStreamFormat}{\XLingPaperdottedtocline{\leveloneindent}{\levelonewidth}{A.2 Data Stream Format}{??}
}\hyperlink{rXLingPapReferences}{\XLingPaperdottedtocline{0pt}{0pt}{References}{??}
}
\vspace{12pt plus 2pt minus 1pt}
\XLingPaperneedspace{5\baselineskip}

\penalty-3000{{\centering{\raisebox{\baselineskip}[0pt]{\protect\hypertarget{rXLingPapAbstract0}{}}\large{\raisebox{\baselineskip}[0pt]{\pdfbookmark[1]{Abstract}{rXLingPapAbstract0}}\textbf{Abstract\\}}}
}\markright{Abstract}
\XLingPaperaddtocontents{rXLingPapAbstract0}}\par{}\penalty10000
\vspace{12pt plus 2pt minus 1pt}
\indent This document describes how to use the {\LangtToolFontFamily{\textbf{\textcolor[rgb]{0,0,0.5019607843137255}{FLExTrans}}}} machine translation system. See section \hyperlink{sReferenceDocs}{11} for other documentation.\par{}
\vspace{12pt}
\XLingPaperneedspace{5\baselineskip}

\penalty-3000{{\centering\raisebox{\baselineskip}[0pt]{\protect\hypertarget{sIntro}{}}\SectionLevelOneFontFamily{\large{\raisebox{\baselineskip}[0pt]{\pdfbookmark[1]{1 Introduction}{sIntro}}\textbf{1 Introduction}}}\\{}}\markright{Introduction}
\XLingPaperaddtocontents{sIntro}}\par{}\penalty10000
\vspace{12pt}
\indent {\LangtToolFontFamily{\textbf{\textcolor[rgb]{0,0,0.5019607843137255}{FLExTrans}}}} is a rule-based machine translation system. It was first developed in 2015 (\hyperlink{rLockwood15}{Lockwood  2015)}. It marries the power of the {\textcolor[rgb]{0,0,0.4}{\textbf{Apertium}}} (\hyperlink{rApertiumSite}{Apertium,  2021}) transfer engine with the outstanding lexical model and usability of {\LangtToolFontFamily{\textbf{\textcolor[rgb]{0,0,0.5019607843137255}{FLEx}}}}.\par{}\vspace{12pt plus 2pt minus 1pt}\raisebox{\baselineskip}[0pt]{\protect\hypertarget{f1}{}}\XLingPaperaddtocontents{f1}{\protect\centering \leavevmode
\vspace*{0pt}{\XeTeXpicfile "../Images/diagram v3.png" scaled 500}\\\vspace{.3em}{\textbf{Figure 1 }}\\}\vspace{12pt plus 2pt minus 1pt}\indent {\LangtToolFontFamily{\textbf{\textcolor[rgb]{0,0,0.5019607843137255}{FLExTrans}}}} is composed basically of two language databases, a list of transfer rules and a series of programs that do the work. The language databases, one for the source language and one for the target language, are in the form of {\LangtToolFontFamily{\textbf{\textcolor[rgb]{0,0,0.5019607843137255}{FLEx}}}} projects. A {\LangtToolFontFamily{\textbf{\textcolor[rgb]{0,0,0.5019607843137255}{FLEx}}}} project stores, among other things, a lexicon, grammar settings and a database of texts. The transfer rules are in the form of a text file in XML format. The core programs are the analysis engine, the transfer engine and the synthesis engine. A text in the source language first goes through the analysis process where words are broken down into morphemes. Then the “analyzed” text is transformed into target morphemes and rearranged as necessary in the transfer process. Lastly, the target morphemes are put together into target words in the synthesis process. The end result is a text in the target language.\par{}\indent {\LangtToolFontFamily{\textbf{\textcolor[rgb]{0,0,0.5019607843137255}{FLExTrans}}}}, as a program, consists of many different modules that are run within the {\LangtToolFontFamily{\textbf{\textcolor[rgb]{0,0,0.5019607843137255}{FLExTools}}}}\protect\footnote[1]{{\leftskip0pt\parindent1em\raisebox{\baselineskip}[0pt]{\protect\hypertarget{nFlexTools}{}}{\LangtToolFontFamily{\textbf{\textcolor[rgb]{0,0,0.5019607843137255}{FLExTools}}}} is a general purpose program for running python scripts against a {\LangtToolFontFamily{\textbf{\textcolor[rgb]{0,0,0.5019607843137255}{FLEx}}}} database.}} program. Upon installation, {\LangtToolFontFamily{\textbf{\textcolor[rgb]{0,0,0.5019607843137255}{FLExTools}}}} is populated with several collections of modules for common {\LangtToolFontFamily{\textbf{\textcolor[rgb]{0,0,0.5019607843137255}{FLExTrans}}}} tasks. \par{}
\vspace{12pt}
\XLingPaperneedspace{5\baselineskip}

\penalty-3000{{\centering\raisebox{\baselineskip}[0pt]{\protect\hypertarget{sGettingStarted}{}}\SectionLevelOneFontFamily{\large{\raisebox{\baselineskip}[0pt]{\pdfbookmark[1]{2 Overview and Getting Started}{sGettingStarted}}\textbf{2 Overview and Getting Started}}}\\{}}\markright{Overview and Getting Started}
\XLingPaperaddtocontents{sGettingStarted}}\par{}\penalty10000
\vspace{12pt}
\indent After you have installed FLExTrans and performed some one-time preparatory steps on your source and target FLEx projects, the basic steps for machine translation with {\LangtToolFontFamily{\textbf{\textcolor[rgb]{0,0,0.5019607843137255}{FLExTrans}}}} are as follows:\par{}{\parskip .5pt plus 1pt minus 1pt
                    
\vspace{\baselineskip}

{\setlength{\XLingPapertempdim}{\XLingPapersingledigitlistitemwidth+\parindent{}}\leftskip\XLingPapertempdim\relax
\interlinepenalty10000
\XLingPaperlistitem{\parindent{}}{\XLingPapersingledigitlistitemwidth}{1.}{Analyze the text you want to translate in the source {\LangtToolFontFamily{\textbf{\textcolor[rgb]{0,0,0.5019607843137255}{FLEx}}}} project.}}
{\setlength{\XLingPapertempdim}{\XLingPapersingledigitlistitemwidth+\parindent{}}\leftskip\XLingPapertempdim\relax
\interlinepenalty10000
\XLingPaperlistitem{\parindent{}}{\XLingPapersingledigitlistitemwidth}{2.}{Link lexical senses that are in the source text to senses in the target {\LangtToolFontFamily{\textbf{\textcolor[rgb]{0,0,0.5019607843137255}{FLEx}}}} project using the {\LangtToolFontFamily{\textbf{\textcolor[rgb]{0,0,0.5019607843137255}{Sense Linker Tool}}}}}}
{\setlength{\XLingPapertempdim}{\XLingPapersingledigitlistitemwidth+\parindent{}}\leftskip\XLingPapertempdim\relax
\interlinepenalty10000
\XLingPaperlistitem{\parindent{}}{\XLingPapersingledigitlistitemwidth}{3.}{Write transfer rules that convert source words and phrases to target words and phrases.}}
{\setlength{\XLingPapertempdim}{\XLingPapersingledigitlistitemwidth+\parindent{}}\leftskip\XLingPapertempdim\relax
\interlinepenalty10000
\XLingPaperlistitem{\parindent{}}{\XLingPapersingledigitlistitemwidth}{4.}{Run the core {\LangtToolFontFamily{\textbf{\textcolor[rgb]{0,0,0.5019607843137255}{FLExTrans}}}} modules.}}
\vspace{\baselineskip}
}
\vspace{12pt}
\XLingPaperneedspace{5\baselineskip}

\penalty-3000{\noindent{\raisebox{\baselineskip}[0pt]{\protect\hypertarget{sPrepSteps}{}}\SectionLevelTwoFontFamily{\normalsize{\raisebox{\baselineskip}[0pt]{\pdfbookmark[2]{2.1 Preparatory Steps}{sPrepSteps}}\textbf{2.1 Preparatory Steps}}}}
\markright{Preparatory Steps}
\XLingPaperaddtocontents{sPrepSteps}}\par{}\penalty10000
\vspace{12pt}
\indent These steps only need to be performed once, when you are setting up your projects to be used with {\LangtToolFontFamily{\textbf{\textcolor[rgb]{0,0,0.5019607843137255}{FLExTrans}}}}.\par{}\indent For the following steps, it is assumed that you have gone through the \href{https://software.sil.org/flextrans/installation/}{installation} steps as given on the {\LangtToolFontFamily{\textbf{\textcolor[rgb]{0,0,0.5019607843137255}{FLExTrans}}}} website and that you have confirmed that {\LangtToolFontFamily{\textbf{\textcolor[rgb]{0,0,0.5019607843137255}{FLExTrans}}}} works for the sample projects. Be sure that the {\LangtToolFontFamily{\textbf{\textcolor[rgb]{0,0,0.5019607843137255}{FLEx}}}} projects for your source and target languages are on your machine and can be opened by {\LangtToolFontFamily{\textbf{\textcolor[rgb]{0,0,0.5019607843137255}{FLEx}}}}.\par{}
\vspace{12pt}
\XLingPaperneedspace{5\baselineskip}

\penalty-3000{\noindent{\raisebox{\baselineskip}[0pt]{\protect\hypertarget{sProjectSetup}{}}\SectionLevelThreeFontFamily{\normalsize{\raisebox{\baselineskip}[0pt]{\pdfbookmark[3]{2.1.1 Set up your projects}{sProjectSetup}}\textit{2.1.1 Set up your projects}}}}
\markright{Set up your projects}
\XLingPaperaddtocontents{sProjectSetup}}\par{}\penalty10000
\vspace{12pt}
\indent The following steps will prepare your {\LangtToolFontFamily{\textbf{\textcolor[rgb]{0,0,0.5019607843137255}{FLEx}}}} projects, as well as the {\LangtToolFontFamily{\textbf{\textcolor[rgb]{0,0,0.5019607843137255}{FLExTrans}}}} setup for your specific pair of languages.\par{}{\parskip .5pt plus 1pt minus 1pt
                    
\vspace{\baselineskip}

{\setlength{\XLingPapertempdim}{\XLingPapersingledigitlistitemwidth+\parindent{}}\leftskip\XLingPapertempdim\relax
\interlinepenalty10000
\XLingPaperlistitem{\parindent{}}{\XLingPapersingledigitlistitemwidth}{1.}{Enable the sharing option in each of your {\LangtToolFontFamily{\textbf{\textcolor[rgb]{0,0,0.5019607843137255}{FLEx}}}} projects. This will allow you to keep {\LangtToolFontFamily{\textbf{\textcolor[rgb]{0,0,0.5019607843137255}{FLEx}}}} projects open while running {\LangtToolFontFamily{\textbf{\textcolor[rgb]{0,0,0.5019607843137255}{FLExTrans}}}}.}{\setlength{\XLingPaperlistitemindent}{\XLingPapersingledigitlistitemwidth + \parindent{}}
{\setlength{\XLingPapertempdim}{\XLingPaperbulletlistitemwidth+\XLingPaperlistitemindent}\leftskip\XLingPapertempdim\relax
\interlinepenalty10000
\XLingPaperlistitem{\XLingPaperlistitemindent}{\XLingPaperbulletlistitemwidth}{•}{From the menu select {\LangtMenuFontFamily{\textbf{\textcolor[rgb]{0,0.6,0.2}{File}}}} \textgreater{} {\LangtMenuFontFamily{\textbf{\textcolor[rgb]{0,0.6,0.2}{Project Management}}}} \textgreater{} {\LangtMenuFontFamily{\textbf{\textcolor[rgb]{0,0.6,0.2}{FieldWorks Project Properties}}}}}}
{\setlength{\XLingPapertempdim}{\XLingPaperbulletlistitemwidth+\XLingPaperlistitemindent}\leftskip\XLingPapertempdim\relax
\interlinepenalty10000
\XLingPaperlistitem{\XLingPaperlistitemindent}{\XLingPaperbulletlistitemwidth}{•}{Click on the {\LangtMenuFontFamily{\textbf{\textcolor[rgb]{0,0.6,0.2}{Sharing}}}} tab.}}
{\setlength{\XLingPapertempdim}{\XLingPaperbulletlistitemwidth+\XLingPaperlistitemindent}\leftskip\XLingPapertempdim\relax
\interlinepenalty10000
\XLingPaperlistitem{\XLingPaperlistitemindent}{\XLingPaperbulletlistitemwidth}{•}{Check the box that says {\uline{Share project contents with programs on this computer}}}}}}
{\setlength{\XLingPapertempdim}{\XLingPapersingledigitlistitemwidth+\parindent{}}\leftskip\XLingPapertempdim\relax
\interlinepenalty10000
\XLingPaperlistitem{\parindent{}}{\XLingPapersingledigitlistitemwidth}{2.}{Add a custom field to your source {\LangtToolFontFamily{\textbf{\textcolor[rgb]{0,0,0.5019607843137255}{FLEx}}}} project. This is where link information to the target senses is kept.}{\setlength{\XLingPaperlistitemindent}{\XLingPapersingledigitlistitemwidth + \parindent{}}
{\setlength{\XLingPapertempdim}{\XLingPapersingleletterlistitemwidth+\XLingPaperlistitemindent}\leftskip\XLingPapertempdim\relax
\interlinepenalty10000
\XLingPaperlistitem{\XLingPaperlistitemindent}{\XLingPapersingleletterlistitemwidth}{a.}{In your source {\LangtToolFontFamily{\textbf{\textcolor[rgb]{0,0,0.5019607843137255}{FLEx}}}} project, go to the {\textbf{Lexicon}} area}}
{\setlength{\XLingPapertempdim}{\XLingPapersingleletterlistitemwidth+\XLingPaperlistitemindent}\leftskip\XLingPapertempdim\relax
\interlinepenalty10000
\XLingPaperlistitem{\XLingPaperlistitemindent}{\XLingPapersingleletterlistitemwidth}{b.}{From the menu select {\LangtMenuFontFamily{\textbf{\textcolor[rgb]{0,0.6,0.2}{Tools}}}} \textgreater{} {\LangtMenuFontFamily{\textbf{\textcolor[rgb]{0,0.6,0.2}{Configure}}}} \textgreater{} {\LangtMenuFontFamily{\textbf{\textcolor[rgb]{0,0.6,0.2}{Custom Fields}}}}}}
{\setlength{\XLingPapertempdim}{\XLingPapersingleletterlistitemwidth+\XLingPaperlistitemindent}\leftskip\XLingPapertempdim\relax
\interlinepenalty10000
\XLingPaperlistitem{\XLingPaperlistitemindent}{\XLingPapersingleletterlistitemwidth}{c.}{Click Add and use the following settings:}{\setlength{\XLingPaperlistitemindent}{\XLingPapersingleletterlistitemwidth + \XLingPapersingledigitlistitemwidth + \parindent{}}
{\setlength{\XLingPapertempdim}{\XLingPaperromanviilistitemwidth+\XLingPaperlistitemindent}\leftskip\XLingPapertempdim\relax
\interlinepenalty10000
\XLingPaperlistitem{\XLingPaperlistitemindent}{\XLingPaperromanviilistitemwidth}{i.}{Custom Field Name: Target Equivalent \protect\footnote[2]{{\leftskip0pt\parindent1em\raisebox{\baselineskip}[0pt]{\protect\hypertarget{nCustomFieldNames}{}}You may choose a different name for the custom field, but you also have to update the {\LangtToolFontFamily{\textbf{\textcolor[rgb]{0,0,0.5019607843137255}{FLExTrans Settings}}}} to match the name. The name {\textbf{\textcolor[rgb]{0.5882352941176471,0.29411764705882354,0}{Target Equivalent}}} is already defined in the settings that get installed with {\LangtToolFontFamily{\textbf{\textcolor[rgb]{0,0,0.5019607843137255}{FLExTrans}}}}.}}}}
{\setlength{\XLingPapertempdim}{\XLingPaperromanviilistitemwidth+\XLingPaperlistitemindent}\leftskip\XLingPapertempdim\relax
\interlinepenalty10000
\XLingPaperlistitem{\XLingPaperlistitemindent}{\XLingPaperromanviilistitemwidth}{ii.}{Location: {\uline{Sense}}}}
{\setlength{\XLingPapertempdim}{\XLingPaperromanviilistitemwidth+\XLingPaperlistitemindent}\leftskip\XLingPapertempdim\relax
\interlinepenalty10000
\XLingPaperlistitem{\XLingPaperlistitemindent}{\XLingPaperromanviilistitemwidth}{iii.}{Type: {\uline{Single-line Text}}}}
{\setlength{\XLingPapertempdim}{\XLingPaperromanviilistitemwidth+\XLingPaperlistitemindent}\leftskip\XLingPapertempdim\relax
\interlinepenalty10000
\XLingPaperlistitem{\XLingPaperlistitemindent}{\XLingPaperromanviilistitemwidth}{iv.}{Writing Systems(s): {\uline{First Analysis Writing System}}}}}}
{\setlength{\XLingPapertempdim}{\XLingPapersingleletterlistitemwidth+\XLingPaperlistitemindent}\leftskip\XLingPapertempdim\relax
\interlinepenalty10000
\XLingPaperlistitem{\XLingPaperlistitemindent}{\XLingPapersingleletterlistitemwidth}{d.}{Click OK}}}}
{\setlength{\XLingPapertempdim}{\XLingPapersingledigitlistitemwidth+\parindent{}}\leftskip\XLingPapertempdim\relax
\interlinepenalty10000
\XLingPaperlistitem{\parindent{}}{\XLingPapersingledigitlistitemwidth}{3.}{Create a new folder for your language pair.}{\setlength{\XLingPaperlistitemindent}{\XLingPapersingledigitlistitemwidth + \parindent{}}
{\setlength{\XLingPapertempdim}{\XLingPapersingleletterlistitemwidth+\XLingPaperlistitemindent}\leftskip\XLingPapertempdim\relax
\interlinepenalty10000
\XLingPaperlistitem{\XLingPaperlistitemindent}{\XLingPapersingleletterlistitemwidth}{a.}{In File Explorer, go to the {\LangtFoldernameFontFamily{{\fontspec[Scale=0.8]{Tahoma}\textup{\textmd{WorkProjects}}}}} subfolder of the installation folder (typically {\LangtFoldernameFontFamily{{\fontspec[Scale=0.8]{Tahoma}\textup{\textmd{Documents\textbackslash{}FLExTrans}}}}}).}}
{\setlength{\XLingPapertempdim}{\XLingPapersingleletterlistitemwidth+\XLingPaperlistitemindent}\leftskip\XLingPapertempdim\relax
\interlinepenalty10000
\XLingPaperlistitem{\XLingPaperlistitemindent}{\XLingPapersingleletterlistitemwidth}{b.}{Copy the {\LangtFoldernameFontFamily{{\fontspec[Scale=0.8]{Tahoma}\textup{\textmd{TemplateProject}}}}} folder and Paste it where it is. You should see a new folder called {\LangtFoldernameFontFamily{{\fontspec[Scale=0.8]{Tahoma}\textup{\textmd{TemplateProject - Copy}}}}}}}
{\setlength{\XLingPapertempdim}{\XLingPapersingleletterlistitemwidth+\XLingPaperlistitemindent}\leftskip\XLingPapertempdim\relax
\interlinepenalty10000
\XLingPaperlistitem{\XLingPaperlistitemindent}{\XLingPapersingleletterlistitemwidth}{c.}{Rename this folder to the name of language pair you will work with, for example, {\LangtFoldernameFontFamily{{\fontspec[Scale=0.8]{Tahoma}\textup{\textmd{French-Spanish}}}}}.}}}}
{\setlength{\XLingPapertempdim}{\XLingPapersingledigitlistitemwidth+\parindent{}}\leftskip\XLingPapertempdim\relax
\interlinepenalty10000
\XLingPaperlistitem{\parindent{}}{\XLingPapersingledigitlistitemwidth}{4.}{Start {\LangtToolFontFamily{\textbf{\textcolor[rgb]{0,0,0.5019607843137255}{FLExTools}}}}. }{\setlength{\XLingPaperlistitemindent}{\XLingPapersingledigitlistitemwidth + \parindent{}}
{\setlength{\XLingPapertempdim}{\XLingPapersingleletterlistitemwidth+\XLingPaperlistitemindent}\leftskip\XLingPapertempdim\relax
\interlinepenalty10000
\XLingPaperlistitem{\XLingPaperlistitemindent}{\XLingPapersingleletterlistitemwidth}{a.}{Navigate to the folder for your project (typically {\textit{Documents\textbackslash{}FLExTrans\textbackslash{}WorkProjects\textbackslash{}{[}YourProject{]}}}).}}
{\setlength{\XLingPapertempdim}{\XLingPapersingleletterlistitemwidth+\XLingPaperlistitemindent}\leftskip\XLingPapertempdim\relax
\interlinepenalty10000
\XLingPaperlistitem{\XLingPaperlistitemindent}{\XLingPapersingleletterlistitemwidth}{b.}{Double-click on the file {\textbf{FlexTools.vbs}} (a Visual Basic script file).}}
{\setlength{\XLingPapertempdim}{\XLingPapersingleletterlistitemwidth+\XLingPaperlistitemindent}\leftskip\XLingPapertempdim\relax
\interlinepenalty10000
\XLingPaperlistitem{\XLingPaperlistitemindent}{\XLingPapersingleletterlistitemwidth}{c.}{It should look like \hyperlink{xFirstStart}{(1)}.}\par{}{\vspace{12pt plus 2pt minus 1pt}\raggedright{}\XLingPaperexample{.125in}{0pt}{2.75em}{\raisebox{\baselineskip}[0pt]{\protect\hypertarget{xFirstStart}{}}(1)}{\parbox[t]{\textwidth - .125in - 0pt}{\vspace*{-\baselineskip}{\XeTeXpicfile "../Images/FLExTransModules.png" scaled 600}}}
\vspace{12pt plus 2pt minus 1pt}}{}}}}
{\setlength{\XLingPapertempdim}{\XLingPapersingledigitlistitemwidth+\parindent{}}\leftskip\XLingPapertempdim\relax
\interlinepenalty10000
\XLingPaperlistitem{\parindent{}}{\XLingPapersingledigitlistitemwidth}{5.}{Click on the \vspace*{0pt}{\XeTeXpicfile "../Images/DatabaseBut.PNG" scaled 750} button and choose your source database, i.e. {\LangtToolFontFamily{\textbf{\textcolor[rgb]{0,0,0.5019607843137255}{FLEx}}}} project. If you did the installation test, instead of "Choose Project", it will show the most recently used source project \vspace*{0pt}{\XeTeXpicfile "../Images/ChooseProject-German.png" scaled 750} before you click on it.}}
{\setlength{\XLingPapertempdim}{\XLingPapersingledigitlistitemwidth+\parindent{}}\leftskip\XLingPapertempdim\relax
\interlinepenalty10000
\XLingPaperlistitem{\parindent{}}{\XLingPapersingledigitlistitemwidth}{6.}{\raisebox{\baselineskip}[0pt]{\protect\hypertarget{gChangeSetings}{}}Change the settings for your new project.}{\setlength{\XLingPaperlistitemindent}{\XLingPapersingledigitlistitemwidth + \parindent{}}
{\setlength{\XLingPapertempdim}{\XLingPapersingleletterlistitemwidth+\XLingPaperlistitemindent}\leftskip\XLingPapertempdim\relax
\interlinepenalty10000
\XLingPaperlistitem{\XLingPaperlistitemindent}{\XLingPapersingleletterlistitemwidth}{a.}{Select {\LangtMenuFontFamily{\textbf{\textcolor[rgb]{0,0.6,0.2}{FLExTrans}}}} \textgreater{} {\LangtMenuFontFamily{\textbf{\textcolor[rgb]{0,0.6,0.2}{Settings}}}} from the menu.}}
{\setlength{\XLingPapertempdim}{\XLingPapersingleletterlistitemwidth+\XLingPaperlistitemindent}\leftskip\XLingPapertempdim\relax
\interlinepenalty10000
\XLingPaperlistitem{\XLingPaperlistitemindent}{\XLingPapersingleletterlistitemwidth}{b.}{You should see the {\LangtToolFontFamily{\textbf{\textcolor[rgb]{0,0,0.5019607843137255}{FLExTrans Settings}}}} window. Note that this window has three choices for the {\textup{\textbf{View Mode}}}, with the {\textit{Basic}} choice seen here: }\par{}{\vspace{12pt plus 2pt minus 1pt}\raggedright{}\XLingPaperexample{.125in}{0pt}{2.75em}{\raisebox{\baselineskip}[0pt]{\protect\hypertarget{xPrepSettings}{}}(2)}{\parbox[t]{\textwidth - .125in - 0pt}{\vspace*{-\baselineskip}{\XeTeXpicfile "../Images/SettingsWindowBasic.png" scaled 600}}}
\vspace{12pt plus 2pt minus 1pt}}{}}
{\setlength{\XLingPapertempdim}{\XLingPapersingleletterlistitemwidth+\XLingPaperlistitemindent}\leftskip\XLingPapertempdim\relax
\interlinepenalty10000
\XLingPaperlistitem{\XLingPaperlistitemindent}{\XLingPapersingleletterlistitemwidth}{c.}{Change these two settings:{
\XLingPaperminmaxcellincolumn{Setting}{\XLingPapermincola}{\textbf{Setting}}{\XLingPapermaxcola}{+0\tabcolsep}
\XLingPaperminmaxcellincolumn{Change}{\XLingPapermincolb}{\textbf{Change to}}{\XLingPapermaxcolb}{+0\tabcolsep}
\XLingPaperminmaxcellincolumn{Source}{\XLingPapermincola}{{\textbf{\textcolor[rgb]{0.5882352941176471,0.29411764705882354,0}{Source Text Name}}}}{\XLingPapermaxcola}{+0\tabcolsep}
\XLingPaperminmaxcellincolumn{translate}{\XLingPapermincolb}{{\textit{The title of the text you want to translate (from the lists of texts in the source }}{\LangtToolFontFamily{\textbf{\textcolor[rgb]{0,0,0.5019607843137255}{FLEx}}}} {\textit{project).}}}{\XLingPapermaxcolb}{+0\tabcolsep}
\XLingPaperminmaxcellincolumn{Project}{\XLingPapermincola}{{\textbf{\textcolor[rgb]{0.5882352941176471,0.29411764705882354,0}{Target Project}}}}{\XLingPapermaxcola}{+0\tabcolsep}
\XLingPaperminmaxcellincolumn{project}{\XLingPapermincolb}{{\textit{The name of your target }}{\LangtToolFontFamily{\textbf{\textcolor[rgb]{0,0,0.5019607843137255}{FLEx}}}}{\textit{ project (from the list of {\LangtToolFontFamily{\textbf{\textcolor[rgb]{0,0,0.5019607843137255}{FLEx}}}} projects).}}}{\XLingPapermaxcolb}{+0\tabcolsep}
\setlength{\XLingPaperavailabletablewidth}{433.62pt}
\setlength{\XLingPapertableminwidth}{\XLingPapermincola+\XLingPapermincolb}
\setlength{\XLingPapertablemaxwidth}{\XLingPapermaxcola+\XLingPapermaxcolb}
\XLingPapercalculatetablewidthratio{}
\XLingPapersetcolumnwidth{\XLingPapercolawidth}{\XLingPapermincola}{\XLingPapermaxcola}{-0\tabcolsep}
\XLingPapersetcolumnwidth{\XLingPapercolbwidth}{\XLingPapermincolb}{\XLingPapermaxcolb}{-2\tabcolsep}\vspace*{-\baselineskip}
\begin{longtable}
[l]{@{}>{\raggedright}p{\XLingPapercolawidth}>{\raggedright}p{\XLingPapercolbwidth}@{}}\specialrule{\heavyrulewidth}{4\aboverulesep}{\belowrulesep}\multicolumn{1}{@{}>{\raggedright}p{\XLingPapercolawidth}}{\textbf{Setting}}&\multicolumn{1}{>{\raggedright}p{\XLingPapercolbwidth}@{}}{\textbf{Change to}}\\%
\midrule\endhead \multicolumn{1}{@{}>{\raggedright}p{\XLingPapercolawidth}}{{\textbf{\textcolor[rgb]{0.5882352941176471,0.29411764705882354,0}{Source Text Name}}}}&\multicolumn{1}{>{\raggedright}p{\XLingPapercolbwidth}@{}}{{\textit{The title of the text you want to translate (from the lists of texts in the source }}{\LangtToolFontFamily{\textbf{\textcolor[rgb]{0,0,0.5019607843137255}{FLEx}}}} {\textit{project).}}}\\%
\multicolumn{1}{@{}>{\raggedright}p{\XLingPapercolawidth}}{{\textbf{\textcolor[rgb]{0.5882352941176471,0.29411764705882354,0}{Target Project}}}}&\multicolumn{1}{>{\raggedright}p{\XLingPapercolbwidth}@{}}{{\textit{The name of your target }}{\LangtToolFontFamily{\textbf{\textcolor[rgb]{0,0,0.5019607843137255}{FLEx}}}}{\textit{ project (from the list of {\LangtToolFontFamily{\textbf{\textcolor[rgb]{0,0,0.5019607843137255}{FLEx}}}} projects).}}}\\\bottomrule%
\end{longtable}
}}}
{\setlength{\XLingPapertempdim}{\XLingPapersingleletterlistitemwidth+\XLingPaperlistitemindent}\leftskip\XLingPapertempdim\relax
\interlinepenalty10000
\XLingPaperlistitem{\XLingPaperlistitemindent}{\XLingPapersingleletterlistitemwidth}{d.}{Click the \vspace*{0pt}{\XeTeXpicfile "../Images/ApplyAndCloseButton.png" scaled 750} button.}}}}
\vspace{\baselineskip}
}
\vspace{12pt}
\XLingPaperneedspace{5\baselineskip}

\penalty-3000{\noindent{\raisebox{\baselineskip}[0pt]{\protect\hypertarget{sHermitCrabSetup}{}}\SectionLevelThreeFontFamily{\normalsize{\raisebox{\baselineskip}[0pt]{\pdfbookmark[3]{2.1.2 Choose your parser for synthesis, XAmple or HermitCrab}{sHermitCrabSetup}}\textit{2.1.2 Choose your parser for synthesis, {\LangtToolFontFamily{\textbf{\textcolor[rgb]{0,0,0.5019607843137255}{XAmple}}}} or {\LangtToolFontFamily{\textbf{\textcolor[rgb]{0,0,0.5019607843137255}{HermitCrab}}}}}}}}
\markright{Choose your parser for synthesis, {\LangtToolFontFamily{\textbf{\textcolor[rgb]{0,0,0.5019607843137255}{XAmple}}}} or {\LangtToolFontFamily{\textbf{\textcolor[rgb]{0,0,0.5019607843137255}{HermitCrab}}}}}
\XLingPaperaddtocontents{sHermitCrabSetup}}\par{}\penalty10000
\vspace{12pt}
\indent The modules related to synthesis (writing out lexicon files, preparing the text, running synthesis) need to know if you are using the {\LangtToolFontFamily{\textbf{\textcolor[rgb]{0,0,0.5019607843137255}{XAmple}}}} parsing model or the {\LangtToolFontFamily{\textbf{\textcolor[rgb]{0,0,0.5019607843137255}{HermitCrab}}}} parsing model in your Target Project for synthesizing target morphemes into surface forms. For more details on the {\LangtToolFontFamily{\textbf{\textcolor[rgb]{0,0,0.5019607843137255}{HermitCrab}}}} parser, see \hyperlink{rBlack}{Black, (2014)}.\par{}\indent By default {\LangtToolFontFamily{\textbf{\textcolor[rgb]{0,0,0.5019607843137255}{FLExTrans}}}} assumes you will use the {\LangtToolFontFamily{\textbf{\textcolor[rgb]{0,0,0.5019607843137255}{XAmple}}}} parsing model. To set up {\LangtToolFontFamily{\textbf{\textcolor[rgb]{0,0,0.5019607843137255}{FLExTrans}}}} for {\LangtToolFontFamily{\textbf{\textcolor[rgb]{0,0,0.5019607843137255}{HermitCrab}}}} parsing, you need to do the following things:\par{}{\parskip .5pt plus 1pt minus 1pt
                    
\vspace{\baselineskip}

{\setlength{\XLingPapertempdim}{\XLingPapersingledigitlistitemwidth+\parindent{}}\leftskip\XLingPapertempdim\relax
\interlinepenalty10000
\XLingPaperlistitem{\parindent{}}{\XLingPapersingledigitlistitemwidth}{1.}{Turn on the setting to enable {\LangtToolFontFamily{\textbf{\textcolor[rgb]{0,0,0.5019607843137255}{HermitCrab}}}} for synthesis.}{\setlength{\XLingPaperlistitemindent}{\XLingPapersingledigitlistitemwidth + \parindent{}}
{\setlength{\XLingPapertempdim}{\XLingPapersingleletterlistitemwidth+\XLingPaperlistitemindent}\leftskip\XLingPapertempdim\relax
\interlinepenalty10000
\XLingPaperlistitem{\XLingPaperlistitemindent}{\XLingPapersingleletterlistitemwidth}{a.}{Launch the {\LangtToolFontFamily{\textbf{\textcolor[rgb]{0,0,0.5019607843137255}{FLExTrans Settings}}}} by selecting {\LangtMenuFontFamily{\textbf{\textcolor[rgb]{0,0.6,0.2}{FLExTrans}}}} \textgreater{} {\LangtMenuFontFamily{\textbf{\textcolor[rgb]{0,0.6,0.2}{Settings}}}} from the {\LangtToolFontFamily{\textbf{\textcolor[rgb]{0,0,0.5019607843137255}{FLExTools}}}} menu.}}
{\setlength{\XLingPapertempdim}{\XLingPapersingleletterlistitemwidth+\XLingPaperlistitemindent}\leftskip\XLingPapertempdim\relax
\interlinepenalty10000
\XLingPaperlistitem{\XLingPaperlistitemindent}{\XLingPapersingleletterlistitemwidth}{b.}{Click {\textit{Yes}} for {\textbf{\textcolor[rgb]{0.5882352941176471,0.29411764705882354,0}{Use HermitCrab synthesis}}} as shown in \hyperlink{xHermitCrabYesNo}{(3)}.}\par{}{\vspace{12pt plus 2pt minus 1pt}\raggedright{}\XLingPaperexample{.125in}{0pt}{2.75em}{\raisebox{\baselineskip}[0pt]{\protect\hypertarget{xHermitCrabYesNo}{}}(3)}{\parbox[t]{\textwidth - .125in - 0pt}{\vspace*{-\baselineskip}{\XeTeXpicfile "../Images/HermitCrabYesNo.png" scaled 600}}}
\vspace{12pt plus 2pt minus 1pt}}{}}}}
{\setlength{\XLingPapertempdim}{\XLingPapersingledigitlistitemwidth+\parindent{}}\leftskip\XLingPapertempdim\relax
\interlinepenalty10000
\XLingPaperlistitem{\parindent{}}{\XLingPapersingledigitlistitemwidth}{2.}{\setlength{\parindent}{1em}\indent If you have any lexical entries that have a space in the {\textcolor[rgb]{0.8,0.6,0}{Lexeme Form}}, do the following in your {\LangtToolFontFamily{\textbf{\textcolor[rgb]{0,0,0.5019607843137255}{FLEx}}}} projects:}{\setlength{\XLingPaperlistitemindent}{\XLingPapersingledigitlistitemwidth + \parindent{}}
{\setlength{\XLingPapertempdim}{\XLingPapersingleletterlistitemwidth+\XLingPaperlistitemindent}\leftskip\XLingPapertempdim\relax
\interlinepenalty10000
\XLingPaperlistitem{\XLingPaperlistitemindent}{\XLingPapersingleletterlistitemwidth}{a.}{Add a "phoneme" with a grapheme of \#.}}
{\setlength{\XLingPapertempdim}{\XLingPapersingleletterlistitemwidth+\XLingPaperlistitemindent}\leftskip\XLingPapertempdim\relax
\interlinepenalty10000
\XLingPaperlistitem{\XLingPaperlistitemindent}{\XLingPapersingleletterlistitemwidth}{b.}{Make sure that the phonological features for this phoneme are unique.}}\setlength{\parindent}{1em}\indent The {\LangtToolFontFamily{\textbf{\textcolor[rgb]{0,0,0.5019607843137255}{HermitCrab}}}} synthesis tool will use \#s internally for spaces.\par{}}}
\vspace{\baselineskip}
}
\vspace{12pt}
\XLingPaperneedspace{5\baselineskip}

\penalty-3000{\noindent{\raisebox{\baselineskip}[0pt]{\protect\hypertarget{sDataPrep}{}}\SectionLevelThreeFontFamily{\normalsize{\raisebox{\baselineskip}[0pt]{\pdfbookmark[3]{2.1.3 Prepare your data}{sDataPrep}}\textit{2.1.3 Prepare your data}}}}
\markright{Prepare your data}
\XLingPaperaddtocontents{sDataPrep}}\par{}\penalty10000
\vspace{12pt}
\indent There are certain steps that are not strictly necessary for {\LangtToolFontFamily{\textbf{\textcolor[rgb]{0,0,0.5019607843137255}{FLEx}}}}, but they are needed for {\LangtToolFontFamily{\textbf{\textcolor[rgb]{0,0,0.5019607843137255}{FLExTrans}}}}. In your {\LangtToolFontFamily{\textbf{\textcolor[rgb]{0,0,0.5019607843137255}{FLEx}}}} projects:\par{}{\parskip .5pt plus 1pt minus 1pt
                    
\vspace{\baselineskip}

{\setlength{\XLingPapertempdim}{\XLingPapersingledigitlistitemwidth+\parindent{}}\leftskip\XLingPapertempdim\relax
\interlinepenalty10000
\XLingPaperlistitem{\parindent{}}{\XLingPapersingledigitlistitemwidth}{1.}{Ensure that the glosses of affixes are unique, within each morpheme type. That is, all suffixes have unique glosses, all prefixes have unique glosses, etc. The glosses of different senses within an entry can be the same, but across separate entries they should be different.}}
{\setlength{\XLingPapertempdim}{\XLingPapersingledigitlistitemwidth+\parindent{}}\leftskip\XLingPapertempdim\relax
\interlinepenalty10000
\XLingPaperlistitem{\parindent{}}{\XLingPapersingledigitlistitemwidth}{2.}{Ensure that glosses of affixes don't have spaces in them. Use period or underscore instead. Note that if you use periods in your affix glosses, you will have to write them with underscores when referencing them in rules.}}
{\setlength{\XLingPapertempdim}{\XLingPapersingledigitlistitemwidth+\parindent{}}\leftskip\XLingPapertempdim\relax
\interlinepenalty10000
\XLingPaperlistitem{\parindent{}}{\XLingPapersingledigitlistitemwidth}{3.}{Ensure that every Part of Speech (category) has both a Name and an Abbreviation in the primary analysis writing system.}}
{\setlength{\XLingPapertempdim}{\XLingPapersingledigitlistitemwidth+\parindent{}}\leftskip\XLingPapertempdim\relax
\interlinepenalty10000
\XLingPaperlistitem{\parindent{}}{\XLingPapersingledigitlistitemwidth}{4.}{Ensure that all templates have a name in the primary analysis writing system, all slots have a name, and all slots have at least one affix in them. If any template doesn't meet these criteria, temporarily deactivate it by unticking the "Active" box in the description of the template.}}
{\setlength{\XLingPapertempdim}{\XLingPapersingledigitlistitemwidth+\parindent{}}\leftskip\XLingPapertempdim\relax
\interlinepenalty10000
\XLingPaperlistitem{\parindent{}}{\XLingPapersingledigitlistitemwidth}{5.}{FLExTrans is intended to work with the primary analysis writing system set to any writing system. If you encounter an area that only works with English as the analysis writing system, please report it to the developers.}}
\vspace{\baselineskip}
}
\vspace{12pt}
\XLingPaperneedspace{5\baselineskip}

\penalty-3000{\noindent{\raisebox{\baselineskip}[0pt]{\protect\hypertarget{sCategorySetup}{}}\SectionLevelThreeFontFamily{\normalsize{\raisebox{\baselineskip}[0pt]{\pdfbookmark[3]{2.1.4 Pre-populate the transfer rules file with Categories and Attributes}{sCategorySetup}}\textit{2.1.4 Pre-populate the transfer rules file with Categories and Attributes}}}}
\markright{Pre-populate the transfer rules file with Categories and Attributes}
\XLingPaperaddtocontents{sCategorySetup}}\par{}\penalty10000
\vspace{12pt}
\indent If you expect to write rules that use more than a few of the categories, attributes, and template slots in your {\LangtToolFontFamily{\textbf{\textcolor[rgb]{0,0,0.5019607843137255}{FLEx}}}} projects, there is a module that can get those into your transfer rules file before beginning. (Find out more about this tool in section \hyperlink{sSetupGramCat}{4.5}. Find more about transfer rules in section \hyperlink{sTransferTutorial}{6}.)\par{}{\parskip .5pt plus 1pt minus 1pt
                    
\vspace{\baselineskip}

{\setlength{\XLingPapertempdim}{\XLingPapersingledigitlistitemwidth+\parindent{}}\leftskip\XLingPapertempdim\relax
\interlinepenalty10000
\XLingPaperlistitem{\parindent{}}{\XLingPapersingledigitlistitemwidth}{1.}{Run the {\LangtToolFontFamily{\textbf{\textcolor[rgb]{0,0,0.5019607843137255}{Set Up Transfer Rule Categories and Attributes Tool}}}} from the {\LangtCollectionFontFamily{{\fontspec[Scale=0.8]{Arial}\textup{\textbf{\textcolor[rgb]{0.4,0,0.4}{Tools}}}}}} collection. You will see a window that looks like \hyperlink{xSetupCatAttrib0}{(4)}.}\par{}{\vspace{12pt plus 2pt minus 1pt}\raggedright{}\XLingPaperexample{.125in}{0pt}{2.75em}{\raisebox{\baselineskip}[0pt]{\protect\hypertarget{xSetupCatAttrib0}{}}(4)}{\parbox[t]{\textwidth - .125in - 0pt}{\vspace*{-\baselineskip}{\XeTeXpicfile "../Images/SetupCatAttrib.png" scaled 750}}}
\vspace{12pt plus 2pt minus 1pt}}{}}
{\setlength{\XLingPapertempdim}{\XLingPapersingledigitlistitemwidth+\parindent{}}\leftskip\XLingPapertempdim\relax
\interlinepenalty10000
\XLingPaperlistitem{\parindent{}}{\XLingPapersingledigitlistitemwidth}{2.}{If you have grammatical templates defined in your {\LangtToolFontFamily{\textbf{\textcolor[rgb]{0,0,0.5019607843137255}{FLEx}}}} projects, click the check box for {\textit{Populate template slots as attributes}}.}}
{\setlength{\XLingPapertempdim}{\XLingPapersingledigitlistitemwidth+\parindent{}}\leftskip\XLingPapertempdim\relax
\interlinepenalty10000
\XLingPaperlistitem{\parindent{}}{\XLingPapersingledigitlistitemwidth}{3.}{Click OK}}
\vspace{\baselineskip}
}
\vspace{12pt}
\XLingPaperneedspace{5\baselineskip}

\penalty-3000{\noindent{\raisebox{\baselineskip}[0pt]{\protect\hypertarget{sWalkthrough}{}}\SectionLevelTwoFontFamily{\normalsize{\raisebox{\baselineskip}[0pt]{\pdfbookmark[2]{2.2 Walkthrough of Basic Machine Translation Steps}{sWalkthrough}}\textbf{2.2 Walkthrough of Basic Machine Translation Steps}}}}
\markright{Walkthrough of Basic Machine Translation Steps}
\XLingPaperaddtocontents{sWalkthrough}}\par{}\penalty10000
\vspace{12pt}

\vspace{12pt}
\XLingPaperneedspace{5\baselineskip}

\penalty-3000{\noindent{\raisebox{\baselineskip}[0pt]{\protect\hypertarget{sAnalyze}{}}\SectionLevelThreeFontFamily{\normalsize{\raisebox{\baselineskip}[0pt]{\pdfbookmark[3]{2.2.1 Analyze the Source Text}{sAnalyze}}\textit{2.2.1 Analyze the Source Text}}}}
\markright{Analyze the Source Text}
\XLingPaperaddtocontents{sAnalyze}}\par{}\penalty10000
\vspace{12pt}
{\parskip .5pt plus 1pt minus 1pt
                    
{\setlength{\XLingPapertempdim}{\XLingPapersingledigitlistitemwidth+\parindent{}}\leftskip\XLingPapertempdim\relax
\interlinepenalty10000
\XLingPaperlistitem{\parindent{}}{\XLingPapersingledigitlistitemwidth}{1.}{Create a new text in {\LangtToolFontFamily{\textbf{\textcolor[rgb]{0,0,0.5019607843137255}{FLEx}}}}.}}
{\setlength{\XLingPapertempdim}{\XLingPapersingledigitlistitemwidth+\parindent{}}\leftskip\XLingPapertempdim\relax
\interlinepenalty10000
\XLingPaperlistitem{\parindent{}}{\XLingPapersingledigitlistitemwidth}{2.}{Paste in some text. (Start with a simple text; perhaps just one sentence.)}}
{\setlength{\XLingPapertempdim}{\XLingPapersingledigitlistitemwidth+\parindent{}}\leftskip\XLingPapertempdim\relax
\interlinepenalty10000
\XLingPaperlistitem{\parindent{}}{\XLingPapersingledigitlistitemwidth}{3.}{Analyze this text.}{\setlength{\XLingPaperlistitemindent}{\XLingPapersingledigitlistitemwidth + \parindent{}}
{\setlength{\XLingPapertempdim}{\XLingPapersingleletterlistitemwidth+\XLingPaperlistitemindent}\leftskip\XLingPapertempdim\relax
\interlinepenalty10000
\XLingPaperlistitem{\XLingPaperlistitemindent}{\XLingPapersingleletterlistitemwidth}{a.}{Divide each word into its morphemes.}}
{\setlength{\XLingPapertempdim}{\XLingPapersingleletterlistitemwidth+\XLingPaperlistitemindent}\leftskip\XLingPapertempdim\relax
\interlinepenalty10000
\XLingPaperlistitem{\XLingPaperlistitemindent}{\XLingPapersingleletterlistitemwidth}{b.}{Approve the analysis of each word.}}}}
\vspace{\baselineskip}
}\indent Word glosses, word categories and free translations are not necessary. A fully analyzed text will look something like \hyperlink{xAnalyzedText}{(5)} below.\par{}{\vspace{12pt plus 2pt minus 1pt}\raggedright{}\XLingPaperexample{.125in}{0pt}{2.75em}{\raisebox{\baselineskip}[0pt]{\protect\hypertarget{xAnalyzedText}{}}(5)}{\parbox[t]{\textwidth - .125in - 0pt}{\vspace*{-\baselineskip}{\XeTeXpicfile "../Images/AnalyzedText.PNG" scaled 600}}}
\vspace{12pt plus 2pt minus 1pt}}\par\indent Note: Word gloss and category are hidden as well as free translation.\par{}
\vspace{12pt}
\XLingPaperneedspace{5\baselineskip}

\penalty-3000{\noindent{\raisebox{\baselineskip}[0pt]{\protect\hypertarget{sLinkSenses}{}}\SectionLevelThreeFontFamily{\normalsize{\raisebox{\baselineskip}[0pt]{\pdfbookmark[3]{2.2.2 Link Senses}{sLinkSenses}}\textit{2.2.2 Link Senses}}}}
\markright{Link Senses}
\XLingPaperaddtocontents{sLinkSenses}}\par{}\penalty10000
\vspace{12pt}
{\parskip .5pt plus 1pt minus 1pt
                    
{\setlength{\XLingPapertempdim}{\XLingPapersingledigitlistitemwidth+\parindent{}}\leftskip\XLingPapertempdim\relax
\interlinepenalty10000
\XLingPaperlistitem{\parindent{}}{\XLingPapersingledigitlistitemwidth}{1.}{Click on {\LangtCollectionFontFamily{{\fontspec[Scale=0.8]{Arial}\textup{\textbf{\textcolor[rgb]{0.4,0,0.4}{Tools}}}}}} tab and you’ll see list of Tools modules:}}
{\setlength{\XLingPapertempdim}{\XLingPapersingledigitlistitemwidth+\parindent{}}\leftskip\XLingPapertempdim\relax
\interlinepenalty10000
\XLingPaperlistitem{\parindent{}}{\XLingPapersingledigitlistitemwidth}{2.}{Click on {\LangtCollectionFontFamily{{\fontspec[Scale=0.8]{Arial}\textup{\textbf{\textcolor[rgb]{0.4,0,0.4}{Sense Linker Tool}}}}}} in the list.}\par{}{\vspace{12pt plus 2pt minus 1pt}\raggedright{}\XLingPaperexample{.125in}{0pt}{2.75em}{\raisebox{\baselineskip}[0pt]{\protect\hypertarget{xLinkerModule}{}}(6)}{\parbox[t]{\textwidth - .125in - 0pt}{\vspace*{-\baselineskip}{\XeTeXpicfile "../Images/LinkerModule.png" scaled 600}}}
\vspace{12pt plus 2pt minus 1pt}}{}}
{\setlength{\XLingPapertempdim}{\XLingPapersingledigitlistitemwidth+\parindent{}}\leftskip\XLingPapertempdim\relax
\interlinepenalty10000
\XLingPaperlistitem{\parindent{}}{\XLingPapersingledigitlistitemwidth}{3.}{Click on the {\textup{\textbf{Run}}} button to run the module. You will now see the {\LangtToolFontFamily{\textbf{\textcolor[rgb]{0,0,0.5019607843137255}{Sense Linker Tool}}}} with no links yet established. The window will look something like this:}\par{}{\vspace{12pt plus 2pt minus 1pt}\raggedright{}\XLingPaperexample{.125in}{0pt}{2.75em}{\raisebox{\baselineskip}[0pt]{\protect\hypertarget{xLinkerNoLinks}{}}(7)}{\parbox[t]{\textwidth - .125in - 0pt}{\vspace*{-\baselineskip}{\XeTeXpicfile "../Images/SenseLinkerNoLinks.png" scaled 750}}}
\vspace{12pt plus 2pt minus 1pt}}{}}
{\setlength{\XLingPapertempdim}{\XLingPapersingledigitlistitemwidth+\parindent{}}\leftskip\XLingPapertempdim\relax
\interlinepenalty10000
\XLingPaperlistitem{\parindent{}}{\XLingPapersingledigitlistitemwidth}{4.}{Link each sense. (More details in section \hyperlink{sLinker}{4.1}) }{\setlength{\XLingPaperlistitemindent}{\XLingPapersingledigitlistitemwidth + \parindent{}}
{\setlength{\XLingPapertempdim}{\XLingPapersingleletterlistitemwidth+\XLingPaperlistitemindent}\leftskip\XLingPapertempdim\relax
\interlinepenalty10000
\XLingPaperlistitem{\XLingPaperlistitemindent}{\XLingPapersingleletterlistitemwidth}{a.}{For blue rows that look good, simply check the check box in the {\textbf{Link It}} column.}}
{\setlength{\XLingPapertempdim}{\XLingPapersingleletterlistitemwidth+\XLingPaperlistitemindent}\leftskip\XLingPapertempdim\relax
\interlinepenalty10000
\XLingPaperlistitem{\XLingPaperlistitemindent}{\XLingPapersingleletterlistitemwidth}{b.}{For red rows pick the appropriate target sense in the drop-down list, then double-click in the {\textbf{Target Head Word}} column for that row.}}}}
{\setlength{\XLingPapertempdim}{\XLingPapersingledigitlistitemwidth+\parindent{}}\leftskip\XLingPapertempdim\relax
\interlinepenalty10000
\XLingPaperlistitem{\parindent{}}{\XLingPapersingledigitlistitemwidth}{5.}{Click OK}}
\vspace{\baselineskip}
}\indent Now you have senses linked from the source project to the target. You haven’t created anything in the target project yet, but you have done the preparatory work.\par{}
\vspace{12pt}
\XLingPaperneedspace{5\baselineskip}

\penalty-3000{\noindent{\raisebox{\baselineskip}[0pt]{\protect\hypertarget{sWriteRules}{}}\SectionLevelThreeFontFamily{\normalsize{\raisebox{\baselineskip}[0pt]{\pdfbookmark[3]{2.2.3 Write Transfer Rules}{sWriteRules}}\textit{2.2.3 Write Transfer Rules}}}}
\markright{Write Transfer Rules}
\XLingPaperaddtocontents{sWriteRules}}\par{}\penalty10000
\vspace{12pt}
\indent In this section, as an exercise, we are going to write a simple rule that removes affixes from words of a certain category. We’ll modify an existing rule to do this.\par{}{\parskip .5pt plus 1pt minus 1pt
                    
\vspace{\baselineskip}

{\setlength{\XLingPapertempdim}{\XLingPaperdoubledigitlistitemwidth+\parindent{}}\leftskip\XLingPapertempdim\relax
\interlinepenalty10000
\XLingPaperlistitem{\parindent{}}{\XLingPaperdoubledigitlistitemwidth}{1.}{Navigate to your work project folder.}}
{\setlength{\XLingPapertempdim}{\XLingPaperdoubledigitlistitemwidth+\parindent{}}\leftskip\XLingPapertempdim\relax
\interlinepenalty10000
\XLingPaperlistitem{\parindent{}}{\XLingPaperdoubledigitlistitemwidth}{2.}{Open the file {\textit{transfer\_rules-Swedish.t1x}} in {\LangtToolFontFamily{\textbf{\textcolor[rgb]{0,0,0.5019607843137255}{XMLmind XML Editor}}}}. It should look something like \hyperlink{xInitialSampleRules}{(8)}. This is the transfer rules file that works with the sample projects.}\par{}{\vspace{12pt plus 2pt minus 1pt}\raggedright{}\XLingPaperexample{.125in}{0pt}{2.75em}{\raisebox{\baselineskip}[0pt]{\protect\hypertarget{xInitialSampleRules}{}}(8)}{\parbox[t]{\textwidth - .125in - 0pt}{\vspace*{-\baselineskip}{\XeTeXpicfile "../Images/RulesSampleInitial.PNG" scaled 600}}}
\vspace{12pt plus 2pt minus 1pt}}{}}
{\setlength{\XLingPapertempdim}{\XLingPaperdoubledigitlistitemwidth+\parindent{}}\leftskip\XLingPapertempdim\relax
\interlinepenalty10000
\XLingPaperlistitem{\parindent{}}{\XLingPaperdoubledigitlistitemwidth}{3.}{Expand the {\LangtRuleElemInXXEFontFamily{{\fontspec[Scale=0.8]{Arial}\textcolor[rgb]{0,0.4,0.2}{\textbf{rule }}}}} element for Verbs and also the {\LangtRuleElemInXXEFontFamily{{\fontspec[Scale=0.8]{Arial}\textcolor[rgb]{0,0.4,0.2}{\textbf{output}}}}} element so that it looks like \hyperlink{xRulesSampleVerbRule}{(9)}.}\par{}{\vspace{12pt plus 2pt minus 1pt}\raggedright{}\XLingPaperexample{.125in}{0pt}{2.75em}{\raisebox{\baselineskip}[0pt]{\protect\hypertarget{xRulesSampleVerbRule}{}}(9)}{\parbox[t]{\textwidth - .125in - 0pt}{\vspace*{-\baselineskip}{\XeTeXpicfile "../Images/RulesSampleVerbRule.PNG" scaled 600}}}
\vspace{12pt plus 2pt minus 1pt}}{}}
{\setlength{\XLingPapertempdim}{\XLingPaperdoubledigitlistitemwidth+\parindent{}}\leftskip\XLingPapertempdim\relax
\interlinepenalty10000
\XLingPaperlistitem{\parindent{}}{\XLingPaperdoubledigitlistitemwidth}{4.}{Pick a category that you are going to process. For this example let’s use noun, abbreviated “n”.}}
{\setlength{\XLingPapertempdim}{\XLingPaperdoubledigitlistitemwidth+\parindent{}}\leftskip\XLingPapertempdim\relax
\interlinepenalty10000
\XLingPaperlistitem{\parindent{}}{\XLingPaperdoubledigitlistitemwidth}{5.}{Change the {\LangtRuleElemInXXEFontFamily{{\fontspec[Scale=0.8]{Arial}\textcolor[rgb]{0,0.4,0.2}{\textbf{rule}}}}} comment shown in blue to “Nouns”}}
{\setlength{\XLingPapertempdim}{\XLingPaperdoubledigitlistitemwidth+\parindent{}}\leftskip\XLingPapertempdim\relax
\interlinepenalty10000
\XLingPaperlistitem{\parindent{}}{\XLingPaperdoubledigitlistitemwidth}{6.}{We need to change the {\LangtRuleElemInXXEFontFamily{{\fontspec[Scale=0.8]{Arial}\textcolor[rgb]{0,0.4,0.2}{\textbf{pattern}}}}} element so it matches nouns. Change the {\LangtRuleElemInXXEFontFamily{{\fontspec[Scale=0.8]{Arial}\textcolor[rgb]{0,0.4,0.2}{\textbf{item}}}}} element to “c\_n”. We will use this to mean: category noun. Category noun gets defined in a later step.}}
{\setlength{\XLingPapertempdim}{\XLingPaperdoubledigitlistitemwidth+\parindent{}}\leftskip\XLingPapertempdim\relax
\interlinepenalty10000
\XLingPaperlistitem{\parindent{}}{\XLingPaperdoubledigitlistitemwidth}{7.}{Remove the two {\LangtRuleElemInXXEFontFamily{{\fontspec[Scale=0.8]{Arial}\textcolor[rgb]{0,0.4,0.2}{\textbf{literal tag}}}}} elements by clicking on the word “literal” and pressing the delete key. The rule should now look like \hyperlink{xRulesSampleNounRule}{(10)}.}\par{}{\vspace{12pt plus 2pt minus 1pt}\raggedright{}\XLingPaperexample{.125in}{0pt}{2.75em}{\raisebox{\baselineskip}[0pt]{\protect\hypertarget{xRulesSampleNounRule}{}}(10)}{\parbox[t]{\textwidth - .125in - 0pt}{\vspace*{-\baselineskip}{\XeTeXpicfile "../Images/RulesSampleNounRule.PNG" scaled 600}}}
\vspace{12pt plus 2pt minus 1pt}}{}}
{\setlength{\XLingPapertempdim}{\XLingPaperdoubledigitlistitemwidth+\parindent{}}\leftskip\XLingPapertempdim\relax
\interlinepenalty10000
\XLingPaperlistitem{\parindent{}}{\XLingPaperdoubledigitlistitemwidth}{8.}{Expand the {\LangtRuleElemInXXEFontFamily{{\fontspec[Scale=0.8]{Arial}\textcolor[rgb]{0,0.4,0.2}{\textbf{Categories}}}}} element (the first item in \hyperlink{xInitialSampleRules}{(8)}) and the second {\LangtRuleElemInXXEFontFamily{{\fontspec[Scale=0.8]{Arial}\textcolor[rgb]{0,0.4,0.2}{\textbf{category}}}}} element and it should look like \hyperlink{xRulesSampleCatsBefore}{(11)}.}\par{}{\vspace{12pt plus 2pt minus 1pt}\raggedright{}\XLingPaperexample{.125in}{0pt}{2.75em}{\raisebox{\baselineskip}[0pt]{\protect\hypertarget{xRulesSampleCatsBefore}{}}(11)}{\parbox[t]{\textwidth - .125in - 0pt}{\vspace*{-\baselineskip}{\XeTeXpicfile "../Images/RulesSampleCatsBefore.PNG" scaled 600}}}
\vspace{12pt plus 2pt minus 1pt}}{}}
{\setlength{\XLingPapertempdim}{\XLingPaperdoubledigitlistitemwidth+\parindent{}}\leftskip\XLingPapertempdim\relax
\interlinepenalty10000
\XLingPaperlistitem{\parindent{}}{\XLingPaperdoubledigitlistitemwidth}{9.}{Change “c\_v” to “c\_n” in the {\LangtRuleElemInXXEFontFamily{{\fontspec[Scale=0.8]{Arial}\textcolor[rgb]{0,0.4,0.2}{\textbf{category}}}}} element and “v” to “n” in both {\LangtRuleElemInXXEFontFamily{{\fontspec[Scale=0.8]{Arial}\textcolor[rgb]{0,0.4,0.2}{\textbf{tags}}}}} elements. Now it should look something like \hyperlink{xRulesSampleCatsAfter}{(12)}. Category “c\_n” is now defined as the single tag “n” or the tag “n” followed by any other tag, i.e. affixes and the like.}\par{}{\vspace{12pt plus 2pt minus 1pt}\raggedright{}\XLingPaperexample{.125in}{0pt}{2.75em}{\raisebox{\baselineskip}[0pt]{\protect\hypertarget{xRulesSampleCatsAfter}{}}(12)}{\parbox[t]{\textwidth - .125in - 0pt}{\vspace*{-\baselineskip}{\XeTeXpicfile "../Images/RulesSampleCatsAfter.PNG" scaled 600}}}
\vspace{12pt plus 2pt minus 1pt}}{}}
{\setlength{\XLingPapertempdim}{\XLingPaperdoubledigitlistitemwidth+\parindent{}}\leftskip\XLingPapertempdim\relax
\interlinepenalty10000
\XLingPaperlistitem{\parindent{}}{\XLingPaperdoubledigitlistitemwidth}{10.}{Expand the {\LangtRuleElemInXXEFontFamily{{\fontspec[Scale=0.8]{Arial}\textcolor[rgb]{0,0.4,0.2}{\textbf{Attributes}}}}} element and the second {\LangtRuleElemInXXEFontFamily{{\fontspec[Scale=0.8]{Arial}\textcolor[rgb]{0,0.4,0.2}{\textbf{attribute}}}}} element and it should look like \hyperlink{xRulesSampleAttribs}{(13)}. Note that we already have a tag for a noun.}\par{}{\vspace{12pt plus 2pt minus 1pt}\raggedright{}\XLingPaperexample{.125in}{0pt}{2.75em}{\raisebox{\baselineskip}[0pt]{\protect\hypertarget{xRulesSampleAttribs}{}}(13)}{\parbox[t]{\textwidth - .125in - 0pt}{\vspace*{-\baselineskip}{\XeTeXpicfile "../Images/RulesSampleAttribs.PNG" scaled 600}}}
\vspace{12pt plus 2pt minus 1pt}}{}}
{\setlength{\XLingPapertempdim}{\XLingPaperdoubledigitlistitemwidth+\parindent{}}\leftskip\XLingPapertempdim\relax
\interlinepenalty10000
\XLingPaperlistitem{\parindent{}}{\XLingPaperdoubledigitlistitemwidth}{11.}{We are done editing the rule. You can save the file.}}
\vspace{\baselineskip}
}\indent In brief what this rule does is set up an action to take place when a word is processed by the {\LangtToolFontFamily{\textbf{\textcolor[rgb]{0,0,0.5019607843137255}{Apertium}}}} engine that has the “n” tag. (Every word coming from {\LangtToolFontFamily{\textbf{\textcolor[rgb]{0,0,0.5019607843137255}{FLEx}}}} has the word’s grammatical category in the form of a tag. It’s always the first tag.)\par{}\indent The {\LangtRuleElemInXXEFontFamily{{\fontspec[Scale=0.8]{Arial}\textcolor[rgb]{0,0.4,0.2}{\textbf{action}}}}} that we execute is to strip off any affixes from the word. This is accomplished by outputting the word itself -- the “lem” part and the grammatical category -- anything that matches a tag specified by in the “a\_gram\_cat” attribute. No other tags (affixes) are explicitly outputted. The {\LangtRuleElemInXXEFontFamily{{\fontspec[Scale=0.8]{Arial}\textcolor[rgb]{0,0.4,0.2}{\textbf{target lang.}}}}} attribute of the {\LangtRuleElemInXXEFontFamily{{\fontspec[Scale=0.8]{Arial}\textcolor[rgb]{0,0.4,0.2}{\textbf{clip}}}}} elements in \hyperlink{xRulesSampleNounRule}{(10)} above is selected so the target word that corresponds to the input source word is outputted.\par{}\indent See sections “\hyperlink{sTransferTutorial}{A Tutorial on Writing Transfer Rules}” and “\hyperlink{sTransferRuleHowTos}{Transfer Rule How To’s}” for more information on transfer rules. The authoritative reference document for transfer rules is found in section 3.5 and appendix 3 of \hyperlink{rApertium}{Forcada et al. (2010)}. \par{}
\vspace{12pt}
\XLingPaperneedspace{5\baselineskip}

\penalty-3000{\noindent{\raisebox{\baselineskip}[0pt]{\protect\hypertarget{sRunModules}{}}\SectionLevelThreeFontFamily{\normalsize{\raisebox{\baselineskip}[0pt]{\pdfbookmark[3]{2.2.4 Run FLExTrans Modules}{sRunModules}}\textit{2.2.4 Run {\LangtToolFontFamily{\textbf{\textcolor[rgb]{0,0,0.5019607843137255}{FLExTrans}}}} Modules}}}}
\markright{Run {\LangtToolFontFamily{\textbf{\textcolor[rgb]{0,0,0.5019607843137255}{FLExTrans}}}} Modules}
\XLingPaperaddtocontents{sRunModules}}\par{}\penalty10000
\vspace{12pt}
{\parskip .5pt plus 1pt minus 1pt
                    
{\setlength{\XLingPapertempdim}{\XLingPapersingledigitlistitemwidth+\parindent{}}\leftskip\XLingPapertempdim\relax
\interlinepenalty10000
\XLingPaperlistitem{\parindent{}}{\XLingPapersingledigitlistitemwidth}{1.}{In {\LangtToolFontFamily{\textbf{\textcolor[rgb]{0,0,0.5019607843137255}{FLExTools}}}}, click on the {\LangtCollectionFontFamily{{\fontspec[Scale=0.8]{Arial}\textup{\textbf{\textcolor[rgb]{0.4,0,0.4}{Drafting}}}}}} tab and you’ll see \hyperlink{xFirstStart}{(1)}.}}
{\setlength{\XLingPapertempdim}{\XLingPapersingledigitlistitemwidth+\parindent{}}\leftskip\XLingPapertempdim\relax
\interlinepenalty10000
\XLingPaperlistitem{\parindent{}}{\XLingPapersingledigitlistitemwidth}{2.}{Click on the {\textup{\textbf{Run All}}} button to run all the modules. You’ll see the modules run and see various lines of output.}}
\vspace{\baselineskip}
}\indent You can see the results by opening the target {\LangtToolFontFamily{\textbf{\textcolor[rgb]{0,0,0.5019607843137255}{FLEx}}}} project. Go to the {\textcolor[rgb]{0.8,0.6,0}{Texts \& Words}} area and verify that a new text has been inserted. (Sometimes it takes a moment before it appears.)\par{}
\vspace{12pt}
\XLingPaperneedspace{5\baselineskip}

\penalty-3000{{\centering\raisebox{\baselineskip}[0pt]{\protect\hypertarget{sInfrastructure}{}}\SectionLevelOneFontFamily{\large{\raisebox{\baselineskip}[0pt]{\pdfbookmark[1]{3 FLExTrans User Interface Orientation}{sInfrastructure}}\textbf{3 {\LangtToolFontFamily{\textbf{\textcolor[rgb]{0,0,0.5019607843137255}{FLExTrans}}}} User Interface Orientation}}}\\{}}\markright{{\LangtToolFontFamily{\textbf{\textcolor[rgb]{0,0,0.5019607843137255}{FLExTrans}}}} User Interface Orientation}
\XLingPaperaddtocontents{sInfrastructure}}\par{}\penalty10000
\vspace{12pt}
\indent The {\LangtToolFontFamily{\textbf{\textcolor[rgb]{0,0,0.5019607843137255}{FLExTrans}}}} User Interface presents modules that can be run to perform various operations. These are organized into collections. There is a menu that allows adjusting settings for the modules.\par{}
\vspace{12pt}
\XLingPaperneedspace{5\baselineskip}

\penalty-3000{\noindent{\raisebox{\baselineskip}[0pt]{\protect\hypertarget{sFTUserInterface}{}}\SectionLevelTwoFontFamily{\normalsize{\raisebox{\baselineskip}[0pt]{\pdfbookmark[2]{3.1 FLExTrans Window}{sFTUserInterface}}\textbf{3.1 FLExTrans Window}}}}
\markright{FLExTrans Window}
\XLingPaperaddtocontents{sFTUserInterface}}\par{}\penalty10000
\vspace{12pt}
\indent The {\LangtToolFontFamily{\textbf{\textcolor[rgb]{0,0,0.5019607843137255}{FLExTrans}}}} user interface is shown in \hyperlink{xFTwindow}{(14)}. The parts of this window are described below.\par{}{\vspace{12pt plus 2pt minus 1pt}\raggedright{}\XLingPaperexample{.125in}{0pt}{2.75em}{\raisebox{\baselineskip}[0pt]{\protect\hypertarget{xFTwindow}{}}(14)}{\parbox[t]{\textwidth - .125in - 0pt}{\vspace*{-\baselineskip}{\XeTeXpicfile "../Images/FLExTransWindow.png" scaled 750}}}
}{\parskip .5pt plus 1pt minus 1pt
                    
\vspace{\baselineskip}

{\setlength{\XLingPapertempdim}{\XLingPapersingledigitlistitemwidth+\parindent{}}\leftskip\XLingPapertempdim\relax
\interlinepenalty10000
\XLingPaperlistitem{\parindent{}}{\XLingPapersingledigitlistitemwidth}{1.}{The {\textit{\textbf{\textcolor[rgb]{0.6392156862745098,0.28627450980392155,0.6431372549019608}{Menu bar}}}} provides a way to run various {\LangtToolFontFamily{\textbf{\textcolor[rgb]{0,0,0.5019607843137255}{FLExTrans}}}} commands by choosing a menu and then selecting a command from that menu.}{\setlength{\XLingPaperlistitemindent}{\XLingPapersingledigitlistitemwidth + \parindent{}}
{\setlength{\XLingPapertempdim}{\XLingPapersingleletterlistitemwidth+\XLingPaperlistitemindent}\leftskip\XLingPapertempdim\relax
\interlinepenalty10000
\XLingPaperlistitem{\XLingPaperlistitemindent}{\XLingPapersingleletterlistitemwidth}{a.}{The {\LangtMenuFontFamily{\textbf{\textcolor[rgb]{0,0.6,0.2}{FLExTools}}}} menu provides a way to change the Source Project, change or manage collections, learn about modules, reload modules, and exit the {\LangtToolFontFamily{\textbf{\textcolor[rgb]{0,0,0.5019607843137255}{FLExTrans}}}} program.}}
{\setlength{\XLingPapertempdim}{\XLingPapersingleletterlistitemwidth+\XLingPaperlistitemindent}\leftskip\XLingPapertempdim\relax
\interlinepenalty10000
\XLingPaperlistitem{\XLingPaperlistitemindent}{\XLingPapersingleletterlistitemwidth}{b.}{The {\LangtMenuFontFamily{\textbf{\textcolor[rgb]{0,0.6,0.2}{Run}}}} menu allows users to run a single module, or all modules in the current collection.}}
{\setlength{\XLingPapertempdim}{\XLingPapersingleletterlistitemwidth+\XLingPaperlistitemindent}\leftskip\XLingPapertempdim\relax
\interlinepenalty10000
\XLingPaperlistitem{\XLingPaperlistitemindent}{\XLingPapersingleletterlistitemwidth}{c.}{The {\LangtMenuFontFamily{\textbf{\textcolor[rgb]{0,0.6,0.2}{Report}}}} menu includes options for copying messages to the clipboard or to clear them.}}
{\setlength{\XLingPapertempdim}{\XLingPapersingleletterlistitemwidth+\XLingPaperlistitemindent}\leftskip\XLingPapertempdim\relax
\interlinepenalty10000
\XLingPaperlistitem{\XLingPaperlistitemindent}{\XLingPapersingleletterlistitemwidth}{d.}{The {\LangtMenuFontFamily{\textbf{\textcolor[rgb]{0,0.6,0.2}{FLExTrans}}}} menu gives access to the settings for {\LangtToolFontFamily{\textbf{\textcolor[rgb]{0,0,0.5019607843137255}{FLExTrans}}}}, as well as the {\LangtToolFontFamily{\textbf{\textcolor[rgb]{0,0,0.5019607843137255}{FLExTrans}}}} documentation, and information about the current version of {\LangtToolFontFamily{\textbf{\textcolor[rgb]{0,0,0.5019607843137255}{FLExTrans}}}}.}}
{\setlength{\XLingPapertempdim}{\XLingPapersingleletterlistitemwidth+\XLingPaperlistitemindent}\leftskip\XLingPapertempdim\relax
\interlinepenalty10000
\XLingPaperlistitem{\XLingPaperlistitemindent}{\XLingPapersingleletterlistitemwidth}{e.}{The {\LangtMenuFontFamily{\textbf{\textcolor[rgb]{0,0.6,0.2}{Help}}}} menu gives access to the {\LangtToolFontFamily{\textbf{\textcolor[rgb]{0,0,0.5019607843137255}{FLExTools}}}} documentation, as well as some additional resources for writing modules for {\LangtToolFontFamily{\textbf{\textcolor[rgb]{0,0,0.5019607843137255}{FLExTools}}}}, and the "About" information for {\LangtToolFontFamily{\textbf{\textcolor[rgb]{0,0,0.5019607843137255}{FLExTools}}}}.}}}}
{\setlength{\XLingPapertempdim}{\XLingPapersingledigitlistitemwidth+\parindent{}}\leftskip\XLingPapertempdim\relax
\interlinepenalty10000
\XLingPaperlistitem{\parindent{}}{\XLingPapersingledigitlistitemwidth}{2.}{The {\textit{\textbf{\textcolor[rgb]{0.6392156862745098,0.28627450980392155,0.6431372549019608}{Toolbar}}}} provides icons that can be clicked to run the most common {\LangtToolFontFamily{\textbf{\textcolor[rgb]{0,0,0.5019607843137255}{FLExTools}}}} commands.}{\setlength{\XLingPaperlistitemindent}{\XLingPapersingledigitlistitemwidth + \parindent{}}
{\setlength{\XLingPapertempdim}{\XLingPapersingleletterlistitemwidth+\XLingPaperlistitemindent}\leftskip\XLingPapertempdim\relax
\interlinepenalty10000
\XLingPaperlistitem{\XLingPaperlistitemindent}{\XLingPapersingleletterlistitemwidth}{a.}{The {\textup{\textbf{Source Project}}} icon allows you to choose from the {\LangtToolFontFamily{\textbf{\textcolor[rgb]{0,0,0.5019607843137255}{FLEx}}}} projects that are available on your device.}}
{\setlength{\XLingPapertempdim}{\XLingPapersingleletterlistitemwidth+\XLingPaperlistitemindent}\leftskip\XLingPapertempdim\relax
\interlinepenalty10000
\XLingPaperlistitem{\XLingPaperlistitemindent}{\XLingPapersingleletterlistitemwidth}{b.}{The {\textup{\textbf{Module Info}}} button shows more details about whichever module is currently selected.}}
{\setlength{\XLingPapertempdim}{\XLingPapersingleletterlistitemwidth+\XLingPaperlistitemindent}\leftskip\XLingPapertempdim\relax
\interlinepenalty10000
\XLingPaperlistitem{\XLingPaperlistitemindent}{\XLingPapersingleletterlistitemwidth}{c.}{The {\textup{\textbf{Run}}} button runs the currently selected module. Note: you can multi-select modules (using ctrl-click or shift-click) and use the {\textup{\textbf{Run}}} button to run all the modules that you have selected.}}
{\setlength{\XLingPapertempdim}{\XLingPapersingleletterlistitemwidth+\XLingPaperlistitemindent}\leftskip\XLingPapertempdim\relax
\interlinepenalty10000
\XLingPaperlistitem{\XLingPaperlistitemindent}{\XLingPapersingleletterlistitemwidth}{d.}{The {\textup{\textbf{Run All}}} button runs all the modules in the current collection, in order. If one of the modules results in an error, operation will stop and the remaining modules will not be run. }}}}
{\setlength{\XLingPapertempdim}{\XLingPapersingledigitlistitemwidth+\parindent{}}\leftskip\XLingPapertempdim\relax
\interlinepenalty10000
\XLingPaperlistitem{\parindent{}}{\XLingPapersingledigitlistitemwidth}{3.}{The {\textit{\textbf{\textcolor[rgb]{0.6392156862745098,0.28627450980392155,0.6431372549019608}{Modules pane}}}} displays all the modules that are in the current collection. Keep in mind that this is a subset of all the modules that are available in {\LangtToolFontFamily{\textbf{\textcolor[rgb]{0,0,0.5019607843137255}{FLExTrans}}}}, and modules may exist in more than one collection.}}
{\setlength{\XLingPapertempdim}{\XLingPapersingledigitlistitemwidth+\parindent{}}\leftskip\XLingPapertempdim\relax
\interlinepenalty10000
\XLingPaperlistitem{\parindent{}}{\XLingPapersingledigitlistitemwidth}{4.}{The {\textit{\textbf{\textcolor[rgb]{0.6392156862745098,0.28627450980392155,0.6431372549019608}{Messages pane}}}} displays any messages, warnings, or errors that result from running modules.}}
{\setlength{\XLingPapertempdim}{\XLingPapersingledigitlistitemwidth+\parindent{}}\leftskip\XLingPapertempdim\relax
\interlinepenalty10000
\XLingPaperlistitem{\parindent{}}{\XLingPapersingledigitlistitemwidth}{5.}{The {\textit{\textbf{\textcolor[rgb]{0.6392156862745098,0.28627450980392155,0.6431372549019608}{Status Bar}}}} gives details that indicate the current context for your operations:}{\setlength{\XLingPaperlistitemindent}{\XLingPapersingledigitlistitemwidth + \parindent{}}
{\setlength{\XLingPapertempdim}{\XLingPapersingleletterlistitemwidth+\XLingPaperlistitemindent}\leftskip\XLingPapertempdim\relax
\interlinepenalty10000
\XLingPaperlistitem{\XLingPaperlistitemindent}{\XLingPapersingleletterlistitemwidth}{a.}{The current collection}}
{\setlength{\XLingPapertempdim}{\XLingPapersingleletterlistitemwidth+\XLingPaperlistitemindent}\leftskip\XLingPapertempdim\relax
\interlinepenalty10000
\XLingPaperlistitem{\XLingPaperlistitemindent}{\XLingPapersingleletterlistitemwidth}{b.}{The name of your project folder}}
{\setlength{\XLingPapertempdim}{\XLingPapersingleletterlistitemwidth+\XLingPaperlistitemindent}\leftskip\XLingPapertempdim\relax
\interlinepenalty10000
\XLingPaperlistitem{\XLingPaperlistitemindent}{\XLingPapersingleletterlistitemwidth}{c.}{The Source Text that is currently selected}}}}
\vspace{\baselineskip}
}
\vspace{12pt}
\XLingPaperneedspace{5\baselineskip}

\penalty-3000{\noindent{\raisebox{\baselineskip}[0pt]{\protect\hypertarget{sSettings}{}}\SectionLevelTwoFontFamily{\normalsize{\raisebox{\baselineskip}[0pt]{\pdfbookmark[2]{3.2 The FLExTrans Settings}{sSettings}}\textbf{3.2 The {\LangtToolFontFamily{\textbf{\textcolor[rgb]{0,0,0.5019607843137255}{FLExTrans Settings}}}}}}}}
\markright{The {\LangtToolFontFamily{\textbf{\textcolor[rgb]{0,0,0.5019607843137255}{FLExTrans Settings}}}}}
\XLingPaperaddtocontents{sSettings}}\par{}\penalty10000
\vspace{12pt}
\indent The {\LangtToolFontFamily{\textbf{\textcolor[rgb]{0,0,0.5019607843137255}{FLExTrans Settings}}}} provide a way to configure various details needed for {\LangtToolFontFamily{\textbf{\textcolor[rgb]{0,0,0.5019607843137255}{FLExTrans}}}}. You launch the {\LangtToolFontFamily{\textbf{\textcolor[rgb]{0,0,0.5019607843137255}{FLExTrans Settings}}}} by selecting {\LangtMenuFontFamily{\textbf{\textcolor[rgb]{0,0.6,0.2}{FLExTrans}}}} \textgreater{} {\LangtMenuFontFamily{\textbf{\textcolor[rgb]{0,0.6,0.2}{Settings}}}} from the {\LangtToolFontFamily{\textbf{\textcolor[rgb]{0,0,0.5019607843137255}{FLExTools}}}} main menu. Hover over a text box or {\textup{\textbf{Browse}}} button to get an explanation of the setting. For file or folder names in the {\LangtToolFontFamily{\textbf{\textcolor[rgb]{0,0,0.5019607843137255}{FLExTrans Settings}}}}, you can change the file to be used by simply typing the changes in the pertinent text box. The {\textit{Basic}} {\textup{\textbf{View Mode}}} is seen here; other options are {\textit{Mini}} and {\textit{Full}}.\par{}{\vspace{12pt plus 2pt minus 1pt}\raggedright{}\XLingPaperexample{.125in}{0pt}{2.75em}{\raisebox{\baselineskip}[0pt]{\protect\hypertarget{xSettings}{}}(15)}{\parbox[t]{\textwidth - .125in - 0pt}{\vspace*{-\baselineskip}{\XeTeXpicfile "../Images/SettingsWindowBasic.png" scaled 600}}}
}
\vspace{12pt}
\XLingPaperneedspace{5\baselineskip}

\penalty-3000{\noindent{\raisebox{\baselineskip}[0pt]{\protect\hypertarget{sCollections}{}}\SectionLevelTwoFontFamily{\normalsize{\raisebox{\baselineskip}[0pt]{\pdfbookmark[2]{3.3 Using Collections}{sCollections}}\textbf{3.3 Using Collections}}}}
\markright{Using Collections}
\XLingPaperaddtocontents{sCollections}}\par{}\penalty10000
\vspace{12pt}
\indent Most of the tasks in {\LangtToolFontFamily{\textbf{\textcolor[rgb]{0,0,0.5019607843137255}{FLExTrans}}}} are accomplished by running several modules. To make it easier to find the modules needed for larger tasks, collections provide a way to see only a subset of the available modules, presented in a set sequence. Modules can be in more than one collection.\par{}\indent To change to a different collection, click the desired tab just below the toolbar buttons.\par{}\indent To manage collections, select {\LangtMenuFontFamily{\textbf{\textcolor[rgb]{0,0.6,0.2}{Manage Collections}}}} from the {\LangtMenuFontFamily{\textbf{\textcolor[rgb]{0,0.6,0.2}{FLExTools}}}} menu.\par{}\indent This will bring up the {\LangtToolFontFamily{\textbf{\textcolor[rgb]{0,0,0.5019607843137255}{Collections Manager}}}}, as shown in \hyperlink{xCollectionsManager}{(16)}\par{}{\vspace{12pt plus 2pt minus 1pt}\raggedright{}\XLingPaperexample{.125in}{0pt}{2.75em}{\raisebox{\baselineskip}[0pt]{\protect\hypertarget{xCollectionsManager}{}}(16)}{\parbox[t]{\textwidth - .125in - 0pt}{\vspace*{-\baselineskip}{\XeTeXpicfile "../Images/CollectionsManager.png" scaled 600}}}
\vspace{12pt plus 2pt minus 1pt}}\par\indent In the upper left pane is a list of the currently defined collections. The upper right pane shows all the modules that are in the currently selected collection. The lower left pane shows all possible modules in FLExTrans. If a module is selected in this pane, the lower right pane will provide information about that module.\par{}\indent If you want to make your own collection to hold modules that you frequently run together, use the \vspace*{0pt}{\XeTeXpicfile "../Images/NewButton.png" scaled 750} button to create a new collection. Select a module in the lower left pane and use the \vspace*{0pt}{\XeTeXpicfile "../Images/AddModuleButton.png" scaled 750} button to add it to your collection. Use the \vspace*{0pt}{\XeTeXpicfile "../Images/MoveUpButton.png" scaled 750} and \vspace*{0pt}{\XeTeXpicfile "../Images/MoveDownButton.png" scaled 750} buttons to change the sequence.\par{}\indent When you are finished making your collection, click the {\textup{\textbf{Select}}} button on the toolbar.\par{}
\vspace{12pt}
\XLingPaperneedspace{5\baselineskip}

\penalty-3000{{\centering\raisebox{\baselineskip}[0pt]{\protect\hypertarget{sTools}{}}\SectionLevelOneFontFamily{\large{\raisebox{\baselineskip}[0pt]{\pdfbookmark[1]{4 FLExTrans Tools}{sTools}}\textbf{4 {\LangtToolFontFamily{\textbf{\textcolor[rgb]{0,0,0.5019607843137255}{FLExTrans}}}} Tools}}}\\{}}\markright{{\LangtToolFontFamily{\textbf{\textcolor[rgb]{0,0,0.5019607843137255}{FLExTrans}}}} Tools}
\XLingPaperaddtocontents{sTools}}\par{}\penalty10000
\vspace{12pt}

\vspace{12pt}
\XLingPaperneedspace{5\baselineskip}

\penalty-3000{\noindent{\raisebox{\baselineskip}[0pt]{\protect\hypertarget{sLinker}{}}\SectionLevelTwoFontFamily{\normalsize{\raisebox{\baselineskip}[0pt]{\pdfbookmark[2]{4.1 The Sense Linker Tool}{sLinker}}\textbf{4.1 The {\LangtToolFontFamily{\textbf{\textcolor[rgb]{0,0,0.5019607843137255}{Sense Linker Tool}}}}}}}}
\markright{The {\LangtToolFontFamily{\textbf{\textcolor[rgb]{0,0,0.5019607843137255}{Sense Linker Tool}}}}}
\XLingPaperaddtocontents{sLinker}}\par{}\penalty10000
\vspace{12pt}
\indent The {\LangtToolFontFamily{\textbf{\textcolor[rgb]{0,0,0.5019607843137255}{Sense Linker Tool}}}} is a tool for linking source senses to target senses. The tool looks through the text specified in the settings and lists each and every sense that was analyzed in that text in order. Here’s a sample screen shot of the {\LangtToolFontFamily{\textbf{\textcolor[rgb]{0,0,0.5019607843137255}{Sense Linker Tool}}}}:\par{}{\vspace{12pt plus 2pt minus 1pt}\raggedright{}\XLingPaperexample{.125in}{0pt}{2.75em}{\raisebox{\baselineskip}[0pt]{\protect\hypertarget{xLinkerToolShot}{}}(17)}{\parbox[t]{\textwidth - .125in - 0pt}{\vspace*{-\baselineskip}{\XeTeXpicfile "../Images/SenseLinkerCapture.png" scaled 600}}}
}
\vspace{12pt}
\XLingPaperneedspace{5\baselineskip}

\penalty-3000{\noindent{\raisebox{\baselineskip}[0pt]{\protect\hypertarget{sColorKey}{}}\SectionLevelThreeFontFamily{\normalsize{\raisebox{\baselineskip}[0pt]{\pdfbookmark[3]{4.1.1 Color Coding Key}{sColorKey}}\textit{4.1.1 Color Coding Key}}}}
\markright{Color Coding Key}
\XLingPaperaddtocontents{sColorKey}}\par{}\penalty10000
\vspace{12pt}
\indent Here’s a key to the color coding:\par{}\hspace*{.125in}{
\XLingPaperminmaxcellincolumn{Color}{\XLingPapermincola}{\textbf{Color}}{\XLingPapermaxcola}{+0\tabcolsep}
\XLingPaperminmaxcellincolumn{}{\XLingPapermincolb}{\textbf{}}{\XLingPapermaxcolb}{+0\tabcolsep}
\XLingPaperminmaxcellincolumn{White}{\XLingPapermincola}{White}{\XLingPapermaxcola}{+0\tabcolsep}
\XLingPaperminmaxcellincolumn{Source}{\XLingPapermincolb}{Source sense has been linked to a target sense.}{\XLingPapermaxcolb}{+0\tabcolsep}
\XLingPaperminmaxcellincolumn{Red}{\XLingPapermincola}{Red}{\XLingPapermaxcola}{+0\tabcolsep}
\XLingPaperminmaxcellincolumn{Source}{\XLingPapermincolb}{Source sense has not been linked to a target sense.}{\XLingPapermaxcolb}{+0\tabcolsep}
\XLingPaperminmaxcellincolumn{Light}{\XLingPapermincola}{Very Light Blue}{\XLingPapermaxcola}{+0\tabcolsep}
\XLingPaperminmaxcellincolumn{suggested}{\XLingPapermincolb}{A target sense has been suggested for this source sense through a partial match of the glosses.}{\XLingPapermaxcolb}{+0\tabcolsep}
\XLingPaperminmaxcellincolumn{Light}{\XLingPapermincola}{Light Blue}{\XLingPapermaxcola}{+0\tabcolsep}
\XLingPaperminmaxcellincolumn{suggested}{\XLingPapermincolb}{A target sense has been suggested for this source sense through an exact match of the glosses.}{\XLingPapermaxcolb}{+0\tabcolsep}
\XLingPaperminmaxcellincolumn{Green}{\XLingPapermincola}{Green}{\XLingPapermaxcola}{+0\tabcolsep}
\XLingPaperminmaxcellincolumn{changed}{\XLingPapermincolb}{The target sense has been set or changed for this source sense.}{\XLingPapermaxcolb}{+0\tabcolsep}
\XLingPaperminmaxcellincolumn{Text}{\XLingPapermincola}{Red Text}{\XLingPapermaxcola}{+0\tabcolsep}
\XLingPaperminmaxcellincolumn{mismatch}{\XLingPapermincolb}{There is a mismatch of grammatical category even though the glosses match.}{\XLingPapermaxcolb}{+0\tabcolsep}
\XLingPaperminmaxcellincolumn{Green}{\XLingPapermincola}{Green Text}{\XLingPapermaxcola}{+0\tabcolsep}
\XLingPaperminmaxcellincolumn{headwords}{\XLingPapermincolb}{Source headwords}{\XLingPapermaxcolb}{+0\tabcolsep}
\XLingPaperminmaxcellincolumn{Blue}{\XLingPapermincola}{Blue Text}{\XLingPapermaxcola}{+0\tabcolsep}
\XLingPaperminmaxcellincolumn{headwords}{\XLingPapermincolb}{Target headwords}{\XLingPapermaxcolb}{+0\tabcolsep}
\setlength{\XLingPaperavailabletablewidth}{433.62pt}
\setlength{\XLingPapertableminwidth}{\XLingPapermincola+\XLingPapermincolb}
\setlength{\XLingPapertablemaxwidth}{\XLingPapermaxcola+\XLingPapermaxcolb}
\XLingPapercalculatetablewidthratio{}
\XLingPapersetcolumnwidth{\XLingPapercolawidth}{\XLingPapermincola}{\XLingPapermaxcola}{-0\tabcolsep}
\XLingPapersetcolumnwidth{\XLingPapercolbwidth}{\XLingPapermincolb}{\XLingPapermaxcolb}{-2\tabcolsep}\vspace*{-\baselineskip}
\begin{longtable}
[l]{@{}>{\raggedright}p{\XLingPapercolawidth}>{\raggedright}p{\XLingPapercolbwidth}@{}}\toprule\multicolumn{1}{@{}>{\raggedright}p{\XLingPapercolawidth}}{\textbf{Color}}&\multicolumn{1}{>{\raggedright}p{\XLingPapercolbwidth}@{}}{\textbf{}}\\%
\midrule\endhead \multicolumn{1}{@{}>{\raggedright}p{\XLingPapercolawidth}}{White}&\multicolumn{1}{>{\raggedright}p{\XLingPapercolbwidth}@{}}{Source sense has been linked to a target sense.}\\%
\multicolumn{1}{@{}>{\raggedright}p{\XLingPapercolawidth}}{Red}&\multicolumn{1}{>{\raggedright}p{\XLingPapercolbwidth}@{}}{Source sense has not been linked to a target sense.}\\%
\multicolumn{1}{@{}>{\raggedright}p{\XLingPapercolawidth}}{Very Light Blue}&\multicolumn{1}{>{\raggedright}p{\XLingPapercolbwidth}@{}}{A target sense has been suggested for this source sense through a partial match of the glosses.}\\%
\multicolumn{1}{@{}>{\raggedright}p{\XLingPapercolawidth}}{Light Blue}&\multicolumn{1}{>{\raggedright}p{\XLingPapercolbwidth}@{}}{A target sense has been suggested for this source sense through an exact match of the glosses.}\\%
\multicolumn{1}{@{}>{\raggedright}p{\XLingPapercolawidth}}{Green}&\multicolumn{1}{>{\raggedright}p{\XLingPapercolbwidth}@{}}{The target sense has been set or changed for this source sense.}\\%
\multicolumn{1}{@{}>{\raggedright}p{\XLingPapercolawidth}}{Red Text}&\multicolumn{1}{>{\raggedright}p{\XLingPapercolbwidth}@{}}{There is a mismatch of grammatical category even though the glosses match.}\\%
\multicolumn{1}{@{}>{\raggedright}p{\XLingPapercolawidth}}{Green Text}&\multicolumn{1}{>{\raggedright}p{\XLingPapercolbwidth}@{}}{Source headwords}\\%
\multicolumn{1}{@{}>{\raggedright}p{\XLingPapercolawidth}}{Blue Text}&\multicolumn{1}{>{\raggedright}p{\XLingPapercolbwidth}@{}}{Target headwords}\\\bottomrule%
\end{longtable}
}

\vspace{12pt}
\XLingPaperneedspace{5\baselineskip}

\penalty-3000{\noindent{\raisebox{\baselineskip}[0pt]{\protect\hypertarget{sHowLink}{}}\SectionLevelThreeFontFamily{\normalsize{\raisebox{\baselineskip}[0pt]{\pdfbookmark[3]{4.1.2 Linking}{sHowLink}}\textit{4.1.2 Linking}}}}
\markright{Linking}
\XLingPaperaddtocontents{sHowLink}}\par{}\penalty10000
\vspace{12pt}
\indent For suggested sense links (colored in blue shades), if you want to link them, simply check the check box in the {\textit{Link It}} column.\par{}\indent To link a source sense with a target sense where there is no suggestion or to change an existing link, you need to do two things. First, select the target sense that you want to link to in the drop down list of {\textup{\textbf{All Target Senses}}} then go to the {\textup{\textbf{Target Head Word}}} column for the source sense in question and double-click there. The target sense information should appear with a green background color. You can use the search text box to find the appropriate target sense.\par{}
\vspace{12pt}
\XLingPaperneedspace{5\baselineskip}

\penalty-3000{\noindent{\raisebox{\baselineskip}[0pt]{\protect\hypertarget{sSLTChangeSource}{}}\SectionLevelThreeFontFamily{\normalsize{\raisebox{\baselineskip}[0pt]{\pdfbookmark[3]{4.1.3 Source Text}{sSLTChangeSource}}\textit{4.1.3 Source Text}}}}
\markright{Source Text}
\XLingPaperaddtocontents{sSLTChangeSource}}\par{}\penalty10000
\vspace{12pt}
\indent If you want to work on linking senses for a different source text, click the {\textup{\textbf{Source Text}}} drop-down box and select the desired text. At this point, the {\LangtToolFontFamily{\textbf{\textcolor[rgb]{0,0,0.5019607843137255}{Sense Linker Tool}}}} will close and restart.\par{}
\vspace{12pt}
\XLingPaperneedspace{5\baselineskip}

\penalty-3000{\noindent{\raisebox{\baselineskip}[0pt]{\protect\hypertarget{sSearchList}{}}\SectionLevelThreeFontFamily{\normalsize{\raisebox{\baselineskip}[0pt]{\pdfbookmark[3]{4.1.4 Search the All Target Senses List}{sSearchList}}\textit{4.1.4 Search the All Target Senses List}}}}
\markright{Search the All Target Senses List}
\XLingPaperaddtocontents{sSearchList}}\par{}\penalty10000
\vspace{12pt}
\indent Use the {\textup{\textbf{Search here}}} field to find a sense. This will find senses in the list by matching the beginning of the headwords in the list. If you want to filter the target sense list to those matching part of a headword or the gloss or the part of speech (using parentheses), click on the {\textup{\textbf{Filter on all fields}}} check box and then type in the {\textup{\textbf{Search here}}} box. Note: if you want the {\textup{\textbf{Filter on all fields}}} check box to be checked by default every time, there's a setting for that. Change the {\textbf{\textcolor[rgb]{0.5882352941176471,0.29411764705882354,0}{Default to filtering on all fields?}}} to {\textit{Yes}} in the {\textup{\textbf{Linker Settings}}} section of the {\LangtToolFontFamily{\textbf{\textcolor[rgb]{0,0,0.5019607843137255}{FLExTrans Settings}}}}.\par{}{\vspace{12pt plus 2pt minus 1pt}\raggedright{}\XLingPaperexample{.125in}{0pt}{2.75em}{\raisebox{\baselineskip}[0pt]{\protect\hypertarget{xLinkerNoLinks-Filter}{}}(18)}{\parbox[t]{\textwidth - .125in - 0pt}{\vspace*{-\baselineskip}{\XeTeXpicfile "../Images/SenseLinkerNoLinks-Filter.png" scaled 750}}}
}
\vspace{12pt}
\XLingPaperneedspace{5\baselineskip}

\penalty-3000{\noindent{\raisebox{\baselineskip}[0pt]{\protect\hypertarget{sAddEntry}{}}\SectionLevelThreeFontFamily{\normalsize{\raisebox{\baselineskip}[0pt]{\pdfbookmark[3]{4.1.5 Add Entry}{sAddEntry}}\textit{4.1.5 Add Entry}}}}
\markright{Add Entry}
\XLingPaperaddtocontents{sAddEntry}}\par{}\penalty10000
\vspace{12pt}
{\raggedright{}\XLingPaperexample{.125in}{0pt}{2.75em}{\raisebox{\baselineskip}[0pt]{\protect\hypertarget{xNewEntry}{}}(19)}{\parbox[t]{\textwidth - .125in - 0pt}{\vspace*{-\baselineskip}{\XeTeXpicfile "../Images/AddEntry.png" scaled 600}}}
\vspace{12pt plus 2pt minus 1pt}}\par\indent At times you may realize the target sense that you want to link to is not yet in the target lexicon. You can click on the {\textup{\textbf{Add Entry }}} button to bring up the dialog shown in \hyperlink{xNewEntry}{(19)} to add the basic information about the new entry and sense. Once you click {\textup{\textbf{OK}}}, the new entry/sense gets added to the Target Sense List and gets selected and now you can use it for linking. Note that the new sense will always be sense number one. If you need to merge the sense into another entry, use the merge capability in {\LangtToolFontFamily{\textbf{\textcolor[rgb]{0,0,0.5019607843137255}{FLEx}}}}. If you do such a merge, the link you have established remains valid since {\LangtToolFontFamily{\textbf{\textcolor[rgb]{0,0,0.5019607843137255}{FLExTrans}}}} links directly to the sense regardless of where it is.\par{}\indent This function supports cluster projects. See \hyperlink{xClusterAddNewEntry}{(35)}.\par{}
\vspace{12pt}
\XLingPaperneedspace{5\baselineskip}

\penalty-3000{\noindent{\raisebox{\baselineskip}[0pt]{\protect\hypertarget{sShowUnlinked}{}}\SectionLevelThreeFontFamily{\normalsize{\raisebox{\baselineskip}[0pt]{\pdfbookmark[3]{4.1.6 Show Only Unlinked}{sShowUnlinked}}\textit{4.1.6 Show Only Unlinked}}}}
\markright{Show Only Unlinked}
\XLingPaperaddtocontents{sShowUnlinked}}\par{}\penalty10000
\vspace{12pt}
\indent If you only want to see the senses that need to be linked, check the box {\textup{\textbf{Show Only Unlinked}}}.\par{}
\vspace{12pt}
\XLingPaperneedspace{5\baselineskip}

\penalty-3000{\noindent{\raisebox{\baselineskip}[0pt]{\protect\hypertarget{sHidePN}{}}\SectionLevelThreeFontFamily{\normalsize{\raisebox{\baselineskip}[0pt]{\pdfbookmark[3]{4.1.7 Hide Proper Nouns}{sHidePN}}\textit{4.1.7 Hide Proper Nouns}}}}
\markright{Hide Proper Nouns}
\XLingPaperaddtocontents{sHidePN}}\par{}\penalty10000
\vspace{12pt}
\indent If you want to hide proper nouns, which you may not want to bother linking, (especially if the form is the same in both languages) click the box {\textup{\textbf{Hide Proper Nouns}}}.\par{}
\vspace{12pt}
\XLingPaperneedspace{5\baselineskip}

\penalty-3000{\noindent{\raisebox{\baselineskip}[0pt]{\protect\hypertarget{sFontControl}{}}\SectionLevelThreeFontFamily{\normalsize{\raisebox{\baselineskip}[0pt]{\pdfbookmark[3]{4.1.8 Zoom and Font Button}{sFontControl}}\textit{4.1.8 Zoom and Font Button}}}}
\markright{Zoom and Font Button}
\XLingPaperaddtocontents{sFontControl}}\par{}\penalty10000
\vspace{12pt}
\indent Quickly change the font size with the plus and minus buttons next to {\textup{\textbf{Zoom}}}. Change all aspects of the font with the {\textup{\textbf{Font}}} button.\par{}
\vspace{12pt}
\XLingPaperneedspace{5\baselineskip}

\penalty-3000{\noindent{\raisebox{\baselineskip}[0pt]{\protect\hypertarget{sRebuildBiling}{}}\SectionLevelThreeFontFamily{\normalsize{\raisebox{\baselineskip}[0pt]{\pdfbookmark[3]{4.1.9 Rebuild Bilingual Lexicon}{sRebuildBiling}}\textit{4.1.9 Rebuild Bilingual Lexicon}}}}
\markright{Rebuild Bilingual Lexicon}
\XLingPaperaddtocontents{sRebuildBiling}}\par{}\penalty10000
\vspace{12pt}
\indent After finishing sense linking, sooner or later you need to rebuild the bilingual lexicon. If the {\textup{\textbf{Rebuild the Bilingual Lexicon}}} check box is checked, this will happen automatically when you click OK. If you would like to change whether by default this box is checked or unchecked when you start the {\LangtToolFontFamily{\textbf{\textcolor[rgb]{0,0,0.5019607843137255}{Sense Linker Tool}}}}, set the corresponding setting in the {\LangtToolFontFamily{\textbf{\textcolor[rgb]{0,0,0.5019607843137255}{FLExTrans Settings}}}}.\par{}
\vspace{12pt}
\XLingPaperneedspace{5\baselineskip}

\penalty-3000{\noindent{\raisebox{\baselineskip}[0pt]{\protect\hypertarget{sExportUnlinked}{}}\SectionLevelThreeFontFamily{\normalsize{\raisebox{\baselineskip}[0pt]{\pdfbookmark[3]{4.1.10 Export Unlinked Senses}{sExportUnlinked}}\textit{4.1.10 Export Unlinked Senses}}}}
\markright{Export Unlinked Senses}
\XLingPaperaddtocontents{sExportUnlinked}}\par{}\penalty10000
\vspace{12pt}
\indent Use this check box to export a list of unlinked senses to a file (after clicking OK). If Hide Proper Nouns is checked, Proper Nouns will not be exported. The results pane will tell you where the file is and its name. This file can be useful to send to a colleague to fill out the correct target words for the given source word. The file is in HTML format, it works well to open this in a word processor and save it as a document. After the completed file is returned for use in {\LangtToolFontFamily{\textbf{\textcolor[rgb]{0,0,0.5019607843137255}{FLExTrans}}}}, it may be necessary to create new {\LangtToolFontFamily{\textbf{\textcolor[rgb]{0,0,0.5019607843137255}{FLEx}}}} entries and use the {\LangtToolFontFamily{\textbf{\textcolor[rgb]{0,0,0.5019607843137255}{Sense Linker Tool}}}}.\par{}
\vspace{12pt}
\XLingPaperneedspace{5\baselineskip}

\penalty-3000{\noindent{\raisebox{\baselineskip}[0pt]{\protect\hypertarget{sSenses2Link}{}}\SectionLevelThreeFontFamily{\normalsize{\raisebox{\baselineskip}[0pt]{\pdfbookmark[3]{4.1.11 Senses to link}{sSenses2Link}}\textit{4.1.11 Senses to link}}}}
\markright{Senses to link}
\XLingPaperaddtocontents{sSenses2Link}}\par{}\penalty10000
\vspace{12pt}
\indent At the bottom left the number of senses left to link is shown. Note that this is not necessarily the number of unchecked boxes, rather the number of unique senses that are remaining to link.\par{}
\vspace{12pt}
\XLingPaperneedspace{5\baselineskip}

\penalty-3000{\noindent{\raisebox{\baselineskip}[0pt]{\protect\hypertarget{sRuleTester}{}}\SectionLevelTwoFontFamily{\normalsize{\raisebox{\baselineskip}[0pt]{\pdfbookmark[2]{4.2 The Live Rule Tester Tool}{sRuleTester}}\textbf{4.2 The {\LangtToolFontFamily{\textbf{\textcolor[rgb]{0,0,0.5019607843137255}{Live Rule Tester Tool}}}}}}}}
\markright{The {\LangtToolFontFamily{\textbf{\textcolor[rgb]{0,0,0.5019607843137255}{Live Rule Tester Tool}}}}}
\XLingPaperaddtocontents{sRuleTester}}\par{}\penalty10000
\vspace{12pt}
\indent The {\LangtToolFontFamily{\textbf{\textcolor[rgb]{0,0,0.5019607843137255}{Live Rule Tester Tool}}}} is a tool that allows you to test source words or sentences live against transfer rules. This tool is especially helpful for finding out why transfer rules not are doing what you expect them to do. You can zero in on the problem by selecting just one source word and applying the pertinent transfer rule. In this way you don’t have to run the whole system against the whole text file and all transfer rules. Here’s sample screen shot of the {\LangtToolFontFamily{\textbf{\textcolor[rgb]{0,0,0.5019607843137255}{Live Rule Tester Tool}}}}:\par{}{\vspace{12pt plus 2pt minus 1pt}\raggedright{}\XLingPaperexample{.125in}{0pt}{2.75em}{\raisebox{\baselineskip}[0pt]{\protect\hypertarget{xTesterImage}{}}(20)}{\parbox[t]{\textwidth - .125in - 0pt}{\vspace*{-\baselineskip}{\XeTeXpicfile "../Images/LiveRuleTester.png" scaled 500}}}
}
\vspace{12pt}
\XLingPaperneedspace{5\baselineskip}

\penalty-3000{\noindent{\raisebox{\baselineskip}[0pt]{\protect\hypertarget{sTesterQuickGuide}{}}\SectionLevelThreeFontFamily{\normalsize{\raisebox{\baselineskip}[0pt]{\pdfbookmark[3]{4.2.1 Quick Guide}{sTesterQuickGuide}}\textit{4.2.1 Quick Guide}}}}
\markright{Quick Guide}
\XLingPaperaddtocontents{sTesterQuickGuide}}\par{}\penalty10000
\vspace{12pt}
\indent Here’s a quick guide on using the {\LangtToolFontFamily{\textbf{\textcolor[rgb]{0,0,0.5019607843137255}{Live Rule Tester Tool}}}}.\par{}{\parskip .5pt plus 1pt minus 1pt
                    
\vspace{\baselineskip}

{\setlength{\XLingPapertempdim}{\XLingPapersingledigitlistitemwidth+\parindent{}}\leftskip\XLingPapertempdim\relax
\interlinepenalty10000
\XLingPaperlistitem{\parindent{}}{\XLingPapersingledigitlistitemwidth}{1.}{First you want to choose something from your source text that you want to test. When you first run the tool, the words of the first sentence are shown both in the drop down list at the top and in the peach-colored area. You can check one or more words and, when you do, the \hyperlink{gDataStream}{data stream format} of the words is shown in the blue-colored area. You can quickly select a whole sentence, by clicking on the {\textit{\textbf{\textcolor[rgb]{0.6392156862745098,0.28627450980392155,0.6431372549019608}{Select Sentences}}}} tab and clicking on the sentence you want.}}
{\setlength{\XLingPapertempdim}{\XLingPapersingledigitlistitemwidth+\parindent{}}\leftskip\XLingPapertempdim\relax
\interlinepenalty10000
\XLingPaperlistitem{\parindent{}}{\XLingPapersingledigitlistitemwidth}{2.}{Next select which rule or rules you want to run against the words you selected. Check the check boxes in the purple area as appropriate. Check them all with the {\textup{\textbf{Select All}}} button. Uncheck them all with the {\textup{\textbf{Unselect All}}} button. Move a rule up or down by clicking on it and clicking the up or down arrow buttons. Moving the rules in the {\LangtToolFontFamily{\textbf{\textcolor[rgb]{0,0,0.5019607843137255}{Live Rule Tester Tool}}}} will not affect the order of the rules in the transfer rules file, and is only possible here for testing purposes. Note that rules are run in the order that they are listed, both in the {\LangtToolFontFamily{\textbf{\textcolor[rgb]{0,0,0.5019607843137255}{Live Rule Tester Tool}}}} and when the transfer rules file is in use.}}
{\setlength{\XLingPapertempdim}{\XLingPapersingledigitlistitemwidth+\parindent{}}\leftskip\XLingPapertempdim\relax
\interlinepenalty10000
\XLingPaperlistitem{\parindent{}}{\XLingPapersingledigitlistitemwidth}{3.}{Now you are ready to test. Click the {\textup{\textbf{Transfer}}} button and you will process the selected words. You will see the results in \hyperlink{gDataStream}{data stream format} in the green-colored {\textit{Target Text}} area.}}
{\setlength{\XLingPapertempdim}{\XLingPapersingledigitlistitemwidth+\parindent{}}\leftskip\XLingPapertempdim\relax
\interlinepenalty10000
\XLingPaperlistitem{\parindent{}}{\XLingPapersingledigitlistitemwidth}{4.}{Check the yellow {\textit{Rule Execution Information}} area. It will show which words were matched for the rules that were executed.}}
{\setlength{\XLingPapertempdim}{\XLingPapersingledigitlistitemwidth+\parindent{}}\leftskip\XLingPapertempdim\relax
\interlinepenalty10000
\XLingPaperlistitem{\parindent{}}{\XLingPapersingledigitlistitemwidth}{5.}{Now you are ready to synthesize. Click the {\textup{\textbf{Synthesize}}} button and you will synthesize the target text into target words in the salmon-colored {\textit{Synthesized Text}} area.}}
{\setlength{\XLingPapertempdim}{\XLingPapersingledigitlistitemwidth+\parindent{}}\leftskip\XLingPapertempdim\relax
\interlinepenalty10000
\XLingPaperlistitem{\parindent{}}{\XLingPapersingledigitlistitemwidth}{6.}{Use the {\textup{\textbf{Add to Testbed}}} button to add the input words and synthesized result to the \hyperlink{sTestbed}{FLExTrans Testbed}. This is essentially a database of expected results for a certain inputs. In another part of {\LangtToolFontFamily{\textbf{\textcolor[rgb]{0,0,0.5019607843137255}{FLExTrans}}}} you can run the {\LangtToolFontFamily{\textbf{\textcolor[rgb]{0,0,0.5019607843137255}{Testbed}}}} to see if you are still getting the results you expect.}}
\vspace{\baselineskip}
}
\vspace{12pt}
\XLingPaperneedspace{5\baselineskip}

\penalty-3000{\noindent{\raisebox{\baselineskip}[0pt]{\protect\hypertarget{sLRTChangeSource}{}}\SectionLevelThreeFontFamily{\normalsize{\raisebox{\baselineskip}[0pt]{\pdfbookmark[3]{4.2.2 Change Source Text}{sLRTChangeSource}}\textit{4.2.2 Change Source Text}}}}
\markright{Change Source Text}
\XLingPaperaddtocontents{sLRTChangeSource}}\par{}\penalty10000
\vspace{12pt}
\indent If you want to work on a different source text, click the {\textup{\textbf{Source Text}}} drop-down box and select the desired text. At this point, the {\LangtToolFontFamily{\textbf{\textcolor[rgb]{0,0,0.5019607843137255}{Live Rule Tester Tool}}}} will close and restart.\par{}
\vspace{12pt}
\XLingPaperneedspace{5\baselineskip}

\penalty-3000{\noindent{\raisebox{\baselineskip}[0pt]{\protect\hypertarget{sHover}{}}\SectionLevelThreeFontFamily{\normalsize{\raisebox{\baselineskip}[0pt]{\pdfbookmark[3]{4.2.3 Hover to See Bilingual Lexicon Entry}{sHover}}\textit{4.2.3 Hover to See Bilingual Lexicon Entry}}}}
\markright{Hover to See Bilingual Lexicon Entry}
\XLingPaperaddtocontents{sHover}}\par{}\penalty10000
\vspace{12pt}
\indent When you are in the {\textit{\textbf{\textcolor[rgb]{0.6392156862745098,0.28627450980392155,0.6431372549019608}{Select Words}}}} tab in the top pane, you can hover over a word to see what the bilingual lexicon entry looks like for that word. As shown in the image above when you hover over the word ‘liebe’, use see that the bilingual lexicon maps ‘lieben’ to ‘älska’. This feature can be very useful when inflection features are present on the target word. These will show in the pop-up as well. If you have used “\hyperlink{sReplEditTool}{The {\LangtToolFontFamily{\textbf{\textcolor[rgb]{0,0,0.5019607843137255}{Replacement Dictionary Editor}}}}}”, you may see multiple mappings in the pop-up image.\par{}
\vspace{12pt}
\XLingPaperneedspace{5\baselineskip}

\penalty-3000{\noindent{\raisebox{\baselineskip}[0pt]{\protect\hypertarget{sTesterOther}{}}\SectionLevelThreeFontFamily{\normalsize{\raisebox{\baselineskip}[0pt]{\pdfbookmark[3]{4.2.4 Other Parts of the Tool}{sTesterOther}}\textit{4.2.4 Other Parts of the Tool}}}}
\markright{Other Parts of the Tool}
\XLingPaperaddtocontents{sTesterOther}}\par{}\penalty10000
\vspace{12pt}
\indent By default the transfer rules file, the bilingual lexicon file and the source text are loaded according to the {\LangtToolFontFamily{\textbf{\textcolor[rgb]{0,0,0.5019607843137255}{FLExTrans Settings}}}}. If you want to rebuild the bilingual lexicon, perhaps after changing something in a lexicon, click the {\textup{\textbf{Rebuild Bilingual Lexicon}}} button.\par{}\indent You may have noticed that there is a {\textit{\textbf{\textcolor[rgb]{0.6392156862745098,0.28627450980392155,0.6431372549019608}{Manual Entry}}}} tab for the source text words. You can enter words (technically word-senses) directly in plain text data stream format\footnote[3]{See footnote \hyperlink{nRawDataStream}{17}.}. This might be useful if you want to test a word-sense that isn’t in a given text.\par{}\indent The {\textup{\textbf{Interchunk}}} and {\textup{\textbf{Postchunk}}} tabs are for use in testing rules using the advanced {\LangtToolFontFamily{\textbf{\textcolor[rgb]{0,0,0.5019607843137255}{Apertium}}}} transfer engine.\par{}\indent You can’t edit the transfer rules in the {\LangtToolFontFamily{\textbf{\textcolor[rgb]{0,0,0.5019607843137255}{Live Rule Tester Tool}}}} and even when you change the order of the rules in the tool that doesn’t change your transfer rules file. Make your changes in the {\LangtToolFontFamily{\textbf{\textcolor[rgb]{0,0,0.5019607843137255}{XMLmind XML Editor}}}}. You can launch the editor by clicking on the {\textup{\textbf{Edit Transfer Rules}}} button. To reload the transfer rules after you have changed them, click on the {\textup{\textbf{Refresh Rules}}} button.\par{}\indent The {\textup{\textbf{View Bilingual Lexicon}}} will start the {\LangtToolFontFamily{\textbf{\textcolor[rgb]{0,0,0.5019607843137255}{XMLmind XML Editor}}}} for viewing the bilingual lexicon. The {\textup{\textbf{Replacement Dictionary Editor}}} button will start the {\LangtToolFontFamily{\textbf{\textcolor[rgb]{0,0,0.5019607843137255}{Replacement Dictionary Editor}}}} module.\par{}\indent If you change the target lexicon, click the {\textup{\textbf{Refresh Target Lexicon}}} button to reload it so that upon synthesis, {\LangtToolFontFamily{\textbf{\textcolor[rgb]{0,0,0.5019607843137255}{FLExTrans}}}} uses the changed target lexicon.\par{}\indent There is a check box that says {\textup{\textbf{Add multiple words to the testbed word by word}}}. Use this when you want to add multiple one-word translations to the testbed in batch. This is useful, for example, if you have a "sentence" that runs through a paradigm. When you click {\textup{\textbf{Add to Testbed}}}, {\LangtToolFontFamily{\textbf{\textcolor[rgb]{0,0,0.5019607843137255}{FLExTrans}}}} will add each individual word and its corresponding synthesis to the {\LangtToolFontFamily{\textbf{\textcolor[rgb]{0,0,0.5019607843137255}{Testbed}}}} in one go. Each pair will be its own test.\par{}\indent The {\textup{\textbf{Open Rule Assistant}}} button will launch the {\LangtToolFontFamily{\textbf{\textcolor[rgb]{0,0,0.5019607843137255}{Rule Assistant}}}} and the test data shown in it will be from the source text that is currently shown in the {\LangtToolFontFamily{\textbf{\textcolor[rgb]{0,0,0.5019607843137255}{Live Rule Tester Tool}}}}.\par{}\indent The {\textup{\textbf{Trace HermitCrab synthesis}}} check box, when checked before clicking the {\textup{\textbf{Synthesize}}} button, will bring up a web page showing trace information. This may be helpful when trying to troubleshoot why some morphemes are not synthesizing correctly. This check box will only be visible if you have already indicated you want to use HermitCrab synthesis in the {\LangtToolFontFamily{\textbf{\textcolor[rgb]{0,0,0.5019607843137255}{FLExTrans Settings}}}}.\par{}\indent The {\textup{\textbf{Apply Text Out rules}}} check box, when checked before clicking the {\textup{\textbf{Synthesize}}} button, will execute all of the search/replace rules you have defined with the {\LangtModuleFontFamily{\textbf{\textcolor[rgb]{0.4,0,0.4}{Text Out Rules}}}} module. These will be applied before showing the result in the {\textup{\textbf{Synthesis Text}}} box. This check box will only be visible if you have already added some rules via the {\LangtModuleFontFamily{\textbf{\textcolor[rgb]{0.4,0,0.4}{Text Out Rules}}}} module.\par{}\indent The {\textup{\textbf{Do not clean up unknown words}}} check box, when checked before clicking the {\textup{\textbf{Synthesize}}} button, will prevent the synthesis process from removing symbols around synthesized words that indicate a word didn't synthesize. For example, {\LangtCourierFontFamily{{\%0\%bilar\%}}} will not get cleaned up to be {\LangtCourierFontFamily{{bilar}}}. For production use of FT it is often helpful to do this clean up for words that you don't expect to synthesize, but for troubleshooting synthesis issues, it is helpful to see any problems. Otherwise, these problems may lie hidden. This check box will only be visible if you have already selected {\textit{Yes}} for the {\textbf{\textcolor[rgb]{0.5882352941176471,0.29411764705882354,0}{Clean Up Unknown Target Words}}} setting in the {\LangtToolFontFamily{\textbf{\textcolor[rgb]{0,0,0.5019607843137255}{FLExTrans Settings}}}}.\par{}
\vspace{12pt}
\XLingPaperneedspace{5\baselineskip}

\penalty-3000{\noindent{\raisebox{\baselineskip}[0pt]{\protect\hypertarget{sViewSrcTgt}{}}\SectionLevelTwoFontFamily{\normalsize{\raisebox{\baselineskip}[0pt]{\pdfbookmark[2]{4.3 The View Source/Target Apertium Text Tool}{sViewSrcTgt}}\textbf{4.3 The {\LangtToolFontFamily{\textbf{\textcolor[rgb]{0,0,0.5019607843137255}{View Source/Target Apertium Text Tool}}}}}}}}
\markright{The {\LangtToolFontFamily{\textbf{\textcolor[rgb]{0,0,0.5019607843137255}{View Source/Target Apertium Text Tool}}}}}
\XLingPaperaddtocontents{sViewSrcTgt}}\par{}\penalty10000
\vspace{12pt}
\indent The {\LangtToolFontFamily{\textbf{\textcolor[rgb]{0,0,0.5019607843137255}{View Source/Target Apertium Text Tool}}}} is a tool that allows you to view the {\LangtToolFontFamily{\textbf{\textcolor[rgb]{0,0,0.5019607843137255}{Apertium}}}} source or target files in an easy-to-read manner. You may not need to examine the {\LangtToolFontFamily{\textbf{\textcolor[rgb]{0,0,0.5019607843137255}{Apertium}}}} files {\textit{source\_text-aper.txt}} or {\textit{target\_text-aper.txt}} very often, but sometimes you may find it useful to see what got extracted from the source {\LangtToolFontFamily{\textbf{\textcolor[rgb]{0,0,0.5019607843137255}{FLEx}}}} project into {\textit{source\_text.txt}} or to see what the transfer rules produced after running the {\LangtModuleFontFamily{\textbf{\textcolor[rgb]{0.4,0,0.4}{RunApertium}}}} module in the form of {\textit{target\_text.txt}}. Run this tool to see the contents of these files. \hyperlink{xViewSrcTgt}{(21)} shows what the tool looks like. The actual file contents look like \hyperlink{xDataStream}{(22)}.\par{}{\vspace{12pt plus 2pt minus 1pt}\raggedright{}\XLingPaperexample{.125in}{0pt}{2.75em}{\raisebox{\baselineskip}[0pt]{\protect\hypertarget{xViewSrcTgt}{}}(21)}{\parbox[t]{\textwidth - .125in - 0pt}{\vspace*{-\baselineskip}{\XeTeXpicfile "../Images/SourceTargetViewer.png" scaled 750}}}
}{\vspace{12pt plus 2pt minus 1pt}\raggedright{}\XLingPaperexample{.125in}{0pt}{2.75em}{\raisebox{\baselineskip}[0pt]{\protect\hypertarget{xDataStream}{}}(22)}{\parbox[t]{\textwidth - .125in - 0pt}{\vspace*{-\baselineskip}{\XeTeXpicfile "../Images/DataStreamFormat.PNG" scaled 750}}}
\vspace{12pt plus 2pt minus 1pt}}\par\indent You have the options of changing the font, increasing the zoom level, displaying the text right-to-left or opening a similar view in the default web browser. If you click on the {\textbf{Target}} button, you will see the contents of {\textit{target\_text.txt}}.\par{}
\vspace{12pt}
\XLingPaperneedspace{5\baselineskip}

\penalty-3000{\noindent{\raisebox{\baselineskip}[0pt]{\protect\hypertarget{sParatext}{}}\SectionLevelTwoFontFamily{\normalsize{\raisebox{\baselineskip}[0pt]{\pdfbookmark[2]{4.4 Paratext Import and Export Tools}{sParatext}}\textbf{4.4 {\LangtToolFontFamily{\textbf{\textcolor[rgb]{0,0,0.5019607843137255}{Paratext}}}} Import and Export Tools}}}}
\markright{{\LangtToolFontFamily{\textbf{\textcolor[rgb]{0,0,0.5019607843137255}{Paratext}}}} Import and Export Tools}
\XLingPaperaddtocontents{sParatext}}\par{}\penalty10000
\vspace{12pt}

\vspace{12pt}
\XLingPaperneedspace{5\baselineskip}

\penalty-3000{\noindent{\raisebox{\baselineskip}[0pt]{\protect\hypertarget{sPtxImport}{}}\SectionLevelThreeFontFamily{\normalsize{\raisebox{\baselineskip}[0pt]{\pdfbookmark[3]{4.4.1 Paratext Import}{sPtxImport}}\textit{4.4.1 {\LangtToolFontFamily{\textbf{\textcolor[rgb]{0,0,0.5019607843137255}{Paratext}}}} Import}}}}
\markright{{\LangtToolFontFamily{\textbf{\textcolor[rgb]{0,0,0.5019607843137255}{Paratext}}}} Import}
\XLingPaperaddtocontents{sPtxImport}}\par{}\penalty10000
\vspace{12pt}
\indent To import {\LangtToolFontFamily{\textbf{\textcolor[rgb]{0,0,0.5019607843137255}{Paratext}}}} chapters into the source {\LangtToolFontFamily{\textbf{\textcolor[rgb]{0,0,0.5019607843137255}{FLEx}}}} project, use the {\LangtToolFontFamily{\textbf{\textcolor[rgb]{0,0,0.5019607843137255}{Import Text From Paratext}}}} module. Fill out the {\LangtToolFontFamily{\textbf{\textcolor[rgb]{0,0,0.5019607843137255}{Paratext}}}} project abbreviation in the first box. Note: this must be a project that you are at least an {\textit{Observer}} on in {\LangtToolFontFamily{\textbf{\textcolor[rgb]{0,0,0.5019607843137255}{Paratext}}}}.\par{}\indent Also, you cannot import from {\LangtToolFontFamily{\textbf{\textcolor[rgb]{0,0,0.5019607843137255}{Paratext}}}} resources. Put the {\LangtToolFontFamily{\textbf{\textcolor[rgb]{0,0,0.5019607843137255}{Paratext}}}} book abbreviation in the second box. Select from and to chapters.\par{}\indent After importing, you will have a new text in the source {\LangtToolFontFamily{\textbf{\textcolor[rgb]{0,0,0.5019607843137255}{FLEx}}}} project. It will be named something like the following: John 03-04. Note: if a text with that name already exists, you will get a text with the same name with '- Copy' appended, unless you have checked the {\textit{Overwrite existing text}} box.\par{}\indent Additional options are the following:\par{}{\parskip .5pt plus 1pt minus 1pt

\vspace{\baselineskip}

{\setlength{\XLingPapertempdim}{\XLingPaperbulletlistitemwidth+\parindent{}}\leftskip\XLingPapertempdim\relax
\interlinepenalty10000
\XLingPaperlistitem{\parindent{}}{\XLingPaperbulletlistitemwidth}{•}{{\textit{\textbf{Create one text per chapter}}} - when more than one chapter is specified, create one text in {\LangtToolFontFamily{\textbf{\textcolor[rgb]{0,0,0.5019607843137255}{FLEx}}}} for each chapter.}}
{\setlength{\XLingPapertempdim}{\XLingPaperbulletlistitemwidth+\parindent{}}\leftskip\XLingPapertempdim\relax
\interlinepenalty10000
\XLingPaperlistitem{\parindent{}}{\XLingPaperbulletlistitemwidth}{•}{{\textit{\textbf{Make a new text the active text}}} - the newly created text will be configured to be the active source text in {\LangtToolFontFamily{\textbf{\textcolor[rgb]{0,0,0.5019607843137255}{FLExTrans}}}}. This means {\LangtToolFontFamily{\textbf{\textcolor[rgb]{0,0,0.5019607843137255}{FLExTrans}}}} modules will use it.}}
{\setlength{\XLingPapertempdim}{\XLingPaperbulletlistitemwidth+\parindent{}}\leftskip\XLingPapertempdim\relax
\interlinepenalty10000
\XLingPaperlistitem{\parindent{}}{\XLingPaperbulletlistitemwidth}{•}{{\textit{\textbf{Use full English book name for title}}} - e.g. JHN will change to John.}}
{\setlength{\XLingPapertempdim}{\XLingPaperbulletlistitemwidth+\parindent{}}\leftskip\XLingPapertempdim\relax
\interlinepenalty10000
\XLingPaperlistitem{\parindent{}}{\XLingPaperbulletlistitemwidth}{•}{{\textit{\textbf{Overwrite existing text}}} - If a text already exists with the same name exists, it will be overwritten. When this is not checked, '- Copy' will be appended to the text name when a text already exists.}}
{\setlength{\XLingPapertempdim}{\XLingPaperbulletlistitemwidth+\parindent{}}\leftskip\XLingPapertempdim\relax
\interlinepenalty10000
\XLingPaperlistitem{\parindent{}}{\XLingPaperbulletlistitemwidth}{•}{{\textit{\textbf{Include footnotes}}} - include all the text involved with footnotes}}
{\setlength{\XLingPapertempdim}{\XLingPaperbulletlistitemwidth+\parindent{}}\leftskip\XLingPapertempdim\relax
\interlinepenalty10000
\XLingPaperlistitem{\parindent{}}{\XLingPaperbulletlistitemwidth}{•}{{\textit{\textbf{Include cross references}}} - include all the text involved with cross references, namely \textbackslash{}x and \textbackslash{}r marked text.}}
{\setlength{\XLingPapertempdim}{\XLingPaperbulletlistitemwidth+\parindent{}}\leftskip\XLingPapertempdim\relax
\interlinepenalty10000
\XLingPaperlistitem{\parindent{}}{\XLingPaperbulletlistitemwidth}{•}{{\textit{\textbf{Include introduction (for chapter 1)}}} - bring in introductory information that is above chapter 1. This only applies to chapter 1.}}
\vspace{\baselineskip}
}{\raggedright{}\XLingPaperexample{.125in}{0pt}{2.75em}{\raisebox{\baselineskip}[0pt]{\protect\hypertarget{xParatextImport}{}}(23)}{\parbox[t]{\textwidth - .125in - 0pt}{\vspace*{-\baselineskip}{\XeTeXpicfile "../Images/ParatextImport.png" scaled 750}}}
\vspace{12pt plus 2pt minus 1pt}}\par\indent This module supports cluster projects. See \hyperlink{xClusterImportParatext}{(37)}.\par{}
\vspace{12pt}
\XLingPaperneedspace{5\baselineskip}

\penalty-3000{\noindent{\raisebox{\baselineskip}[0pt]{\protect\hypertarget{sPtxExport}{}}\SectionLevelThreeFontFamily{\normalsize{\raisebox{\baselineskip}[0pt]{\pdfbookmark[3]{4.4.2 Paratext Export}{sPtxExport}}\textit{4.4.2 {\LangtToolFontFamily{\textbf{\textcolor[rgb]{0,0,0.5019607843137255}{Paratext}}}} Export}}}}
\markright{{\LangtToolFontFamily{\textbf{\textcolor[rgb]{0,0,0.5019607843137255}{Paratext}}}} Export}
\XLingPaperaddtocontents{sPtxExport}}\par{}\penalty10000
\vspace{12pt}
\indent There are two tools that help you export to {\LangtToolFontFamily{\textbf{\textcolor[rgb]{0,0,0.5019607843137255}{Paratext}}}}. The difference between them is where you want to export your text from.\par{}{\parskip .5pt plus 1pt minus 1pt
                    
\vspace{\baselineskip}

{\setlength{\XLingPapertempdim}{\XLingPapersingledigitlistitemwidth+\parindent{}}\leftskip\XLingPapertempdim\relax
\interlinepenalty10000
\XLingPaperlistitem{\parindent{}}{\XLingPapersingledigitlistitemwidth}{1.}{{\LangtExportDraftBPtxToolFontFamily{\textbf{\textcolor[rgb]{0,0,0.5019607843137255}{Export FLExTrans Draft to Paratext}}}} - takes a newly synthesized text and exports it to {\LangtToolFontFamily{\textbf{\textcolor[rgb]{0,0,0.5019607843137255}{Paratext}}}}.}{\setlength{\XLingPaperlistitemindent}{\XLingPapersingledigitlistitemwidth + \parindent{}}
{\setlength{\XLingPapertempdim}{\XLingPapersingleletterlistitemwidth+\XLingPaperlistitemindent}\leftskip\XLingPapertempdim\relax
\interlinepenalty10000
\XLingPaperlistitem{\XLingPaperlistitemindent}{\XLingPapersingleletterlistitemwidth}{a.}{You can substitute this module for the {\LangtToolFontFamily{\textbf{\textcolor[rgb]{0,0,0.5019607843137255}{Insert Target Text}}}} module that is normally the last module in the {\LangtCollectionFontFamily{{\fontspec[Scale=0.8]{Arial}\textup{\textbf{\textcolor[rgb]{0.4,0,0.4}{Drafting}}}}}} collection.}}
{\setlength{\XLingPapertempdim}{\XLingPapersingleletterlistitemwidth+\XLingPaperlistitemindent}\leftskip\XLingPapertempdim\relax
\interlinepenalty10000
\XLingPaperlistitem{\XLingPaperlistitemindent}{\XLingPapersingleletterlistitemwidth}{b.}{By doing this, after a text gets translated, it will immediately get put into the target {\LangtToolFontFamily{\textbf{\textcolor[rgb]{0,0,0.5019607843137255}{Paratext}}}} project for further editing.}}
{\setlength{\XLingPapertempdim}{\XLingPapersingleletterlistitemwidth+\XLingPaperlistitemindent}\leftskip\XLingPapertempdim\relax
\interlinepenalty10000
\XLingPaperlistitem{\XLingPaperlistitemindent}{\XLingPapersingleletterlistitemwidth}{c.}{This is the best practice once your translation system is working smoothly.}}}}
{\setlength{\XLingPapertempdim}{\XLingPapersingledigitlistitemwidth+\parindent{}}\leftskip\XLingPapertempdim\relax
\interlinepenalty10000
\XLingPaperlistitem{\parindent{}}{\XLingPapersingledigitlistitemwidth}{2.}{{\LangtToolFontFamily{\textbf{\textcolor[rgb]{0,0,0.5019607843137255}{Export Text from Target FLEx to Paratext}}}} - takes a text that is in {\LangtToolFontFamily{\textbf{\textcolor[rgb]{0,0,0.5019607843137255}{FLEx}}}} and exports it to {\LangtToolFontFamily{\textbf{\textcolor[rgb]{0,0,0.5019607843137255}{Paratext}}}}.}{\setlength{\XLingPaperlistitemindent}{\XLingPapersingledigitlistitemwidth + \parindent{}}
{\setlength{\XLingPapertempdim}{\XLingPapersingleletterlistitemwidth+\XLingPaperlistitemindent}\leftskip\XLingPapertempdim\relax
\interlinepenalty10000
\XLingPaperlistitem{\XLingPaperlistitemindent}{\XLingPapersingleletterlistitemwidth}{a.}{Use this module if you have made changes or normally make changes to the target text in {\LangtToolFontFamily{\textbf{\textcolor[rgb]{0,0,0.5019607843137255}{FLEx}}}} and you want those changes to end up in {\LangtToolFontFamily{\textbf{\textcolor[rgb]{0,0,0.5019607843137255}{Paratext}}}}.}}}}
\vspace{\baselineskip}
}
\vspace{12pt}
\XLingPaperneedspace{5\baselineskip}

\penalty-3000{\noindent{\raisebox{\baselineskip}[0pt]{\protect\hypertarget{sPtxExportFromSynth}{}}\SectionLevelFourFontFamily{\normalsize{\raisebox{\baselineskip}[0pt]{\pdfbookmark[4]{4.4.2.1 Export FLExTrans Draft to Paratext}{sPtxExportFromSynth}}\textit{4.4.2.1 {\LangtExportDraftBPtxToolFontFamily{\textbf{\textcolor[rgb]{0,0,0.5019607843137255}{Export FLExTrans Draft to Paratext}}}}}}}}
\markright{{\LangtExportDraftBPtxToolFontFamily{\textbf{\textcolor[rgb]{0,0,0.5019607843137255}{Export FLExTrans Draft to Paratext}}}}}
\XLingPaperaddtocontents{sPtxExportFromSynth}}\par{}\penalty10000
\vspace{12pt}
\indent To export chapters to {\LangtToolFontFamily{\textbf{\textcolor[rgb]{0,0,0.5019607843137255}{Paratext}}}} from the newly synthesized text, use this tool. It is designed to work well with a text that you have previously imported with the {\LangtToolFontFamily{\textbf{\textcolor[rgb]{0,0,0.5019607843137255}{Import Text From Paratext}}}} module and translated. The export tool expects the name of the active source text to be in the form John 03-04 or JHN 03-04. Use the {\LangtToolFontFamily{\textbf{\textcolor[rgb]{0,0,0.5019607843137255}{FLExTrans Settings}}}} if you need to change the active source text. {\LangtExportDraftBPtxToolFontFamily{\textbf{\textcolor[rgb]{0,0,0.5019607843137255}{Export FLExTrans Draft to Paratext}}}} will find the book and chapters in the given {\LangtToolFontFamily{\textbf{\textcolor[rgb]{0,0,0.5019607843137255}{Paratext}}}} project and replace the text that was there with the text created by {\LangtToolFontFamily{\textbf{\textcolor[rgb]{0,0,0.5019607843137255}{FLExTrans}}}}.\protect\footnote[4]{{\leftskip0pt\parindent1em\raisebox{\baselineskip}[0pt]{\protect\hypertarget{nSynFile}{}}The text will be taken from the {\textbf{\textcolor[rgb]{0.5882352941176471,0.29411764705882354,0}{Target Output Synthesis File}}} as defined in the settings. This is the file created by the {\LangtToolFontFamily{\textbf{\textcolor[rgb]{0,0,0.5019607843137255}{Synthesize Text}}}} module.}} Note: the {\LangtToolFontFamily{\textbf{\textcolor[rgb]{0,0,0.5019607843137255}{Paratext}}}} project must be a project that you are at least a {\textit{Translator}} on. Also, the {\LangtToolFontFamily{\textbf{\textcolor[rgb]{0,0,0.5019607843137255}{Paratext}}}} book must already exist.\par{}\indent Normally this module should be run after the {\LangtToolFontFamily{\textbf{\textcolor[rgb]{0,0,0.5019607843137255}{Synthesize Text}}}} module.\par{}{\vspace{12pt plus 2pt minus 1pt}\raggedright{}\XLingPaperexample{.125in}{0pt}{2.75em}{\raisebox{\baselineskip}[0pt]{\protect\hypertarget{xParatextExport}{}}(24)}{\parbox[t]{\textwidth - .125in - 0pt}{\vspace*{-\baselineskip}{\XeTeXpicfile "../Images/ParatextExport.png" scaled 750}}}
}
\vspace{12pt}
\XLingPaperneedspace{5\baselineskip}

\penalty-3000{\noindent{\raisebox{\baselineskip}[0pt]{\protect\hypertarget{sPtxExportFromFLEx}{}}\SectionLevelFourFontFamily{\normalsize{\raisebox{\baselineskip}[0pt]{\pdfbookmark[4]{4.4.2.2 Export Text from Target FLEx to Paratext}{sPtxExportFromFLEx}}\textit{4.4.2.2 {\LangtExportFLExBPtxToolFontFamily{\textbf{\textcolor[rgb]{0,0,0.5019607843137255}{Export Text from Target FLEx to Paratext}}}}}}}}
\markright{{\LangtExportFLExBPtxToolFontFamily{\textbf{\textcolor[rgb]{0,0,0.5019607843137255}{Export Text from Target FLEx to Paratext}}}}}
\XLingPaperaddtocontents{sPtxExportFromFLEx}}\par{}\penalty10000
\vspace{12pt}
\indent To export chapters to {\LangtToolFontFamily{\textbf{\textcolor[rgb]{0,0,0.5019607843137255}{Paratext}}}} from a {\LangtToolFontFamily{\textbf{\textcolor[rgb]{0,0,0.5019607843137255}{FLEx}}}} project, use this tool. It is designed to work well with a text that you have previously inserted into the target {\LangtToolFontFamily{\textbf{\textcolor[rgb]{0,0,0.5019607843137255}{FLEx}}}} project with the {\LangtToolFontFamily{\textbf{\textcolor[rgb]{0,0,0.5019607843137255}{Insert Target Text}}}} tool. {\LangtExportFLExBPtxToolFontFamily{\textbf{\textcolor[rgb]{0,0,0.5019607843137255}{Export Text from Target FLEx to Paratext}}}} will find all scripture texts in {\LangtToolFontFamily{\textbf{\textcolor[rgb]{0,0,0.5019607843137255}{FLEx}}}} and will show you the filtered list of texts. It looks for texts with names in the format John 03-04 or JHN 01. You can select multiple texts to be exported. Also, if you want to export all chapters for a book, first check the box that says: {\textit{Clicking any chapter of a book selects all chapters of the book}}. Then when you click one of the book's chapters in the Scripture Texts list, all chapters will get selected.\par{}\indent Note: the {\LangtToolFontFamily{\textbf{\textcolor[rgb]{0,0,0.5019607843137255}{Paratext}}}} project must be a project that you are at least a {\textit{Translator}} on. Also, the {\LangtToolFontFamily{\textbf{\textcolor[rgb]{0,0,0.5019607843137255}{Paratext}}}} book must already exist.\par{}{\vspace{12pt plus 2pt minus 1pt}\raggedright{}\XLingPaperexample{.125in}{0pt}{2.75em}{\raisebox{\baselineskip}[0pt]{\protect\hypertarget{xParatextExportFromFLEx}{}}(25)}{\parbox[t]{\textwidth - .125in - 0pt}{\vspace*{-\baselineskip}{\XeTeXpicfile "../Images/ParatextExportFromFLEx.png" scaled 750}}}
\vspace{12pt plus 2pt minus 1pt}}\par\indent This module supports cluster projects. See \hyperlink{xClusterExportFromFlex}{(36)}.\par{}
\vspace{12pt}
\XLingPaperneedspace{5\baselineskip}

\penalty-3000{\noindent{\raisebox{\baselineskip}[0pt]{\protect\hypertarget{sSetupGramCat}{}}\SectionLevelTwoFontFamily{\normalsize{\raisebox{\baselineskip}[0pt]{\pdfbookmark[2]{4.5 The Set Up Transfer Rule Categories and Attributes Tool}{sSetupGramCat}}\textbf{4.5 The {\LangtToolFontFamily{\textbf{\textcolor[rgb]{0,0,0.5019607843137255}{Set Up Transfer Rule Categories and Attributes Tool}}}}}}}}
\markright{The {\LangtToolFontFamily{\textbf{\textcolor[rgb]{0,0,0.5019607843137255}{Set Up Transfer Rule Categories and Attributes Tool}}}}}
\XLingPaperaddtocontents{sSetupGramCat}}\par{}\penalty10000
\vspace{12pt}
\indent The {\LangtToolFontFamily{\textbf{\textcolor[rgb]{0,0,0.5019607843137255}{Set Up Transfer Rule Categories and Attributes Tool}}}} will populate categories and attributes in your transfer rule file. Use this tool when you are first setting up your transfer rules. You are given a choice for populating attributes as shown in \hyperlink{xSetupCatAttrib}{(26)}. Source categories \& the {\LangtCourierFontFamily{{a\_gram\_cat}}} attribute are always processed.\par{}{\vspace{12pt plus 2pt minus 1pt}\raggedright{}\XLingPaperexample{.125in}{0pt}{2.75em}{\raisebox{\baselineskip}[0pt]{\protect\hypertarget{xSetupCatAttrib}{}}(26)}{\parbox[t]{\textwidth - .125in - 0pt}{\vspace*{-\baselineskip}{\XeTeXpicfile "../Images/SetupCatAttrib.png" scaled 750}}}
}
\vspace{12pt}
\XLingPaperneedspace{5\baselineskip}

\penalty-3000{\noindent{\raisebox{\baselineskip}[0pt]{\protect\hypertarget{sSrcCats}{}}\SectionLevelThreeFontFamily{\normalsize{\raisebox{\baselineskip}[0pt]{\pdfbookmark[3]{4.5.1 Source Categories}{sSrcCats}}\textit{4.5.1 Source Categories}}}}
\markright{Source Categories}
\XLingPaperaddtocontents{sSrcCats}}\par{}\penalty10000
\vspace{12pt}
\indent This tool finds all the grammatical categories in the source {\LangtToolFontFamily{\textbf{\textcolor[rgb]{0,0,0.5019607843137255}{FLEx}}}} project and creates corresponding categories under the Categories element in the transfer rule file. Source categories will always be created with this tool, unless an existing category of the same name is found.\par{}
\vspace{12pt}
\XLingPaperneedspace{5\baselineskip}

\penalty-3000{\noindent{\raisebox{\baselineskip}[0pt]{\protect\hypertarget{sGramCat}{}}\SectionLevelThreeFontFamily{\normalsize{\raisebox{\baselineskip}[0pt]{\pdfbookmark[3]{4.5.2 a\_gram\_cat Attribute}{sGramCat}}\textit{4.5.2 {\LangtCourierFontFamily{{a\_gram\_cat}}} Attribute}}}}
\markright{{\LangtCourierFontFamily{{a\_gram\_cat}}} Attribute}
\XLingPaperaddtocontents{sGramCat}}\par{}\penalty10000
\vspace{12pt}
\indent This tool that will put the necessary attribute values for grammatical categories into your transfer rules file. Every transfer rule will likely need to reference the grammatical category of a word either in the source or target language. The initial transfer rules file that comes with {\LangtToolFontFamily{\textbf{\textcolor[rgb]{0,0,0.5019607843137255}{FLExTrans}}}} has just a few sample grammatical categories. This tool will take the categories that are in the source and target {\LangtToolFontFamily{\textbf{\textcolor[rgb]{0,0,0.5019607843137255}{FLEx}}}} projects and insert them as tags under the specific attribute called {\LangtCourierFontFamily{{a\_gram\_cat}}} in the transfer rules file. The list of grammatical categories will be a complete list of all the unique categories in both projects.\par{}\indent \hyperlink{xSetupGramCat}{(27)} shows what the {\LangtCourierFontFamily{{a\_gram\_cat}}} attribute might look like after running this tool.\par{}{\vspace{12pt plus 2pt minus 1pt}\raggedright{}\XLingPaperexample{.125in}{0pt}{2.75em}{\raisebox{\baselineskip}[0pt]{\protect\hypertarget{xSetupGramCat}{}}(27)}{\parbox[t]{\textwidth - .125in - 0pt}{\vspace*{-\baselineskip}{\XeTeXpicfile "../Images/SetupGramCat.PNG" scaled 750}}}
}
\vspace{12pt}
\XLingPaperneedspace{5\baselineskip}

\penalty-3000{\noindent{\raisebox{\baselineskip}[0pt]{\protect\hypertarget{sInflectionFeatures}{}}\SectionLevelThreeFontFamily{\normalsize{\raisebox{\baselineskip}[0pt]{\pdfbookmark[3]{4.5.3 Inflection Features}{sInflectionFeatures}}\textit{4.5.3 Inflection Features}}}}
\markright{Inflection Features}
\XLingPaperaddtocontents{sInflectionFeatures}}\par{}\penalty10000
\vspace{12pt}
\indent Click the appropriate check box as shown in \hyperlink{xSetupCatAttrib}{(26)} to populate attributes in your transfer rules that correspond to the inflection features in both of your {\LangtToolFontFamily{\textbf{\textcolor[rgb]{0,0,0.5019607843137255}{FLEx}}}} projects. Click the overwrite check box to change existing attributes in your transfer rule file with the inflection features from the {\LangtToolFontFamily{\textbf{\textcolor[rgb]{0,0,0.5019607843137255}{FLEx}}}} projects.\par{}
\vspace{12pt}
\XLingPaperneedspace{5\baselineskip}

\penalty-3000{\noindent{\raisebox{\baselineskip}[0pt]{\protect\hypertarget{sInflectionClasses}{}}\SectionLevelThreeFontFamily{\normalsize{\raisebox{\baselineskip}[0pt]{\pdfbookmark[3]{4.5.4 Inflection Classes}{sInflectionClasses}}\textit{4.5.4 Inflection Classes}}}}
\markright{Inflection Classes}
\XLingPaperaddtocontents{sInflectionClasses}}\par{}\penalty10000
\vspace{12pt}
\indent Click the appropriate check box as shown in \hyperlink{xSetupCatAttrib}{(26)} to populate attributes in your transfer rules that correspond to the inflection classes in both of your {\LangtToolFontFamily{\textbf{\textcolor[rgb]{0,0,0.5019607843137255}{FLEx}}}} projects. Click the overwrite check box to change existing attributes in your transfer rule file with the inflection classes from the {\LangtToolFontFamily{\textbf{\textcolor[rgb]{0,0,0.5019607843137255}{FLEx}}}} projects.\par{}
\vspace{12pt}
\XLingPaperneedspace{5\baselineskip}

\penalty-3000{\noindent{\raisebox{\baselineskip}[0pt]{\protect\hypertarget{sTemplateSlots}{}}\SectionLevelThreeFontFamily{\normalsize{\raisebox{\baselineskip}[0pt]{\pdfbookmark[3]{4.5.5 Template Slots}{sTemplateSlots}}\textit{4.5.5 Template Slots}}}}
\markright{Template Slots}
\XLingPaperaddtocontents{sTemplateSlots}}\par{}\penalty10000
\vspace{12pt}
\indent Click the appropriate check box as shown in \hyperlink{xSetupCatAttrib}{(26)} to populate attributes in your transfer rules that correspond to template slots in both of your {\LangtToolFontFamily{\textbf{\textcolor[rgb]{0,0,0.5019607843137255}{FLEx}}}} projects. Click the overwrite check box to change existing attributes in your transfer rule file with the template slots from the {\LangtToolFontFamily{\textbf{\textcolor[rgb]{0,0,0.5019607843137255}{FLEx}}}} projects.\par{}
\vspace{12pt}
\XLingPaperneedspace{5\baselineskip}

\penalty-3000{\noindent{\raisebox{\baselineskip}[0pt]{\protect\hypertarget{sSynthTest}{}}\SectionLevelTwoFontFamily{\normalsize{\raisebox{\baselineskip}[0pt]{\pdfbookmark[2]{4.6 The Synthesis Test Tool}{sSynthTest}}\textbf{4.6 The {\LangtToolFontFamily{\textbf{\textcolor[rgb]{0,0,0.5019607843137255}{Synthesis Test Tool}}}}}}}}
\markright{The {\LangtToolFontFamily{\textbf{\textcolor[rgb]{0,0,0.5019607843137255}{Synthesis Test Tool}}}}}
\XLingPaperaddtocontents{sSynthTest}}\par{}\penalty10000
\vspace{12pt}
\indent The {\LangtToolFontFamily{\textbf{\textcolor[rgb]{0,0,0.5019607843137255}{Generate All Parses Tool}}}} provides a way to test whether your Target Project is ready for the Synthesis stage of {\LangtToolFontFamily{\textbf{\textcolor[rgb]{0,0,0.5019607843137255}{FLExTrans}}}}. It allows you to choose a specific inflectional template from your {\LangtToolFontFamily{\textbf{\textcolor[rgb]{0,0,0.5019607843137255}{FLEx}}}} project and generate all possible forms that are allowed by that template. This allows you to review the forms to ensure they are all valid forms. If not, that is an indication of further work needed on the morphological description in the {\LangtToolFontFamily{\textbf{\textcolor[rgb]{0,0,0.5019607843137255}{FLEx}}}} project. It does not attempt to generate for any category without a template, and it does not attempt to generate with derivational affixes.\par{}\indent The output can be limited to a specific part of speech, a certain number of roots, and/or a specific Citation Form. It is a good idea to limit the output when starting out, because it can easily become quite large.\par{}\indent Note that for this test, any active template must have at least one affix in each of its slots. If you have any slots with no affixes in them, it is recommended to set that template to "inactive" for this test. (Untick the box labeled "Active" in the description of the template in Category Edit in {\LangtToolFontFamily{\textbf{\textcolor[rgb]{0,0,0.5019607843137255}{FLEx}}}}.)\par{}
\vspace{12pt}
\XLingPaperneedspace{5\baselineskip}

\penalty-3000{\noindent{\raisebox{\baselineskip}[0pt]{\protect\hypertarget{sRunSynthTest}{}}\SectionLevelThreeFontFamily{\normalsize{\raisebox{\baselineskip}[0pt]{\pdfbookmark[3]{4.6.1 Running the Test}{sRunSynthTest}}\textit{4.6.1 Running the Test}}}}
\markright{Running the Test}
\XLingPaperaddtocontents{sRunSynthTest}}\par{}\penalty10000
\vspace{12pt}
\indent To use this tool, change to the {\LangtCollectionFontFamily{{\fontspec[Scale=0.8]{Arial}\textup{\textbf{\textcolor[rgb]{0.4,0,0.4}{Synthesis Test}}}}}} collection. This will bring up the standard {\LangtToolFontFamily{\textbf{\textcolor[rgb]{0,0,0.5019607843137255}{FLExTrans}}}} modules that are needed for synthesis, plus the {\LangtToolFontFamily{\textbf{\textcolor[rgb]{0,0,0.5019607843137255}{Generate All Parses}}}} module, as shown in \hyperlink{xSynthTestSteps}{(28)}.\par{}{\vspace{12pt plus 2pt minus 1pt}\raggedright{}\XLingPaperexample{.125in}{0pt}{2.75em}{\raisebox{\baselineskip}[0pt]{\protect\hypertarget{xSynthTestSteps}{}}(28)}{\parbox[t]{\textwidth - .125in - 0pt}{\vspace*{-\baselineskip}{\XeTeXpicfile "../Images/SynthTestSteps.png" scaled 750}}}
\vspace{12pt plus 2pt minus 1pt}}\par\indent To run the Synthesis Test:\par{}{\parskip .5pt plus 1pt minus 1pt
                    
\vspace{\baselineskip}

{\setlength{\XLingPapertempdim}{\XLingPapersingledigitlistitemwidth+\parindent{}}\leftskip\XLingPapertempdim\relax
\interlinepenalty10000
\XLingPaperlistitem{\parindent{}}{\XLingPapersingledigitlistitemwidth}{1.}{First open the {\LangtToolFontFamily{\textbf{\textcolor[rgb]{0,0,0.5019607843137255}{FLExTrans Settings}}}} (as described in \hyperlink{sSettings}{3.2}).}}
{\setlength{\XLingPapertempdim}{\XLingPapersingledigitlistitemwidth+\parindent{}}\leftskip\XLingPapertempdim\relax
\interlinepenalty10000
\XLingPaperlistitem{\parindent{}}{\XLingPapersingledigitlistitemwidth}{2.}{In the top section, set the Target Project to the {\LangtToolFontFamily{\textbf{\textcolor[rgb]{0,0,0.5019607843137255}{FLEx}}}} project you want to use for the synthesis test. Click the \vspace*{0pt}{\XeTeXpicfile "../Images/ApplyAndCloseButton.png" scaled 750} button to register the change. Then reopen the {\LangtToolFontFamily{\textbf{\textcolor[rgb]{0,0,0.5019607843137255}{FLExTrans Settings}}}} to update the setting values that apply to the Target Project. (The Source Project has no effect for the Synthesis Test.)}}
{\setlength{\XLingPapertempdim}{\XLingPapersingledigitlistitemwidth+\parindent{}}\leftskip\XLingPapertempdim\relax
\interlinepenalty10000
\XLingPaperlistitem{\parindent{}}{\XLingPapersingledigitlistitemwidth}{3.}{Scroll down to the Synthesis Test section, as shown in \hyperlink{xSynthTestSettings}{(29)}.}\par{}{\vspace{12pt plus 2pt minus 1pt}\raggedright{}\XLingPaperexample{.125in}{0pt}{2.75em}{\raisebox{\baselineskip}[0pt]{\protect\hypertarget{xSynthTestSettings}{}}(29)}{\parbox[t]{\textwidth - .125in - 0pt}{\vspace*{-\baselineskip}{\XeTeXpicfile "../Images/SynthTestSettings.png" scaled 750}}}
\vspace{12pt plus 2pt minus 1pt}}{}}
{\setlength{\XLingPapertempdim}{\XLingPapersingledigitlistitemwidth+\parindent{}}\leftskip\XLingPapertempdim\relax
\interlinepenalty10000
\XLingPaperlistitem{\parindent{}}{\XLingPapersingledigitlistitemwidth}{4.}{Select the Part of Speech you want to test. The test currently requires at least one POS value to be selected.}}
{\setlength{\XLingPapertempdim}{\XLingPapersingledigitlistitemwidth+\parindent{}}\leftskip\XLingPapertempdim\relax
\interlinepenalty10000
\XLingPaperlistitem{\parindent{}}{\XLingPapersingledigitlistitemwidth}{5.}{You can further restrict how much is generated, either by limiting the number of stems to use for generation, or limiting it to a specific Citation Form. (When limiting the number of stems, the logic it uses for choosing stems is not obvious, and it uses the same criteria each time.) Click the \vspace*{0pt}{\XeTeXpicfile "../Images/ApplyAndCloseButton.png" scaled 750} button.}}
{\setlength{\XLingPapertempdim}{\XLingPapersingledigitlistitemwidth+\parindent{}}\leftskip\XLingPapertempdim\relax
\interlinepenalty10000
\XLingPaperlistitem{\parindent{}}{\XLingPapersingledigitlistitemwidth}{6.}{Back in the {\LangtToolFontFamily{\textbf{\textcolor[rgb]{0,0,0.5019607843137255}{FLExTrans}}}} window, click the {\textup{\textbf{Run All}}} button to run all the modules. It will choose which synthesis module to use based on the value for the {\textbf{\textcolor[rgb]{0.5882352941176471,0.29411764705882354,0}{Use HermitCrab synthesis}}} setting (see section \hyperlink{sHermitCrabSetup}{2.1.2}).}}
{\setlength{\XLingPapertempdim}{\XLingPapersingledigitlistitemwidth+\parindent{}}\leftskip\XLingPapertempdim\relax
\interlinepenalty10000
\XLingPaperlistitem{\parindent{}}{\XLingPapersingledigitlistitemwidth}{7.}{Output will go to two files, as specified in the settings. If you have not changed the defaults, the underlying parses will go to {\textit{Output\textbackslash{}words-uf.txt}} and the list of surface words will go to {\textit{Output\textbackslash{}target\_text-syn.txt}}. The {\textit{Output\textbackslash{}words-SIG.txt}} file will contain text in SIGMORPHON format.}}
\vspace{\baselineskip}
}\indent If you wish to try this module out on the Swedish sample project, use the settings to limit the POS to Noun (n), since that is the only category with a template in this project. When you run it, the output will overgenerate. That is, it will create forms that are not valid forms in Swedish. However, this is a good example of what the output will look like when you have started setting up your template, but have not yet added all the constraints (such as Inflection Features or Inflection Classes) that will fine tune the output.\par{}
\vspace{12pt}
\XLingPaperneedspace{5\baselineskip}

\penalty-3000{\noindent{\raisebox{\baselineskip}[0pt]{\protect\hypertarget{sDebugSynth}{}}\SectionLevelThreeFontFamily{\normalsize{\raisebox{\baselineskip}[0pt]{\pdfbookmark[3]{4.6.2 Debugging Synthesis}{sDebugSynth}}\textit{4.6.2 Debugging Synthesis}}}}
\markright{Debugging Synthesis}
\XLingPaperaddtocontents{sDebugSynth}}\par{}\penalty10000
\vspace{12pt}
\indent There are various reasons synthesis may fail or may generate incorrect parses. Below are a few debugging steps, but there are many more possible causes.\par{}{\parskip .5pt plus 1pt minus 1pt
                    
\vspace{\baselineskip}

{\setlength{\XLingPapertempdim}{\XLingPapersingledigitlistitemwidth+\parindent{}}\leftskip\XLingPapertempdim\relax
\interlinepenalty10000
\XLingPaperlistitem{\parindent{}}{\XLingPapersingledigitlistitemwidth}{1.}{Make sure glosses of affixes are unique.}}
{\setlength{\XLingPapertempdim}{\XLingPapersingledigitlistitemwidth+\parindent{}}\leftskip\XLingPapertempdim\relax
\interlinepenalty10000
\XLingPaperlistitem{\parindent{}}{\XLingPapersingledigitlistitemwidth}{2.}{If you have allomorphs, make sure they have appropriate environments and are ordered from most specific to least specific.}{\setlength{\XLingPaperlistitemindent}{\XLingPapersingledigitlistitemwidth + \parindent{}}
{\setlength{\XLingPapertempdim}{\XLingPapersingleletterlistitemwidth+\XLingPaperlistitemindent}\leftskip\XLingPapertempdim\relax
\interlinepenalty10000
\XLingPaperlistitem{\XLingPaperlistitemindent}{\XLingPapersingleletterlistitemwidth}{a.}{The {\LangtToolFontFamily{\textbf{\textcolor[rgb]{0,0,0.5019607843137255}{XAmple}}}} synthesis module tests allomorphs in the order they are listed, testing the {\textcolor[rgb]{0.8,0.6,0}{Lexeme Form}} last, as the "elsewhere allomorph". If an allomorph with an overly generous environment occurs before one that describes a subset of that environment, it will block application of the later allomorph.}}
{\setlength{\XLingPapertempdim}{\XLingPapersingleletterlistitemwidth+\XLingPaperlistitemindent}\leftskip\XLingPapertempdim\relax
\interlinepenalty10000
\XLingPaperlistitem{\XLingPaperlistitemindent}{\XLingPapersingleletterlistitemwidth}{b.}{If any allomorph has no environment, then once the synthesis module reaches it, that one will be accepted, and no further allomorphs will be tested. It becomes the de facto elsewhere allomorph, and blocks any later ones.}}}}
{\setlength{\XLingPapertempdim}{\XLingPapersingledigitlistitemwidth+\parindent{}}\leftskip\XLingPapertempdim\relax
\interlinepenalty10000
\XLingPaperlistitem{\parindent{}}{\XLingPapersingledigitlistitemwidth}{3.}{Even if two morphemes are allowed by the templates to co-occur, if they have conflicting inflection features, that combination will be disallowed. These may come out as {\LangtCourierFontFamily{{\%0\%word\%}}} in the output, indicating that a sequence was created based on the templates, but it was disallowed based on other constraints. It is not necessarily a problem if you see some of these lines in the output.}}
\vspace{\baselineskip}
}\indent Although having a thorough parser setup is not necessary to use a project for synthesis, having at least some of it working provides a really good start for synthesis. See \hyperlink{rBlack}{Black, (2014)} for more details about setting up parsing effectively for either parser.\par{}
\vspace{12pt}
\XLingPaperneedspace{5\baselineskip}

\penalty-3000{\noindent{\raisebox{\baselineskip}[0pt]{\protect\hypertarget{sRuleAssist}{}}\SectionLevelTwoFontFamily{\normalsize{\raisebox{\baselineskip}[0pt]{\pdfbookmark[2]{4.7 The Rule Assistant}{sRuleAssist}}\textbf{4.7 The {\LangtToolFontFamily{\textbf{\textcolor[rgb]{0,0,0.5019607843137255}{Rule Assistant}}}}}}}}
\markright{The {\LangtToolFontFamily{\textbf{\textcolor[rgb]{0,0,0.5019607843137255}{Rule Assistant}}}}}
\XLingPaperaddtocontents{sRuleAssist}}\par{}\penalty10000
\vspace{12pt}
\indent The {\LangtToolFontFamily{\textbf{\textcolor[rgb]{0,0,0.5019607843137255}{Rule Assistant}}}} helps you write Apertium transfer rules for common situations such as agreement in a phrase. It can help for the following situations:\par{}{\parskip .5pt plus 1pt minus 1pt
                    
\vspace{\baselineskip}

{\setlength{\XLingPapertempdim}{\XLingPapersingledigitlistitemwidth+\parindent{}}\leftskip\XLingPapertempdim\relax
\interlinepenalty10000
\XLingPaperlistitem{\parindent{}}{\XLingPapersingledigitlistitemwidth}{1.}{Agreement in the phrase such as gender and number agreement between adjective and noun.}}
{\setlength{\XLingPapertempdim}{\XLingPapersingledigitlistitemwidth+\parindent{}}\leftskip\XLingPapertempdim\relax
\interlinepenalty10000
\XLingPaperlistitem{\parindent{}}{\XLingPapersingledigitlistitemwidth}{2.}{Bantu gender agreement.}}
{\setlength{\XLingPapertempdim}{\XLingPapersingledigitlistitemwidth+\parindent{}}\leftskip\XLingPapertempdim\relax
\interlinepenalty10000
\XLingPaperlistitem{\parindent{}}{\XLingPapersingledigitlistitemwidth}{3.}{Reordering constituents in a phrase.}}
{\setlength{\XLingPapertempdim}{\XLingPapersingledigitlistitemwidth+\parindent{}}\leftskip\XLingPapertempdim\relax
\interlinepenalty10000
\XLingPaperlistitem{\parindent{}}{\XLingPapersingledigitlistitemwidth}{4.}{Turning a source word into a target affix.}}
{\setlength{\XLingPapertempdim}{\XLingPapersingledigitlistitemwidth+\parindent{}}\leftskip\XLingPapertempdim\relax
\interlinepenalty10000
\XLingPaperlistitem{\parindent{}}{\XLingPapersingledigitlistitemwidth}{5.}{Turning a source affix into a target word.}}
{\setlength{\XLingPapertempdim}{\XLingPapersingledigitlistitemwidth+\parindent{}}\leftskip\XLingPapertempdim\relax
\interlinepenalty10000
\XLingPaperlistitem{\parindent{}}{\XLingPapersingledigitlistitemwidth}{6.}{Rearranging affix order – including deleting affixes or adding affixes.}}
\vspace{\baselineskip}
}{\raggedright{}\XLingPaperexample{.125in}{0pt}{2.75em}{\raisebox{\baselineskip}[0pt]{\protect\hypertarget{xRAimage}{}}(30)}{\parbox[t]{\textwidth - .125in - 0pt}{\vspace*{-\baselineskip}{\XeTeXpicfile "../Images/RuleAssistant.png" scaled 500}}}
\vspace{12pt plus 2pt minus 1pt}}\par\indent For enforcing agreement between phrase constituents you must assign the appropriate feature in {\LangtToolFontFamily{\textbf{\textcolor[rgb]{0,0,0.5019607843137255}{FLEx}}}} to the affixes or stems involved.\par{}\indent For detailed help on using the {\LangtToolFontFamily{\textbf{\textcolor[rgb]{0,0,0.5019607843137255}{Rule Assistant}}}}, consult the help file the comes with the {\LangtToolFontFamily{\textbf{\textcolor[rgb]{0,0,0.5019607843137255}{Rule Assistant}}}}. You can get to this help file by clicking on the {\textup{\textbf{Help}}} button, then on {\textup{\textbf{User Documentation}}}.\par{}
\vspace{12pt}
\XLingPaperneedspace{5\baselineskip}

\penalty-3000{\noindent{\raisebox{\baselineskip}[0pt]{\protect\hypertarget{sTextIn}{}}\SectionLevelTwoFontFamily{\normalsize{\raisebox{\baselineskip}[0pt]{\pdfbookmark[2]{4.8 The Text In Rules Tool}{sTextIn}}\textbf{4.8 The {\LangtToolFontFamily{\textbf{\textcolor[rgb]{0,0,0.5019607843137255}{Text In Rules Tool}}}}}}}}
\markright{The {\LangtToolFontFamily{\textbf{\textcolor[rgb]{0,0,0.5019607843137255}{Text In Rules Tool}}}}}
\XLingPaperaddtocontents{sTextIn}}\par{}\penalty10000
\vspace{12pt}
\indent The {\LangtToolFontFamily{\textbf{\textcolor[rgb]{0,0,0.5019607843137255}{Text In Rules Tool}}}} helps you test a set of search and replace operations to be used to fix up the text that comes out of {\LangtToolFontFamily{\textbf{\textcolor[rgb]{0,0,0.5019607843137255}{Paratext}}}}. Regular expressions can be used if desired.\par{}\indent All the active rules defined in this tool are used in the {\LangtToolFontFamily{\textbf{\textcolor[rgb]{0,0,0.5019607843137255}{Import Text From Paratext}}}} module. The newly imported text in {\LangtToolFontFamily{\textbf{\textcolor[rgb]{0,0,0.5019607843137255}{FLEx}}}} will have the changes applied. This is very helpful for those situations where you may not have editing permission of the {\LangtToolFontFamily{\textbf{\textcolor[rgb]{0,0,0.5019607843137255}{Paratext}}}} text, but some cleanup of the text is needed.\par{}{\vspace{12pt plus 2pt minus 1pt}\raggedright{}\XLingPaperexample{.125in}{0pt}{2.75em}{\raisebox{\baselineskip}[0pt]{\protect\hypertarget{xTextInimage}{}}(31)}{\parbox[t]{\textwidth - .125in - 0pt}{\vspace*{-\baselineskip}{\XeTeXpicfile "../Images/TextInRules.png" scaled 500}}}
\vspace{12pt plus 2pt minus 1pt}}\par\indent The rules defined in this tool are run sequentially with the output of one rule being the input for the next one. Use the up and down arrow buttons to change the order of the rules. The two buttons next to the arrow buttons will select or unselect all the rules for testing.\par{}\indent The interface is pretty self explanatory. To edit a rule first click on it in the list, make your changes and then click the {\textup{\textbf{Update}}} button. You will see the edited form now in the list.\par{}\indent If the search and replace lines are Regular Expressions, check the {\textup{\textbf{Regular Expression}}} check box. See the {\LangtToolFontFamily{\textbf{\textcolor[rgb]{0,0,0.5019607843137255}{FLEx}}}} help on how to use Regular Expressions or various websites that explain the Regular Expression language.\par{}\indent The {\LangtToolFontFamily{\textbf{\textcolor[rgb]{0,0,0.5019607843137255}{Wildebeest cleanup tool}}}} is something that can fix or normalize many character problems. For example, it can repair characters that have encoding errors. Click the link to find out more about what can be normalized with it. You can use the default cleanup steps that come with the {\LangtToolFontFamily{\textbf{\textcolor[rgb]{0,0,0.5019607843137255}{Wildebeest cleanup tool}}}}, all of the steps or mix and match steps using the {\textup{\textbf{Add}}} and {\textup{\textbf{Skip}}} boxes. For example, put the step {\textit{\textbf{look-alike}}} into the {\textup{\textbf{Add}}} box to add this normalization step. For multiple {\textup{\textbf{Add}}} or {\textup{\textbf{Skip}}} steps, separate the steps with spaces. Some steps require an ISO 639-3 language code. Note: {\LangtToolFontFamily{\textbf{\textcolor[rgb]{0,0,0.5019607843137255}{Wildebeest cleanup tool}}}} steps are run {\textit{before}} the rules you define.\par{}\indent Use the {\textup{\textbf{Test}}} button to check that your rules are doing what you expect.\par{}
\vspace{12pt}
\XLingPaperneedspace{5\baselineskip}

\penalty-3000{\noindent{\raisebox{\baselineskip}[0pt]{\protect\hypertarget{sTextOut}{}}\SectionLevelTwoFontFamily{\normalsize{\raisebox{\baselineskip}[0pt]{\pdfbookmark[2]{4.9 The Text Out Rules Tool}{sTextOut}}\textbf{4.9 The {\LangtToolFontFamily{\textbf{\textcolor[rgb]{0,0,0.5019607843137255}{Text Out Rules Tool}}}}}}}}
\markright{The {\LangtToolFontFamily{\textbf{\textcolor[rgb]{0,0,0.5019607843137255}{Text Out Rules Tool}}}}}
\XLingPaperaddtocontents{sTextOut}}\par{}\penalty10000
\vspace{12pt}
\indent The {\LangtToolFontFamily{\textbf{\textcolor[rgb]{0,0,0.5019607843137255}{Text Out Rules Tool}}}} helps you test a set of search and replace operations to be used to fix up the text that comes out of the {\LangtModuleFontFamily{\textbf{\textcolor[rgb]{0.4,0,0.4}{Synthesize Text}}}} module. It basically is identical to the {\LangtToolFontFamily{\textbf{\textcolor[rgb]{0,0,0.5019607843137255}{Text In Rules Tool}}}}, except it is used in a different part of the {\LangtToolFontFamily{\textbf{\textcolor[rgb]{0,0,0.5019607843137255}{FLExTrans}}}} process, namely the end of the process. The rules created here are distinct from the other rules. Use this tool to make any changes to the translated text before you bring it into another tool like {\LangtToolFontFamily{\textbf{\textcolor[rgb]{0,0,0.5019607843137255}{FLEx}}}} or {\LangtToolFontFamily{\textbf{\textcolor[rgb]{0,0,0.5019607843137255}{Paratext}}}}. This tool does not have the {\LangtToolFontFamily{\textbf{\textcolor[rgb]{0,0,0.5019607843137255}{Wildebeest cleanup tool}}}}, since it is assumed the text has been normalized previously.\par{}\indent There is a separate module that must be added to the {\LangtCollectionFontFamily{{\fontspec[Scale=0.8]{Arial}\textup{\textbf{\textcolor[rgb]{0.4,0,0.4}{Drafting}}}}}} collection of modules (typically after {\LangtModuleFontFamily{\textbf{\textcolor[rgb]{0.4,0,0.4}{Synthesize Text}}}}) to make use of the rules defined in the {\LangtToolFontFamily{\textbf{\textcolor[rgb]{0,0,0.5019607843137255}{Text Out Rules Tool}}}}. This module is called {\LangtModuleFontFamily{\textbf{\textcolor[rgb]{0.4,0,0.4}{Fix Up Synthesis Text}}}}. This module will run all the rules you have defined. See Section \hyperlink{sCollections}{3.3} for help on adding a module to a collection.\par{}{\vspace{12pt plus 2pt minus 1pt}\raggedright{}\XLingPaperexample{.125in}{0pt}{2.75em}{\raisebox{\baselineskip}[0pt]{\protect\hypertarget{xTextOutimage}{}}(32)}{\parbox[t]{\textwidth - .125in - 0pt}{\vspace*{-\baselineskip}{\XeTeXpicfile "../Images/TextOutRules.png" scaled 500}}}
}
\vspace{12pt}
\XLingPaperneedspace{5\baselineskip}

\penalty-3000{\noindent{\raisebox{\baselineskip}[0pt]{\protect\hypertarget{sReplEditTool}{}}\SectionLevelTwoFontFamily{\normalsize{\raisebox{\baselineskip}[0pt]{\pdfbookmark[2]{4.10 The Replacement Dictionary Editor}{sReplEditTool}}\textbf{4.10 The {\LangtToolFontFamily{\textbf{\textcolor[rgb]{0,0,0.5019607843137255}{Replacement Dictionary Editor}}}}}}}}
\markright{The {\LangtToolFontFamily{\textbf{\textcolor[rgb]{0,0,0.5019607843137255}{Replacement Dictionary Editor}}}}}
\XLingPaperaddtocontents{sReplEditTool}}\par{}\penalty10000
\vspace{12pt}
\indent The {\LangtToolFontFamily{\textbf{\textcolor[rgb]{0,0,0.5019607843137255}{Replacement Dictionary Editor}}}} allows you to link source senses in certain inflections to target senses that are different than the normal link for the sense. Likewise you can link certain uninflected source senses to inflected target senses that are different than the normal link. For example, in the last line of \hyperlink{xReplEdit}{(33)} you can see that quëda1.1 with the affixes CAUS and REF is linked to chari1.1 with affixes REF and UP. Whereas normally, quëda1.1 (uninflected or in other inflections) links to kïda1.1. In line 1 you can see an OBJ affix is added to the target word rimëru1.1 is linked to. In line 6 you can see an affix is removed from the target word that sinö1.1 is linked to. (These last two examples could be done with a very specific transfer rule, but making the change at the lexical transfer level makes good sense.)\par{}\indent The rows that are listed in this tool augment the bilingual lexicon. If the source information in a row is exactly the same as the default bilingual lexicon entry, this replacement dictionary entry will override the default entry.\par{}\indent The buttons for the tool are pretty self-explanatory. When editing or adding information, start typing in the Lemma cell and a list will appear as you type. Tab or click elsewhere to have the tool automatically fill in the grammatical category and/or features and classes.\par{}\indent The information displayed in this tool is stored in a replacement dictionary file typically called {\textit{replace.dix}}.\par{}{\vspace{12pt plus 2pt minus 1pt}\raggedright{}\XLingPaperexample{.125in}{0pt}{2.75em}{\raisebox{\baselineskip}[0pt]{\protect\hypertarget{xReplEdit}{}}(33)}{\parbox[t]{\textwidth - .125in - 0pt}{\vspace*{-\baselineskip}{\XeTeXpicfile "../Images/ReplacementEditor.png" scaled 500}}}
}
\vspace{12pt}
\XLingPaperneedspace{5\baselineskip}

\penalty-3000{{\centering\raisebox{\baselineskip}[0pt]{\protect\hypertarget{sCluster}{}}\SectionLevelOneFontFamily{\large{\raisebox{\baselineskip}[0pt]{\pdfbookmark[1]{5 Cluster Project Support}{sCluster}}\textbf{5 Cluster Project Support}}}\\{}}\markright{Cluster Project Support}
\XLingPaperaddtocontents{sCluster}}\par{}\penalty10000
\vspace{12pt}
\indent {\LangtToolFontFamily{\textbf{\textcolor[rgb]{0,0,0.5019607843137255}{FLExTrans}}}} works great in cluster projects. You can leverage the approval of an interlinear text in the source language to produce translations of the text in multiple target languages.\par{}\indent {\LangtToolFontFamily{\textbf{\textcolor[rgb]{0,0,0.5019607843137255}{FLExTrans}}}} supports cluster projects by allowing you to define which {\LangtToolFontFamily{\textbf{\textcolor[rgb]{0,0,0.5019607843137255}{FLEx}}}} projects - both source and targets - are in your cluster. You do this in the {\LangtToolFontFamily{\textbf{\textcolor[rgb]{0,0,0.5019607843137255}{FLExTrans Settings}}}}. See \hyperlink{xClusterSettings}{(34)}. This information is used in various tools to allow you to apply the tool to multiple {\LangtToolFontFamily{\textbf{\textcolor[rgb]{0,0,0.5019607843137255}{FLEx}}}} projects at once.\par{}{\vspace{12pt plus 2pt minus 1pt}\raggedright{}\XLingPaperexample{.125in}{0pt}{2.75em}{\raisebox{\baselineskip}[0pt]{\protect\hypertarget{xClusterSettings}{}}(34)}{\parbox[t]{\textwidth - .125in - 0pt}{\vspace*{-\baselineskip}{\XeTeXpicfile "../Images/ClusterSettings.png" scaled 750}}}
}
\vspace{12pt}
\XLingPaperneedspace{5\baselineskip}

\penalty-3000{\noindent{\raisebox{\baselineskip}[0pt]{\protect\hypertarget{sClustExisting}{}}\SectionLevelTwoFontFamily{\normalsize{\raisebox{\baselineskip}[0pt]{\pdfbookmark[2]{5.1 Cluster Support in Existing Tools}{sClustExisting}}\textbf{5.1 Cluster Support in Existing Tools}}}}
\markright{Cluster Support in Existing Tools}
\XLingPaperaddtocontents{sClustExisting}}\par{}\penalty10000
\vspace{12pt}
\indent See tool images below that show support for cluster projects. \hyperlink{xClusterAddNewEntry}{(35)} shows the {\textbf{Add New Entry}} window that is launched from the {\LangtToolFontFamily{\textbf{\textcolor[rgb]{0,0,0.5019607843137255}{Sense Linker Tool}}}}.\par{}{\vspace{12pt plus 2pt minus 1pt}\raggedright{}\XLingPaperexample{.125in}{0pt}{2.75em}{\raisebox{\baselineskip}[0pt]{\protect\hypertarget{xClusterAddNewEntry}{}}(35)}{\parbox[t]{\textwidth - .125in - 0pt}{\vspace*{-\baselineskip}{\XeTeXpicfile "../Images/ClusterAddNewEntry.png" scaled 750}}}
}{\vspace{12pt plus 2pt minus 1pt}\raggedright{}\XLingPaperexample{.125in}{0pt}{2.75em}{\raisebox{\baselineskip}[0pt]{\protect\hypertarget{xClusterExportFromFlex}{}}(36)}{\parbox[t]{\textwidth - .125in - 0pt}{\vspace*{-\baselineskip}{\XeTeXpicfile "../Images/ClusterExportFromFLEx.png" scaled 750}}}
}{\vspace{12pt plus 2pt minus 1pt}\raggedright{}\XLingPaperexample{.125in}{0pt}{2.75em}{\raisebox{\baselineskip}[0pt]{\protect\hypertarget{xClusterImportParatext}{}}(37)}{\parbox[t]{\textwidth - .125in - 0pt}{\vspace*{-\baselineskip}{\XeTeXpicfile "../Images/ClusterImportParatext.png" scaled 750}}}
}
\vspace{12pt}
\XLingPaperneedspace{5\baselineskip}

\penalty-3000{\noindent{\raisebox{\baselineskip}[0pt]{\protect\hypertarget{sClustNew}{}}\SectionLevelTwoFontFamily{\normalsize{\raisebox{\baselineskip}[0pt]{\pdfbookmark[2]{5.2 Cluster Support in New Tools}{sClustNew}}\textbf{5.2 Cluster Support in New Tools}}}}
\markright{Cluster Support in New Tools}
\XLingPaperaddtocontents{sClustNew}}\par{}\penalty10000
\vspace{12pt}

\vspace{12pt}
\XLingPaperneedspace{5\baselineskip}

\penalty-3000{\noindent{\raisebox{\baselineskip}[0pt]{\protect\hypertarget{sOpenMultProjs}{}}\SectionLevelThreeFontFamily{\normalsize{\raisebox{\baselineskip}[0pt]{\pdfbookmark[3]{5.2.1 Open Multiple FLEx Projects Tool}{sOpenMultProjs}}\textit{5.2.1 Open Multiple {\LangtToolFontFamily{\textbf{\textcolor[rgb]{0,0,0.5019607843137255}{FLEx}}}} Projects Tool}}}}
\markright{Open Multiple {\LangtToolFontFamily{\textbf{\textcolor[rgb]{0,0,0.5019607843137255}{FLEx}}}} Projects Tool}
\XLingPaperaddtocontents{sOpenMultProjs}}\par{}\penalty10000
\vspace{12pt}
\indent This tool allows you to open {\LangtToolFontFamily{\textbf{\textcolor[rgb]{0,0,0.5019607843137255}{FLEx}}}} projects you choose.\par{}{\vspace{12pt plus 2pt minus 1pt}\raggedright{}\XLingPaperexample{.125in}{0pt}{2.75em}{\raisebox{\baselineskip}[0pt]{\protect\hypertarget{xOpenMultProjs}{}}(38)}{\parbox[t]{\textwidth - .125in - 0pt}{\vspace*{-\baselineskip}{\XeTeXpicfile "../Images/OpenMultipleProjects.png" scaled 750}}}
}
\vspace{12pt}
\XLingPaperneedspace{5\baselineskip}

\penalty-3000{\noindent{\raisebox{\baselineskip}[0pt]{\protect\hypertarget{sRestoreProjs}{}}\SectionLevelThreeFontFamily{\normalsize{\raisebox{\baselineskip}[0pt]{\pdfbookmark[3]{5.2.2 Restore Multiple FLEx Projects Tool}{sRestoreProjs}}\textit{5.2.2 Restore Multiple {\LangtToolFontFamily{\textbf{\textcolor[rgb]{0,0,0.5019607843137255}{FLEx}}}} Projects Tool}}}}
\markright{Restore Multiple {\LangtToolFontFamily{\textbf{\textcolor[rgb]{0,0,0.5019607843137255}{FLEx}}}} Projects Tool}
\XLingPaperaddtocontents{sRestoreProjs}}\par{}\penalty10000
\vspace{12pt}
\indent This tool allows you to restore {\LangtToolFontFamily{\textbf{\textcolor[rgb]{0,0,0.5019607843137255}{FLEx}}}} projects you choose.\par{}{\vspace{12pt plus 2pt minus 1pt}\raggedright{}\XLingPaperexample{.125in}{0pt}{2.75em}{\raisebox{\baselineskip}[0pt]{\protect\hypertarget{xRestoreProjs}{}}(39)}{\parbox[t]{\textwidth - .125in - 0pt}{\vspace*{-\baselineskip}{\XeTeXpicfile "../Images/RestoreFLExProjects.png" scaled 750}}}
}
\vspace{12pt}
\XLingPaperneedspace{5\baselineskip}

\penalty-3000{\noindent{\raisebox{\baselineskip}[0pt]{\protect\hypertarget{sFixClusterProjs}{}}\SectionLevelThreeFontFamily{\normalsize{\raisebox{\baselineskip}[0pt]{\pdfbookmark[3]{5.2.3 Fix FLEx Projects Tool}{sFixClusterProjs}}\textit{5.2.3 Fix {\LangtToolFontFamily{\textbf{\textcolor[rgb]{0,0,0.5019607843137255}{FLEx}}}} Projects Tool}}}}
\markright{Fix {\LangtToolFontFamily{\textbf{\textcolor[rgb]{0,0,0.5019607843137255}{FLEx}}}} Projects Tool}
\XLingPaperaddtocontents{sFixClusterProjs}}\par{}\penalty10000
\vspace{12pt}
\indent This tool allows you to run the Find and Fix utility on the projects you choose. This is the same utility that is available in {\LangtToolFontFamily{\textbf{\textcolor[rgb]{0,0,0.5019607843137255}{FLEx}}}}. You cannot run this utility on a project that is currently open in {\LangtToolFontFamily{\textbf{\textcolor[rgb]{0,0,0.5019607843137255}{FLEx}}}} or on the current source project even if it is not open. Fixed errors are logged to the report pane.\par{}{\vspace{12pt plus 2pt minus 1pt}\raggedright{}\XLingPaperexample{.125in}{0pt}{2.75em}{\raisebox{\baselineskip}[0pt]{\protect\hypertarget{xFixClusterProjs}{}}(40)}{\parbox[t]{\textwidth - .125in - 0pt}{\vspace*{-\baselineskip}{\XeTeXpicfile "../Images/FixClusterProjects.png" scaled 750}}}
}
\vspace{12pt}
\XLingPaperneedspace{5\baselineskip}

\penalty-3000{\noindent{\raisebox{\baselineskip}[0pt]{\protect\hypertarget{sAddClusterAdHoc}{}}\SectionLevelThreeFontFamily{\normalsize{\raisebox{\baselineskip}[0pt]{\pdfbookmark[3]{5.2.4 Add Ad Hoc Constraint for a Cluster Tool}{sAddClusterAdHoc}}\textit{5.2.4 Add Ad Hoc Constraint for a Cluster Tool}}}}
\markright{Add Ad Hoc Constraint for a Cluster Tool}
\XLingPaperaddtocontents{sAddClusterAdHoc}}\par{}\penalty10000
\vspace{12pt}
\indent This tool allows you to add an ad hoc constraint to multiple cluster projects.\par{}{\vspace{12pt plus 2pt minus 1pt}\raggedright{}\XLingPaperexample{.125in}{0pt}{2.75em}{\raisebox{\baselineskip}[0pt]{\protect\hypertarget{xClusterAdHoc}{}}(41)}{\parbox[t]{\textwidth - .125in - 0pt}{\vspace*{-\baselineskip}{\XeTeXpicfile "../Images/ClusterAdHoc.png" scaled 750}}}
}
\vspace{12pt}
\XLingPaperneedspace{5\baselineskip}

\penalty-3000{{\centering\raisebox{\baselineskip}[0pt]{\protect\hypertarget{sTransferTutorial}{}}\SectionLevelOneFontFamily{\large{\raisebox{\baselineskip}[0pt]{\pdfbookmark[1]{6 A Tutorial on Writing Transfer Rules}{sTransferTutorial}}\textbf{6 A Tutorial on Writing Transfer Rules\protect\footnote[5]{{\leftskip0pt\parindent1em\raisebox{\baselineskip}[0pt]{\protect\hypertarget{nReproducing}{}}For this tutorial I am basically reproducing the {\LangtToolFontFamily{\textbf{\textcolor[rgb]{0,0,0.5019607843137255}{Apertium}}}} article \href{http://wiki.apertium.org/wiki/A\_long\_introduction\_to\_transfer\_rules}{A long introduction to transfer rules}. I am mainly modifying it to show the {\LangtToolFontFamily{\textbf{\textcolor[rgb]{0,0,0.5019607843137255}{XMLmind XML Editor}}}} method of editing the transfer rules, but also the {\LangtToolFontFamily{\textbf{\textcolor[rgb]{0,0,0.5019607843137255}{FLExTrans}}}} process is assumed instead of the {\LangtToolFontFamily{\textbf{\textcolor[rgb]{0,0,0.5019607843137255}{Apertium}}}} process.}}}}}\\{}}\markright{A Tutorial on Writing Transfer Rules}
\XLingPaperaddtocontents{sTransferTutorial}}\par{}\penalty10000
\vspace{12pt}
\indent Writing transfer rules is not as tricky as it might seem. People generally understand the basic concepts, but they sometimes struggle with the formalism. Maybe one of the reasons people struggle is that the formalism mixes declarative and procedural statements. This tutorial should help you get used to writing transfer rules.\par{}
\vspace{12pt}
\XLingPaperneedspace{5\baselineskip}

\penalty-3000{\noindent{\raisebox{\baselineskip}[0pt]{\protect\hypertarget{sOverview}{}}\SectionLevelTwoFontFamily{\normalsize{\raisebox{\baselineskip}[0pt]{\pdfbookmark[2]{6.1 Overview}{sOverview}}\textbf{6.1 Overview}}}}
\markright{Overview}
\XLingPaperaddtocontents{sOverview}}\par{}\penalty10000
\vspace{12pt}
\indent Here’s a brief overview:\par{}
\vspace{12pt}
\XLingPaperneedspace{5\baselineskip}

\penalty-3000{\noindent{\raisebox{\baselineskip}[0pt]{\protect\hypertarget{sFormalities}{}}\SectionLevelThreeFontFamily{\normalsize{\raisebox{\baselineskip}[0pt]{\pdfbookmark[3]{6.1.1 Some Formalities}{sFormalities}}\textit{6.1.1 Some Formalities}}}}
\markright{Some Formalities}
\XLingPaperaddtocontents{sFormalities}}\par{}\penalty10000
\vspace{12pt}
\indent Before starting, it is important to get an idea of what we can’t do, before explaining what we can.\par{}{\parskip .5pt plus 1pt minus 1pt

\vspace{\baselineskip}

{\setlength{\XLingPapertempdim}{\XLingPaperbulletlistitemwidth+\parindent{}}\leftskip\XLingPapertempdim\relax
\interlinepenalty10000
\XLingPaperlistitem{\parindent{}}{\XLingPaperbulletlistitemwidth}{•}{There are no recursive rules. Rules match fixed-length patterns. There is no optionality at the level of words. There is no way of saying one or more, it’s just one.}}
{\setlength{\XLingPapertempdim}{\XLingPaperbulletlistitemwidth+\parindent{}}\leftskip\XLingPapertempdim\relax
\interlinepenalty10000
\XLingPaperlistitem{\parindent{}}{\XLingPaperbulletlistitemwidth}{•}{Rules contain both declarative parts and procedural parts. You can’t just expect to say what you want or how you want to do it. You need to do both -- but in different places (but it’s quite intuitive).}}
{\setlength{\XLingPapertempdim}{\XLingPaperbulletlistitemwidth+\parindent{}}\leftskip\XLingPapertempdim\relax
\interlinepenalty10000
\XLingPaperlistitem{\parindent{}}{\XLingPaperbulletlistitemwidth}{•}{Patterns match only on the source side, not on the target side.}}
{\setlength{\XLingPapertempdim}{\XLingPaperbulletlistitemwidth+\parindent{}}\leftskip\XLingPapertempdim\relax
\interlinepenalty10000
\XLingPaperlistitem{\parindent{}}{\XLingPaperbulletlistitemwidth}{•}{The transfer process has no access to the information in the target language dictionary. This means that if the transfer needs some information about the available forms of a particular word, e.g. if it is only singular, or only plural, then this information needs to go in the bilingual dictionary.}}
\vspace{\baselineskip}
}
\vspace{12pt}
\XLingPaperneedspace{5\baselineskip}

\penalty-3000{\noindent{\raisebox{\baselineskip}[0pt]{\protect\hypertarget{sApproach}{}}\SectionLevelThreeFontFamily{\normalsize{\raisebox{\baselineskip}[0pt]{\pdfbookmark[3]{6.1.2 Approaching the Process of Writing Transfer Rules}{sApproach}}\textit{6.1.2 Approaching the Process of Writing Transfer Rules}}}}
\markright{Approaching the Process of Writing Transfer Rules}
\XLingPaperaddtocontents{sApproach}}\par{}\penalty10000
\vspace{12pt}
\indent Think bottom up, not top down. Start with a detailed question like “How can I deal with the source language having possession indicated on nouns, but the target language having possession indicated on verbs?” not “How can I change SOV order to SVO order?”\par{}
\vspace{12pt}
\XLingPaperneedspace{5\baselineskip}

\penalty-3000{\noindent{\raisebox{\baselineskip}[0pt]{\protect\hypertarget{sLexicalStruct}{}}\SectionLevelThreeFontFamily{\normalsize{\raisebox{\baselineskip}[0pt]{\pdfbookmark[3]{6.1.3 Lexical Transfer and Structural Transfer}{sLexicalStruct}}\textit{6.1.3 Lexical Transfer and Structural Transfer}}}}
\markright{Lexical Transfer and Structural Transfer}
\XLingPaperaddtocontents{sLexicalStruct}}\par{}\penalty10000
\vspace{12pt}
\indent We don’t want to confuse the roles of lexical transfer and structural transfer. There is a grey area between the two, but there are also big parts that don’t overlap.\par{}{\parskip .5pt plus 1pt minus 1pt

\vspace{\baselineskip}

{\setlength{\XLingPapertempdim}{\XLingPaperbulletlistitemwidth+\parindent{}}\leftskip\XLingPapertempdim\relax
\interlinepenalty10000
\XLingPaperlistitem{\parindent{}}{\XLingPaperbulletlistitemwidth}{•}{{\textbf{Lexical transfer}} (the {\textit{bilingual.dix}} file -- produced by the {\LangtToolFontFamily{\textbf{\textcolor[rgb]{0,0,0.5019607843137255}{Sense Linker Tool}}}} and the {\LangtModuleFontFamily{\textbf{\textcolor[rgb]{0.4,0,0.4}{Extract Bilingual Lexicon}}}} module)}{\setlength{\XLingPaperlistitemindent}{\XLingPaperbulletlistitemwidth + \parindent{}}
{\setlength{\XLingPapertempdim}{\XLingPaperbulletlistitemwidth+\XLingPaperlistitemindent}\leftskip\XLingPapertempdim\relax
\interlinepenalty10000
\XLingPaperlistitem{\XLingPaperlistitemindent}{\XLingPaperbulletlistitemwidth}{•}{Nearly always gives translations between words, not tags.}}
{\setlength{\XLingPapertempdim}{\XLingPaperbulletlistitemwidth+\XLingPaperlistitemindent}\leftskip\XLingPapertempdim\relax
\interlinepenalty10000
\XLingPaperlistitem{\XLingPaperlistitemindent}{\XLingPaperbulletlistitemwidth}{•}{Can add or change tags, on a per-sense basis.}}
{\setlength{\XLingPapertempdim}{\XLingPaperbulletlistitemwidth+\XLingPaperlistitemindent}\leftskip\XLingPapertempdim\relax
\interlinepenalty10000
\XLingPaperlistitem{\XLingPaperlistitemindent}{\XLingPaperbulletlistitemwidth}{•}{Doesn’t do reordering.}}
{\setlength{\XLingPapertempdim}{\XLingPaperbulletlistitemwidth+\XLingPaperlistitemindent}\leftskip\XLingPapertempdim\relax
\interlinepenalty10000
\XLingPaperlistitem{\XLingPaperlistitemindent}{\XLingPaperbulletlistitemwidth}{•}{Can be used to give a head’s up to the structural transfer to draw attention to missing features, or features that cannot be decided on a no-context basis.}}}}
{\setlength{\XLingPapertempdim}{\XLingPaperbulletlistitemwidth+\parindent{}}\leftskip\XLingPapertempdim\relax
\interlinepenalty10000
\XLingPaperlistitem{\parindent{}}{\XLingPaperbulletlistitemwidth}{•}{{\textbf{Structural transfer}} (the {\textit{transfer\_rules.t1x}} file -- edited with {\LangtToolFontFamily{\textbf{\textcolor[rgb]{0,0,0.5019607843137255}{XMLmind XML Editor}}}})}{\setlength{\XLingPaperlistitemindent}{\XLingPaperbulletlistitemwidth + \parindent{}}
{\setlength{\XLingPapertempdim}{\XLingPaperbulletlistitemwidth+\XLingPaperlistitemindent}\leftskip\XLingPapertempdim\relax
\interlinepenalty10000
\XLingPaperlistitem{\XLingPaperlistitemindent}{\XLingPaperbulletlistitemwidth}{•}{Rarely gives translations between single words.}}
{\setlength{\XLingPapertempdim}{\XLingPaperbulletlistitemwidth+\XLingPaperlistitemindent}\leftskip\XLingPapertempdim\relax
\interlinepenalty10000
\XLingPaperlistitem{\XLingPaperlistitemindent}{\XLingPaperbulletlistitemwidth}{•}{Often adds or changes tags on a per-category (groups of words) basis.}}
{\setlength{\XLingPapertempdim}{\XLingPaperbulletlistitemwidth+\XLingPaperlistitemindent}\leftskip\XLingPapertempdim\relax
\interlinepenalty10000
\XLingPaperlistitem{\XLingPaperlistitemindent}{\XLingPaperbulletlistitemwidth}{•}{Can change the order of words.}}}}
\vspace{\baselineskip}
}\indent A rule-of-thumb is, if the rule applies to all words in a category, it probably wants to be treated in the structural transfer; if it applies to just part of those words, then maybe it needs to be dealt with in the lexical transfer.\par{}\indent Lexical transfer for a three word lexicon looks like \hyperlink{xLexXfer}{(42)}. In other words, the bilingual dictionary maps a source word to target word. In {\LangtToolFontFamily{\textbf{\textcolor[rgb]{0,0,0.5019607843137255}{FLExTrans}}}} you don’t normally need to examine the bilingual dictionary, but conceptually it looks like this. Open the {\textit{bilingual.dix}} in the XML Mind editor for the details.\par{}{\vspace{12pt plus 2pt minus 1pt}\raggedright{}\XLingPaperexample{.125in}{0pt}{2.75em}{\raisebox{\baselineskip}[0pt]{\protect\hypertarget{xLexXfer}{}}(42)}{{\LanglVernacularFontFamily{{\fontspec[Scale=0.9]{Courier New}\textup{\textbf{\vbox{\hbox{\strut{}slword{{\fontspec[Scale=0.65]{Times New Roman}\textsubscript{1.1}}} {\LangtluGrammCatFontFamily{{\fontspec[Scale=0.9]{Courier New}\textcolor[rgb]{0,0.4392156862745098,0.7529411764705882}{somecat}}}} -\textgreater{} tlword{{\fontspec[Scale=0.65]{Times New Roman}\textsubscript{1.1}}} {\LangtluGrammCatFontFamily{{\fontspec[Scale=0.9]{Courier New}\textcolor[rgb]{0,0.4392156862745098,0.7529411764705882}{somecat}}}} }\hbox{\strut{}slword1{{\fontspec[Scale=0.65]{Times New Roman}\textsubscript{2.1}}} {\LangtluGrammCatFontFamily{{\fontspec[Scale=0.9]{Courier New}\textcolor[rgb]{0,0.4392156862745098,0.7529411764705882}{somecat}}}} {\LangtluAffixFontFamily{{\fontspec[Scale=0.9]{Courier New}\textcolor[rgb]{0,0.6901960784313725,0.3137254901960784}{blah}}}} -\textgreater{} tlword3{{\fontspec[Scale=0.65]{Times New Roman}\textsubscript{1.2}}} {\LangtluGrammCatFontFamily{{\fontspec[Scale=0.9]{Courier New}\textcolor[rgb]{0,0.4392156862745098,0.7529411764705882}{somecat}}}} {\LangtluAffixFontFamily{{\fontspec[Scale=0.9]{Courier New}\textcolor[rgb]{0,0.6901960784313725,0.3137254901960784}{foo}}}}}\hbox{\strut{}slword3{{\fontspec[Scale=0.65]{Times New Roman}\textsubscript{3.1}}} {\LangtluGrammCatFontFamily{{\fontspec[Scale=0.9]{Courier New}\textcolor[rgb]{0,0.4392156862745098,0.7529411764705882}{somecat}}}} {\LangtluAffixFontFamily{{\fontspec[Scale=0.9]{Courier New}\textcolor[rgb]{0,0.6901960784313725,0.3137254901960784}{blah}}}} -\textgreater{} tlword2{{\fontspec[Scale=0.65]{Times New Roman}\textsubscript{1.1}}} {\LangtluGrammCatFontFamily{{\fontspec[Scale=0.9]{Courier New}\textcolor[rgb]{0,0.4392156862745098,0.7529411764705882}{somecat}}}} {\LangtluAffixFontFamily{{\fontspec[Scale=0.9]{Courier New}\textcolor[rgb]{0,0.6901960784313725,0.3137254901960784}{GD}}}}}}}}}}}\ }
\vspace{12pt plus 2pt minus 1pt}}\par\indent The output of the structural transfer would look like \hyperlink{xStrXfer}{(43)}.\par{}{\vspace{12pt plus 2pt minus 1pt}\raggedright{}\XLingPaperexample{.125in}{0pt}{2.75em}{\raisebox{\baselineskip}[0pt]{\protect\hypertarget{xStrXfer}{}}(43)}{{\LanglVernacularFontFamily{{\fontspec[Scale=0.9]{Courier New}\textup{\textbf{tlword{{\fontspec[Scale=0.65]{Times New Roman}\textsubscript{1.1}}} {\LangtluGrammCatFontFamily{{\fontspec[Scale=0.9]{Courier New}\textcolor[rgb]{0,0.4392156862745098,0.7529411764705882}{somecat}}}} tlword3{{\fontspec[Scale=0.65]{Times New Roman}\textsubscript{1.2}}} {\LangtluGrammCatFontFamily{{\fontspec[Scale=0.9]{Courier New}\textcolor[rgb]{0,0.4392156862745098,0.7529411764705882}{somecat}}}} {\LangtluAffixFontFamily{{\fontspec[Scale=0.9]{Courier New}\textcolor[rgb]{0,0.6901960784313725,0.3137254901960784}{foo}}}} tlword2{{\fontspec[Scale=0.65]{Times New Roman}\textsubscript{1.1}}} {\LangtluGrammCatFontFamily{{\fontspec[Scale=0.9]{Courier New}\textcolor[rgb]{0,0.4392156862745098,0.7529411764705882}{somecat}}}} {\LangtluAffixFontFamily{{\fontspec[Scale=0.9]{Courier New}\textcolor[rgb]{0,0.6901960784313725,0.3137254901960784}{GD}}}}}}}}}\ }
\vspace{12pt plus 2pt minus 1pt}}\par\indent When you are in the structural transfer stage, you have access to both the source and target sides.\par{}
\vspace{12pt}
\XLingPaperneedspace{5\baselineskip}

\penalty-3000{\noindent{\raisebox{\baselineskip}[0pt]{\protect\hypertarget{sLexTransProc}{}}\SectionLevelFourFontFamily{\normalsize{\raisebox{\baselineskip}[0pt]{\pdfbookmark[4]{6.1.3.1 How Lexical Transfer is Processed}{sLexTransProc}}\textit{6.1.3.1 How Lexical Transfer is Processed}}}}
\markright{How Lexical Transfer is Processed}
\XLingPaperaddtocontents{sLexTransProc}}\par{}\penalty10000
\vspace{12pt}
\indent This isn’t critical information that you have to know, but gives insight into what’s going on. Feel free to skip this section.\par{}\indent Given an input lexical unit shown in \hyperlink{xInputLexUnit}{(44)}:\par{}{\vspace{12pt plus 2pt minus 1pt}\raggedright{}\XLingPaperexample{.125in}{0pt}{2.75em}{\raisebox{\baselineskip}[0pt]{\protect\hypertarget{xInputLexUnit}{}}(44)}{{\LanglVernacularFontFamily{{\fontspec[Scale=0.9]{Courier New}\textup{\textbf{slword{{\fontspec[Scale=0.65]{Times New Roman}\textsubscript{1.1}}} {\LangtluGrammCatFontFamily{{\fontspec[Scale=0.9]{Courier New}\textcolor[rgb]{0,0.4392156862745098,0.7529411764705882}{cat1}}}} {\LangtluAffixFontFamily{{\fontspec[Scale=0.9]{Courier New}\textcolor[rgb]{0,0.6901960784313725,0.3137254901960784}{aff2}}}} {\LangtluAffixFontFamily{{\fontspec[Scale=0.9]{Courier New}\textcolor[rgb]{0,0.6901960784313725,0.3137254901960784}{aff3}}}}}}}}}\ }
\vspace{12pt plus 2pt minus 1pt}}\par\indent If we have the following in the bilingual dictionary:\par{}{\vspace{12pt plus 2pt minus 1pt}\raggedright{}\XLingPaperexample{.125in}{0pt}{2.75em}{\raisebox{\baselineskip}[0pt]{\protect\hypertarget{xBiLingEntry}{}}(45)}{\parbox[t]{\textwidth - .125in - 0pt}{\vspace*{-\baselineskip}{\XeTeXpicfile "../Images/RulesTutSampBilEntry.PNG" scaled 600}}}
\vspace{12pt plus 2pt minus 1pt}}\par\indent Which gives the following mapping in terms of lexical units:\par{}{\vspace{12pt plus 2pt minus 1pt}\raggedright{}\XLingPaperexample{.125in}{0pt}{2.75em}{\raisebox{\baselineskip}[0pt]{\protect\hypertarget{xMappingSample}{}}(46)}{{\LanglVernacularFontFamily{{\fontspec[Scale=0.9]{Courier New}\textup{\textbf{slword{{\fontspec[Scale=0.65]{Times New Roman}\textsubscript{1.1}}} {\LangtluGrammCatFontFamily{{\fontspec[Scale=0.9]{Courier New}\textcolor[rgb]{0,0.4392156862745098,0.7529411764705882}{cat1}}}} {\LangtluAffixFontFamily{{\fontspec[Scale=0.9]{Courier New}\textcolor[rgb]{0,0.6901960784313725,0.3137254901960784}{aff2}}}} -\textgreater{} tlword{{\fontspec[Scale=0.65]{Times New Roman}\textsubscript{1.1}}} {\LangtluGrammCatFontFamily{{\fontspec[Scale=0.9]{Courier New}\textcolor[rgb]{0,0.4392156862745098,0.7529411764705882}{cat1}}}}}}}}}\ }
\vspace{12pt plus 2pt minus 1pt}}\par\indent We will get this target-language output from the lexical-transfer module:\par{}{\vspace{12pt plus 2pt minus 1pt}\raggedright{}\XLingPaperexample{.125in}{0pt}{2.75em}{\raisebox{\baselineskip}[0pt]{\protect\hypertarget{xLexXferResult}{}}(47)}{{\LanglVernacularFontFamily{{\fontspec[Scale=0.9]{Courier New}\textup{\textbf{tlword{{\fontspec[Scale=0.65]{Times New Roman}\textsubscript{1.1}}} {\LangtluGrammCatFontFamily{{\fontspec[Scale=0.9]{Courier New}\textcolor[rgb]{0,0.4392156862745098,0.7529411764705882}{cat1}}}} {\LangtluAffixFontFamily{{\fontspec[Scale=0.9]{Courier New}\textcolor[rgb]{0,0.6901960784313725,0.3137254901960784}{aff3}}}}}}}}}\ }
\vspace{12pt plus 2pt minus 1pt}}\par\indent Note that the target-language lexical form, as defined in the bilingual dictionary entry in \hyperlink{xBiLingEntry}{(45)}, is produced by replacing two tags on the source side (i.e. {\LanglVernacularFontFamily{{\fontspec[Scale=0.9]{Courier New}\textup{\textbf{cat1 aff2}}}}}) with one tag on the target side (i.e. {\LanglVernacularFontFamily{{\fontspec[Scale=0.9]{Courier New}\textup{\textbf{cat1}}}}}).\par{}\indent Important: any source language tags not matched in the bilingual dictionary entry are copied into the output on the target language side. In our example, {\LanglVernacularFontFamily{{\fontspec[Scale=0.9]{Courier New}\textup{\textbf{{\LangtluAffixFontFamily{{\fontspec[Scale=0.9]{Courier New}\textcolor[rgb]{0,0.6901960784313725,0.3137254901960784}{aff3}}}}}}}}} in \hyperlink{xInputLexUnit}{(44)} gets copied to the final output in \hyperlink{xLexXferResult}{(47)}.\par{}
\vspace{12pt}
\XLingPaperneedspace{5\baselineskip}

\penalty-3000{\noindent{\raisebox{\baselineskip}[0pt]{\protect\hypertarget{sPrelim}{}}\SectionLevelThreeFontFamily{\normalsize{\raisebox{\baselineskip}[0pt]{\pdfbookmark[3]{6.1.4 Some Preliminaries}{sPrelim}}\textit{6.1.4 Some Preliminaries}}}}
\markright{Some Preliminaries}
\XLingPaperaddtocontents{sPrelim}}\par{}\penalty10000
\vspace{12pt}
\indent The transfer rule file is written in an XML format and in {\LangtToolFontFamily{\textbf{\textcolor[rgb]{0,0,0.5019607843137255}{FLExTrans}}}} it is named {\textit{transfer\_rules.t1x}}\protect\footnote[6]{{\leftskip0pt\parindent1em\raisebox{\baselineskip}[0pt]{\protect\hypertarget{nT1X}{}}The 1 in .t1x stands for the first structural transfer pass. This is the only file we worry about when doing a shallow-transfer system. For advanced-transfer, you use three rule files, and the extensions .t2x and .t3x are also used.}}. We could do all the editing of the transfer rules in a text editor, but by using the structured editor {\LangtToolFontFamily{\textbf{\textcolor[rgb]{0,0,0.5019607843137255}{XMLmind XML Editor}}}} we not only get a graphical user interface, but also verification that the rule file we are writing is valid. In fact the {\LangtToolFontFamily{\textbf{\textcolor[rgb]{0,0,0.5019607843137255}{XMLmind XML Editor}}}} and the add-ons for {\LangtToolFontFamily{\textbf{\textcolor[rgb]{0,0,0.5019607843137255}{FLExTrans}}}} make it hard to write an invalid rule file.\par{}
\vspace{12pt}
\XLingPaperneedspace{5\baselineskip}

\penalty-3000{\noindent{\raisebox{\baselineskip}[0pt]{\protect\hypertarget{sFileOverview}{}}\SectionLevelThreeFontFamily{\normalsize{\raisebox{\baselineskip}[0pt]{\pdfbookmark[3]{6.1.5 Overview of a Transfer File}{sFileOverview}}\textit{6.1.5 Overview of a Transfer File}}}}
\markright{Overview of a Transfer File}
\XLingPaperaddtocontents{sFileOverview}}\par{}\penalty10000
\vspace{12pt}
\indent It’s hard to give a step-by-step overview of what a transfer file looks like because there is quite a lot of obligatory parts that need to go into even the most basic file. But, it’s important to get a general view before we go into the details. Here is an example in which I’m deliberately not going to use linguistic names for the different parts to try and avoid assumptions.\par{}{\vspace{12pt plus 2pt minus 1pt}\raggedright{}\XLingPaperexample{.125in}{0pt}{2.75em}{\raisebox{\baselineskip}[0pt]{\protect\hypertarget{xOverviewRuleFile}{}}(48)}{\parbox[t]{\textwidth - .125in - 0pt}{\vspace*{-\baselineskip}{\XeTeXpicfile "../Images/RulesTutOverview.PNG" scaled 600}}}
\vspace{12pt plus 2pt minus 1pt}}\par\indent You should see this same thing when you open the file {\textit{transfer\_rules.t1x}} ({\LangtFoldernameFontFamily{{\fontspec[Scale=0.8]{Tahoma}\textup{\textmd{FLExTrans Documentation\textbackslash{}Transfer Rules Tutorial\textbackslash{}Croatian-English}}}}} folder) in the {\LangtToolFontFamily{\textbf{\textcolor[rgb]{0,0,0.5019607843137255}{XMLmind XML Editor}}}}. (Expand all the elements.)\par{}\indent The transfer file is divided into two main parts: a declaration section and a rules section. The rules section uses information from the declaration section. In fact, every kind of reference you make in the rules section, you have to declare in the declaration section. The rules section is clearly denoted by the element {\LangtRuleElemInXXEFontFamily{{\fontspec[Scale=0.8]{Arial}\textcolor[rgb]{0,0.4,0.2}{\textbf{Rules}}}}}. All of the elements above {\LangtRuleElemInXXEFontFamily{{\fontspec[Scale=0.8]{Arial}\textcolor[rgb]{0,0.4,0.2}{\textbf{Rules}}}}} are part of the declaration section. In \hyperlink{xOverviewRuleFile}{(48)} a minimal amount of declarations are shown, namely the {\LangtRuleElemInXXEFontFamily{{\fontspec[Scale=0.8]{Arial}\textcolor[rgb]{0,0.4,0.2}{\textbf{Categories}}}}}, {\LangtRuleElemInXXEFontFamily{{\fontspec[Scale=0.8]{Arial}\textcolor[rgb]{0,0.4,0.2}{\textbf{Attributes}}}}}, and {\LangtRuleElemInXXEFontFamily{{\fontspec[Scale=0.8]{Arial}\textcolor[rgb]{0,0.4,0.2}{\textbf{Variables}}}}} elements. (The elements {\LangtRuleElemInXXEFontFamily{{\fontspec[Scale=0.8]{Arial}\textcolor[rgb]{0,0.4,0.2}{\textbf{Lists}}}}} and {\LangtRuleElemInXXEFontFamily{{\fontspec[Scale=0.8]{Arial}\textcolor[rgb]{0,0.4,0.2}{\textbf{Macros}}}}} are not shown.)\par{}\indent Rules have two main parts: the pattern definition and the actions to be carried out.\par{}\indent The {\LangtRuleElemInXXEFontFamily{{\fontspec[Scale=0.8]{Arial}\textcolor[rgb]{0,0.4,0.2}{\textbf{pattern}}}}} element is what determines if {\LangtToolFontFamily{\textbf{\textcolor[rgb]{0,0,0.5019607843137255}{Apertium}}}} runs the rule or not. If the input word or words match the pattern, {\LangtToolFontFamily{\textbf{\textcolor[rgb]{0,0,0.5019607843137255}{Apertium}}}} will run the rule. The thing that goes in the {\LangtRuleElemInXXEFontFamily{{\fontspec[Scale=0.8]{Arial}\textcolor[rgb]{0,0.4,0.2}{\textbf{item}}}}} sub-element of the {\LangtRuleElemInXXEFontFamily{{\fontspec[Scale=0.8]{Arial}\textcolor[rgb]{0,0.4,0.2}{\textbf{pattern}}}}} element is a list of one or more categories (from the declaration section) you want to match.\par{}\indent The {\LangtRuleElemInXXEFontFamily{{\fontspec[Scale=0.8]{Arial}\textcolor[rgb]{0,0.4,0.2}{\textbf{action}}}}} element contains the steps you want to perform in the rule, generally ending with an {\LangtRuleElemInXXEFontFamily{{\fontspec[Scale=0.8]{Arial}\textcolor[rgb]{0,0.4,0.2}{\textbf{output}}}}} element where you output lexical units into the data stream. Note how declared attributes are used in the {\LangtRuleElemInXXEFontFamily{{\fontspec[Scale=0.8]{Arial}\textcolor[rgb]{0,0.4,0.2}{\textbf{action}}}}} element. Categories and attributes are discussed more below.\par{}
\vspace{12pt}
\XLingPaperneedspace{5\baselineskip}

\penalty-3000{\noindent{\raisebox{\baselineskip}[0pt]{\protect\hypertarget{sRulesApplied}{}}\SectionLevelThreeFontFamily{\normalsize{\raisebox{\baselineskip}[0pt]{\pdfbookmark[3]{6.1.6 How Rules are Applied}{sRulesApplied}}\textit{6.1.6 How Rules are Applied}}}}
\markright{How Rules are Applied}
\XLingPaperaddtocontents{sRulesApplied}}\par{}\penalty10000
\vspace{12pt}
\indent Rules are applied when a pattern is matched in the source language input data stream. Which patterns are matched in which order goes like this: the longest patterns are attempted first and then successive shorter patterns are attempted. When there is more than one pattern of the same length, the pattern that comes first in the file is attempted first followed by the rest in order.\par{}\indent Patterns do not overlap. Once a pattern is matched, the matched words are processed and not considered again for any other patterns. In other words, the source language words are processed sequentially in chunks. Cf. “\hyperlink{sPatternMatch}{How does pattern matching work?}”.\par{}
\vspace{12pt}
\XLingPaperneedspace{5\baselineskip}

\penalty-3000{\noindent{\raisebox{\baselineskip}[0pt]{\protect\hypertarget{sPracticalEx}{}}\SectionLevelTwoFontFamily{\normalsize{\raisebox{\baselineskip}[0pt]{\pdfbookmark[2]{6.2 Practical Example}{sPracticalEx}}\textbf{6.2 Practical Example}}}}
\markright{Practical Example}
\XLingPaperaddtocontents{sPracticalEx}}\par{}\penalty10000
\vspace{12pt}
{\raggedright{}\XLingPaperneedspace{4\baselineskip}\XLingPaperexample{.125in}{0pt}{2.75em}{\raisebox{\baselineskip}[0pt]{\protect\hypertarget{xCroatEngTable}{}}(49)}{{
\XLingPaperminmaxcellincolumn{Input:}{\XLingPapermincola}{\textbf{Input:}}{\XLingPapermaxcola}{+0\tabcolsep}
\XLingPaperminmaxcellincolumn{Otišla}{\XLingPapermincolb}{Otišla si}{\XLingPapermaxcolb}{+0\tabcolsep}
\XLingPaperminmaxcellincolumn{tiho}{\XLingPapermincolc}{tiho}{\XLingPapermaxcolc}{+0\tabcolsep}
\XLingPaperminmaxcellincolumn{i}{\XLingPapermincold}{i}{\XLingPapermaxcold}{+0\tabcolsep}
\XLingPaperminmaxcellincolumn{bez}{\XLingPapermincole}{bez}{\XLingPapermaxcole}{+0\tabcolsep}
\XLingPaperminmaxcellincolumn{pozdrava}{\XLingPapermincolf}{pozdrava}{\XLingPapermaxcolf}{+0\tabcolsep}
\XLingPaperminmaxcellincolumn{Output:}{\XLingPapermincola}{\textbf{Output:}}{\XLingPapermaxcola}{+0\tabcolsep}
\XLingPaperminmaxcellincolumn{left}{\XLingPapermincolb}{You left}{\XLingPapermaxcolb}{+0\tabcolsep}
\XLingPaperminmaxcellincolumn{quietly}{\XLingPapermincolc}{quietly}{\XLingPapermaxcolc}{+0\tabcolsep}
\XLingPaperminmaxcellincolumn{and}{\XLingPapermincold}{and}{\XLingPapermaxcold}{+0\tabcolsep}
\XLingPaperminmaxcellincolumn{without}{\XLingPapermincole}{without}{\XLingPapermaxcole}{+0\tabcolsep}
\XLingPaperminmaxcellincolumn{word}{\XLingPapermincolf}{a word}{\XLingPapermaxcolf}{+0\tabcolsep}
\setlength{\XLingPaperavailabletablewidth}{433.62pt - .125in - 0pt - 2.75em}
\setlength{\XLingPapertableminwidth}{\XLingPapermincola+\XLingPapermincolb+\XLingPapermincolc+\XLingPapermincold+\XLingPapermincole+\XLingPapermincolf}
\setlength{\XLingPapertablemaxwidth}{\XLingPapermaxcola+\XLingPapermaxcolb+\XLingPapermaxcolc+\XLingPapermaxcold+\XLingPapermaxcole+\XLingPapermaxcolf}
\XLingPapercalculatetablewidthratio{}
\XLingPapersetcolumnwidth{\XLingPapercolawidth}{\XLingPapermincola}{\XLingPapermaxcola}{-0\tabcolsep}
\XLingPapersetcolumnwidth{\XLingPapercolbwidth}{\XLingPapermincolb}{\XLingPapermaxcolb}{-2\tabcolsep}
\XLingPapersetcolumnwidth{\XLingPapercolcwidth}{\XLingPapermincolc}{\XLingPapermaxcolc}{-2\tabcolsep}
\XLingPapersetcolumnwidth{\XLingPapercoldwidth}{\XLingPapermincold}{\XLingPapermaxcold}{-2\tabcolsep}
\XLingPapersetcolumnwidth{\XLingPapercolewidth}{\XLingPapermincole}{\XLingPapermaxcole}{-2\tabcolsep}
\XLingPapersetcolumnwidth{\XLingPapercolfwidth}{\XLingPapermincolf}{\XLingPapermaxcolf}{-2\tabcolsep}\setlength{\LTpre}{-.5\baselineskip}\setlength{\LTleft}{.125in + 2.75em}\setlength{\LTpost}{0pt}
\begin{longtable}
[t]{@{}>{\raggedright}p{\XLingPapercolawidth}>{\raggedright}p{\XLingPapercolbwidth}>{\raggedright}p{\XLingPapercolcwidth}>{\raggedright}p{\XLingPapercoldwidth}>{\raggedright}p{\XLingPapercolewidth}>{\raggedright}p{\XLingPapercolfwidth}@{}}\specialrule{\heavyrulewidth}{-4\aboverulesep}{\belowrulesep}\specialrule{\heavyrulewidth}{-4\aboverulesep}{\belowrulesep}\multicolumn{1}{@{}>{\raggedright}p{\XLingPapercolawidth}}{\textbf{Input:}}&\multicolumn{1}{>{\raggedright}p{\XLingPapercolbwidth}}{Otišla si}&\multicolumn{1}{>{\raggedright}p{\XLingPapercolcwidth}}{tiho}&\multicolumn{1}{>{\raggedright}p{\XLingPapercoldwidth}}{i}&\multicolumn{1}{>{\raggedright}p{\XLingPapercolewidth}}{bez}&\multicolumn{1}{>{\raggedright}p{\XLingPapercolfwidth}@{}}{pozdrava}\\%
\multicolumn{1}{@{}>{\raggedright}p{\XLingPapercolawidth}}{\textbf{Output:}}&\multicolumn{1}{>{\raggedright}p{\XLingPapercolbwidth}}{You left}&\multicolumn{1}{>{\raggedright}p{\XLingPapercolcwidth}}{quietly}&\multicolumn{1}{>{\raggedright}p{\XLingPapercoldwidth}}{and}&\multicolumn{1}{>{\raggedright}p{\XLingPapercolewidth}}{without}&\multicolumn{1}{>{\raggedright}p{\XLingPapercolfwidth}@{}}{a word}\\\bottomrule%
\end{longtable}
}}
\vspace{12pt plus 2pt minus 1pt}}\par\indent Let’s do an exercise where the goal is to turn a sentence in Croatian into English. The input and desired output is shown in \hyperlink{xCroatEngTable}{(49)}.\par{}
\vspace{12pt}
\XLingPaperneedspace{5\baselineskip}

\penalty-3000{\noindent{\raisebox{\baselineskip}[0pt]{\protect\hypertarget{sTutSetup}{}}\SectionLevelThreeFontFamily{\normalsize{\raisebox{\baselineskip}[0pt]{\pdfbookmark[3]{6.2.1 Getting Set Up}{sTutSetup}}\textit{6.2.1 Getting Set Up}}}}
\markright{Getting Set Up}
\XLingPaperaddtocontents{sTutSetup}}\par{}\penalty10000
\vspace{12pt}
{\parskip .5pt plus 1pt minus 1pt
                    
{\setlength{\XLingPapertempdim}{\XLingPapersingledigitlistitemwidth+\parindent{}}\leftskip\XLingPapertempdim\relax
\interlinepenalty10000
\XLingPaperlistitem{\parindent{}}{\XLingPapersingledigitlistitemwidth}{1.}{There are two {\LangtToolFontFamily{\textbf{\textcolor[rgb]{0,0,0.5019607843137255}{FLEx}}}} projects already set up that we are going to use. You will find them in the {\LangtFoldernameFontFamily{{\fontspec[Scale=0.8]{Tahoma}\textup{\textmd{FLExTrans Documentation\textbackslash{}Transfer Rules Tutorial}}}}} folder. They are called {\textit{Croatian-FLExTrans-Sample ... .fwbackup}} and {\textit{English-FLExTrans-Sample ... .fwbackup}}. Double-click on them one by one to restore them into {\LangtToolFontFamily{\textbf{\textcolor[rgb]{0,0,0.5019607843137255}{FLEx}}}}.}}
{\setlength{\XLingPapertempdim}{\XLingPapersingledigitlistitemwidth+\parindent{}}\leftskip\XLingPapertempdim\relax
\interlinepenalty10000
\XLingPaperlistitem{\parindent{}}{\XLingPapersingledigitlistitemwidth}{2.}{In the same folder, you'll find a folder named {\LangtFoldernameFontFamily{{\fontspec[Scale=0.8]{Tahoma}\textup{\textmd{Croatian-English}}}}}. Copy the folder and navigate to the {\LangtFoldernameFontFamily{{\fontspec[Scale=0.8]{Tahoma}\textup{\textmd{FLExTrans\textbackslash{}WorkProjects}}}}} folder and Paste it there.}}
{\setlength{\XLingPapertempdim}{\XLingPapersingledigitlistitemwidth+\parindent{}}\leftskip\XLingPapertempdim\relax
\interlinepenalty10000
\XLingPaperlistitem{\parindent{}}{\XLingPapersingledigitlistitemwidth}{3.}{Start {\LangtToolFontFamily{\textbf{\textcolor[rgb]{0,0,0.5019607843137255}{FLExTools}}}}. See \hyperlink{sStartFlextools}{A.1} for step-by-step instructions.}}
{\setlength{\XLingPapertempdim}{\XLingPapersingledigitlistitemwidth+\parindent{}}\leftskip\XLingPapertempdim\relax
\interlinepenalty10000
\XLingPaperlistitem{\parindent{}}{\XLingPapersingledigitlistitemwidth}{4.}{Verify that the status bar shows the current collection is set to {\LangtCollectionFontFamily{{\fontspec[Scale=0.8]{Arial}\textup{\textbf{\textcolor[rgb]{0.4,0,0.4}{Drafting}}}}}}.}}
\vspace{\baselineskip}
}
\vspace{12pt}
\XLingPaperneedspace{5\baselineskip}

\penalty-3000{\noindent{\raisebox{\baselineskip}[0pt]{\protect\hypertarget{sLexicalTransfer}{}}\SectionLevelThreeFontFamily{\normalsize{\raisebox{\baselineskip}[0pt]{\pdfbookmark[3]{6.2.2 Lexical Transfer}{sLexicalTransfer}}\textit{6.2.2 Lexical Transfer}}}}
\markright{Lexical Transfer}
\XLingPaperaddtocontents{sLexicalTransfer}}\par{}\penalty10000
\vspace{12pt}
\indent For the purposes of this example, we are going to use some prepared data. We have a Croatian text that is already analyzed and we have small Croatian and English lexicons with all of the entries and senses that we need.\par{}\indent Let’s see what our input and output looks like initially if we run the system. Our input of course, according to \hyperlink{xCroatEngTable}{(49)} is: {\textit{Otišla si tiho i bez pozdrava.}}\par{}{\parskip .5pt plus 1pt minus 1pt
                    
\vspace{\baselineskip}

{\setlength{\XLingPapertempdim}{\XLingPapersingledigitlistitemwidth+\parindent{}}\leftskip\XLingPapertempdim\relax
\interlinepenalty10000
\XLingPaperlistitem{\parindent{}}{\XLingPapersingledigitlistitemwidth}{1.}{Go to the {\LangtToolFontFamily{\textbf{\textcolor[rgb]{0,0,0.5019607843137255}{FLExTools}}}} app.}}
{\setlength{\XLingPapertempdim}{\XLingPapersingledigitlistitemwidth+\parindent{}}\leftskip\XLingPapertempdim\relax
\interlinepenalty10000
\XLingPaperlistitem{\parindent{}}{\XLingPapersingledigitlistitemwidth}{2.}{Ensure the {\LangtCollectionFontFamily{{\fontspec[Scale=0.8]{Arial}\textup{\textbf{\textcolor[rgb]{0.4,0,0.4}{Drafting}}}}}} collection is selected.}}
{\setlength{\XLingPapertempdim}{\XLingPapersingledigitlistitemwidth+\parindent{}}\leftskip\XLingPapertempdim\relax
\interlinepenalty10000
\XLingPaperlistitem{\parindent{}}{\XLingPapersingledigitlistitemwidth}{3.}{Click the {\textup{\textbf{Run All}}} button to run all the {\LangtToolFontFamily{\textbf{\textcolor[rgb]{0,0,0.5019607843137255}{FLExTrans}}}} modules.}}
{\setlength{\XLingPapertempdim}{\XLingPapersingledigitlistitemwidth+\parindent{}}\leftskip\XLingPapertempdim\relax
\interlinepenalty10000
\XLingPaperlistitem{\parindent{}}{\XLingPapersingledigitlistitemwidth}{4.}{Open the {\textbf{English-FLExTrans-Sample}} {\LangtToolFontFamily{\textbf{\textcolor[rgb]{0,0,0.5019607843137255}{FLEx}}}} project.}}
{\setlength{\XLingPapertempdim}{\XLingPapersingledigitlistitemwidth+\parindent{}}\leftskip\XLingPapertempdim\relax
\interlinepenalty10000
\XLingPaperlistitem{\parindent{}}{\XLingPapersingledigitlistitemwidth}{5.}{In the Texts and Words view, refresh the screen and click on the text titled “Left Behind”.}}
\vspace{\baselineskip}
}\indent You should see:\par{}{\vspace{12pt plus 2pt minus 1pt}\raggedright{}\XLingPaperexample{.125in}{0pt}{2.75em}{\raisebox{\baselineskip}[0pt]{\protect\hypertarget{xOutputInitial}{}}(50)}{\LanglGlossFontFamily{\textit{Leavepfvptcp f sg beprssg2 quietly and without wordgen sg}}\ }
\vspace{12pt plus 2pt minus 1pt}}\par\indent Clearly we have some work to do, but at least three words look good. \hyperlink{xOutputInitial}{(50)} is what we get using straight lexical transfer. In other words, English word-senses are substituted for Croatian word-senses. The sample transfer rule we are using (shown in \hyperlink{xOverviewRuleFile}{(48)}) is having no influence during the transfer process.\par{}\indent Let’s look at the input and output in data stream format.\par{}{\vspace{12pt plus 2pt minus 1pt}\raggedright{}\XLingPaperneedspace{4\baselineskip}\XLingPaperexample{.125in}{0pt}{2.75em}{\raisebox{\baselineskip}[0pt]{\protect\hypertarget{xCroatEngDataStreamTable}{}}(51)}{{\setcounter{footnote}{6}
\XLingPaperminmaxcellincolumn{Input:You}{\XLingPapermincola}{\textbf{Input:}}{\XLingPapermaxcola}{+0\tabcolsep}
\XLingPaperminmaxcellincolumn{Otići1.1}{\XLingPapermincolb}{{\LanglVernacularFontFamily{{\fontspec[Scale=0.9]{Courier New}\textup{\textbf{Otići{{\fontspec[Scale=0.65]{Times New Roman}\textsubscript{1.1}}} {\LangtluGrammCatFontFamily{{\fontspec[Scale=0.9]{Courier New}\textcolor[rgb]{0,0.4392156862745098,0.7529411764705882}{v}}}} {\LangtluAffixFontFamily{{\fontspec[Scale=0.9]{Courier New}\textcolor[rgb]{0,0.6901960784313725,0.3137254901960784}{pfv}}}} {\LangtluAffixFontFamily{{\fontspec[Scale=0.9]{Courier New}\textcolor[rgb]{0,0.6901960784313725,0.3137254901960784}{ptcp\_f\_sg}}}}}}}}}}{\XLingPapermaxcolb}{+0\tabcolsep}
\XLingPaperminmaxcellincolumn{biti1.1}{\XLingPapermincolc}{{\LanglVernacularFontFamily{{\fontspec[Scale=0.9]{Courier New}\textup{\textbf{biti{{\fontspec[Scale=0.65]{Times New Roman}\textsubscript{1.1}}} {\LangtluGrammCatFontFamily{{\fontspec[Scale=0.9]{Courier New}\textcolor[rgb]{0,0.4392156862745098,0.7529411764705882}{cop}}}} {\LangtluAffixFontFamily{{\fontspec[Scale=0.9]{Courier New}\textcolor[rgb]{0,0.6901960784313725,0.3137254901960784}{prs}}}} {\LangtluAffixFontFamily{{\fontspec[Scale=0.9]{Courier New}\textcolor[rgb]{0,0.6901960784313725,0.3137254901960784}{sg}}}} {\LangtluAffixFontFamily{{\fontspec[Scale=0.9]{Courier New}\textcolor[rgb]{0,0.6901960784313725,0.3137254901960784}{2}}}}}}}}}}{\XLingPapermaxcolc}{+0\tabcolsep}
\XLingPaperminmaxcellincolumn{tiho1.1}{\XLingPapermincold}{{\LanglVernacularFontFamily{{\fontspec[Scale=0.9]{Courier New}\textup{\textbf{tiho{{\fontspec[Scale=0.65]{Times New Roman}\textsubscript{1.1}}} {\LangtluGrammCatFontFamily{{\fontspec[Scale=0.9]{Courier New}\textcolor[rgb]{0,0.4392156862745098,0.7529411764705882}{adv}}}}}}}}}}{\XLingPapermaxcold}{+0\tabcolsep}
\XLingPaperminmaxcellincolumn{coordconn}{\XLingPapermincole}{{\LanglVernacularFontFamily{{\fontspec[Scale=0.9]{Courier New}\textup{\textbf{i{{\fontspec[Scale=0.65]{Times New Roman}\textsubscript{1.1}}} {\LangtluGrammCatFontFamily{{\fontspec[Scale=0.9]{Courier New}\textcolor[rgb]{0,0.4392156862745098,0.7529411764705882}{coordconn}}}}}}}}}}{\XLingPapermaxcole}{+0\tabcolsep}
\XLingPaperminmaxcellincolumn{bez1.1}{\XLingPapermincolf}{{\LanglVernacularFontFamily{{\fontspec[Scale=0.9]{Courier New}\textup{\textbf{bez{{\fontspec[Scale=0.65]{Times New Roman}\textsubscript{1.1}}} {\LangtluGrammCatFontFamily{{\fontspec[Scale=0.9]{Courier New}\textcolor[rgb]{0,0.4392156862745098,0.7529411764705882}{prep}}}}}}}}}}{\XLingPapermaxcolf}{+0\tabcolsep}
\XLingPaperminmaxcellincolumn{gen\_sg}{\XLingPapermincolg}{{\LanglVernacularFontFamily{{\fontspec[Scale=0.9]{Courier New}\textup{\textbf{pozdrav{{\fontspec[Scale=0.65]{Times New Roman}\textsubscript{1.1}}} {\LangtluGrammCatFontFamily{{\fontspec[Scale=0.9]{Courier New}\textcolor[rgb]{0,0.4392156862745098,0.7529411764705882}{n}}}} {\LangtluAffixFontFamily{{\fontspec[Scale=0.9]{Courier New}\textcolor[rgb]{0,0.6901960784313725,0.3137254901960784}{gen\_sg}}}} }}}}}}{\XLingPapermaxcolg}{+0\tabcolsep}
\XLingPaperminmaxcellincolumn{Apertium}{\XLingPapermincola}{\textbf{Output:}}{\XLingPapermaxcola}{+0\tabcolsep}
\XLingPaperminmaxcellincolumn{ptcp\_f\_sg}{\XLingPapermincolb}{{\LanglVernacularFontFamily{{\fontspec[Scale=0.9]{Courier New}\textup{\textbf{Leave{{\fontspec[Scale=0.65]{Times New Roman}\textsubscript{1.1}}} {\LangtluGrammCatFontFamily{{\fontspec[Scale=0.9]{Courier New}\textcolor[rgb]{0,0.4392156862745098,0.7529411764705882}{v}}}} {\LangtluAffixFontFamily{{\fontspec[Scale=0.9]{Courier New}\textcolor[rgb]{0,0.6901960784313725,0.3137254901960784}{pfv}}}} {\LangtluAffixFontFamily{{\fontspec[Scale=0.9]{Courier New}\textcolor[rgb]{0,0.6901960784313725,0.3137254901960784}{ptcp\_f\_sg}}}} }}}}}}{\XLingPapermaxcolb}{+0\tabcolsep}
\XLingPaperminmaxcellincolumn{be1.1}{\XLingPapermincolc}{{\LanglVernacularFontFamily{{\fontspec[Scale=0.9]{Courier New}\textup{\textbf{be{{\fontspec[Scale=0.65]{Times New Roman}\textsubscript{1.1}}} {\LangtluGrammCatFontFamily{{\fontspec[Scale=0.9]{Courier New}\textcolor[rgb]{0,0.4392156862745098,0.7529411764705882}{cop}}}} {\LangtluAffixFontFamily{{\fontspec[Scale=0.9]{Courier New}\textcolor[rgb]{0,0.6901960784313725,0.3137254901960784}{prs}}}} {\LangtluAffixFontFamily{{\fontspec[Scale=0.9]{Courier New}\textcolor[rgb]{0,0.6901960784313725,0.3137254901960784}{sg}}}} {\LangtluAffixFontFamily{{\fontspec[Scale=0.9]{Courier New}\textcolor[rgb]{0,0.6901960784313725,0.3137254901960784}{2}}}} }}}}}}{\XLingPapermaxcolc}{+0\tabcolsep}
\XLingPaperminmaxcellincolumn{adv}{\XLingPapermincold}{{\LanglVernacularFontFamily{{\fontspec[Scale=0.9]{Courier New}\textup{\textbf{quietly{{\fontspec[Scale=0.65]{Times New Roman}\textsubscript{1.1}}} {\LangtluGrammCatFontFamily{{\fontspec[Scale=0.9]{Courier New}\textcolor[rgb]{0,0.4392156862745098,0.7529411764705882}{adv}}}} }}}}}}{\XLingPapermaxcold}{+0\tabcolsep}
\XLingPaperminmaxcellincolumn{coordconn}{\XLingPapermincole}{{\LanglVernacularFontFamily{{\fontspec[Scale=0.9]{Courier New}\textup{\textbf{and{{\fontspec[Scale=0.65]{Times New Roman}\textsubscript{1.1}}} {\LangtluGrammCatFontFamily{{\fontspec[Scale=0.9]{Courier New}\textcolor[rgb]{0,0.4392156862745098,0.7529411764705882}{coordconn}}}} }}}}}}{\XLingPapermaxcole}{+0\tabcolsep}
\XLingPaperminmaxcellincolumn{prep}{\XLingPapermincolf}{{\LanglVernacularFontFamily{{\fontspec[Scale=0.9]{Courier New}\textup{\textbf{without{{\fontspec[Scale=0.65]{Times New Roman}\textsubscript{1.1}}} {\LangtluGrammCatFontFamily{{\fontspec[Scale=0.9]{Courier New}\textcolor[rgb]{0,0.4392156862745098,0.7529411764705882}{prep}}}} }}}}}}{\XLingPapermaxcolf}{+0\tabcolsep}
\XLingPaperminmaxcellincolumn{word1.1}{\XLingPapermincolg}{{\LanglVernacularFontFamily{{\fontspec[Scale=0.9]{Courier New}\textup{\textbf{word{{\fontspec[Scale=0.65]{Times New Roman}\textsubscript{1.1}}} {\LangtluGrammCatFontFamily{{\fontspec[Scale=0.9]{Courier New}\textcolor[rgb]{0,0.4392156862745098,0.7529411764705882}{n}}}} {\LangtluAffixFontFamily{{\fontspec[Scale=0.9]{Courier New}\textcolor[rgb]{0,0.6901960784313725,0.3137254901960784}{gen\_sg}}}} }}}}}}{\XLingPapermaxcolg}{+0\tabcolsep}
\setlength{\XLingPaperavailabletablewidth}{433.62pt - .125in - 0pt - 2.75em}
\setlength{\XLingPapertableminwidth}{\XLingPapermincola+\XLingPapermincolb+\XLingPapermincolc+\XLingPapermincold+\XLingPapermincole+\XLingPapermincolf+\XLingPapermincolg}
\setlength{\XLingPapertablemaxwidth}{\XLingPapermaxcola+\XLingPapermaxcolb+\XLingPapermaxcolc+\XLingPapermaxcold+\XLingPapermaxcole+\XLingPapermaxcolf+\XLingPapermaxcolg}
\XLingPapercalculatetablewidthratio{}
\XLingPapersetcolumnwidth{\XLingPapercolawidth}{\XLingPapermincola}{\XLingPapermaxcola}{-0\tabcolsep}
\XLingPapersetcolumnwidth{\XLingPapercolbwidth}{\XLingPapermincolb}{\XLingPapermaxcolb}{-2\tabcolsep}
\XLingPapersetcolumnwidth{\XLingPapercolcwidth}{\XLingPapermincolc}{\XLingPapermaxcolc}{-2\tabcolsep}
\XLingPapersetcolumnwidth{\XLingPapercoldwidth}{\XLingPapermincold}{\XLingPapermaxcold}{-2\tabcolsep}
\XLingPapersetcolumnwidth{\XLingPapercolewidth}{\XLingPapermincole}{\XLingPapermaxcole}{-2\tabcolsep}
\XLingPapersetcolumnwidth{\XLingPapercolfwidth}{\XLingPapermincolf}{\XLingPapermaxcolf}{-2\tabcolsep}
\XLingPapersetcolumnwidth{\XLingPapercolgwidth}{\XLingPapermincolg}{\XLingPapermaxcolg}{-2\tabcolsep}\setcounter{footnote}{6}\setlength{\LTpre}{-.5\baselineskip}\setlength{\LTleft}{.125in + 2.75em}\setlength{\LTpost}{0pt}
\begin{longtable}
[t]{@{}>{\raggedright}p{\XLingPapercolawidth}>{\raggedright}p{\XLingPapercolbwidth}>{\raggedright}p{\XLingPapercolcwidth}>{\raggedright}p{\XLingPapercoldwidth}>{\raggedright}p{\XLingPapercolewidth}>{\raggedright}p{\XLingPapercolfwidth}>{\raggedright}p{\XLingPapercolgwidth}@{}}\specialrule{\heavyrulewidth}{-4\aboverulesep}{\belowrulesep}\specialrule{\heavyrulewidth}{-4\aboverulesep}{\belowrulesep}\multicolumn{1}{@{}>{\raggedright}p{\XLingPapercolawidth}}{\textbf{Input:\protect\footnote{{\leftskip0pt\parindent1em\raisebox{\baselineskip}[0pt]{\protect\hypertarget{nInputFile}{}}You would see this if you use the {\LangtModuleFontFamily{\textbf{\textcolor[rgb]{0.4,0,0.4}{View Source/Target Apertium File Tool}}}}. It is the friendly view of the file {\textit{source\_text-aper.txt}} in the {\LangtFoldernameFontFamily{{\fontspec[Scale=0.8]{Tahoma}\textup{\textmd{Build}}}}} folder. The actual content looks like this: {{\LanglVernacularFontFamily{{\fontspec[Scale=0.9]{Courier New}\textup{\textbf{\^{}Otići1.1\textless{}v\textgreater{}\textless{}pfv\textgreater{}\textless{}ptcp\_f\_sg\textgreater{}\textdollar{} \^{}biti1.1\textless{}cop\textgreater{}\textless{}prs\textgreater{}\textless{}sg\textgreater{}\textless{}2\textgreater{}\textdollar{} \^{}tiho1.1\textless{}adv\textgreater{}\textdollar{} \^{}i1.1\textless{}coordconn\textgreater{}\textdollar{} \^{}bez1.1\textless{}prep\textgreater{}\textdollar{} \^{}pozdrav1.1\textless{}n\textgreater{}\textless{}gen\_sg\textgreater{}\textdollar{}}}}}}}}}}}&\multicolumn{1}{>{\raggedright}p{\XLingPapercolbwidth}}{{\LanglVernacularFontFamily{{\fontspec[Scale=0.9]{Courier New}\textup{\textbf{Otići{{\fontspec[Scale=0.65]{Times New Roman}\textsubscript{1.1}}} {\LangtluGrammCatFontFamily{{\fontspec[Scale=0.9]{Courier New}\textcolor[rgb]{0,0.4392156862745098,0.7529411764705882}{v}}}} {\LangtluAffixFontFamily{{\fontspec[Scale=0.9]{Courier New}\textcolor[rgb]{0,0.6901960784313725,0.3137254901960784}{pfv}}}} {\LangtluAffixFontFamily{{\fontspec[Scale=0.9]{Courier New}\textcolor[rgb]{0,0.6901960784313725,0.3137254901960784}{ptcp\_f\_sg}}}}}}}}}}&\multicolumn{1}{>{\raggedright}p{\XLingPapercolcwidth}}{{\LanglVernacularFontFamily{{\fontspec[Scale=0.9]{Courier New}\textup{\textbf{biti{{\fontspec[Scale=0.65]{Times New Roman}\textsubscript{1.1}}} {\LangtluGrammCatFontFamily{{\fontspec[Scale=0.9]{Courier New}\textcolor[rgb]{0,0.4392156862745098,0.7529411764705882}{cop}}}} {\LangtluAffixFontFamily{{\fontspec[Scale=0.9]{Courier New}\textcolor[rgb]{0,0.6901960784313725,0.3137254901960784}{prs}}}} {\LangtluAffixFontFamily{{\fontspec[Scale=0.9]{Courier New}\textcolor[rgb]{0,0.6901960784313725,0.3137254901960784}{sg}}}} {\LangtluAffixFontFamily{{\fontspec[Scale=0.9]{Courier New}\textcolor[rgb]{0,0.6901960784313725,0.3137254901960784}{2}}}}}}}}}}&\multicolumn{1}{>{\raggedright}p{\XLingPapercoldwidth}}{{\LanglVernacularFontFamily{{\fontspec[Scale=0.9]{Courier New}\textup{\textbf{tiho{{\fontspec[Scale=0.65]{Times New Roman}\textsubscript{1.1}}} {\LangtluGrammCatFontFamily{{\fontspec[Scale=0.9]{Courier New}\textcolor[rgb]{0,0.4392156862745098,0.7529411764705882}{adv}}}}}}}}}}&\multicolumn{1}{>{\raggedright}p{\XLingPapercolewidth}}{{\LanglVernacularFontFamily{{\fontspec[Scale=0.9]{Courier New}\textup{\textbf{i{{\fontspec[Scale=0.65]{Times New Roman}\textsubscript{1.1}}} {\LangtluGrammCatFontFamily{{\fontspec[Scale=0.9]{Courier New}\textcolor[rgb]{0,0.4392156862745098,0.7529411764705882}{coordconn}}}}}}}}}}&\multicolumn{1}{>{\raggedright}p{\XLingPapercolfwidth}}{{\LanglVernacularFontFamily{{\fontspec[Scale=0.9]{Courier New}\textup{\textbf{bez{{\fontspec[Scale=0.65]{Times New Roman}\textsubscript{1.1}}} {\LangtluGrammCatFontFamily{{\fontspec[Scale=0.9]{Courier New}\textcolor[rgb]{0,0.4392156862745098,0.7529411764705882}{prep}}}}}}}}}}&\multicolumn{1}{>{\raggedright}p{\XLingPapercolgwidth}@{}}{{\LanglVernacularFontFamily{{\fontspec[Scale=0.9]{Courier New}\textup{\textbf{pozdrav{{\fontspec[Scale=0.65]{Times New Roman}\textsubscript{1.1}}} {\LangtluGrammCatFontFamily{{\fontspec[Scale=0.9]{Courier New}\textcolor[rgb]{0,0.4392156862745098,0.7529411764705882}{n}}}} {\LangtluAffixFontFamily{{\fontspec[Scale=0.9]{Courier New}\textcolor[rgb]{0,0.6901960784313725,0.3137254901960784}{gen\_sg}}}} }}}}}}\\%
\multicolumn{1}{@{}>{\raggedright}p{\XLingPapercolawidth}}{\textbf{Output:\protect\footnote{{\leftskip0pt\parindent1em\raisebox{\baselineskip}[0pt]{\protect\hypertarget{nOutputFile}{}}You would see this if you use the {\LangtModuleFontFamily{\textbf{\textcolor[rgb]{0.4,0,0.4}{View Source/Target Apertium File Tool}}}} and click the {\textup{\textbf{Target}}} button. It is the friendly view of the file {\textit{target\_text-aper.txt}} in the {\LangtFoldernameFontFamily{{\fontspec[Scale=0.8]{Tahoma}\textup{\textmd{Output}}}}} folder. The actual content looks like this: {{\LanglVernacularFontFamily{{\fontspec[Scale=0.9]{Courier New}\textup{\textbf{\^{}Leave1.1\textless{}v\textgreater{}\textless{}pfv\textgreater{}\textless{}ptcp\_f\_sg\textgreater{}\textdollar{} \^{}be1.1\textless{}cop\textgreater{}\textless{}prs\textgreater{}\textless{}sg\textgreater{}\textless{}2\textgreater{}\textdollar{} \^{}quietly1.1\textless{}adv\textgreater{}\textdollar{} \^{}and1.1\textless{}coordconn\textgreater{}\textdollar{} \^{}without1.1\textless{}prep\textgreater{}\textdollar{} \^{}word1.1\textless{}n\textgreater{}\textless{}gen\_sg\textgreater{}\textdollar{}}}}}}}}}}}&\multicolumn{1}{>{\raggedright}p{\XLingPapercolbwidth}}{{\LanglVernacularFontFamily{{\fontspec[Scale=0.9]{Courier New}\textup{\textbf{Leave{{\fontspec[Scale=0.65]{Times New Roman}\textsubscript{1.1}}} {\LangtluGrammCatFontFamily{{\fontspec[Scale=0.9]{Courier New}\textcolor[rgb]{0,0.4392156862745098,0.7529411764705882}{v}}}} {\LangtluAffixFontFamily{{\fontspec[Scale=0.9]{Courier New}\textcolor[rgb]{0,0.6901960784313725,0.3137254901960784}{pfv}}}} {\LangtluAffixFontFamily{{\fontspec[Scale=0.9]{Courier New}\textcolor[rgb]{0,0.6901960784313725,0.3137254901960784}{ptcp\_f\_sg}}}} }}}}}}&\multicolumn{1}{>{\raggedright}p{\XLingPapercolcwidth}}{{\LanglVernacularFontFamily{{\fontspec[Scale=0.9]{Courier New}\textup{\textbf{be{{\fontspec[Scale=0.65]{Times New Roman}\textsubscript{1.1}}} {\LangtluGrammCatFontFamily{{\fontspec[Scale=0.9]{Courier New}\textcolor[rgb]{0,0.4392156862745098,0.7529411764705882}{cop}}}} {\LangtluAffixFontFamily{{\fontspec[Scale=0.9]{Courier New}\textcolor[rgb]{0,0.6901960784313725,0.3137254901960784}{prs}}}} {\LangtluAffixFontFamily{{\fontspec[Scale=0.9]{Courier New}\textcolor[rgb]{0,0.6901960784313725,0.3137254901960784}{sg}}}} {\LangtluAffixFontFamily{{\fontspec[Scale=0.9]{Courier New}\textcolor[rgb]{0,0.6901960784313725,0.3137254901960784}{2}}}} }}}}}}&\multicolumn{1}{>{\raggedright}p{\XLingPapercoldwidth}}{{\LanglVernacularFontFamily{{\fontspec[Scale=0.9]{Courier New}\textup{\textbf{quietly{{\fontspec[Scale=0.65]{Times New Roman}\textsubscript{1.1}}} {\LangtluGrammCatFontFamily{{\fontspec[Scale=0.9]{Courier New}\textcolor[rgb]{0,0.4392156862745098,0.7529411764705882}{adv}}}} }}}}}}&\multicolumn{1}{>{\raggedright}p{\XLingPapercolewidth}}{{\LanglVernacularFontFamily{{\fontspec[Scale=0.9]{Courier New}\textup{\textbf{and{{\fontspec[Scale=0.65]{Times New Roman}\textsubscript{1.1}}} {\LangtluGrammCatFontFamily{{\fontspec[Scale=0.9]{Courier New}\textcolor[rgb]{0,0.4392156862745098,0.7529411764705882}{coordconn}}}} }}}}}}&\multicolumn{1}{>{\raggedright}p{\XLingPapercolfwidth}}{{\LanglVernacularFontFamily{{\fontspec[Scale=0.9]{Courier New}\textup{\textbf{without{{\fontspec[Scale=0.65]{Times New Roman}\textsubscript{1.1}}} {\LangtluGrammCatFontFamily{{\fontspec[Scale=0.9]{Courier New}\textcolor[rgb]{0,0.4392156862745098,0.7529411764705882}{prep}}}} }}}}}}&\multicolumn{1}{>{\raggedright}p{\XLingPapercolgwidth}@{}}{{\LanglVernacularFontFamily{{\fontspec[Scale=0.9]{Courier New}\textup{\textbf{word{{\fontspec[Scale=0.65]{Times New Roman}\textsubscript{1.1}}} {\LangtluGrammCatFontFamily{{\fontspec[Scale=0.9]{Courier New}\textcolor[rgb]{0,0.4392156862745098,0.7529411764705882}{n}}}} {\LangtluAffixFontFamily{{\fontspec[Scale=0.9]{Courier New}\textcolor[rgb]{0,0.6901960784313725,0.3137254901960784}{gen\_sg}}}} }}}}}}\\\bottomrule%
\end{longtable}
}}
}
\vspace{12pt}
\XLingPaperneedspace{5\baselineskip}

\penalty-3000{\noindent{\raisebox{\baselineskip}[0pt]{\protect\hypertarget{sThinking}{}}\SectionLevelThreeFontFamily{\normalsize{\raisebox{\baselineskip}[0pt]{\pdfbookmark[3]{6.2.3 Thinking it Through}{sThinking}}\textit{6.2.3 Thinking it Through}}}}
\markright{Thinking it Through}
\XLingPaperaddtocontents{sThinking}}\par{}\penalty10000
\vspace{12pt}
\indent Let’s think about what changes we need to make in order to convert the output shown in \hyperlink{xCroatEngDataStreamTable}{(51)} into an adequate form for target language synthesis. NB: If we want to change information, it’s a procedure; if we want to output information or not output information, it’s a declaration.\par{}
\vspace{12pt}
\XLingPaperneedspace{5\baselineskip}

\penalty-3000{\noindent{\raisebox{\baselineskip}[0pt]{\protect\hypertarget{sProc}{}}\SectionLevelFourFontFamily{\normalsize{\raisebox{\baselineskip}[0pt]{\pdfbookmark[4]{6.2.3.1 Procedures}{sProc}}\textit{6.2.3.1 Procedures}}}}
\markright{Procedures}
\XLingPaperaddtocontents{sProc}}\par{}\penalty10000
\vspace{12pt}
{\parskip .5pt plus 1pt minus 1pt
                    
{\setlength{\XLingPapertempdim}{\XLingPapersingledigitlistitemwidth+\parindent{}}\leftskip\XLingPapertempdim\relax
\interlinepenalty10000
\XLingPaperlistitem{\parindent{}}{\XLingPapersingledigitlistitemwidth}{1.}{If the source language grammatical category tag is{\LanglVernacularFontFamily{{\fontspec[Scale=0.9]{Courier New}\textup{\textbf{ {\LangtluAffixFontFamily{{\fontspec[Scale=0.9]{Courier New}\textcolor[rgb]{0,0.6901960784313725,0.3137254901960784}{ptcp\_f\_sg}}}}}}}}}, change the target language tag to {\LanglVernacularFontFamily{{\fontspec[Scale=0.9]{Courier New}\textup{\textbf{{\LangtluAffixFontFamily{{\fontspec[Scale=0.9]{Courier New}\textcolor[rgb]{0,0.6901960784313725,0.3137254901960784}{pst}}}}}}}}}\protect\footnote[9]{{\leftskip0pt\parindent1em\raisebox{\baselineskip}[0pt]{\protect\hypertarget{nPastSuf}{}}{\LanglVernacularFontFamily{{\fontspec[Scale=0.9]{Courier New}\textup{\textbf{ {\LangtluAffixFontFamily{{\fontspec[Scale=0.9]{Courier New}\textcolor[rgb]{0,0.6901960784313725,0.3137254901960784}{pst}}}}}}}}} corresponds to the past suffix in the English {\LangtToolFontFamily{\textbf{\textcolor[rgb]{0,0,0.5019607843137255}{FLEx}}}} project (and also the past tense feature).}}}}
\vspace{\baselineskip}
}
\vspace{12pt}
\XLingPaperneedspace{5\baselineskip}

\penalty-3000{\noindent{\raisebox{\baselineskip}[0pt]{\protect\hypertarget{sDecl}{}}\SectionLevelFourFontFamily{\normalsize{\raisebox{\baselineskip}[0pt]{\pdfbookmark[4]{6.2.3.2 Declarations}{sDecl}}\textit{6.2.3.2 Declarations}}}}
\markright{Declarations}
\XLingPaperaddtocontents{sDecl}}\par{}\penalty10000
\vspace{12pt}
{\parskip .5pt plus 1pt minus 1pt
                    
{\setlength{\XLingPapertempdim}{\XLingPapersingledigitlistitemwidth+\parindent{}}\leftskip\XLingPapertempdim\relax
\interlinepenalty10000
\XLingPaperlistitem{\parindent{}}{\XLingPapersingledigitlistitemwidth}{1.}{Output a subject pronoun which takes its person and number information from the auxiliary verb.}}
{\setlength{\XLingPapertempdim}{\XLingPapersingledigitlistitemwidth+\parindent{}}\leftskip\XLingPapertempdim\relax
\interlinepenalty10000
\XLingPaperlistitem{\parindent{}}{\XLingPapersingledigitlistitemwidth}{2.}{Output the main verb with information on category and tense (but not aspect).}}
{\setlength{\XLingPapertempdim}{\XLingPapersingledigitlistitemwidth+\parindent{}}\leftskip\XLingPapertempdim\relax
\interlinepenalty10000
\XLingPaperlistitem{\parindent{}}{\XLingPapersingledigitlistitemwidth}{3.}{Do not output the auxiliary verb {\textit{biti}}, “be”.}}
{\setlength{\XLingPapertempdim}{\XLingPapersingledigitlistitemwidth+\parindent{}}\leftskip\XLingPapertempdim\relax
\interlinepenalty10000
\XLingPaperlistitem{\parindent{}}{\XLingPapersingledigitlistitemwidth}{4.}{Output nouns with their grammatical category (but not number or case).}}
\vspace{\baselineskip}
}
\vspace{12pt}
\XLingPaperneedspace{5\baselineskip}

\penalty-3000{\noindent{\raisebox{\baselineskip}[0pt]{\protect\hypertarget{sWorkOrder}{}}\SectionLevelFourFontFamily{\normalsize{\raisebox{\baselineskip}[0pt]{\pdfbookmark[4]{6.2.3.3 Work Order}{sWorkOrder}}\textit{6.2.3.3 Work Order}}}}
\markright{Work Order}
\XLingPaperaddtocontents{sWorkOrder}}\par{}\penalty10000
\vspace{12pt}
\indent So, what order do we do these in? Well it doesn’t really matter. An experienced {\LangtToolFontFamily{\textbf{\textcolor[rgb]{0,0,0.5019607843137255}{FLExTrans}}}} linguist would probably do it in two steps, but for pedagogical purposes, we’re going to split it up into five steps:\par{}{\parskip .5pt plus 1pt minus 1pt

\vspace{\baselineskip}

{\setlength{\XLingPapertempdim}{\XLingPaperbulletlistitemwidth+\parindent{}}\leftskip\XLingPapertempdim\relax
\interlinepenalty10000
\XLingPaperlistitem{\parindent{}}{\XLingPaperbulletlistitemwidth}{•}{First, we’re going to write a rule which matches the participle and auxiliary construction and outputs only the main verb (declarations: 2 \& 3). }{\setlength{\XLingPaperlistitemindent}{\XLingPaperbulletlistitemwidth + \parindent{}}
{\setlength{\XLingPapertempdim}{\XLingPaperbulletlistitemwidth+\XLingPaperlistitemindent}\leftskip\XLingPapertempdim\relax
\interlinepenalty10000
\XLingPaperlistitem{\XLingPaperlistitemindent}{\XLingPaperbulletlistitemwidth}{•}{Define the categories of “ptcp\_f\_sg” and “biti”.}}}}
{\setlength{\XLingPapertempdim}{\XLingPaperbulletlistitemwidth+\parindent{}}\leftskip\XLingPapertempdim\relax
\interlinepenalty10000
\XLingPaperlistitem{\parindent{}}{\XLingPaperbulletlistitemwidth}{•}{Second, we’re going to edit that rule to change the source language tag from{\LanglVernacularFontFamily{{\fontspec[Scale=0.9]{Courier New}\textup{\textbf{ {\LangtluAffixFontFamily{{\fontspec[Scale=0.9]{Courier New}\textcolor[rgb]{0,0.6901960784313725,0.3137254901960784}{ptcp\_f\_sg}}}}}}}}} to{\LanglVernacularFontFamily{{\fontspec[Scale=0.9]{Courier New}\textup{\textbf{ {\LangtluAffixFontFamily{{\fontspec[Scale=0.9]{Courier New}\textcolor[rgb]{0,0.6901960784313725,0.3137254901960784}{pst}}}}}}}}} (procedure: 1).}{\setlength{\XLingPaperlistitemindent}{\XLingPaperbulletlistitemwidth + \parindent{}}
{\setlength{\XLingPapertempdim}{\XLingPaperbulletlistitemwidth+\XLingPaperlistitemindent}\leftskip\XLingPapertempdim\relax
\interlinepenalty10000
\XLingPaperlistitem{\XLingPaperlistitemindent}{\XLingPaperbulletlistitemwidth}{•}{Define the attribute of “tense”.}}}}
{\setlength{\XLingPapertempdim}{\XLingPaperbulletlistitemwidth+\parindent{}}\leftskip\XLingPapertempdim\relax
\interlinepenalty10000
\XLingPaperlistitem{\parindent{}}{\XLingPaperbulletlistitemwidth}{•}{Third, we’re going to edit the same rule to not output aspect (declaration: 2).}{\setlength{\XLingPaperlistitemindent}{\XLingPaperbulletlistitemwidth + \parindent{}}
{\setlength{\XLingPapertempdim}{\XLingPaperbulletlistitemwidth+\XLingPaperlistitemindent}\leftskip\XLingPapertempdim\relax
\interlinepenalty10000
\XLingPaperlistitem{\XLingPaperlistitemindent}{\XLingPaperbulletlistitemwidth}{•}{Define the attribute for grammatical category and add the tag for “verb”.}}}}
{\setlength{\XLingPapertempdim}{\XLingPaperbulletlistitemwidth+\parindent{}}\leftskip\XLingPapertempdim\relax
\interlinepenalty10000
\XLingPaperlistitem{\parindent{}}{\XLingPaperbulletlistitemwidth}{•}{Fourth, we’re going to edit the same rule to output a subject pronoun before the verb (declaration: 1).}{\setlength{\XLingPaperlistitemindent}{\XLingPaperbulletlistitemwidth + \parindent{}}
{\setlength{\XLingPapertempdim}{\XLingPaperbulletlistitemwidth+\XLingPaperlistitemindent}\leftskip\XLingPapertempdim\relax
\interlinepenalty10000
\XLingPaperlistitem{\XLingPaperlistitemindent}{\XLingPaperbulletlistitemwidth}{•}{Define the attributes of “person” and “number”.}}}}
{\setlength{\XLingPapertempdim}{\XLingPaperbulletlistitemwidth+\parindent{}}\leftskip\XLingPapertempdim\relax
\interlinepenalty10000
\XLingPaperlistitem{\parindent{}}{\XLingPaperbulletlistitemwidth}{•}{Fifth, we’re going to write a new rule which matches the noun construction and output only the grammatical category (declaration: 4) }{\setlength{\XLingPaperlistitemindent}{\XLingPaperbulletlistitemwidth + \parindent{}}
{\setlength{\XLingPapertempdim}{\XLingPaperbulletlistitemwidth+\XLingPaperlistitemindent}\leftskip\XLingPapertempdim\relax
\interlinepenalty10000
\XLingPaperlistitem{\XLingPaperlistitemindent}{\XLingPaperbulletlistitemwidth}{•}{Define the category of “noun”.}}
{\setlength{\XLingPapertempdim}{\XLingPaperbulletlistitemwidth+\XLingPaperlistitemindent}\leftskip\XLingPapertempdim\relax
\interlinepenalty10000
\XLingPaperlistitem{\XLingPaperlistitemindent}{\XLingPaperbulletlistitemwidth}{•}{Add the tag for nouns to the grammatical category attribute.}}}}
\vspace{\baselineskip}
}
\vspace{12pt}
\XLingPaperneedspace{5\baselineskip}

\penalty-3000{\noindent{\raisebox{\baselineskip}[0pt]{\protect\hypertarget{sCheatsheet}{}}\SectionLevelFourFontFamily{\normalsize{\raisebox{\baselineskip}[0pt]{\pdfbookmark[4]{6.2.3.4 Cheat Sheet}{sCheatsheet}}\textit{6.2.3.4 Cheat Sheet}}}}
\markright{Cheat Sheet}
\XLingPaperaddtocontents{sCheatsheet}}\par{}\penalty10000
\vspace{12pt}
\indent Here is what the input and output data stream of each of the above steps will look like:\par{}{\vspace{12pt plus 2pt minus 1pt}\raggedright{}\XLingPaperneedspace{8\baselineskip}\XLingPaperexample{.125in}{0pt}{2.75em}{\raisebox{\baselineskip}[0pt]{\protect\hypertarget{xCheatsheet}{}}(52)}{{
\XLingPaperminmaxcellincolumn{Step}{\XLingPapermincola}{\textbf{Step}}{\XLingPapermaxcola}{+0\tabcolsep}
\XLingPaperminmaxcellincolumn{Input}{\XLingPapermincolb}{\textbf{Input}}{\XLingPapermaxcolb}{+0\tabcolsep}
\XLingPaperminmaxcellincolumn{Output}{\XLingPapermincolc}{\textbf{Output}}{\XLingPapermaxcolc}{+0\tabcolsep}
\XLingPaperminmaxcellincolumn{1}{\XLingPapermincola}{1}{\XLingPapermaxcola}{+0\tabcolsep}
\XLingPaperminmaxcellincolumn{Otići1.1}{\XLingPapermincolb}{{\LanglVernacularFontFamily{{\fontspec[Scale=0.9]{Courier New}\textup{\textbf{Otići{{\fontspec[Scale=0.65]{Times New Roman}\textsubscript{1.1}}} {\LangtluGrammCatFontFamily{{\fontspec[Scale=0.9]{Courier New}\textcolor[rgb]{0,0.4392156862745098,0.7529411764705882}{v}}}} {\LangtluAffixFontFamily{{\fontspec[Scale=0.9]{Courier New}\textcolor[rgb]{0,0.6901960784313725,0.3137254901960784}{pfv}}}} {\LangtluAffixFontFamily{{\fontspec[Scale=0.9]{Courier New}\textcolor[rgb]{0,0.6901960784313725,0.3137254901960784}{ptcp\_f\_sg}}}} biti{{\fontspec[Scale=0.65]{Times New Roman}\textsubscript{1.1}}} {\LangtluGrammCatFontFamily{{\fontspec[Scale=0.9]{Courier New}\textcolor[rgb]{0,0.4392156862745098,0.7529411764705882}{cop}}}} {\LangtluAffixFontFamily{{\fontspec[Scale=0.9]{Courier New}\textcolor[rgb]{0,0.6901960784313725,0.3137254901960784}{prs sg 2}}}}}}}}}}{\XLingPapermaxcolb}{+0\tabcolsep}
\XLingPaperminmaxcellincolumn{ptcp\_f\_sg}{\XLingPapermincolc}{{\LanglVernacularFontFamily{{\fontspec[Scale=0.9]{Courier New}\textup{\textbf{Leave{{\fontspec[Scale=0.65]{Times New Roman}\textsubscript{1.1}}} {\LangtluGrammCatFontFamily{{\fontspec[Scale=0.9]{Courier New}\textcolor[rgb]{0,0.4392156862745098,0.7529411764705882}{v}}}} {\LangtluAffixFontFamily{{\fontspec[Scale=0.9]{Courier New}\textcolor[rgb]{0,0.6901960784313725,0.3137254901960784}{pfv}}}} {\LangtluAffixFontFamily{{\fontspec[Scale=0.9]{Courier New}\textcolor[rgb]{0,0.6901960784313725,0.3137254901960784}{ptcp\_f\_sg}}}}}}}}}}{\XLingPapermaxcolc}{+0\tabcolsep}
\XLingPaperminmaxcellincolumn{2}{\XLingPapermincola}{2}{\XLingPapermaxcola}{+0\tabcolsep}
\XLingPaperminmaxcellincolumn{Otići1.1}{\XLingPapermincolb}{{\LanglVernacularFontFamily{{\fontspec[Scale=0.9]{Courier New}\textup{\textbf{Otići{{\fontspec[Scale=0.65]{Times New Roman}\textsubscript{1.1}}} {\LangtluGrammCatFontFamily{{\fontspec[Scale=0.9]{Courier New}\textcolor[rgb]{0,0.4392156862745098,0.7529411764705882}{v}}}} {\LangtluAffixFontFamily{{\fontspec[Scale=0.9]{Courier New}\textcolor[rgb]{0,0.6901960784313725,0.3137254901960784}{pfv}}}} {\LangtluAffixFontFamily{{\fontspec[Scale=0.9]{Courier New}\textcolor[rgb]{0,0.6901960784313725,0.3137254901960784}{ptcp\_f\_sg}}}} biti{{\fontspec[Scale=0.65]{Times New Roman}\textsubscript{1.1}}} {\LangtluGrammCatFontFamily{{\fontspec[Scale=0.9]{Courier New}\textcolor[rgb]{0,0.4392156862745098,0.7529411764705882}{cop}}}} {\LangtluAffixFontFamily{{\fontspec[Scale=0.9]{Courier New}\textcolor[rgb]{0,0.6901960784313725,0.3137254901960784}{prs sg 2}}}}}}}}}}{\XLingPapermaxcolb}{+0\tabcolsep}
\XLingPaperminmaxcellincolumn{Leave1.1}{\XLingPapermincolc}{{\LanglVernacularFontFamily{{\fontspec[Scale=0.9]{Courier New}\textup{\textbf{Leave{{\fontspec[Scale=0.65]{Times New Roman}\textsubscript{1.1}}} {\LangtluGrammCatFontFamily{{\fontspec[Scale=0.9]{Courier New}\textcolor[rgb]{0,0.4392156862745098,0.7529411764705882}{v}}}} {\LangtluAffixFontFamily{{\fontspec[Scale=0.9]{Courier New}\textcolor[rgb]{0,0.6901960784313725,0.3137254901960784}{pfv}}}} {\LangtluAffixFontFamily{{\fontspec[Scale=0.9]{Courier New}\textcolor[rgb]{0,0.6901960784313725,0.3137254901960784}{pst}}}} }}}}}}{\XLingPapermaxcolc}{+0\tabcolsep}
\XLingPaperminmaxcellincolumn{3}{\XLingPapermincola}{3}{\XLingPapermaxcola}{+0\tabcolsep}
\XLingPaperminmaxcellincolumn{Otići1.1}{\XLingPapermincolb}{{\LanglVernacularFontFamily{{\fontspec[Scale=0.9]{Courier New}\textup{\textbf{Otići{{\fontspec[Scale=0.65]{Times New Roman}\textsubscript{1.1}}} {\LangtluGrammCatFontFamily{{\fontspec[Scale=0.9]{Courier New}\textcolor[rgb]{0,0.4392156862745098,0.7529411764705882}{v}}}} {\LangtluAffixFontFamily{{\fontspec[Scale=0.9]{Courier New}\textcolor[rgb]{0,0.6901960784313725,0.3137254901960784}{pfv}}}} {\LangtluAffixFontFamily{{\fontspec[Scale=0.9]{Courier New}\textcolor[rgb]{0,0.6901960784313725,0.3137254901960784}{ptcp\_f\_sg}}}} biti{{\fontspec[Scale=0.65]{Times New Roman}\textsubscript{1.1}}} {\LangtluGrammCatFontFamily{{\fontspec[Scale=0.9]{Courier New}\textcolor[rgb]{0,0.4392156862745098,0.7529411764705882}{cop}}}} {\LangtluAffixFontFamily{{\fontspec[Scale=0.9]{Courier New}\textcolor[rgb]{0,0.6901960784313725,0.3137254901960784}{prs sg 2}}}}}}}}}}{\XLingPapermaxcolb}{+0\tabcolsep}
\XLingPaperminmaxcellincolumn{Leave1.1}{\XLingPapermincolc}{{\LanglVernacularFontFamily{{\fontspec[Scale=0.9]{Courier New}\textup{\textbf{Leave{{\fontspec[Scale=0.65]{Times New Roman}\textsubscript{1.1}}} {\LangtluGrammCatFontFamily{{\fontspec[Scale=0.9]{Courier New}\textcolor[rgb]{0,0.4392156862745098,0.7529411764705882}{v}}}} {\LangtluAffixFontFamily{{\fontspec[Scale=0.9]{Courier New}\textcolor[rgb]{0,0.6901960784313725,0.3137254901960784}{pst}}}} }}}}}}{\XLingPapermaxcolc}{+0\tabcolsep}
\XLingPaperminmaxcellincolumn{4}{\XLingPapermincola}{4}{\XLingPapermaxcola}{+0\tabcolsep}
\XLingPaperminmaxcellincolumn{Otići1.1}{\XLingPapermincolb}{{\LanglVernacularFontFamily{{\fontspec[Scale=0.9]{Courier New}\textup{\textbf{Otići{{\fontspec[Scale=0.65]{Times New Roman}\textsubscript{1.1}}} {\LangtluGrammCatFontFamily{{\fontspec[Scale=0.9]{Courier New}\textcolor[rgb]{0,0.4392156862745098,0.7529411764705882}{v}}}} {\LangtluAffixFontFamily{{\fontspec[Scale=0.9]{Courier New}\textcolor[rgb]{0,0.6901960784313725,0.3137254901960784}{pfv}}}} {\LangtluAffixFontFamily{{\fontspec[Scale=0.9]{Courier New}\textcolor[rgb]{0,0.6901960784313725,0.3137254901960784}{ptcp\_f\_sg}}}} biti{{\fontspec[Scale=0.65]{Times New Roman}\textsubscript{1.1}}} {\LangtluGrammCatFontFamily{{\fontspec[Scale=0.9]{Courier New}\textcolor[rgb]{0,0.4392156862745098,0.7529411764705882}{cop}}}} {\LangtluAffixFontFamily{{\fontspec[Scale=0.9]{Courier New}\textcolor[rgb]{0,0.6901960784313725,0.3137254901960784}{prs sg 2}}}}}}}}}}{\XLingPapermaxcolb}{+0\tabcolsep}
\XLingPaperminmaxcellincolumn{leave1.1}{\XLingPapermincolc}{{\LanglVernacularFontFamily{{\fontspec[Scale=0.9]{Courier New}\textup{\textbf{Propers{{\fontspec[Scale=0.65]{Times New Roman}\textsubscript{1.1}}} {\LangtluGrammCatFontFamily{{\fontspec[Scale=0.9]{Courier New}\textcolor[rgb]{0,0.4392156862745098,0.7529411764705882}{pro}}}} {\LangtluAffixFontFamily{{\fontspec[Scale=0.9]{Courier New}\textcolor[rgb]{0,0.6901960784313725,0.3137254901960784}{nom}}}} {\LangtluAffixFontFamily{{\fontspec[Scale=0.9]{Courier New}\textcolor[rgb]{0,0.6901960784313725,0.3137254901960784}{2}}}} {\LangtluAffixFontFamily{{\fontspec[Scale=0.9]{Courier New}\textcolor[rgb]{0,0.6901960784313725,0.3137254901960784}{sg}}}} leave{{\fontspec[Scale=0.65]{Times New Roman}\textsubscript{1.1}}} {\LangtluGrammCatFontFamily{{\fontspec[Scale=0.9]{Courier New}\textcolor[rgb]{0,0.4392156862745098,0.7529411764705882}{v}}}} {\LangtluAffixFontFamily{{\fontspec[Scale=0.9]{Courier New}\textcolor[rgb]{0,0.6901960784313725,0.3137254901960784}{pst}}}} }}}}}}{\XLingPapermaxcolc}{+0\tabcolsep}
\XLingPaperminmaxcellincolumn{5}{\XLingPapermincola}{5}{\XLingPapermaxcola}{+0\tabcolsep}
\XLingPaperminmaxcellincolumn{gen\_sg}{\XLingPapermincolb}{{\LanglVernacularFontFamily{{\fontspec[Scale=0.9]{Courier New}\textup{\textbf{pozdrav{{\fontspec[Scale=0.65]{Times New Roman}\textsubscript{1.1}}} {\LangtluGrammCatFontFamily{{\fontspec[Scale=0.9]{Courier New}\textcolor[rgb]{0,0.4392156862745098,0.7529411764705882}{n}}}} {\LangtluAffixFontFamily{{\fontspec[Scale=0.9]{Courier New}\textcolor[rgb]{0,0.6901960784313725,0.3137254901960784}{gen\_sg}}}}}}}}}}{\XLingPapermaxcolb}{+0\tabcolsep}
\XLingPaperminmaxcellincolumn{word1.1}{\XLingPapermincolc}{{\LanglVernacularFontFamily{{\fontspec[Scale=0.9]{Courier New}\textup{\textbf{word{{\fontspec[Scale=0.65]{Times New Roman}\textsubscript{1.1}}} {\LangtluGrammCatFontFamily{{\fontspec[Scale=0.9]{Courier New}\textcolor[rgb]{0,0.4392156862745098,0.7529411764705882}{n}}}}}}}}}}{\XLingPapermaxcolc}{+0\tabcolsep}
\setlength{\XLingPaperavailabletablewidth}{433.62pt - .125in - 0pt - 2.75em}
\setlength{\XLingPapertableminwidth}{\XLingPapermincola+\XLingPapermincolb+\XLingPapermincolc}
\setlength{\XLingPapertablemaxwidth}{\XLingPapermaxcola+\XLingPapermaxcolb+\XLingPapermaxcolc}
\XLingPapercalculatetablewidthratio{}
\XLingPapersetcolumnwidth{\XLingPapercolawidth}{\XLingPapermincola}{\XLingPapermaxcola}{-0\tabcolsep}
\XLingPapersetcolumnwidth{\XLingPapercolbwidth}{\XLingPapermincolb}{\XLingPapermaxcolb}{-2\tabcolsep}
\XLingPapersetcolumnwidth{\XLingPapercolcwidth}{\XLingPapermincolc}{\XLingPapermaxcolc}{-2\tabcolsep}\setlength{\LTpre}{-.5\baselineskip}\setlength{\LTleft}{.125in + 2.75em}\setlength{\LTpost}{0pt}
\begin{longtable}
[t]{@{}>{\raggedright}p{\XLingPapercolawidth}>{\raggedright}p{\XLingPapercolbwidth}>{\raggedright}p{\XLingPapercolcwidth}@{}}\specialrule{\heavyrulewidth}{-4\aboverulesep}{\belowrulesep}\multicolumn{1}{@{}>{\raggedright}p{\XLingPapercolawidth}}{\textbf{Step}}&\multicolumn{1}{>{\raggedright}p{\XLingPapercolbwidth}}{\textbf{Input}}&\multicolumn{1}{>{\raggedright}p{\XLingPapercolcwidth}@{}}{\textbf{Output}}\\%
\midrule\multicolumn{1}{@{}>{\raggedright}p{\XLingPapercolawidth}}{1}&\multicolumn{1}{>{\raggedright}p{\XLingPapercolbwidth}}{{\LanglVernacularFontFamily{{\fontspec[Scale=0.9]{Courier New}\textup{\textbf{Otići{{\fontspec[Scale=0.65]{Times New Roman}\textsubscript{1.1}}} {\LangtluGrammCatFontFamily{{\fontspec[Scale=0.9]{Courier New}\textcolor[rgb]{0,0.4392156862745098,0.7529411764705882}{v}}}} {\LangtluAffixFontFamily{{\fontspec[Scale=0.9]{Courier New}\textcolor[rgb]{0,0.6901960784313725,0.3137254901960784}{pfv}}}} {\LangtluAffixFontFamily{{\fontspec[Scale=0.9]{Courier New}\textcolor[rgb]{0,0.6901960784313725,0.3137254901960784}{ptcp\_f\_sg}}}} biti{{\fontspec[Scale=0.65]{Times New Roman}\textsubscript{1.1}}} {\LangtluGrammCatFontFamily{{\fontspec[Scale=0.9]{Courier New}\textcolor[rgb]{0,0.4392156862745098,0.7529411764705882}{cop}}}} {\LangtluAffixFontFamily{{\fontspec[Scale=0.9]{Courier New}\textcolor[rgb]{0,0.6901960784313725,0.3137254901960784}{prs sg 2}}}}}}}}}}&\multicolumn{1}{>{\raggedright}p{\XLingPapercolcwidth}@{}}{{\LanglVernacularFontFamily{{\fontspec[Scale=0.9]{Courier New}\textup{\textbf{Leave{{\fontspec[Scale=0.65]{Times New Roman}\textsubscript{1.1}}} {\LangtluGrammCatFontFamily{{\fontspec[Scale=0.9]{Courier New}\textcolor[rgb]{0,0.4392156862745098,0.7529411764705882}{v}}}} {\LangtluAffixFontFamily{{\fontspec[Scale=0.9]{Courier New}\textcolor[rgb]{0,0.6901960784313725,0.3137254901960784}{pfv}}}} {\LangtluAffixFontFamily{{\fontspec[Scale=0.9]{Courier New}\textcolor[rgb]{0,0.6901960784313725,0.3137254901960784}{ptcp\_f\_sg}}}}}}}}}}\\%
\multicolumn{1}{@{}>{\raggedright}p{\XLingPapercolawidth}}{2}&\multicolumn{1}{>{\raggedright}p{\XLingPapercolbwidth}}{{\LanglVernacularFontFamily{{\fontspec[Scale=0.9]{Courier New}\textup{\textbf{Otići{{\fontspec[Scale=0.65]{Times New Roman}\textsubscript{1.1}}} {\LangtluGrammCatFontFamily{{\fontspec[Scale=0.9]{Courier New}\textcolor[rgb]{0,0.4392156862745098,0.7529411764705882}{v}}}} {\LangtluAffixFontFamily{{\fontspec[Scale=0.9]{Courier New}\textcolor[rgb]{0,0.6901960784313725,0.3137254901960784}{pfv}}}} {\LangtluAffixFontFamily{{\fontspec[Scale=0.9]{Courier New}\textcolor[rgb]{0,0.6901960784313725,0.3137254901960784}{ptcp\_f\_sg}}}} biti{{\fontspec[Scale=0.65]{Times New Roman}\textsubscript{1.1}}} {\LangtluGrammCatFontFamily{{\fontspec[Scale=0.9]{Courier New}\textcolor[rgb]{0,0.4392156862745098,0.7529411764705882}{cop}}}} {\LangtluAffixFontFamily{{\fontspec[Scale=0.9]{Courier New}\textcolor[rgb]{0,0.6901960784313725,0.3137254901960784}{prs sg 2}}}}}}}}}}&\multicolumn{1}{>{\raggedright}p{\XLingPapercolcwidth}@{}}{{\LanglVernacularFontFamily{{\fontspec[Scale=0.9]{Courier New}\textup{\textbf{Leave{{\fontspec[Scale=0.65]{Times New Roman}\textsubscript{1.1}}} {\LangtluGrammCatFontFamily{{\fontspec[Scale=0.9]{Courier New}\textcolor[rgb]{0,0.4392156862745098,0.7529411764705882}{v}}}} {\LangtluAffixFontFamily{{\fontspec[Scale=0.9]{Courier New}\textcolor[rgb]{0,0.6901960784313725,0.3137254901960784}{pfv}}}} {\LangtluAffixFontFamily{{\fontspec[Scale=0.9]{Courier New}\textcolor[rgb]{0,0.6901960784313725,0.3137254901960784}{pst}}}} }}}}}}\\%
\multicolumn{1}{@{}>{\raggedright}p{\XLingPapercolawidth}}{3}&\multicolumn{1}{>{\raggedright}p{\XLingPapercolbwidth}}{{\LanglVernacularFontFamily{{\fontspec[Scale=0.9]{Courier New}\textup{\textbf{Otići{{\fontspec[Scale=0.65]{Times New Roman}\textsubscript{1.1}}} {\LangtluGrammCatFontFamily{{\fontspec[Scale=0.9]{Courier New}\textcolor[rgb]{0,0.4392156862745098,0.7529411764705882}{v}}}} {\LangtluAffixFontFamily{{\fontspec[Scale=0.9]{Courier New}\textcolor[rgb]{0,0.6901960784313725,0.3137254901960784}{pfv}}}} {\LangtluAffixFontFamily{{\fontspec[Scale=0.9]{Courier New}\textcolor[rgb]{0,0.6901960784313725,0.3137254901960784}{ptcp\_f\_sg}}}} biti{{\fontspec[Scale=0.65]{Times New Roman}\textsubscript{1.1}}} {\LangtluGrammCatFontFamily{{\fontspec[Scale=0.9]{Courier New}\textcolor[rgb]{0,0.4392156862745098,0.7529411764705882}{cop}}}} {\LangtluAffixFontFamily{{\fontspec[Scale=0.9]{Courier New}\textcolor[rgb]{0,0.6901960784313725,0.3137254901960784}{prs sg 2}}}}}}}}}}&\multicolumn{1}{>{\raggedright}p{\XLingPapercolcwidth}@{}}{{\LanglVernacularFontFamily{{\fontspec[Scale=0.9]{Courier New}\textup{\textbf{Leave{{\fontspec[Scale=0.65]{Times New Roman}\textsubscript{1.1}}} {\LangtluGrammCatFontFamily{{\fontspec[Scale=0.9]{Courier New}\textcolor[rgb]{0,0.4392156862745098,0.7529411764705882}{v}}}} {\LangtluAffixFontFamily{{\fontspec[Scale=0.9]{Courier New}\textcolor[rgb]{0,0.6901960784313725,0.3137254901960784}{pst}}}} }}}}}}\\%
\multicolumn{1}{@{}>{\raggedright}p{\XLingPapercolawidth}}{4}&\multicolumn{1}{>{\raggedright}p{\XLingPapercolbwidth}}{{\LanglVernacularFontFamily{{\fontspec[Scale=0.9]{Courier New}\textup{\textbf{Otići{{\fontspec[Scale=0.65]{Times New Roman}\textsubscript{1.1}}} {\LangtluGrammCatFontFamily{{\fontspec[Scale=0.9]{Courier New}\textcolor[rgb]{0,0.4392156862745098,0.7529411764705882}{v}}}} {\LangtluAffixFontFamily{{\fontspec[Scale=0.9]{Courier New}\textcolor[rgb]{0,0.6901960784313725,0.3137254901960784}{pfv}}}} {\LangtluAffixFontFamily{{\fontspec[Scale=0.9]{Courier New}\textcolor[rgb]{0,0.6901960784313725,0.3137254901960784}{ptcp\_f\_sg}}}} biti{{\fontspec[Scale=0.65]{Times New Roman}\textsubscript{1.1}}} {\LangtluGrammCatFontFamily{{\fontspec[Scale=0.9]{Courier New}\textcolor[rgb]{0,0.4392156862745098,0.7529411764705882}{cop}}}} {\LangtluAffixFontFamily{{\fontspec[Scale=0.9]{Courier New}\textcolor[rgb]{0,0.6901960784313725,0.3137254901960784}{prs sg 2}}}}}}}}}}&\multicolumn{1}{>{\raggedright}p{\XLingPapercolcwidth}@{}}{{\LanglVernacularFontFamily{{\fontspec[Scale=0.9]{Courier New}\textup{\textbf{Propers{{\fontspec[Scale=0.65]{Times New Roman}\textsubscript{1.1}}} {\LangtluGrammCatFontFamily{{\fontspec[Scale=0.9]{Courier New}\textcolor[rgb]{0,0.4392156862745098,0.7529411764705882}{pro}}}} {\LangtluAffixFontFamily{{\fontspec[Scale=0.9]{Courier New}\textcolor[rgb]{0,0.6901960784313725,0.3137254901960784}{nom}}}} {\LangtluAffixFontFamily{{\fontspec[Scale=0.9]{Courier New}\textcolor[rgb]{0,0.6901960784313725,0.3137254901960784}{2}}}} {\LangtluAffixFontFamily{{\fontspec[Scale=0.9]{Courier New}\textcolor[rgb]{0,0.6901960784313725,0.3137254901960784}{sg}}}} leave{{\fontspec[Scale=0.65]{Times New Roman}\textsubscript{1.1}}} {\LangtluGrammCatFontFamily{{\fontspec[Scale=0.9]{Courier New}\textcolor[rgb]{0,0.4392156862745098,0.7529411764705882}{v}}}} {\LangtluAffixFontFamily{{\fontspec[Scale=0.9]{Courier New}\textcolor[rgb]{0,0.6901960784313725,0.3137254901960784}{pst}}}} }}}}}}\\%
\multicolumn{1}{@{}>{\raggedright}p{\XLingPapercolawidth}}{5}&\multicolumn{1}{>{\raggedright}p{\XLingPapercolbwidth}}{{\LanglVernacularFontFamily{{\fontspec[Scale=0.9]{Courier New}\textup{\textbf{pozdrav{{\fontspec[Scale=0.65]{Times New Roman}\textsubscript{1.1}}} {\LangtluGrammCatFontFamily{{\fontspec[Scale=0.9]{Courier New}\textcolor[rgb]{0,0.4392156862745098,0.7529411764705882}{n}}}} {\LangtluAffixFontFamily{{\fontspec[Scale=0.9]{Courier New}\textcolor[rgb]{0,0.6901960784313725,0.3137254901960784}{gen\_sg}}}}}}}}}}&\multicolumn{1}{>{\raggedright}p{\XLingPapercolcwidth}@{}}{{\LanglVernacularFontFamily{{\fontspec[Scale=0.9]{Courier New}\textup{\textbf{word{{\fontspec[Scale=0.65]{Times New Roman}\textsubscript{1.1}}} {\LangtluGrammCatFontFamily{{\fontspec[Scale=0.9]{Courier New}\textcolor[rgb]{0,0.4392156862745098,0.7529411764705882}{n}}}}}}}}}}\\\bottomrule%
\end{longtable}
}}
}
\vspace{12pt}
\XLingPaperneedspace{5\baselineskip}

\penalty-3000{\noindent{\raisebox{\baselineskip}[0pt]{\protect\hypertarget{sImplementation}{}}\SectionLevelThreeFontFamily{\normalsize{\raisebox{\baselineskip}[0pt]{\pdfbookmark[3]{6.2.4 Implementation}{sImplementation}}\textit{6.2.4 Implementation}}}}
\markright{Implementation}
\XLingPaperaddtocontents{sImplementation}}\par{}\penalty10000
\vspace{12pt}
\indent Let’s see what data stream output we get for the first two words right now. To do this we will use the {\LangtToolFontFamily{\textbf{\textcolor[rgb]{0,0,0.5019607843137255}{Live Rule Tester Tool}}}} which is a great tool for applying a single rule to one or more words and checking the result. See section \hyperlink{sRuleTester}{4.2}.\par{}{\parskip .5pt plus 1pt minus 1pt
                    
\vspace{\baselineskip}

{\setlength{\XLingPapertempdim}{\XLingPapersingledigitlistitemwidth+\parindent{}}\leftskip\XLingPapertempdim\relax
\interlinepenalty10000
\XLingPaperlistitem{\parindent{}}{\XLingPapersingledigitlistitemwidth}{1.}{Close the {\textbf{English-FLExTrans-Sample}} {\LangtToolFontFamily{\textbf{\textcolor[rgb]{0,0,0.5019607843137255}{FLEx}}}} project.}}
{\setlength{\XLingPapertempdim}{\XLingPapersingledigitlistitemwidth+\parindent{}}\leftskip\XLingPapertempdim\relax
\interlinepenalty10000
\XLingPaperlistitem{\parindent{}}{\XLingPapersingledigitlistitemwidth}{2.}{Bring up the {\LangtToolFontFamily{\textbf{\textcolor[rgb]{0,0,0.5019607843137255}{FLExTools}}}} app.}}
{\setlength{\XLingPapertempdim}{\XLingPapersingledigitlistitemwidth+\parindent{}}\leftskip\XLingPapertempdim\relax
\interlinepenalty10000
\XLingPaperlistitem{\parindent{}}{\XLingPapersingledigitlistitemwidth}{3.}{Click the {\LangtCollectionFontFamily{{\fontspec[Scale=0.8]{Arial}\textup{\textbf{\textcolor[rgb]{0.4,0,0.4}{Tools}}}}}} collection.}}
{\setlength{\XLingPapertempdim}{\XLingPapersingledigitlistitemwidth+\parindent{}}\leftskip\XLingPapertempdim\relax
\interlinepenalty10000
\XLingPaperlistitem{\parindent{}}{\XLingPapersingledigitlistitemwidth}{4.}{Select the {\LangtModuleFontFamily{\textbf{\textcolor[rgb]{0.4,0,0.4}{Live Rule Tester Tool}}}} module.}}
{\setlength{\XLingPapertempdim}{\XLingPapersingledigitlistitemwidth+\parindent{}}\leftskip\XLingPapertempdim\relax
\interlinepenalty10000
\XLingPaperlistitem{\parindent{}}{\XLingPapersingledigitlistitemwidth}{5.}{Click on the {\textup{\textbf{Run}}} button.}}
{\setlength{\XLingPapertempdim}{\XLingPapersingledigitlistitemwidth+\parindent{}}\leftskip\XLingPapertempdim\relax
\interlinepenalty10000
\XLingPaperlistitem{\parindent{}}{\XLingPapersingledigitlistitemwidth}{6.}{Check the first two Croatian words.}}
{\setlength{\XLingPapertempdim}{\XLingPapersingledigitlistitemwidth+\parindent{}}\leftskip\XLingPapertempdim\relax
\interlinepenalty10000
\XLingPaperlistitem{\parindent{}}{\XLingPapersingledigitlistitemwidth}{7.}{Click on the {\textup{\textbf{Transfer}}} button.}}
\vspace{\baselineskip}
}\indent Your {\LangtToolFontFamily{\textbf{\textcolor[rgb]{0,0,0.5019607843137255}{Live Rule Tester Tool}}}} should look like this:\par{}{\vspace{12pt plus 2pt minus 1pt}\raggedright{}\XLingPaperexample{.125in}{0pt}{2.75em}{\raisebox{\baselineskip}[0pt]{\protect\hypertarget{xTesterStep0}{}}(53)}{\parbox[t]{\textwidth - .125in - 0pt}{\vspace*{-\baselineskip}{\XeTeXpicfile "../Images/RulesTutTesterStep0.PNG" scaled 600}}}
\vspace{12pt plus 2pt minus 1pt}}\par\indent In the blue box, in the lower section of the Source Text area, we have the input data stream for the first two (checked) Croatian words:\par{}{\vspace{12pt plus 2pt minus 1pt}\raggedright{}\XLingPaperexample{.125in}{0pt}{2.75em}{\raisebox{\baselineskip}[0pt]{\protect\hypertarget{xStep0Source}{}}(54)}{{\LanglVernacularFontFamily{{\fontspec[Scale=0.9]{Courier New}\textup{\textbf{Otići{{\fontspec[Scale=0.65]{Times New Roman}\textsubscript{1.1}}} {\LangtluGrammCatFontFamily{{\fontspec[Scale=0.9]{Courier New}\textcolor[rgb]{0,0.4392156862745098,0.7529411764705882}{v}}}} {\LangtluAffixFontFamily{{\fontspec[Scale=0.9]{Courier New}\textcolor[rgb]{0,0.6901960784313725,0.3137254901960784}{pfv ptcp\_f\_sg}}}} biti{{\fontspec[Scale=0.65]{Times New Roman}\textsubscript{1.1}}} {\LangtluGrammCatFontFamily{{\fontspec[Scale=0.9]{Courier New}\textcolor[rgb]{0,0.4392156862745098,0.7529411764705882}{cop}}}} {\LangtluAffixFontFamily{{\fontspec[Scale=0.9]{Courier New}\textcolor[rgb]{0,0.6901960784313725,0.3137254901960784}{prs sg 2}}}}}}}}}\ }
\vspace{12pt plus 2pt minus 1pt}}\par\indent This matches what we have in the cheat sheet in \hyperlink{xCheatsheet}{(52)}.\par{}\indent In the green box, in the Target Text area, we have the output data stream after the sample rule is applied.\par{}{\vspace{12pt plus 2pt minus 1pt}\raggedright{}\XLingPaperexample{.125in}{0pt}{2.75em}{\raisebox{\baselineskip}[0pt]{\protect\hypertarget{xStep0Target}{}}(55)}{{\LanglVernacularFontFamily{{\fontspec[Scale=0.9]{Courier New}\textup{\textbf{Leave{{\fontspec[Scale=0.65]{Times New Roman}\textsubscript{1.1}}} {\LangtluGrammCatFontFamily{{\fontspec[Scale=0.9]{Courier New}\textcolor[rgb]{0,0.4392156862745098,0.7529411764705882}{v}}}} {\LangtluAffixFontFamily{{\fontspec[Scale=0.9]{Courier New}\textcolor[rgb]{0,0.6901960784313725,0.3137254901960784}{pfv ptcp\_f\_sg}}}} be{{\fontspec[Scale=0.65]{Times New Roman}\textsubscript{1.1}}} {\LangtluGrammCatFontFamily{{\fontspec[Scale=0.9]{Courier New}\textcolor[rgb]{0,0.4392156862745098,0.7529411764705882}{cop}}}} {\LangtluAffixFontFamily{{\fontspec[Scale=0.9]{Courier New}\textcolor[rgb]{0,0.6901960784313725,0.3137254901960784}{prs sg 2}}}}}}}}}\ }
\vspace{12pt plus 2pt minus 1pt}}\par\indent Since we haven’t done anything, it doesn’t yet match our output goal for step one in the cheat sheet.\par{}
\vspace{12pt}
\XLingPaperneedspace{5\baselineskip}

\penalty-3000{\noindent{\raisebox{\baselineskip}[0pt]{\protect\hypertarget{step1}{}}\SectionLevelFourFontFamily{\normalsize{\raisebox{\baselineskip}[0pt]{\pdfbookmark[4]{6.2.4.1 Step 1}{step1}}\textit{6.2.4.1 Step 1}}}}
\markright{Step 1}
\XLingPaperaddtocontents{step1}}\par{}\penalty10000
\vspace{12pt}
{\parskip .5pt plus 1pt minus 1pt
                    
{\setlength{\XLingPapertempdim}{\XLingPapersingledigitlistitemwidth+\parindent{}}\leftskip\XLingPapertempdim\relax
\interlinepenalty10000
\XLingPaperlistitem{\parindent{}}{\XLingPapersingledigitlistitemwidth}{1.}{\setlength{\parindent}{1em}\indent Define the categories we need in {\textit{transfer\_rules.t1x}}. Replace the dummy {\LangtRuleElemInXXEFontFamily{{\fontspec[Scale=0.8]{Arial}\textcolor[rgb]{0,0.4,0.2}{\textbf{Categories}}}}} element with:}\par{}{\vspace{12pt plus 2pt minus 1pt}\raggedright{}\XLingPaperexample{.125in}{0pt}{2.75em}{\raisebox{\baselineskip}[0pt]{\protect\hypertarget{xCat1}{}}(56)}{\parbox[t]{\textwidth - .125in - 0pt}{\vspace*{-\baselineskip}{\XeTeXpicfile "../Images/RulesTutCat1.PNG" scaled 750}}}
\vspace{12pt plus 2pt minus 1pt}}{\setlength{\parindent}{1em}\par\indent You may be wondering how you get a new {\LangtRuleElemInXXEFontFamily{{\fontspec[Scale=0.8]{Arial}\textcolor[rgb]{0,0.4,0.2}{\textbf{category}}}}} element inserted. It is straightforward:\par{}}{\setlength{\XLingPaperlistitemindent}{\XLingPapersingledigitlistitemwidth + \parindent{}}
{\setlength{\XLingPapertempdim}{\XLingPapersingleletterlistitemwidth+\XLingPaperlistitemindent}\leftskip\XLingPapertempdim\relax
\interlinepenalty10000
\XLingPaperlistitem{\XLingPaperlistitemindent}{\XLingPapersingleletterlistitemwidth}{a.}{Click on the word “category”. The whole element becomes outlined in a red rectangle.}}
{\setlength{\XLingPapertempdim}{\XLingPapersingleletterlistitemwidth+\XLingPaperlistitemindent}\leftskip\XLingPapertempdim\relax
\interlinepenalty10000
\XLingPaperlistitem{\XLingPaperlistitemindent}{\XLingPapersingleletterlistitemwidth}{b.}{Now either right-click and choose Insert After... or click on the \vspace*{0pt}{\XeTeXpicfile "../Images/ButInsertAfter.PNG" scaled 750} button (upper right side) or press Ctrl-J.}}
{\setlength{\XLingPapertempdim}{\XLingPapersingleletterlistitemwidth+\XLingPaperlistitemindent}\leftskip\XLingPapertempdim\relax
\interlinepenalty10000
\XLingPaperlistitem{\XLingPaperlistitemindent}{\XLingPapersingleletterlistitemwidth}{c.}{At this point a list of elements to insert shows up in the upper right pane as shown in \hyperlink{xInsertListCat}{(57)}.}\par{}{\vspace{12pt plus 2pt minus 1pt}\raggedright{}\XLingPaperexample{.125in}{0pt}{2.75em}{\raisebox{\baselineskip}[0pt]{\protect\hypertarget{xInsertListCat}{}}(57)}{\parbox[t]{\textwidth - .125in - 0pt}{\vspace*{-\baselineskip}{\XeTeXpicfile "../Images/InsertListCat.PNG" scaled 750}}}
\vspace{12pt plus 2pt minus 1pt}}{}}
{\setlength{\XLingPapertempdim}{\XLingPapersingleletterlistitemwidth+\XLingPaperlistitemindent}\leftskip\XLingPapertempdim\relax
\interlinepenalty10000
\XLingPaperlistitem{\XLingPaperlistitemindent}{\XLingPapersingleletterlistitemwidth}{d.}{Choose in this case {\textit{def-cat(category)}} since we only need one {\LangtRuleElemInXXEFontFamily{{\fontspec[Scale=0.8]{Arial}\textcolor[rgb]{0,0.4,0.2}{\textbf{tags}}}}} element. (By the way, this insert procedure will present only valid elements that can be inserted at this point in the rules file.)}}\indent Why do we need {\LanglVernacularFontFamily{{\fontspec[Scale=0.9]{Courier New}\textup{\textbf{{\textbf{.*}}}}}}} in the {\LangtRuleElemInXXEFontFamily{{\fontspec[Scale=0.8]{Arial}\textcolor[rgb]{0,0.4,0.2}{\textbf{tags}}}}} elements? This is because of how the matching system for categories works. In the middle of tag sequences, a {\LanglVernacularFontFamily{{\fontspec[Scale=0.9]{Courier New}\textup{\textbf{{\textbf{*}}}}}}} is counted as a single tag. At the end, it is counted as any sequence of tags. So, what we have is {\LanglVernacularFontFamily{{\fontspec[Scale=0.9]{Courier New}\textup{\textbf{{\LangtluGrammCatFontFamily{{\fontspec[Scale=0.9]{Courier New}\textcolor[rgb]{0,0.4392156862745098,0.7529411764705882}{v}}}}}}}}} followed by any tag, followed by {\LanglVernacularFontFamily{{\fontspec[Scale=0.9]{Courier New}\textup{\textbf{{\LangtluAffixFontFamily{{\fontspec[Scale=0.9]{Courier New}\textcolor[rgb]{0,0.6901960784313725,0.3137254901960784}{ptcp\_f\_sg}}}}}}}}}.\par{}\indent Note: we are using the convention of starting all categories with “{\textbf{c\_}}”.\par{}}}
{\setlength{\XLingPapertempdim}{\XLingPapersingledigitlistitemwidth+\parindent{}}\leftskip\XLingPapertempdim\relax
\interlinepenalty10000
\XLingPaperlistitem{\parindent{}}{\XLingPapersingledigitlistitemwidth}{2.}{Edit the example rule and replace the pattern as shown. Also change the rule comment to “Past Construction”.}\par{}{\vspace{12pt plus 2pt minus 1pt}\raggedright{}\XLingPaperexample{.125in}{0pt}{2.75em}{\raisebox{\baselineskip}[0pt]{\protect\hypertarget{xPattern1}{}}(58)}{\parbox[t]{\textwidth - .125in - 0pt}{\vspace*{-\baselineskip}{\XeTeXpicfile "../Images/RulesTutPattern1.PNG" scaled 750}}}
\vspace{12pt plus 2pt minus 1pt}}{\par{}\setlength{\parindent}{1em}\par\indent {\textit{Trick: Click in the pink area of the {\LangtRuleElemInXXEFontFamily{{\fontspec[Scale=0.8]{Arial}\textcolor[rgb]{0,0.4,0.2}{\textbf{item}}}}} element and press {\textup{\textbf{F11}}}. You will get a list of possible categories that can be used in this element. The list looks like}} \hyperlink{xCatList}{(59)}.\par{}}\par{}{\vspace{12pt plus 2pt minus 1pt}\raggedright{}\XLingPaperexample{.125in}{0pt}{2.75em}{\raisebox{\baselineskip}[0pt]{\protect\hypertarget{xCatList}{}}(59)}{\parbox[t]{\textwidth - .125in - 0pt}{\vspace*{-\baselineskip}{\XeTeXpicfile "../Images/RulesTutCatList.PNG" scaled 750}}}
\vspace{12pt plus 2pt minus 1pt}}{\setlength{\parindent}{1em}\par\indent By using this pattern we will detect a participle followed by the auxiliary. Once detected, the rule will run.\par{}\indent Notice that the example rule as written outputs only the first among two lexical units as seen in \hyperlink{xOutput1}{(60)} (“item: 1” means the first item or word). We are not outputting the 2nd word. The result is that our {\LanglVernacularFontFamily{{\fontspec[Scale=0.9]{Courier New}\textup{\textbf{biti{{\fontspec[Scale=0.65]{Times New Roman}\textsubscript{1.1}}}}}}}} word will disappear.\par{}}\par{}{\vspace{12pt plus 2pt minus 1pt}\raggedright{}\XLingPaperexample{.125in}{0pt}{2.75em}{\raisebox{\baselineskip}[0pt]{\protect\hypertarget{xOutput1}{}}(60)}{}
\vspace{12pt plus 2pt minus 1pt}}{}}
{\setlength{\XLingPapertempdim}{\XLingPapersingledigitlistitemwidth+\parindent{}}\leftskip\XLingPapertempdim\relax
\interlinepenalty10000
\XLingPaperlistitem{\parindent{}}{\XLingPapersingledigitlistitemwidth}{3.}{Test it in the {\LangtToolFontFamily{\textbf{\textcolor[rgb]{0,0,0.5019607843137255}{Live Rule Tester Tool}}}}.}{\setlength{\XLingPaperlistitemindent}{\XLingPapersingledigitlistitemwidth + \parindent{}}
{\setlength{\XLingPapertempdim}{\XLingPapersingleletterlistitemwidth+\XLingPaperlistitemindent}\leftskip\XLingPapertempdim\relax
\interlinepenalty10000
\XLingPaperlistitem{\XLingPaperlistitemindent}{\XLingPapersingleletterlistitemwidth}{a.}{Save the rule file.}}
{\setlength{\XLingPapertempdim}{\XLingPapersingleletterlistitemwidth+\XLingPaperlistitemindent}\leftskip\XLingPapertempdim\relax
\interlinepenalty10000
\XLingPaperlistitem{\XLingPaperlistitemindent}{\XLingPapersingleletterlistitemwidth}{b.}{Bring up the {\LangtToolFontFamily{\textbf{\textcolor[rgb]{0,0,0.5019607843137255}{Live Rule Tester Tool}}}}.}}
{\setlength{\XLingPapertempdim}{\XLingPapersingleletterlistitemwidth+\XLingPaperlistitemindent}\leftskip\XLingPapertempdim\relax
\interlinepenalty10000
\XLingPaperlistitem{\XLingPaperlistitemindent}{\XLingPapersingleletterlistitemwidth}{c.}{Click the {\textup{\textbf{Refresh Rules}}} button to reload the modified rule file.}}
{\setlength{\XLingPapertempdim}{\XLingPapersingleletterlistitemwidth+\XLingPaperlistitemindent}\leftskip\XLingPapertempdim\relax
\interlinepenalty10000
\XLingPaperlistitem{\XLingPaperlistitemindent}{\XLingPapersingleletterlistitemwidth}{d.}{Click on the {\textup{\textbf{Transfer}}} button. (This time you will notice some text in the yellow information box. This shows which rules matched which input words.)}}\par{}\setlength{\parindent}{1em}\indent The result is:\par{}}\par{}{\vspace{12pt plus 2pt minus 1pt}\raggedright{}\XLingPaperexample{.125in}{0pt}{2.75em}{\raisebox{\baselineskip}[0pt]{\protect\hypertarget{xStep1Target}{}}(61)}{{\LanglVernacularFontFamily{{\fontspec[Scale=0.9]{Courier New}\textup{\textbf{Leave{{\fontspec[Scale=0.65]{Times New Roman}\textsubscript{1.1}}} {\LangtluGrammCatFontFamily{{\fontspec[Scale=0.9]{Courier New}\textcolor[rgb]{0,0.4392156862745098,0.7529411764705882}{v}}}} {\LangtluAffixFontFamily{{\fontspec[Scale=0.9]{Courier New}\textcolor[rgb]{0,0.6901960784313725,0.3137254901960784}{pfv ptcp\_f\_sg}}}}}}}}}\ }
\vspace{12pt plus 2pt minus 1pt}}{\setlength{\parindent}{1em}\par\indent Great!\par{}}}
\vspace{\baselineskip}
}
\vspace{12pt}
\XLingPaperneedspace{5\baselineskip}

\penalty-3000{\noindent{\raisebox{\baselineskip}[0pt]{\protect\hypertarget{step2}{}}\SectionLevelFourFontFamily{\normalsize{\raisebox{\baselineskip}[0pt]{\pdfbookmark[4]{6.2.4.2 Step 2}{step2}}\textit{6.2.4.2 Step 2}}}}
\markright{Step 2}
\XLingPaperaddtocontents{step2}}\par{}\penalty10000
\vspace{12pt}
{\parskip .5pt plus 1pt minus 1pt
                    
{\setlength{\XLingPapertempdim}{\XLingPapersingledigitlistitemwidth+\parindent{}}\leftskip\XLingPapertempdim\relax
\interlinepenalty10000
\XLingPaperlistitem{\parindent{}}{\XLingPapersingledigitlistitemwidth}{1.}{Replace the dummy {\LangtRuleElemInXXEFontFamily{{\fontspec[Scale=0.8]{Arial}\textcolor[rgb]{0,0.4,0.2}{\textbf{attribute}}}}} element with an attribute that gives possible values of the tense feature. \\ \\So, now that we’ve gotten rid of the verb “be”, we want to change the tag for tense from {\LanglVernacularFontFamily{{\fontspec[Scale=0.9]{Courier New}\textup{\textbf{{\LangtluAffixFontFamily{{\fontspec[Scale=0.9]{Courier New}\textcolor[rgb]{0,0.6901960784313725,0.3137254901960784}{ptcp\_f\_sg}}}}}}}}} to {\LanglVernacularFontFamily{{\fontspec[Scale=0.9]{Courier New}\textup{\textbf{{\LangtluAffixFontFamily{{\fontspec[Scale=0.9]{Courier New}\textcolor[rgb]{0,0.6901960784313725,0.3137254901960784}{pst}}}}}}}}}. This will involve using a procedure statement, explicitly telling the transfer what we want to change. But before that, we need to define which possible values our feature can take. We do this with the {\LangtRuleElemInXXEFontFamily{{\fontspec[Scale=0.8]{Arial}\textcolor[rgb]{0,0.4,0.2}{\textbf{attribute}}}}} element. Let’s call it “{\textbf{a\_tense}}”. Note we are using the convention of starting all attributes with “{\textbf{a\_}}”.\\ \\Add the following under the {\LangtRuleElemInXXEFontFamily{{\fontspec[Scale=0.8]{Arial}\textcolor[rgb]{0,0.4,0.2}{\textbf{Attributes}}}}} element:}\par{}{\vspace{12pt plus 2pt minus 1pt}\raggedright{}\XLingPaperexample{.125in}{0pt}{2.75em}{\raisebox{\baselineskip}[0pt]{\protect\hypertarget{xAttrTense}{}}(62)}{\parbox[t]{\textwidth - .125in - 0pt}{\vspace*{-\baselineskip}{\XeTeXpicfile "../Images/RulesTutAttrTense.PNG" scaled 750}}}
\vspace{12pt plus 2pt minus 1pt}}{}}
{\setlength{\XLingPapertempdim}{\XLingPapersingledigitlistitemwidth+\parindent{}}\leftskip\XLingPapertempdim\relax
\interlinepenalty10000
\XLingPaperlistitem{\parindent{}}{\XLingPapersingledigitlistitemwidth}{2.}{Write a statement which changes the tense on the target language. This statement should go immediately after the {\LangtRuleElemInXXEFontFamily{{\fontspec[Scale=0.8]{Arial}\textcolor[rgb]{0,0.4,0.2}{\textbf{action}}}}} element and before the {\LangtRuleElemInXXEFontFamily{{\fontspec[Scale=0.8]{Arial}\textcolor[rgb]{0,0.4,0.2}{\textbf{output}}}}} element. Make sure you select the {\LangtRuleElemInXXEFontFamily{{\fontspec[Scale=0.8]{Arial}\textcolor[rgb]{0,0.4,0.2}{\textbf{target lang.}}}}} property.}\par{}{\vspace{12pt plus 2pt minus 1pt}\raggedright{}\XLingPaperexample{.125in}{0pt}{2.75em}{\raisebox{\baselineskip}[0pt]{\protect\hypertarget{xLet1}{}}(63)}{\parbox[t]{\textwidth - .125in - 0pt}{\vspace*{-\baselineskip}{\XeTeXpicfile "../Images/RulesTutLet1.PNG" scaled 750}}}
\vspace{12pt plus 2pt minus 1pt}}{{\textit{Trick: Click in the brown area of the {\LangtRuleElemInXXEFontFamily{{\fontspec[Scale=0.8]{Arial}\textcolor[rgb]{0,0.4,0.2}{\textbf{part}}}}} property and press {\textup{\textbf{F11}}}. You will get a list of possible attributes that can be used in this element. The list looks like}} \hyperlink{xAttrList}{(64)}.}\par{}{\vspace{12pt plus 2pt minus 1pt}\raggedright{}\XLingPaperexample{.125in}{0pt}{2.75em}{\raisebox{\baselineskip}[0pt]{\protect\hypertarget{xAttrList}{}}(64)}{\parbox[t]{\textwidth - .125in - 0pt}{\vspace*{-\baselineskip}{\XeTeXpicfile "../Images/RulesTutAttrList.PNG" scaled 750}}}
\vspace{12pt plus 2pt minus 1pt}}{(Note: the statement in \hyperlink{xLet1}{(63)} will always change the target language tag to {\LanglVernacularFontFamily{{\fontspec[Scale=0.9]{Courier New}\textup{\textbf{{\LangtluAffixFontFamily{{\fontspec[Scale=0.9]{Courier New}\textcolor[rgb]{0,0.6901960784313725,0.3137254901960784}{pst}}}}}}}}}, even if the input is not {\LanglVernacularFontFamily{{\fontspec[Scale=0.9]{Courier New}\textup{\textbf{{\LangtluAffixFontFamily{{\fontspec[Scale=0.9]{Courier New}\textcolor[rgb]{0,0.6901960784313725,0.3137254901960784}{ptcp\_f\_sg}}}}}}}}}. This is good enough for our current example, but for a description of how conditional logic works, see Section \hyperlink{sCondLogic}{8.2.8}.\\ \\}}
{\setlength{\XLingPapertempdim}{\XLingPapersingledigitlistitemwidth+\parindent{}}\leftskip\XLingPapertempdim\relax
\interlinepenalty10000
\XLingPaperlistitem{\parindent{}}{\XLingPapersingledigitlistitemwidth}{3.}{Now test it in the {\LangtToolFontFamily{\textbf{\textcolor[rgb]{0,0,0.5019607843137255}{Live Rule Tester Tool}}}}. (Don’t forget to refresh the rules first!)\\\par{}\setlength{\parindent}{1em}\indent The result is:\par{}}\par{}{\vspace{12pt plus 2pt minus 1pt}\raggedright{}\XLingPaperexample{.125in}{0pt}{2.75em}{\raisebox{\baselineskip}[0pt]{\protect\hypertarget{xStep2Target}{}}(65)}{{\LanglVernacularFontFamily{{\fontspec[Scale=0.9]{Courier New}\textup{\textbf{Leave{{\fontspec[Scale=0.65]{Times New Roman}\textsubscript{1.1}}} {\LangtluGrammCatFontFamily{{\fontspec[Scale=0.9]{Courier New}\textcolor[rgb]{0,0.4392156862745098,0.7529411764705882}{v}}}} {\LangtluAffixFontFamily{{\fontspec[Scale=0.9]{Courier New}\textcolor[rgb]{0,0.6901960784313725,0.3137254901960784}{pfv pst}}}}}}}}}\ }
\vspace{12pt plus 2pt minus 1pt}}{\setlength{\parindent}{1em}\par\indent Excellent!\par{}}}
\vspace{\baselineskip}
}
\vspace{12pt}
\XLingPaperneedspace{5\baselineskip}

\penalty-3000{\noindent{\raisebox{\baselineskip}[0pt]{\protect\hypertarget{step3}{}}\SectionLevelFourFontFamily{\normalsize{\raisebox{\baselineskip}[0pt]{\pdfbookmark[4]{6.2.4.3 Step 3}{step3}}\textit{6.2.4.3 Step 3}}}}
\markright{Step 3}
\XLingPaperaddtocontents{step3}}\par{}\penalty10000
\vspace{12pt}
\indent So far we have been outputting the whole target language lexical unit, as seen in \hyperlink{xLexUnitWhole}{(67)} below.\par{}\indent An important thing to know about lexical transfer is that any source language tags not matched in the bilingual dictionary are copied into the output on the target language side. In our case, {\LanglVernacularFontFamily{{\fontspec[Scale=0.9]{Courier New}\textup{\textbf{{\LangtluAffixFontFamily{{\fontspec[Scale=0.9]{Courier New}\textcolor[rgb]{0,0.6901960784313725,0.3137254901960784}{pfv pst}}}}}}}}} are the tags that get copied from source to target because, in the bilingual dictionary ({\textit{bilingual.dix}}), {\LanglVernacularFontFamily{{\fontspec[Scale=0.9]{Courier New}\textup{\textbf{otići{{\fontspec[Scale=0.65]{Times New Roman}\textsubscript{1.1}}} {\LangtluGrammCatFontFamily{{\fontspec[Scale=0.9]{Courier New}\textcolor[rgb]{0,0.4392156862745098,0.7529411764705882}{v}}}}}}}}} maps to {\LanglVernacularFontFamily{{\fontspec[Scale=0.9]{Courier New}\textup{\textbf{leave{{\fontspec[Scale=0.65]{Times New Roman}\textsubscript{1.1}}} {\LangtluGrammCatFontFamily{{\fontspec[Scale=0.9]{Courier New}\textcolor[rgb]{0,0.4392156862745098,0.7529411764705882}{v}}}}}}}}}. All we need is the {\LanglVernacularFontFamily{{\fontspec[Scale=0.9]{Courier New}\textup{\textbf{{\LangtluAffixFontFamily{{\fontspec[Scale=0.9]{Courier New}\textcolor[rgb]{0,0.6901960784313725,0.3137254901960784}{pst}}}}}}}}} tag; we don’t need the {\LanglVernacularFontFamily{{\fontspec[Scale=0.9]{Courier New}\textup{\textbf{{\LangtluAffixFontFamily{{\fontspec[Scale=0.9]{Courier New}\textcolor[rgb]{0,0.6901960784313725,0.3137254901960784}{pfv}}}}}}}}} tag, so we need to declare that it should not be outputted.\par{}{\parskip .5pt plus 1pt minus 1pt
                    
\vspace{\baselineskip}

{\setlength{\XLingPapertempdim}{\XLingPapersingledigitlistitemwidth+\parindent{}}\leftskip\XLingPapertempdim\relax
\interlinepenalty10000
\XLingPaperlistitem{\parindent{}}{\XLingPapersingledigitlistitemwidth}{1.}{Define the features that we want to output.\\ \\We have already defined the attribute “tense”: the only remaining tag we need to output is the grammatical category, i.e. {\LanglVernacularFontFamily{{\fontspec[Scale=0.9]{Courier New}\textup{\textbf{{\LangtluGrammCatFontFamily{{\fontspec[Scale=0.9]{Courier New}\textcolor[rgb]{0,0.4392156862745098,0.7529411764705882}{v}}}}}}}}} meaning verb. So, make a new attribute element for grammatical category.\protect\footnote[10]{{\leftskip0pt\parindent1em\raisebox{\baselineskip}[0pt]{\protect\hypertarget{nUseSetUpTxfrRulGramCatTool}{}}It can be arduous to type in all the grammatical categories that you need for your transfer rules. That’s why there is the {\LangtToolFontFamily{\textbf{\textcolor[rgb]{0,0,0.5019607843137255}{FLExTrans}}}} tool called {\LangtToolFontFamily{\textbf{\textcolor[rgb]{0,0,0.5019607843137255}{Set Up Transfer Rule Grammatical Categories}}}}. This tool will insert all possible source and target categories for the {\textbf{a\_gram\_cat}} attribute. See section \hyperlink{sSetupGramCat}{4.5} for more details.}}}\par{}{\vspace{12pt plus 2pt minus 1pt}\raggedright{}\XLingPaperexample{.125in}{0pt}{2.75em}{\raisebox{\baselineskip}[0pt]{\protect\hypertarget{xAttrGramCat}{}}(66)}{\parbox[t]{\textwidth - .125in - 0pt}{\vspace*{-\baselineskip}{\XeTeXpicfile "../Images/RulesTutAttrGramCat.PNG" scaled 750}}}
\vspace{12pt plus 2pt minus 1pt}}{}}
{\setlength{\XLingPapertempdim}{\XLingPapersingledigitlistitemwidth+\parindent{}}\leftskip\XLingPapertempdim\relax
\interlinepenalty10000
\XLingPaperlistitem{\parindent{}}{\XLingPapersingledigitlistitemwidth}{2.}{Declare which attributes we want to output. \\ \\So, now that we’ve got our two attributes defined, let’s change our {\LangtRuleElemInXXEFontFamily{{\fontspec[Scale=0.8]{Arial}\textcolor[rgb]{0,0.4,0.2}{\textbf{output}}}}} statement: (You will have to open the {\LangtRuleElemInXXEFontFamily{{\fontspec[Scale=0.8]{Arial}\textcolor[rgb]{0,0.4,0.2}{\textbf{output}}}}} element with the plus sign.) \\ \\Where we previously had:}\par{}{\vspace{12pt plus 2pt minus 1pt}\raggedright{}\XLingPaperexample{.125in}{0pt}{2.75em}{\raisebox{\baselineskip}[0pt]{\protect\hypertarget{xLexUnitWhole}{}}(67)}{\parbox[t]{\textwidth - .125in - 0pt}{\vspace*{-\baselineskip}{\XeTeXpicfile "../Images/RulesTutLexUnitWhole.PNG" scaled 750}}}
}{Replace it with:}\par{}{\vspace{12pt plus 2pt minus 1pt}\raggedright{}\XLingPaperexample{.125in}{0pt}{2.75em}{\raisebox{\baselineskip}[0pt]{\protect\hypertarget{xLexUnit3Clips}{}}(68)}{\parbox[t]{\textwidth - .125in - 0pt}{\vspace*{-\baselineskip}{\XeTeXpicfile "../Images/RulesTutLexUnit3Clips.PNG" scaled 750}}}
\vspace{12pt plus 2pt minus 1pt}}{The last two attributes we know, as we’ve just added them. The “lem” attribute is a predefined feature which corresponds to lemma, i.e. the part before the tags. In this example it would be {\LanglVernacularFontFamily{{\fontspec[Scale=0.9]{Courier New}\textup{\textbf{leave{{\fontspec[Scale=0.65]{Times New Roman}\textsubscript{1.1}}}}}}}}. There are other predefined features, but if you remember this one and “whole” which corresponds to the whole lexical unit, you have the main ones.\\ \\}}
{\setlength{\XLingPapertempdim}{\XLingPapersingledigitlistitemwidth+\parindent{}}\leftskip\XLingPapertempdim\relax
\interlinepenalty10000
\XLingPaperlistitem{\parindent{}}{\XLingPapersingledigitlistitemwidth}{3.}{Now test it in the {\LangtToolFontFamily{\textbf{\textcolor[rgb]{0,0,0.5019607843137255}{Live Rule Tester Tool}}}}. (Don’t forget to refresh the rules first!)\\\par{}\setlength{\parindent}{1em}\indent The result is:\par{}}\par{}{\vspace{12pt plus 2pt minus 1pt}\raggedright{}\XLingPaperexample{.125in}{0pt}{2.75em}{\raisebox{\baselineskip}[0pt]{\protect\hypertarget{xStep3Target}{}}(69)}{{\LanglVernacularFontFamily{{\fontspec[Scale=0.9]{Courier New}\textup{\textbf{Leave{{\fontspec[Scale=0.65]{Times New Roman}\textsubscript{1.1}}} {\LangtluGrammCatFontFamily{{\fontspec[Scale=0.9]{Courier New}\textcolor[rgb]{0,0.4392156862745098,0.7529411764705882}{v}}}} {\LangtluAffixFontFamily{{\fontspec[Scale=0.9]{Courier New}\textcolor[rgb]{0,0.6901960784313725,0.3137254901960784}{pst}}}}}}}}}\ }
\vspace{12pt plus 2pt minus 1pt}}{\setlength{\parindent}{1em}\par\indent Woo! What just happened was that we magically deleted the unwanted {\LanglVernacularFontFamily{{\fontspec[Scale=0.9]{Courier New}\textup{\textbf{{\LangtluAffixFontFamily{{\fontspec[Scale=0.9]{Courier New}\textcolor[rgb]{0,0.6901960784313725,0.3137254901960784}{pfv}}}}}}}}} tag by not specifying that we wanted it. Tag deletion is thus not stated as tag deletion (tag → 0), but by not declaring it.\par{}\indent Now we should have a lexical form for “leave” which we can synthesize. Let’s give it a go. Click the {\textup{\textbf{Synthesize}}} button.\par{}\indent In the red box you should see:\par{}}\par{}{\vspace{12pt plus 2pt minus 1pt}\raggedright{}\XLingPaperexample{.125in}{0pt}{2.75em}{\raisebox{\baselineskip}[0pt]{\protect\hypertarget{xOutputStep3a}{}}(70)}{{\LanglVernacularFontFamily{{\fontspec[Scale=0.9]{Courier New}\textup{\textbf{Left}}}}}\ }
\vspace{12pt plus 2pt minus 1pt}}{\par\indent And if you click on the rest of the check boxes to select all of the Croatian words and click the {\textup{\textbf{Transfer}}} and {\textup{\textbf{Synthesize}}} buttons again you should see:\par{}}\par{}{\vspace{12pt plus 2pt minus 1pt}\raggedright{}\XLingPaperexample{.125in}{0pt}{2.75em}{\raisebox{\baselineskip}[0pt]{\protect\hypertarget{xOutputStep3}{}}(71)}{{\LanglVernacularFontFamily{{\fontspec[Scale=0.9]{Courier New}\textup{\textbf{Left quietly and without wordgen sg}}}}}\ }
\vspace{12pt plus 2pt minus 1pt}}{\par\indent This looks a lot better than our initial results in \hyperlink{xOutputInitial}{(50)}. The word {\textit{Left}} is now there correctly. It’s beyond the scope of this exercise to explain in detail how the synthesis works in this case, but the tag {\LanglVernacularFontFamily{{\fontspec[Scale=0.9]{Courier New}\textup{\textbf{{\LangtluAffixFontFamily{{\fontspec[Scale=0.9]{Courier New}\textcolor[rgb]{0,0.6901960784313725,0.3137254901960784}{pst}}}}}}}}} matches the feature {\LanglVernacularFontFamily{{\fontspec[Scale=0.9]{Courier New}\textup{\textbf{{\LangtluAffixFontFamily{{\fontspec[Scale=0.9]{Courier New}\textcolor[rgb]{0,0.6901960784313725,0.3137254901960784}{pst}}}}}}}}} on the variant form of the verb {\textit{leave}}. See section \hyperlink{sInflVariantTgt}{8.3.4} for more details.\par{}\indent We’re getting there, now let’s move on to the next step.\par{}}}
\vspace{\baselineskip}
}
\vspace{12pt}
\XLingPaperneedspace{5\baselineskip}

\penalty-3000{\noindent{\raisebox{\baselineskip}[0pt]{\protect\hypertarget{step4}{}}\SectionLevelFourFontFamily{\normalsize{\raisebox{\baselineskip}[0pt]{\pdfbookmark[4]{6.2.4.4 Step 4}{step4}}\textit{6.2.4.4 Step 4}}}}
\markright{Step 4}
\XLingPaperaddtocontents{step4}}\par{}\penalty10000
\vspace{12pt}
\indent To recap, the input we’ve been working with is:\par{}{\vspace{12pt plus 2pt minus 1pt}\raggedright{}\XLingPaperexample{.125in}{0pt}{2.75em}{\raisebox{\baselineskip}[0pt]{\protect\hypertarget{xStep4Input}{}}(72)}{{\LanglVernacularFontFamily{{\fontspec[Scale=0.9]{Courier New}\textup{\textbf{Otići{{\fontspec[Scale=0.65]{Times New Roman}\textsubscript{1.1}}} {\LangtluGrammCatFontFamily{{\fontspec[Scale=0.9]{Courier New}\textcolor[rgb]{0,0.4392156862745098,0.7529411764705882}{v}}}} {\LangtluAffixFontFamily{{\fontspec[Scale=0.9]{Courier New}\textcolor[rgb]{0,0.6901960784313725,0.3137254901960784}{pfv ptcp\_f\_sg}}}} biti{{\fontspec[Scale=0.65]{Times New Roman}\textsubscript{1.1}}} {\LangtluGrammCatFontFamily{{\fontspec[Scale=0.9]{Courier New}\textcolor[rgb]{0,0.4392156862745098,0.7529411764705882}{cop}}}} {\LangtluAffixFontFamily{{\fontspec[Scale=0.9]{Courier New}\textcolor[rgb]{0,0.6901960784313725,0.3137254901960784}{prs sg 2}}}}}}}}}\ }
\vspace{12pt plus 2pt minus 1pt}}\par\indent And our current output is:\par{}{\vspace{12pt plus 2pt minus 1pt}\raggedright{}\XLingPaperexample{.125in}{0pt}{2.75em}{\raisebox{\baselineskip}[0pt]{\protect\hypertarget{xStep4CurrentTarget}{}}(73)}{{\LanglVernacularFontFamily{{\fontspec[Scale=0.9]{Courier New}\textup{\textbf{Leave{{\fontspec[Scale=0.65]{Times New Roman}\textsubscript{1.1}}} {\LangtluGrammCatFontFamily{{\fontspec[Scale=0.9]{Courier New}\textcolor[rgb]{0,0.4392156862745098,0.7529411764705882}{v}}}} {\LangtluAffixFontFamily{{\fontspec[Scale=0.9]{Courier New}\textcolor[rgb]{0,0.6901960784313725,0.3137254901960784}{pst}}}}}}}}}\ }
\vspace{12pt plus 2pt minus 1pt}}\par\indent What we now need to do is take the information from the (not outputted) verb {\textit{biti}} “be” and use it to output a personal pronoun before the verb. Remember that when we do not output something we are not deleting it. It is still there in the input, we are just choosing not to put it in the output.\par{}\indent The pronoun that we want to output looks like \hyperlink{xStep4TargetGoal}{(74)}. If we get it in this form, it will synthesize into the word {\textit{you}}.\par{}{\vspace{12pt plus 2pt minus 1pt}\raggedright{}\XLingPaperexample{.125in}{0pt}{2.75em}{\raisebox{\baselineskip}[0pt]{\protect\hypertarget{xStep4TargetGoal}{}}(74)}{{\LanglVernacularFontFamily{{\fontspec[Scale=0.9]{Courier New}\textup{\textbf{propers{{\fontspec[Scale=0.65]{Times New Roman}\textsubscript{1.1}}} {\LangtluGrammCatFontFamily{{\fontspec[Scale=0.9]{Courier New}\textcolor[rgb]{0,0.4392156862745098,0.7529411764705882}{pro}}}} {\LangtluAffixFontFamily{{\fontspec[Scale=0.9]{Courier New}\textcolor[rgb]{0,0.6901960784313725,0.3137254901960784}{nom 2 sg}}}}}}}}}\ }
\vspace{12pt plus 2pt minus 1pt}}\par\indent {\uline{Important}}: We need to remember to output the {\LanglVernacularFontFamily{{\fontspec[Scale=0.9]{Courier New}\textup{\textbf{1.1}}}}} after the word and not just {\LanglVernacularFontFamily{{\fontspec[Scale=0.9]{Courier New}\textup{\textbf{propers}}}}}. Even though in the target {\LangtToolFontFamily{\textbf{\textcolor[rgb]{0,0,0.5019607843137255}{FLEx}}}} project the lexeme form is {\LanglVernacularFontFamily{{\fontspec[Scale=0.9]{Courier New}\textup{\textbf{propers}}}}}, {\LangtToolFontFamily{\textbf{\textcolor[rgb]{0,0,0.5019607843137255}{FLExTrans}}}} automatically adds the homograph number ({\LanglVernacularFontFamily{{\fontspec[Scale=0.9]{Courier New}\textup{\textbf{1}}}}} because it’s the first and only homograph) and the sense number ({\LanglVernacularFontFamily{{\fontspec[Scale=0.9]{Courier New}\textup{\textbf{1}}}}} because it’s the first sense). If we mistakenly use {\LanglVernacularFontFamily{{\fontspec[Scale=0.9]{Courier New}\textup{\textbf{propers}}}}} instead of {\LanglVernacularFontFamily{{\fontspec[Scale=0.9]{Courier New}\textup{\textbf{propers1.1}}}}}, the word won’t synthesize because {\LangtToolFontFamily{\textbf{\textcolor[rgb]{0,0,0.5019607843137255}{FLExTrans}}}} won’t find {\LanglVernacularFontFamily{{\fontspec[Scale=0.9]{Courier New}\textup{\textbf{propers}}}}} in its internal dictionary. This use of {\LanglVernacularFontFamily{{\fontspec[Scale=0.9]{Courier New}\textup{\textbf{x.x}}}}} after the word helps {\LangtToolFontFamily{\textbf{\textcolor[rgb]{0,0,0.5019607843137255}{FLExTrans}}}} identify each unique sense in the dictionary.\par{}{\parskip .5pt plus 1pt minus 1pt
                    
\vspace{\baselineskip}

{\setlength{\XLingPapertempdim}{\XLingPapersingledigitlistitemwidth+\parindent{}}\leftskip\XLingPapertempdim\relax
\interlinepenalty10000
\XLingPaperlistitem{\parindent{}}{\XLingPapersingledigitlistitemwidth}{1.}{Output a skeleton pronoun.\\ \\So, we need to take the person and number information from the verb “be”, and the rest we need to just declare to be outputted. Let’s start with outputting the string {\LanglVernacularFontFamily{{\fontspec[Scale=0.9]{Courier New}\textup{\textbf{propers{{\fontspec[Scale=0.65]{Times New Roman}\textsubscript{1.1}}} {\LangtluGrammCatFontFamily{{\fontspec[Scale=0.9]{Courier New}\textcolor[rgb]{0,0.4392156862745098,0.7529411764705882}{pro}}}} {\LangtluAffixFontFamily{{\fontspec[Scale=0.9]{Courier New}\textcolor[rgb]{0,0.6901960784313725,0.3137254901960784}{nom}}}}}}}}}; we won’t worry about the {\LanglVernacularFontFamily{{\fontspec[Scale=0.9]{Courier New}\textup{\textbf{{\LangtluAffixFontFamily{{\fontspec[Scale=0.9]{Courier New}\textcolor[rgb]{0,0.6901960784313725,0.3137254901960784}{2 sg}}}}}}}}} tags yet.\\ \\}}
{\setlength{\XLingPapertempdim}{\XLingPapersingledigitlistitemwidth+\parindent{}}\leftskip\XLingPapertempdim\relax
\interlinepenalty10000
\XLingPaperlistitem{\parindent{}}{\XLingPapersingledigitlistitemwidth}{2.}{In the {\LangtRuleElemInXXEFontFamily{{\fontspec[Scale=0.8]{Arial}\textcolor[rgb]{0,0.4,0.2}{\textbf{output}}}}} section, insert a new {\LangtRuleElemInXXEFontFamily{{\fontspec[Scale=0.8]{Arial}\textcolor[rgb]{0,0.4,0.2}{\textbf{lexical unit}}}}} before our current {\LangtRuleElemInXXEFontFamily{{\fontspec[Scale=0.8]{Arial}\textcolor[rgb]{0,0.4,0.2}{\textbf{lexical unit}}}}} element and make it look like \hyperlink{xLexUnitPron}{(75)}.\\\par{}\setlength{\parindent}{1em}\indent You can do an Insert Before... and choose {\textit{lu(lexical\_unit\_lit-string)}} then click on the {\LangtRuleElemInXXEFontFamily{{\fontspec[Scale=0.8]{Arial}\textcolor[rgb]{0,0.4,0.2}{\textbf{literal string}}}}}, do an Insert After... and choose {\textit{lit-tag(literal\_tag)}}. The {\LangtRuleElemInXXEFontFamily{{\fontspec[Scale=0.8]{Arial}\textcolor[rgb]{0,0.4,0.2}{\textbf{blank space}}}}} element comes after and is {\uline{indented at the same level}} as {\LangtRuleElemInXXEFontFamily{{\fontspec[Scale=0.8]{Arial}\textcolor[rgb]{0,0.4,0.2}{\textbf{lexical unit}}}}}.\par{}}\par{}{\vspace{12pt plus 2pt minus 1pt}\raggedright{}\XLingPaperexample{.125in}{0pt}{2.75em}{\raisebox{\baselineskip}[0pt]{\protect\hypertarget{xLexUnitPron}{}}(75)}{\parbox[t]{\textwidth - .125in - 0pt}{\vspace*{-\baselineskip}{\XeTeXpicfile "../Images/RulesTutLexUnitPron.PNG" scaled 750}}}
\vspace{12pt plus 2pt minus 1pt}}{The {\LangtRuleElemInXXEFontFamily{{\fontspec[Scale=0.8]{Arial}\textcolor[rgb]{0,0.4,0.2}{\textbf{lexical unit}}}}} statement means we are outputting a word\protect\footnote[11]{{\leftskip0pt\parindent1em\raisebox{\baselineskip}[0pt]{\protect\hypertarget{nLexUnit}{}}In our {\textit{target\_text.txt}} file it puts a {\LanglVernacularFontFamily{{\fontspec[Scale=0.9]{Courier New}\textup{\textbf{\^{}}}}}} and {\LanglVernacularFontFamily{{\fontspec[Scale=0.9]{Courier New}\textup{\textbf{\textdollar{}}}}}} before and after the contents contained within it.}}. The {\LangtRuleElemInXXEFontFamily{{\fontspec[Scale=0.8]{Arial}\textcolor[rgb]{0,0.4,0.2}{\textbf{literal string}}}}} element outputs the given string of characters, and the {\LangtRuleElemInXXEFontFamily{{\fontspec[Scale=0.8]{Arial}\textcolor[rgb]{0,0.4,0.2}{\textbf{literal tag}}}}} element outputs a tag sequence\protect\footnote[12]{{\leftskip0pt\parindent1em\raisebox{\baselineskip}[0pt]{\protect\hypertarget{nTagSeq}{}}In our {\textit{target\_text.txt}} file each tag starts with a {\LanglVernacularFontFamily{{\fontspec[Scale=0.9]{Courier New}\textup{\textbf{\textless{}}}}}}, ends with a {\LanglVernacularFontFamily{{\fontspec[Scale=0.9]{Courier New}\textup{\textbf{\textgreater{}.}}}}}}}. Multiple tags are separated by a period, so in this case we are outputting two tags: {\LanglVernacularFontFamily{{\fontspec[Scale=0.9]{Courier New}\textup{\textbf{{\LangtluGrammCatFontFamily{{\fontspec[Scale=0.9]{Courier New}\textcolor[rgb]{0,0.4392156862745098,0.7529411764705882}{pro}}}}}}}}} and {\LanglVernacularFontFamily{{\fontspec[Scale=0.9]{Courier New}\textup{\textbf{{\LangtluAffixFontFamily{{\fontspec[Scale=0.9]{Courier New}\textcolor[rgb]{0,0.6901960784313725,0.3137254901960784}{nom}}}}}}}}}. The {\LangtRuleElemInXXEFontFamily{{\fontspec[Scale=0.8]{Arial}\textcolor[rgb]{0,0.4,0.2}{\textbf{blank space}}}}} element outputs a single space character.\\ \\}}
{\setlength{\XLingPapertempdim}{\XLingPapersingledigitlistitemwidth+\parindent{}}\leftskip\XLingPapertempdim\relax
\interlinepenalty10000
\XLingPaperlistitem{\parindent{}}{\XLingPapersingledigitlistitemwidth}{3.}{Define features for person and number. \\ \\The next step is to declare the features from the verb “be” to output. To do this we need to again define some attributes and their possible values. Add this:}\par{}{\vspace{12pt plus 2pt minus 1pt}\raggedright{}\XLingPaperexample{.125in}{0pt}{2.75em}{\raisebox{\baselineskip}[0pt]{\protect\hypertarget{xAttrNumPers}{}}(76)}{\parbox[t]{\textwidth - .125in - 0pt}{\vspace*{-\baselineskip}{\XeTeXpicfile "../Images/RulesTutAttrNumPer.PNG" scaled 750}}}
\vspace{12pt plus 2pt minus 1pt}}{}}
{\setlength{\XLingPapertempdim}{\XLingPapersingledigitlistitemwidth+\parindent{}}\leftskip\XLingPapertempdim\relax
\interlinepenalty10000
\XLingPaperlistitem{\parindent{}}{\XLingPapersingledigitlistitemwidth}{4.}{We then go back to our skeleton pronoun and add the missing features. So that it looks like \hyperlink{xLexUnitPronPerNum}{(77)}.\\\par{}\setlength{\parindent}{1em}\indent Click on the {\LangtRuleElemInXXEFontFamily{{\fontspec[Scale=0.8]{Arial}\textcolor[rgb]{0,0.4,0.2}{\textbf{literal tag}}}}} element and Insert After... and choose {\textit{clip(clip\_target\_language)}}. Do this twice.\par{}}\par{}{\vspace{12pt plus 2pt minus 1pt}\raggedright{}\XLingPaperexample{.125in}{0pt}{2.75em}{\raisebox{\baselineskip}[0pt]{\protect\hypertarget{xLexUnitPronPerNum}{}}(77)}{\parbox[t]{\textwidth - .125in - 0pt}{\vspace*{-\baselineskip}{\XeTeXpicfile "../Images/RulesTutLexUnitPronPerNum.PNG" scaled 750}}}
\vspace{12pt plus 2pt minus 1pt}}{Note that we are “clipping” (copying a part of a string which matches a pattern) from position 2, i.e. the verb “be”. To be clearer, the first clip statement in \hyperlink{xLexUnitPronPerNum}{(77)} makes a copy of the part of the lexical unit in position 2, on the target language side, which matches one of the patterns defined in the attribute element “a\_pers” (which can be either {\LanglVernacularFontFamily{{\fontspec[Scale=0.9]{Courier New}\textup{\textbf{ 1}}}}}, {\LanglVernacularFontFamily{{\fontspec[Scale=0.9]{Courier New}\textup{\textbf{ 2}}}}} or {\LanglVernacularFontFamily{{\fontspec[Scale=0.9]{Courier New}\textup{\textbf{ 3}}}}}). If we look at the lexical transfer output which is internally coming from the bilingual dictionary for position 2, this will become clearer:\\ \\}\par{}{\vspace{12pt plus 2pt minus 1pt}\raggedright{}\XLingPaperexample{.125in}{0pt}{2.75em}{\raisebox{\baselineskip}[0pt]{\protect\hypertarget{xStep4Diagram}{}}(78)}{{\LanglVernacularFontFamily{{\fontspec[Scale=0.9]{Courier New}\textup{\textbf{\vbox{\hbox{\strut{}biti{{\fontspec[Scale=0.65]{Times New Roman}\textsubscript{1.1}}} {\LangtluGrammCatFontFamily{{\fontspec[Scale=0.9]{Courier New}\textcolor[rgb]{0,0.4392156862745098,0.7529411764705882}{cop}}}} {\LangtluAffixFontFamily{{\fontspec[Scale=0.9]{Courier New}\textcolor[rgb]{0,0.6901960784313725,0.3137254901960784}{prs sg 2}}}} -\textgreater{} be{{\fontspec[Scale=0.65]{Times New Roman}\textsubscript{1.1}}} {\LangtluGrammCatFontFamily{{\fontspec[Scale=0.9]{Courier New}\textcolor[rgb]{0,0.4392156862745098,0.7529411764705882}{cop}}}} {\LangtluAffixFontFamily{{\fontspec[Scale=0.9]{Courier New}\textcolor[rgb]{0,0.6901960784313725,0.3137254901960784}{prs sg 2}}}}}\hbox{\strut{}\textbar{}                \textbar{}\textbar{}   \textbar{}              \textbar{}}\hbox{\strut{}\textbar{}           person\textbar{}   \textbar{}              \textbar{}}\hbox{\strut{}\textbar{}\_\_\_\_\_\_\_\_\_\_\_\_\_\_\_\_\_\textbar{}   \textbar{}\_\_\_\_\_\_\_\_\_\_\_\_\_\_\textbar{}}\hbox{\strut{}  source language      target language}}}}}}}\ {\LanglVernacularFontFamily{{\fontspec[Scale=0.9]{Courier New}\textup{\textbf{}}}}}\ }
\vspace{12pt plus 2pt minus 1pt}}{If we tried to clip from position 1 ({\LanglVernacularFontFamily{{\fontspec[Scale=0.9]{Courier New}\textup{\textbf{Otići{{\fontspec[Scale=0.65]{Times New Roman}\textsubscript{1.1}}} {\LangtluGrammCatFontFamily{{\fontspec[Scale=0.9]{Courier New}\textcolor[rgb]{0,0.4392156862745098,0.7529411764705882}{v}}}} {\LangtluAffixFontFamily{{\fontspec[Scale=0.9]{Courier New}\textcolor[rgb]{0,0.6901960784313725,0.3137254901960784}{pfv ptcp\_f\_sg}}}}}}}}}), we would get no result for the person attribute because the verb in position 1 does not contain any tags which match the tags in the definition for person.\\ \\}}
{\setlength{\XLingPapertempdim}{\XLingPapersingledigitlistitemwidth+\parindent{}}\leftskip\XLingPapertempdim\relax
\interlinepenalty10000
\XLingPaperlistitem{\parindent{}}{\XLingPapersingledigitlistitemwidth}{5.}{Now test it in the {\LangtToolFontFamily{\textbf{\textcolor[rgb]{0,0,0.5019607843137255}{Live Rule Tester Tool}}}}. (Go back to having just {\textit{Otišla}} and {\textit{si}} checked.)\\\par{}\setlength{\parindent}{1em}\indent The result is:\par{}}\par{}{\vspace{12pt plus 2pt minus 1pt}\raggedright{}\XLingPaperexample{.125in}{0pt}{2.75em}{\raisebox{\baselineskip}[0pt]{\protect\hypertarget{xStep4Target}{}}(79)}{{\LanglVernacularFontFamily{{\fontspec[Scale=0.9]{Courier New}\textup{\textbf{propers{{\fontspec[Scale=0.65]{Times New Roman}\textsubscript{1.1}}} {\LangtluGrammCatFontFamily{{\fontspec[Scale=0.9]{Courier New}\textcolor[rgb]{0,0.4392156862745098,0.7529411764705882}{pro}}}} {\LangtluAffixFontFamily{{\fontspec[Scale=0.9]{Courier New}\textcolor[rgb]{0,0.6901960784313725,0.3137254901960784}{nom 2 sg}}}} Leave{{\fontspec[Scale=0.65]{Times New Roman}\textsubscript{1.1}}} {\LangtluGrammCatFontFamily{{\fontspec[Scale=0.9]{Courier New}\textcolor[rgb]{0,0.4392156862745098,0.7529411764705882}{v}}}} {\LangtluAffixFontFamily{{\fontspec[Scale=0.9]{Courier New}\textcolor[rgb]{0,0.6901960784313725,0.3137254901960784}{pst}}}}}}}}}\ }
\vspace{12pt plus 2pt minus 1pt}}{\setlength{\parindent}{1em}\par\indent Select all the Croatian words, do {\textbf{Transfer}} and {\textbf{Synthesize}} and see what our English sentence looks like.\par{}\indent Now we get:\par{}}\par{}{\vspace{12pt plus 2pt minus 1pt}\raggedright{}\XLingPaperexample{.125in}{0pt}{2.75em}{\raisebox{\baselineskip}[0pt]{\protect\hypertarget{xOutputStep4}{}}(80)}{{\LanglVernacularFontFamily{{\fontspec[Scale=0.9]{Courier New}\textup{\textbf{you Left quietly and without wordgen sg}}}}}\ }
\vspace{12pt plus 2pt minus 1pt}}{\par\indent We get the second person singular pronoun that we wanted, but this is not totally satisfying because {\textit{you}} should be capitalized and {\textit{Left}} should not be. Let’s fix that.\par{}}{\setlength{\XLingPaperlistitemindent}{\XLingPapersingledigitlistitemwidth + \parindent{}}
{\setlength{\XLingPapertempdim}{\XLingPapersingleletterlistitemwidth+\XLingPaperlistitemindent}\leftskip\XLingPapertempdim\relax
\interlinepenalty10000
\XLingPaperlistitem{\XLingPaperlistitemindent}{\XLingPapersingleletterlistitemwidth}{a.}{We want to be careful, because we don’t want {\textit{you}} capitalized in all circumstances. For example, if the Croatian past tense clause is embedded in another phrase, the first word may not be capitalized. What we can do is check what the capitalization of the verb is and apply that same capitalization to the pronoun. Make your {\LangtRuleElemInXXEFontFamily{{\fontspec[Scale=0.8]{Arial}\textcolor[rgb]{0,0.4,0.2}{\textbf{lexical unit}}}}} element look like \hyperlink{xGetCase1}{(81)}:}\par{}{\vspace{12pt plus 2pt minus 1pt}\raggedright{}\XLingPaperexample{.125in}{0pt}{2.75em}{\raisebox{\baselineskip}[0pt]{\protect\hypertarget{xGetCase1}{}}(81)}{\parbox[t]{\textwidth - .125in - 0pt}{\vspace*{-\baselineskip}{\XeTeXpicfile "../Images/RulesTutGetCase1.PNG" scaled 750}}}
\vspace{12pt plus 2pt minus 1pt}}{Notice how the literal string element is indented under the {\LangtRuleElemInXXEFontFamily{{\fontspec[Scale=0.8]{Arial}\textcolor[rgb]{0,0.4,0.2}{\textbf{get case from}}}}} element. This means that it is a child element of the {\LangtRuleElemInXXEFontFamily{{\fontspec[Scale=0.8]{Arial}\textcolor[rgb]{0,0.4,0.2}{\textbf{get case from}}}}} element. To accomplish this, {\LangtToolFontFamily{\textbf{\textcolor[rgb]{0,0,0.5019607843137255}{XMLmind XML Editor}}}} has a way to wrap an element around another element. This is what you do.}{\setlength{\XLingPaperlistitemindent}{\XLingPapersingleletterlistitemwidth + \XLingPapersingledigitlistitemwidth + \parindent{}}
{\setlength{\XLingPapertempdim}{\XLingPaperromanviilistitemwidth+\XLingPaperlistitemindent}\leftskip\XLingPapertempdim\relax
\interlinepenalty10000
\XLingPaperlistitem{\XLingPaperlistitemindent}{\XLingPaperromanviilistitemwidth}{i.}{Click on the words “literal string”. The whole element becomes outlined in a red rectangle.}}
{\setlength{\XLingPapertempdim}{\XLingPaperromanviilistitemwidth+\XLingPaperlistitemindent}\leftskip\XLingPapertempdim\relax
\interlinepenalty10000
\XLingPaperlistitem{\XLingPaperlistitemindent}{\XLingPaperromanviilistitemwidth}{ii.}{Either right-click and choose Convert {[}wrap{]}... or click on the \vspace*{0pt}{\XeTeXpicfile "../Images/ButWrap.PNG" scaled 750} button (upper right side) or press Ctrl-Shift-T.}}
{\setlength{\XLingPapertempdim}{\XLingPaperromanviilistitemwidth+\XLingPaperlistitemindent}\leftskip\XLingPapertempdim\relax
\interlinepenalty10000
\XLingPaperlistitem{\XLingPaperlistitemindent}{\XLingPaperromanviilistitemwidth}{iii.}{At this point a list of elements to insert shows up in the upper right pane. Choose {\textit{get-case-from}}.}}
{\setlength{\XLingPapertempdim}{\XLingPaperromanviilistitemwidth+\XLingPaperlistitemindent}\leftskip\XLingPapertempdim\relax
\interlinepenalty10000
\XLingPaperlistitem{\XLingPaperlistitemindent}{\XLingPaperromanviilistitemwidth}{iv.}{Now set the {\LangtRuleElemInXXEFontFamily{{\fontspec[Scale=0.8]{Arial}\textcolor[rgb]{0,0.4,0.2}{\textbf{item}}}}} property to 1 and we will be getting the capitalization from the first word.\\ \\}}}}
{\setlength{\XLingPapertempdim}{\XLingPapersingleletterlistitemwidth+\XLingPaperlistitemindent}\leftskip\XLingPapertempdim\relax
\interlinepenalty10000
\XLingPaperlistitem{\XLingPaperlistitemindent}{\XLingPapersingleletterlistitemwidth}{b.}{Now we need to make the second lexical unit come out with the first letter in lower case. We could use the {\LangtRuleElemInXXEFontFamily{{\fontspec[Scale=0.8]{Arial}\textcolor[rgb]{0,0.4,0.2}{\textbf{modify-case}}}}} element, but we would have to add a new statement above the {\LangtRuleElemInXXEFontFamily{{\fontspec[Scale=0.8]{Arial}\textcolor[rgb]{0,0.4,0.2}{\textbf{output}}}}} element. Instead we can just use the same method as above to get the case from the second word. This time wrap the {\LangtRuleElemInXXEFontFamily{{\fontspec[Scale=0.8]{Arial}\textcolor[rgb]{0,0.4,0.2}{\textbf{get case from}}}}} element around the {\LangtRuleElemInXXEFontFamily{{\fontspec[Scale=0.8]{Arial}\textcolor[rgb]{0,0.4,0.2}{\textbf{clip}}}}} element that copies the lemma. This is shown below: }\par{}{\vspace{12pt plus 2pt minus 1pt}\raggedright{}\XLingPaperexample{.125in}{0pt}{2.75em}{\raisebox{\baselineskip}[0pt]{\protect\hypertarget{xGetCase2}{}}(82)}{\parbox[t]{\textwidth - .125in - 0pt}{\vspace*{-\baselineskip}{\XeTeXpicfile "../Images/RulesTutGetCase2.PNG" scaled 750}}}
\vspace{12pt plus 2pt minus 1pt}}{{\textit{Trick: Click on the word "clip" and type Ctrl-a. This will execute the last command you did in the {\LangtToolFontFamily{\textbf{\textcolor[rgb]{0,0,0.5019607843137255}{XMLmind XML Editor}}}} again (namely wrap an element with the {\LangtRuleElemInXXEFontFamily{{\fontspec[Scale=0.8]{Arial}\textcolor[rgb]{0,0.4,0.2}{\textbf{get case from}}}}} element.)}}}}\indent {\textbf{Transfer}} and {\textbf{Synthesize}} again.\par{}\indent Now we get:\par{}}\par{}{\vspace{12pt plus 2pt minus 1pt}\raggedright{}\XLingPaperexample{.125in}{0pt}{2.75em}{\raisebox{\baselineskip}[0pt]{\protect\hypertarget{xOutputStep4b}{}}(83)}{{\LanglVernacularFontFamily{{\fontspec[Scale=0.9]{Courier New}\textup{\textbf{You left quietly and without wordgen sg}}}}}\ }
\vspace{12pt plus 2pt minus 1pt}}{\par\indent Looking good!\par{}}}
\vspace{\baselineskip}
}
\vspace{12pt}
\XLingPaperneedspace{5\baselineskip}

\penalty-3000{\noindent{\raisebox{\baselineskip}[0pt]{\protect\hypertarget{step5}{}}\SectionLevelFourFontFamily{\normalsize{\raisebox{\baselineskip}[0pt]{\pdfbookmark[4]{6.2.4.5 Step 5}{step5}}\textit{6.2.4.5 Step 5}}}}
\markright{Step 5}
\XLingPaperaddtocontents{step5}}\par{}\penalty10000
\vspace{12pt}
\indent The last remaining thing to do is to not output the genitive or singular tags on the noun. We’re going to have to make a whole new rule to match nouns.\par{}{\parskip .5pt plus 1pt minus 1pt
                    
\vspace{\baselineskip}

{\setlength{\XLingPapertempdim}{\XLingPapersingledigitlistitemwidth+\parindent{}}\leftskip\XLingPapertempdim\relax
\interlinepenalty10000
\XLingPaperlistitem{\parindent{}}{\XLingPapersingledigitlistitemwidth}{1.}{Define a category for nouns.}\par{}{\vspace{12pt plus 2pt minus 1pt}\raggedright{}\XLingPaperexample{.125in}{0pt}{2.75em}{\raisebox{\baselineskip}[0pt]{\protect\hypertarget{xCatN}{}}(84)}{\parbox[t]{\textwidth - .125in - 0pt}{\vspace*{-\baselineskip}{\XeTeXpicfile "../Images/RulesTutCatN.PNG" scaled 750}}}
\vspace{12pt plus 2pt minus 1pt}}{}}
{\setlength{\XLingPapertempdim}{\XLingPapersingledigitlistitemwidth+\parindent{}}\leftskip\XLingPapertempdim\relax
\interlinepenalty10000
\XLingPaperlistitem{\parindent{}}{\XLingPapersingledigitlistitemwidth}{2.}{Make a new rule which matches nouns as defined by the previous category. To start with, we’ll just output the whole lexical unit. }\par{}{\vspace{12pt plus 2pt minus 1pt}\raggedright{}\XLingPaperexample{.125in}{0pt}{2.75em}{\raisebox{\baselineskip}[0pt]{\protect\hypertarget{xRuleN}{}}(85)}{\parbox[t]{\textwidth - .125in - 0pt}{\vspace*{-\baselineskip}{\XeTeXpicfile "../Images/RulesTutRuleN.PNG" scaled 750}}}
\vspace{12pt plus 2pt minus 1pt}}{}}
{\setlength{\XLingPapertempdim}{\XLingPapersingledigitlistitemwidth+\parindent{}}\leftskip\XLingPapertempdim\relax
\interlinepenalty10000
\XLingPaperlistitem{\parindent{}}{\XLingPapersingledigitlistitemwidth}{3.}{Add nouns to the grammatical {\LangtRuleElemInXXEFontFamily{{\fontspec[Scale=0.8]{Arial}\textcolor[rgb]{0,0.4,0.2}{\textbf{category attribute}}}}} element. Now it looks like this:}\par{}{\vspace{12pt plus 2pt minus 1pt}\raggedright{}\XLingPaperexample{.125in}{0pt}{2.75em}{\raisebox{\baselineskip}[0pt]{\protect\hypertarget{xAttrWithN}{}}(86)}{\parbox[t]{\textwidth - .125in - 0pt}{\vspace*{-\baselineskip}{\XeTeXpicfile "../Images/RulesTutAttrGramCatWNoun.PNG" scaled 750}}}
\vspace{12pt plus 2pt minus 1pt}}{}}
{\setlength{\XLingPapertempdim}{\XLingPapersingledigitlistitemwidth+\parindent{}}\leftskip\XLingPapertempdim\relax
\interlinepenalty10000
\XLingPaperlistitem{\parindent{}}{\XLingPapersingledigitlistitemwidth}{4.}{Adjust the rule to output only the lemma and grammatical category.}\par{}{\vspace{12pt plus 2pt minus 1pt}\raggedright{}\XLingPaperexample{.125in}{0pt}{2.75em}{\raisebox{\baselineskip}[0pt]{\protect\hypertarget{xLexUnitNoun}{}}(87)}{\parbox[t]{\textwidth - .125in - 0pt}{\vspace*{-\baselineskip}{\XeTeXpicfile "../Images/RulesTutLexUnitNoun1.PNG" scaled 750}}}
\vspace{12pt plus 2pt minus 1pt}}{\par{}\setlength{\parindent}{1em}\par\indent Now test it in the {\LangtToolFontFamily{\textbf{\textcolor[rgb]{0,0,0.5019607843137255}{Live Rule Tester Tool}}}}, but this time check only the Croatian word {\textit{pozdrava}}. Note: if you refresh the rules, the new Nouns rule will not automatically get checked in the rule list. Check the box for the new rule before testing it.\par{}\setlength{\parindent}{1em}\indent Starting with:\par{}}\par{}{\vspace{12pt plus 2pt minus 1pt}\raggedright{}\XLingPaperexample{.125in}{0pt}{2.75em}{\raisebox{\baselineskip}[0pt]{\protect\hypertarget{xStep5Source}{}}(88)}{{\LanglVernacularFontFamily{{\fontspec[Scale=0.9]{Courier New}\textup{\textbf{pozdrav{{\fontspec[Scale=0.65]{Times New Roman}\textsubscript{1.1}}} {\LangtluGrammCatFontFamily{{\fontspec[Scale=0.9]{Courier New}\textcolor[rgb]{0,0.4392156862745098,0.7529411764705882}{n}}}} {\LangtluAffixFontFamily{{\fontspec[Scale=0.9]{Courier New}\textcolor[rgb]{0,0.6901960784313725,0.3137254901960784}{gen\_sg}}}}}}}}}\ }
\vspace{12pt plus 2pt minus 1pt}}{\par\indent The result is:\par{}}\par{}{\vspace{12pt plus 2pt minus 1pt}\raggedright{}\XLingPaperexample{.125in}{0pt}{2.75em}{\raisebox{\baselineskip}[0pt]{\protect\hypertarget{xStep5Target}{}}(89)}{{\LanglVernacularFontFamily{{\fontspec[Scale=0.9]{Courier New}\textup{\textbf{word{{\fontspec[Scale=0.65]{Times New Roman}\textsubscript{1.1}}} {\LangtluGrammCatFontFamily{{\fontspec[Scale=0.9]{Courier New}\textcolor[rgb]{0,0.4392156862745098,0.7529411764705882}{n}}}}}}}}}\ }
\vspace{12pt plus 2pt minus 1pt}}{\par\indent And if we {\textbf{Synthesize}} we get:\par{}}\par{}{\vspace{12pt plus 2pt minus 1pt}\raggedright{}\XLingPaperexample{.125in}{0pt}{2.75em}{\raisebox{\baselineskip}[0pt]{\protect\hypertarget{xStep5aTarget}{}}(90)}{{\LanglVernacularFontFamily{{\fontspec[Scale=0.9]{Courier New}\textup{\textbf{word}}}}}\ }
\vspace{12pt plus 2pt minus 1pt}}{\par\indent Close the {\LangtToolFontFamily{\textbf{\textcolor[rgb]{0,0,0.5019607843137255}{Live Rule Tester Tool}}}} and run the modules in the {\LangtCollectionFontFamily{{\fontspec[Scale=0.8]{Arial}\textup{\textbf{\textcolor[rgb]{0.4,0,0.4}{Drafting}}}}}} to see what our English sentence looks like.\par{}\indent Now we get:\par{}}\par{}{\vspace{12pt plus 2pt minus 1pt}\raggedright{}\XLingPaperexample{.125in}{0pt}{2.75em}{\raisebox{\baselineskip}[0pt]{\protect\hypertarget{xOutputStep5}{}}(91)}{\LanglGlossFontFamily{\textit{You left quietly and without word}}\ }
\vspace{12pt plus 2pt minus 1pt}}{\par\indent Success!\par{}}}
\vspace{\baselineskip}
}
\vspace{12pt}
\XLingPaperneedspace{5\baselineskip}

\penalty-3000{\noindent{\raisebox{\baselineskip}[0pt]{\protect\hypertarget{step6}{}}\SectionLevelFourFontFamily{\normalsize{\raisebox{\baselineskip}[0pt]{\pdfbookmark[4]{6.2.4.6 Step 6}{step6}}\textit{6.2.4.6 Step 6}}}}
\markright{Step 6}
\XLingPaperaddtocontents{step6}}\par{}\penalty10000
\vspace{12pt}
\indent Step 6 is left as an exercise for the reader. Change the noun rule to output an indefinite article before the noun. You can do this by following the instructions for adding the pronoun, see “\hyperlink{step4}{Step 4}”. The string you need to output is {\LanglVernacularFontFamily{{\fontspec[Scale=0.9]{Courier New}\textup{\textbf{a{{\fontspec[Scale=0.65]{Times New Roman}\textsubscript{1.1}}} {\LangtluGrammCatFontFamily{{\fontspec[Scale=0.9]{Courier New}\textcolor[rgb]{0,0.4392156862745098,0.7529411764705882}{indf}}}}}}}}}.\par{}\indent You can find the solution transfer file called {\textit{solution.t1x}} in the {\LangtFoldernameFontFamily{{\fontspec[Scale=0.8]{Tahoma}\textup{\textmd{FLExTrans\textbackslash{}WorkProjects\textbackslash{}Croatian-English}}}}} folder.\par{}\indent You may think this was a lot of work to translate one sentence from Croatian to English, but consider the fact that now every Croatian past tense construction that occurs in a text will correctly be translated to English and every indefinite noun will also be handled.\par{}
\vspace{12pt}
\XLingPaperneedspace{5\baselineskip}

\penalty-3000{\noindent{\raisebox{\baselineskip}[0pt]{\protect\hypertarget{sBestPractices}{}}\SectionLevelTwoFontFamily{\normalsize{\raisebox{\baselineskip}[0pt]{\pdfbookmark[2]{6.3 Best Practices}{sBestPractices}}\textbf{6.3 Best Practices}}}}
\markright{Best Practices}
\XLingPaperaddtocontents{sBestPractices}}\par{}\penalty10000
\vspace{12pt}
\indent Here are some tips for best practices when writing rules.\par{}{\parskip .5pt plus 1pt minus 1pt
                    
\vspace{\baselineskip}

{\setlength{\XLingPapertempdim}{\XLingPaperdoubledigitlistitemwidth+\parindent{}}\leftskip\XLingPapertempdim\relax
\interlinepenalty10000
\XLingPaperlistitem{\parindent{}}{\XLingPaperdoubledigitlistitemwidth}{1.}{Check that the categories and attributes contain all the needed tags. Failure to do this may mean that the rules cannot apply to all situations.}}
{\setlength{\XLingPapertempdim}{\XLingPaperdoubledigitlistitemwidth+\parindent{}}\leftskip\XLingPapertempdim\relax
\interlinepenalty10000
\XLingPaperlistitem{\parindent{}}{\XLingPaperdoubledigitlistitemwidth}{2.}{Use categories to screen out which rules get run rather than having complex conditional logic in one rule. For example, a category could be used to group together all the words from different grammatical categories that have a locational suffix.}}
{\setlength{\XLingPapertempdim}{\XLingPaperdoubledigitlistitemwidth+\parindent{}}\leftskip\XLingPapertempdim\relax
\interlinepenalty10000
\XLingPaperlistitem{\parindent{}}{\XLingPaperdoubledigitlistitemwidth}{3.}{Use a macro when sections of rules are repeated. This makes the rules easier to maintain, and read.}}
{\setlength{\XLingPapertempdim}{\XLingPaperdoubledigitlistitemwidth+\parindent{}}\leftskip\XLingPapertempdim\relax
\interlinepenalty10000
\XLingPaperlistitem{\parindent{}}{\XLingPaperdoubledigitlistitemwidth}{4.}{Add comments into complex rules in order to help yourself, and others, know/remember how the different sections are being used. Insert comments from the {\LangtToolFontFamily{\textbf{\textcolor[rgb]{0,0,0.5019607843137255}{XMLmind XML Editor}}}} menus - Edit / Comment / Insert Comment.}}
{\setlength{\XLingPapertempdim}{\XLingPaperdoubledigitlistitemwidth+\parindent{}}\leftskip\XLingPapertempdim\relax
\interlinepenalty10000
\XLingPaperlistitem{\parindent{}}{\XLingPaperdoubledigitlistitemwidth}{5.}{Use literal string (blank) when deleting something (not literal tag). If literal tag is used, then the start tag marker (\textless{}) and end tag marker (\textgreater{}) will be used, essentially creating a blank tag rather than 'nothing'.}}
{\setlength{\XLingPapertempdim}{\XLingPaperdoubledigitlistitemwidth+\parindent{}}\leftskip\XLingPapertempdim\relax
\interlinepenalty10000
\XLingPaperlistitem{\parindent{}}{\XLingPaperdoubledigitlistitemwidth}{6.}{In a choose statement, remember that all the blocks of logic below the test block will get run. Check the indentation levels to ensure that all required blocks are inside the choose statement.}}
{\setlength{\XLingPapertempdim}{\XLingPaperdoubledigitlistitemwidth+\parindent{}}\leftskip\XLingPapertempdim\relax
\interlinepenalty10000
\XLingPaperlistitem{\parindent{}}{\XLingPaperdoubledigitlistitemwidth}{7.}{It is helpful to group rules together by categories. e.g. keep all the verb rules together, and all the noun rules together. This makes it easier to find the required rule.}}
{\setlength{\XLingPapertempdim}{\XLingPaperdoubledigitlistitemwidth+\parindent{}}\leftskip\XLingPapertempdim\relax
\interlinepenalty10000
\XLingPaperlistitem{\parindent{}}{\XLingPaperdoubledigitlistitemwidth}{8.}{Remember it is possible to test re-ordering the rules in the {\LangtToolFontFamily{\textbf{\textcolor[rgb]{0,0,0.5019607843137255}{Live Rule Tester Tool}}}}. Changing the order here does not change the order in the rules file, so if the re-ordering in the {\LangtToolFontFamily{\textbf{\textcolor[rgb]{0,0,0.5019607843137255}{Live Rule Tester Tool}}}} is successful, the rules file should also be re-ordered in the same manner.}}
{\setlength{\XLingPapertempdim}{\XLingPaperdoubledigitlistitemwidth+\parindent{}}\leftskip\XLingPapertempdim\relax
\interlinepenalty10000
\XLingPaperlistitem{\parindent{}}{\XLingPaperdoubledigitlistitemwidth}{9.}{Once a rule is working for some sample data, add it to the {\LangtToolFontFamily{\textbf{\textcolor[rgb]{0,0,0.5019607843137255}{Testbed}}}}. This allows tests to be easily re-run on the same data later on, after the addition of rules, or changes to rules.}}
{\setlength{\XLingPapertempdim}{\XLingPaperdoubledigitlistitemwidth+\parindent{}}\leftskip\XLingPapertempdim\relax
\interlinepenalty10000
\XLingPaperlistitem{\parindent{}}{\XLingPaperdoubledigitlistitemwidth}{10.}{After making changes to a rule, re-run the {\LangtToolFontFamily{\textbf{\textcolor[rgb]{0,0,0.5019607843137255}{Testbed}}}} to ensure that all the rules are still producing the expected results.}}
\vspace{\baselineskip}
}
\vspace{12pt}
\XLingPaperneedspace{5\baselineskip}

\penalty-3000{{\centering\raisebox{\baselineskip}[0pt]{\protect\hypertarget{sTestbed}{}}\SectionLevelOneFontFamily{\large{\raisebox{\baselineskip}[0pt]{\pdfbookmark[1]{7 The FLExTrans Testbed}{sTestbed}}\textbf{7 The {\LangtToolFontFamily{\textbf{\textcolor[rgb]{0,0,0.5019607843137255}{FLExTrans}}}} {\LangtToolFontFamily{\textbf{\textcolor[rgb]{0,0,0.5019607843137255}{Testbed}}}}}}}\\{}}\markright{The {\LangtToolFontFamily{\textbf{\textcolor[rgb]{0,0,0.5019607843137255}{FLExTrans}}}} {\LangtToolFontFamily{\textbf{\textcolor[rgb]{0,0,0.5019607843137255}{Testbed}}}}}
\XLingPaperaddtocontents{sTestbed}}\par{}\penalty10000
\vspace{12pt}
\indent The {\LangtToolFontFamily{\textbf{\textcolor[rgb]{0,0,0.5019607843137255}{FLExTrans}}}} {\LangtToolFontFamily{\textbf{\textcolor[rgb]{0,0,0.5019607843137255}{Testbed}}}} is a system designed to help you maintain the quality of your translations. It not uncommon, as you add new transfer rules to the system or change the lexicons, you mess up the results you were getting before. To help prevent this, it is very helpful to have a database of expected results for a certain inputs. This is called a testbed. After making significant changes to your translation system, you can re-run the testbed to make sure you are still getting the results you expect for words or sentences that you earlier deemed correct.\par{}
\vspace{12pt}
\XLingPaperneedspace{5\baselineskip}

\penalty-3000{\noindent{\raisebox{\baselineskip}[0pt]{\protect\hypertarget{sTestbedWorkflow}{}}\SectionLevelTwoFontFamily{\normalsize{\raisebox{\baselineskip}[0pt]{\pdfbookmark[2]{7.1 Workflow with the Testbed}{sTestbedWorkflow}}\textbf{7.1 Workflow with the {\LangtToolFontFamily{\textbf{\textcolor[rgb]{0,0,0.5019607843137255}{Testbed}}}}}}}}
\markright{Workflow with the {\LangtToolFontFamily{\textbf{\textcolor[rgb]{0,0,0.5019607843137255}{Testbed}}}}}
\XLingPaperaddtocontents{sTestbedWorkflow}}\par{}\penalty10000
\vspace{12pt}
\indent This is the way you normally work with the testbed:\par{}{\parskip .5pt plus 1pt minus 1pt
                    
\vspace{\baselineskip}

{\setlength{\XLingPapertempdim}{\XLingPapersingledigitlistitemwidth+\parindent{}}\leftskip\XLingPapertempdim\relax
\interlinepenalty10000
\XLingPaperlistitem{\parindent{}}{\XLingPapersingledigitlistitemwidth}{1.}{Make a change to the system, e.g. add a new rule or other changes.}}
{\setlength{\XLingPapertempdim}{\XLingPapersingledigitlistitemwidth+\parindent{}}\leftskip\XLingPapertempdim\relax
\interlinepenalty10000
\XLingPaperlistitem{\parindent{}}{\XLingPapersingledigitlistitemwidth}{2.}{Use the {\LangtToolFontFamily{\textbf{\textcolor[rgb]{0,0,0.5019607843137255}{Live Rule Tester Tool}}}} to verify you are getting the results you expect.}}
{\setlength{\XLingPapertempdim}{\XLingPapersingledigitlistitemwidth+\parindent{}}\leftskip\XLingPapertempdim\relax
\interlinepenalty10000
\XLingPaperlistitem{\parindent{}}{\XLingPapersingledigitlistitemwidth}{3.}{Click the {\textup{\textbf{Add to Testbed}}} button to add one or more tests that prove source words are correctly translating to the target.}}
{\setlength{\XLingPapertempdim}{\XLingPapersingledigitlistitemwidth+\parindent{}}\leftskip\XLingPapertempdim\relax
\interlinepenalty10000
\XLingPaperlistitem{\parindent{}}{\XLingPapersingledigitlistitemwidth}{4.}{Now run the testbed to verify that all your previous tests still give the expected result.}{\setlength{\XLingPaperlistitemindent}{\XLingPapersingledigitlistitemwidth + \parindent{}}
{\setlength{\XLingPapertempdim}{\XLingPapersingleletterlistitemwidth+\XLingPaperlistitemindent}\leftskip\XLingPapertempdim\relax
\interlinepenalty10000
\XLingPaperlistitem{\XLingPaperlistitemindent}{\XLingPapersingleletterlistitemwidth}{a.}{Switch to the {\LangtCollectionFontFamily{{\fontspec[Scale=0.8]{Arial}\textup{\textbf{\textcolor[rgb]{0.4,0,0.4}{Run Testbed}}}}}} collection.}}
{\setlength{\XLingPapertempdim}{\XLingPapersingleletterlistitemwidth+\XLingPaperlistitemindent}\leftskip\XLingPapertempdim\relax
\interlinepenalty10000
\XLingPaperlistitem{\XLingPaperlistitemindent}{\XLingPapersingleletterlistitemwidth}{b.}{Click the {\textup{\textbf{Run All}}} button to run all the modules.}}
{\setlength{\XLingPapertempdim}{\XLingPapersingleletterlistitemwidth+\XLingPaperlistitemindent}\leftskip\XLingPapertempdim\relax
\interlinepenalty10000
\XLingPaperlistitem{\XLingPaperlistitemindent}{\XLingPapersingleletterlistitemwidth}{c.}{Review the {\LangtToolFontFamily{\textbf{\textcolor[rgb]{0,0,0.5019607843137255}{Testbed Log}}}} which will open at the end and show you the results of running the testbed.}}}}
{\setlength{\XLingPapertempdim}{\XLingPapersingledigitlistitemwidth+\parindent{}}\leftskip\XLingPapertempdim\relax
\interlinepenalty10000
\XLingPaperlistitem{\parindent{}}{\XLingPapersingledigitlistitemwidth}{5.}{Make any adjustments as needed. Repeat. If you need to edit the testbed, open the file {\textit{testbed.xml}} which is in the project folder in the {\LangtToolFontFamily{\textbf{\textcolor[rgb]{0,0,0.5019607843137255}{XMLmind XML Editor}}}}.}}
\vspace{\baselineskip}
}
\vspace{12pt}
\XLingPaperneedspace{5\baselineskip}

\penalty-3000{\noindent{\raisebox{\baselineskip}[0pt]{\protect\hypertarget{sTestbedTools}{}}\SectionLevelTwoFontFamily{\normalsize{\raisebox{\baselineskip}[0pt]{\pdfbookmark[2]{7.2 Testbed Tools}{sTestbedTools}}\textbf{7.2 {\LangtToolFontFamily{\textbf{\textcolor[rgb]{0,0,0.5019607843137255}{Testbed}}}} Tools}}}}
\markright{{\LangtToolFontFamily{\textbf{\textcolor[rgb]{0,0,0.5019607843137255}{Testbed}}}} Tools}
\XLingPaperaddtocontents{sTestbedTools}}\par{}\penalty10000
\vspace{12pt}
\indent {\LangtToolFontFamily{\textbf{\textcolor[rgb]{0,0,0.5019607843137255}{FLExTrans}}}} has a collection of tools to help you maintain the testbed, run the testbed and see the results.\par{}
\vspace{12pt}
\XLingPaperneedspace{5\baselineskip}

\penalty-3000{\noindent{\raisebox{\baselineskip}[0pt]{\protect\hypertarget{sTestbedLogViewer}{}}\SectionLevelThreeFontFamily{\normalsize{\raisebox{\baselineskip}[0pt]{\pdfbookmark[3]{7.2.1 Testbed Log Viewer}{sTestbedLogViewer}}\textit{7.2.1 {\LangtToolFontFamily{\textbf{\textcolor[rgb]{0,0,0.5019607843137255}{Testbed Log Viewer}}}}}}}}
\markright{{\LangtToolFontFamily{\textbf{\textcolor[rgb]{0,0,0.5019607843137255}{Testbed Log Viewer}}}}}
\XLingPaperaddtocontents{sTestbedLogViewer}}\par{}\penalty10000
\vspace{12pt}
\indent This module you will find in both the {\LangtCollectionFontFamily{{\fontspec[Scale=0.8]{Arial}\textup{\textbf{\textcolor[rgb]{0.4,0,0.4}{Tools}}}}}} and {\LangtCollectionFontFamily{{\fontspec[Scale=0.8]{Arial}\textup{\textbf{\textcolor[rgb]{0.4,0,0.4}{Run Testbed}}}}}} collections. Simply run the module and you will see a log of all the times the testbed has been run with a summary of the results. The most recently run test opens and lets you see the details for each test. Hover over the Source Lexical Units of a test to see some pop-up text of which {\LangtToolFontFamily{\textbf{\textcolor[rgb]{0,0,0.5019607843137255}{FLEx}}}} source text the words came from. If the rule was invalid, the pop-up text will also tell you why {\LangtToolFontFamily{\textbf{\textcolor[rgb]{0,0,0.5019607843137255}{FLExTrans}}}} found the test invalid.\par{}{\vspace{12pt plus 2pt minus 1pt}\raggedright{}\XLingPaperexample{.125in}{0pt}{2.75em}{\raisebox{\baselineskip}[0pt]{\protect\hypertarget{xViewerImage}{}}(92)}{\parbox[t]{\textwidth - .125in - 0pt}{\vspace*{-\baselineskip}{\XeTeXpicfile "../Images/TestbedLogViewer.PNG" scaled 700}}}
}
\vspace{12pt}
\XLingPaperneedspace{5\baselineskip}

\penalty-3000{\noindent{\raisebox{\baselineskip}[0pt]{\protect\hypertarget{sStartTestbed}{}}\SectionLevelThreeFontFamily{\normalsize{\raisebox{\baselineskip}[0pt]{\pdfbookmark[3]{7.2.2 Start Testbed}{sStartTestbed}}\textit{7.2.2 {\LangtToolFontFamily{\textbf{\textcolor[rgb]{0,0,0.5019607843137255}{Start Testbed}}}}}}}}
\markright{{\LangtToolFontFamily{\textbf{\textcolor[rgb]{0,0,0.5019607843137255}{Start Testbed}}}}}
\XLingPaperaddtocontents{sStartTestbed}}\par{}\penalty10000
\vspace{12pt}
\indent This module you will find at the top of the {\LangtCollectionFontFamily{{\fontspec[Scale=0.8]{Arial}\textup{\textbf{\textcolor[rgb]{0.4,0,0.4}{Run Testbed}}}}}} collection. It initializes a test run and dumps all of the source lexical units in the testbed into the {\textbf{\textcolor[rgb]{0.5882352941176471,0.29411764705882354,0}{Analyzed Text Output}}} as defined in the {\LangtToolFontFamily{\textbf{\textcolor[rgb]{0,0,0.5019607843137255}{FLExTrans}}}} {\LangtToolFontFamily{\textbf{\textcolor[rgb]{0,0,0.5019607843137255}{FLExTrans Settings}}}} (the default is the{\textit{source\_text.txt}} file).\par{}
\vspace{12pt}
\XLingPaperneedspace{5\baselineskip}

\penalty-3000{\noindent{\raisebox{\baselineskip}[0pt]{\protect\hypertarget{sEndTestbed}{}}\SectionLevelThreeFontFamily{\normalsize{\raisebox{\baselineskip}[0pt]{\pdfbookmark[3]{7.2.3 End Testbed}{sEndTestbed}}\textit{7.2.3 {\LangtToolFontFamily{\textbf{\textcolor[rgb]{0,0,0.5019607843137255}{End Testbed}}}}}}}}
\markright{{\LangtToolFontFamily{\textbf{\textcolor[rgb]{0,0,0.5019607843137255}{End Testbed}}}}}
\XLingPaperaddtocontents{sEndTestbed}}\par{}\penalty10000
\vspace{12pt}
\indent This module you will find as the next to last module in the {\LangtCollectionFontFamily{{\fontspec[Scale=0.8]{Arial}\textup{\textbf{\textcolor[rgb]{0.4,0,0.4}{Run Testbed}}}}}} collection. It takes the results of running the transfer and synthesis processes (the {\textit{source\_text.txt}} file) and adds the tests run and whether they passed or not to the testbed log.\par{}
\vspace{12pt}
\XLingPaperneedspace{5\baselineskip}

\penalty-3000{\noindent{\raisebox{\baselineskip}[0pt]{\protect\hypertarget{sTestbedLRT}{}}\SectionLevelThreeFontFamily{\normalsize{\raisebox{\baselineskip}[0pt]{\pdfbookmark[3]{7.2.4 Live Rule Tester Tool}{sTestbedLRT}}\textit{7.2.4 {\LangtToolFontFamily{\textbf{\textcolor[rgb]{0,0,0.5019607843137255}{Live Rule Tester Tool}}}}}}}}
\markright{{\LangtToolFontFamily{\textbf{\textcolor[rgb]{0,0,0.5019607843137255}{Live Rule Tester Tool}}}}}
\XLingPaperaddtocontents{sTestbedLRT}}\par{}\penalty10000
\vspace{12pt}
\indent The {\LangtToolFontFamily{\textbf{\textcolor[rgb]{0,0,0.5019607843137255}{Live Rule Tester Tool}}}} is a good place to add new tests to the testbed. See “\hyperlink{sRuleTester}{The {\LangtToolFontFamily{\textbf{\textcolor[rgb]{0,0,0.5019607843137255}{Live Rule Tester Tool}}}}}” for more details.\par{}
\vspace{12pt}
\XLingPaperneedspace{5\baselineskip}

\penalty-3000{\noindent{\raisebox{\baselineskip}[0pt]{\protect\hypertarget{sTestbedEditor}{}}\SectionLevelThreeFontFamily{\normalsize{\raisebox{\baselineskip}[0pt]{\pdfbookmark[3]{7.2.5 Testbed Editing}{sTestbedEditor}}\textit{7.2.5 Testbed Editing}}}}
\markright{Testbed Editing}
\XLingPaperaddtocontents{sTestbedEditor}}\par{}\penalty10000
\vspace{12pt}
\indent It’s easiest to add new tests to the testbed while in the {\LangtToolFontFamily{\textbf{\textcolor[rgb]{0,0,0.5019607843137255}{Live Rule Tester Tool}}}}. If you need to edit or delete a test, edit the file {\textit{testbed.xml}} in the project folder with the {\LangtToolFontFamily{\textbf{\textcolor[rgb]{0,0,0.5019607843137255}{XMLmind XML Editor}}}}.\par{}
\vspace{12pt}
\XLingPaperneedspace{5\baselineskip}

\penalty-3000{{\centering\raisebox{\baselineskip}[0pt]{\protect\hypertarget{sHowTo}{}}\SectionLevelOneFontFamily{\large{\raisebox{\baselineskip}[0pt]{\pdfbookmark[1]{8 FLExTrans How To’s}{sHowTo}}\textbf{8 {\LangtToolFontFamily{\textbf{\textcolor[rgb]{0,0,0.5019607843137255}{FLExTrans}}}} How To’s}}}\\{}}\markright{{\LangtToolFontFamily{\textbf{\textcolor[rgb]{0,0,0.5019607843137255}{FLExTrans}}}} How To’s}
\XLingPaperaddtocontents{sHowTo}}\par{}\penalty10000
\vspace{12pt}

\vspace{12pt}
\XLingPaperneedspace{5\baselineskip}

\penalty-3000{\noindent{\raisebox{\baselineskip}[0pt]{\protect\hypertarget{sOneVerse}{}}\SectionLevelTwoFontFamily{\normalsize{\raisebox{\baselineskip}[0pt]{\pdfbookmark[2]{8.1 Complete Process for Translating One Verse (video)}{sOneVerse}}\textbf{8.1 Complete Process for Translating One Verse (video)}}}}
\markright{Complete Process for Translating One Verse (video)}
\XLingPaperaddtocontents{sOneVerse}}\par{}\penalty10000
\vspace{12pt}
\indent \href{https://vimeo.com/934390273}{\vspace*{0pt}{\XeTeXpicfile "../Images/CompleteVerseVideo.png" scaled 750}}\par{}
\vspace{12pt}
\XLingPaperneedspace{5\baselineskip}

\penalty-3000{\noindent{\raisebox{\baselineskip}[0pt]{\protect\hypertarget{sTransferRuleHowTos}{}}\SectionLevelTwoFontFamily{\normalsize{\raisebox{\baselineskip}[0pt]{\pdfbookmark[2]{8.2 Transfer Rule How To’s}{sTransferRuleHowTos}}\textbf{8.2 Transfer Rule How To’s}}}}
\markright{Transfer Rule How To’s}
\XLingPaperaddtocontents{sTransferRuleHowTos}}\par{}\penalty10000
\vspace{12pt}

\vspace{12pt}
\XLingPaperneedspace{5\baselineskip}

\penalty-3000{\noindent{\raisebox{\baselineskip}[0pt]{\protect\hypertarget{sAffixRef}{}}\SectionLevelThreeFontFamily{\normalsize{\raisebox{\baselineskip}[0pt]{\pdfbookmark[3]{8.2.1 How do I refer to affix glosses in transfer rules?}{sAffixRef}}\textit{8.2.1 How do I refer to affix glosses in transfer rules?}}}}
\markright{How do I refer to affix glosses in transfer rules?}
\XLingPaperaddtocontents{sAffixRef}}\par{}\penalty10000
\vspace{12pt}
\indent Important: if the glosses of any of your affixes have a period in them, you refer to them replacing the period with an underscore. For example, {\LangtluAffixFontFamily{{\fontspec[Scale=0.9]{Courier New}\textcolor[rgb]{0,0.6901960784313725,0.3137254901960784}{3s.S}}}} becomes {\LangtluAffixFontFamily{{\fontspec[Scale=0.9]{Courier New}\textcolor[rgb]{0,0.6901960784313725,0.3137254901960784}{3s\_S}}}}.\par{}\indent You generally refer to an affix using the {\LangtRuleElemInXXEFontFamily{{\fontspec[Scale=0.8]{Arial}\textcolor[rgb]{0,0.4,0.2}{\textbf{clip}}}}} element in your transfer rules. The {\LangtRuleAttribInXXEFontFamily{{\fontspec[Scale=0.8]{Arial}\textcolor[rgb]{1,0.6,0.4}{\textbf{part}}}}} attribute of the {\LangtRuleElemInXXEFontFamily{{\fontspec[Scale=0.8]{Arial}\textcolor[rgb]{0,0.4,0.2}{\textbf{clip}}}}} element references an attribute usually and this attribute consists of potential tags which can be features or affix glosses.\par{}\indent If you are comparing a {\LangtRuleElemInXXEFontFamily{{\fontspec[Scale=0.8]{Arial}\textcolor[rgb]{0,0.4,0.2}{\textbf{clip}}}}} element to see if it matches an affix, you should use the {\LangtRuleElemInXXEFontFamily{{\fontspec[Scale=0.8]{Arial}\textcolor[rgb]{0,0.4,0.2}{\textbf{literal tag}}}}} element and write the gloss there. Remember the period-underscore substitution above.\par{}
\vspace{12pt}
\XLingPaperneedspace{5\baselineskip}

\penalty-3000{\noindent{\raisebox{\baselineskip}[0pt]{\protect\hypertarget{sCategory}{}}\SectionLevelThreeFontFamily{\normalsize{\raisebox{\baselineskip}[0pt]{\pdfbookmark[3]{8.2.2 How do I use the category element in the transfer rules?}{sCategory}}\textit{8.2.2 How do I use the category element in the transfer rules?}}}}
\markright{How do I use the category element in the transfer rules?}
\XLingPaperaddtocontents{sCategory}}\par{}\penalty10000
\vspace{12pt}
\indent Lexical categories are used to group words together. The grouping can be very broad such as verbs or very specific such as pronouns with feminine suffixes. The categories are used in the pattern matching system of transfer rules.\par{}{\vspace{12pt plus 2pt minus 1pt}\raggedright{}\XLingPaperexample{.125in}{0pt}{2.75em}{\raisebox{\baselineskip}[0pt]{\protect\hypertarget{xCat}{}}(93)}{\parbox[t]{\textwidth - .125in - 0pt}{\vspace*{-\baselineskip}{\XeTeXpicfile "../Images/CategoryEx.PNG" scaled 750}}}
\vspace{12pt plus 2pt minus 1pt}}\par\indent \hyperlink{xCat}{(93)} is an example of a lexical category for indefinite nominals. The period indicates where a new tag begins. The asterisk is a wildcard indicator. It means that anything can fill that position. An asterisk at the end of the item matches one or more final tags. For example, if I have the lexical unit {\LanglVernacularFontFamily{{\fontspec[Scale=0.9]{Courier New}\textup{\textbf{book}}}}}{{\fontspec[Scale=0.65]{Times New Roman}\textsubscript{1.1}}} {\LangtluGrammCatFontFamily{{\fontspec[Scale=0.9]{Courier New}\textcolor[rgb]{0,0.4392156862745098,0.7529411764705882}{n}}}} {\LangtluAffixFontFamily{{\fontspec[Scale=0.9]{Courier New}\textcolor[rgb]{0,0.6901960784313725,0.3137254901960784}{f}}}} {\LangtluAffixFontFamily{{\fontspec[Scale=0.9]{Courier New}\textcolor[rgb]{0,0.6901960784313725,0.3137254901960784}{sg}}}}, it could be precisely matched by a {\LangtRuleElemInXXEFontFamily{{\fontspec[Scale=0.8]{Arial}\textcolor[rgb]{0,0.4,0.2}{\textbf{tags}}}}} element containing {\LangtCourierFontFamily{{n.f.sg}}}. More generally we could match {\LanglVernacularFontFamily{{\fontspec[Scale=0.9]{Courier New}\textup{\textbf{book}}}}} with the {\LangtRuleElemInXXEFontFamily{{\fontspec[Scale=0.8]{Arial}\textcolor[rgb]{0,0.4,0.2}{\textbf{tags}}}}} element containing {\LangtCourierFontFamily{{n.f.*}}}. This would refer to all words that have the grammatical category {\LangtluGrammCatFontFamily{{\fontspec[Scale=0.9]{Courier New}\textcolor[rgb]{0,0.4392156862745098,0.7529411764705882}{n}}}} followed by the tag {\LangtluAffixFontFamily{{\fontspec[Scale=0.9]{Courier New}\textcolor[rgb]{0,0.6901960784313725,0.3137254901960784}{f}}}} followed by anything else, i.e. feminine nouns. \hyperlink{xCat}{(93)} defines the set of all indefinite nouns — words that have the grammatical category {\LangtluGrammCatFontFamily{{\fontspec[Scale=0.9]{Courier New}\textcolor[rgb]{0,0.4392156862745098,0.7529411764705882}{n}}}} or {\LangtluGrammCatFontFamily{{\fontspec[Scale=0.9]{Courier New}\textcolor[rgb]{0,0.4392156862745098,0.7529411764705882}{n-irreg}}}} followed by {\LangtluAffixFontFamily{{\fontspec[Scale=0.9]{Courier New}\textcolor[rgb]{0,0.6901960784313725,0.3137254901960784}{ind}}}} and optionally something else afterward. Note that {\LangtCourierFontFamily{{n.ind}}} and {\LangtCourierFontFamily{{n.ind.*}}} are both necessary because {\LangtCourierFontFamily{{n.ind.*}}} would require some affix after {\LangtluAffixFontFamily{{\fontspec[Scale=0.9]{Courier New}\textcolor[rgb]{0,0.6901960784313725,0.3137254901960784}{ind}}}} and not match an indefinite noun like {\LanglVernacularFontFamily{{\fontspec[Scale=0.9]{Courier New}\textup{\textbf{car}}}}}{{\fontspec[Scale=0.65]{Times New Roman}\textsubscript{1.1}}} {\LangtluGrammCatFontFamily{{\fontspec[Scale=0.9]{Courier New}\textcolor[rgb]{0,0.4392156862745098,0.7529411764705882}{n}}}} {\LangtluAffixFontFamily{{\fontspec[Scale=0.9]{Courier New}\textcolor[rgb]{0,0.6901960784313725,0.3137254901960784}{ind}}}} that has no additional suffixes.\par{}\indent If you use {\LangtCourierFontFamily{{*}}} in the middle of the {\LangtRuleElemInXXEFontFamily{{\fontspec[Scale=0.8]{Arial}\textcolor[rgb]{0,0.4,0.2}{\textbf{tags}}}}} definition, it means one or more tags occur in that position. For example, {\LangtCourierFontFamily{{n.*.pl}}} would match both {\LangtluGrammCatFontFamily{{\fontspec[Scale=0.9]{Courier New}\textcolor[rgb]{0,0.4392156862745098,0.7529411764705882}{n}}}} {\LangtluAffixFontFamily{{\fontspec[Scale=0.9]{Courier New}\textcolor[rgb]{0,0.6901960784313725,0.3137254901960784}{f pl}}}} and {\LangtluGrammCatFontFamily{{\fontspec[Scale=0.9]{Courier New}\textcolor[rgb]{0,0.4392156862745098,0.7529411764705882}{n}}}} {\LangtluAffixFontFamily{{\fontspec[Scale=0.9]{Courier New}\textcolor[rgb]{0,0.6901960784313725,0.3137254901960784}{ind f pl}}}}. Important: unlike the meaning of {\LangtCourierFontFamily{{*}}} in regular expressions, it never means zero occurrences. So, {\LangtCourierFontFamily{{n.*.pl}}} would not match {\LangtluGrammCatFontFamily{{\fontspec[Scale=0.9]{Courier New}\textcolor[rgb]{0,0.4392156862745098,0.7529411764705882}{n}}}} {\LangtluAffixFontFamily{{\fontspec[Scale=0.9]{Courier New}\textcolor[rgb]{0,0.6901960784313725,0.3137254901960784}{pl}}}}. To match this you would need an additional {\LangtRuleElemInXXEFontFamily{{\fontspec[Scale=0.8]{Arial}\textcolor[rgb]{0,0.4,0.2}{\textbf{tags}}}}} element {\LangtCourierFontFamily{{n.pl}}} to cover this case.\par{}\indent You can use the lemma element to identify a specific word-sense. \hyperlink{xCatLem}{(94)} shows how the category {\LangtCourierFontFamily{{dem\_this}}} can be defined as a word that has a grammatical category of {\LangtluGrammCatFontFamily{{\fontspec[Scale=0.9]{Courier New}\textcolor[rgb]{0,0.4392156862745098,0.7529411764705882}{dem}}}} and lemma {\LanglVernacularFontFamily{{\fontspec[Scale=0.9]{Courier New}\textup{\textbf{this}}}}}{{\fontspec[Scale=0.65]{Times New Roman}\textsubscript{1.1}}} (in {\LangtToolFontFamily{\textbf{\textcolor[rgb]{0,0,0.5019607843137255}{FLEx}}}}: headword {\LanglVernacularFontFamily{{\fontspec[Scale=0.9]{Courier New}\textup{\textbf{this}}}}}{{\fontspec[Scale=0.65]{Times New Roman}\textsubscript{1}}} , sense 1). This might be useful when you want to match phrases containing the word {\LangtCourierFontFamily{{this}}}.\par{}{\vspace{12pt plus 2pt minus 1pt}\raggedright{}\XLingPaperexample{.125in}{0pt}{2.75em}{\raisebox{\baselineskip}[0pt]{\protect\hypertarget{xCatLem}{}}(94)}{\parbox[t]{\textwidth - .125in - 0pt}{\vspace*{-\baselineskip}{\XeTeXpicfile "../Images/CategoryLemmaEx.PNG" scaled 750}}}
}
\vspace{12pt}
\XLingPaperneedspace{5\baselineskip}

\penalty-3000{\noindent{\raisebox{\baselineskip}[0pt]{\protect\hypertarget{sAttribute}{}}\SectionLevelThreeFontFamily{\normalsize{\raisebox{\baselineskip}[0pt]{\pdfbookmark[3]{8.2.3 How do I use the attribute element in the transfer rules?}{sAttribute}}\textit{8.2.3 How do I use the attribute element in the transfer rules?}}}}
\markright{How do I use the attribute element in the transfer rules?}
\XLingPaperaddtocontents{sAttribute}}\par{}\penalty10000
\vspace{12pt}
\indent Attributes are defined in the rule file in order to identify possible values for word characteristics. An example is shown in \hyperlink{xAttrib}{(95)}.\par{}{\vspace{12pt plus 2pt minus 1pt}\raggedright{}\XLingPaperexample{.125in}{0pt}{2.75em}{\raisebox{\baselineskip}[0pt]{\protect\hypertarget{xAttrib}{}}(95)}{\parbox[t]{\textwidth - .125in - 0pt}{\vspace*{-\baselineskip}{\XeTeXpicfile "../Images/AttributeEx.PNG" scaled 750}}}
\vspace{12pt plus 2pt minus 1pt}}\par\indent Certain words may have affixes that indicate grammatical number. \hyperlink{xAttrib}{(95)} defines the possible values for the number attribute as being sg (singular) or pl (plural). Like lexical categories, these refer to tags in the data stream. Attributes are used within rules to get or set values.\par{}
\vspace{12pt}
\XLingPaperneedspace{5\baselineskip}

\penalty-3000{\noindent{\raisebox{\baselineskip}[0pt]{\protect\hypertarget{sPatternMatch}{}}\SectionLevelThreeFontFamily{\normalsize{\raisebox{\baselineskip}[0pt]{\pdfbookmark[3]{8.2.4 How does pattern matching work?}{sPatternMatch}}\textit{8.2.4 How does pattern matching work?}}}}
\markright{How does pattern matching work?}
\XLingPaperaddtocontents{sPatternMatch}}\par{}\penalty10000
\vspace{12pt}
\indent The {\LangtToolFontFamily{\textbf{\textcolor[rgb]{0,0,0.5019607843137255}{Apertium}}}} transfer engine searches for patterns in your source text. The patterns it searches for are those defined for each rule you have. Some patterns may be one word, some patterns may be multiple words. The transfer engine tries to match the longest patterns first and then progressively goes to shorter and shorter patterns. This means a five-word pattern would be used before a three-word pattern (assuming a set of words would match both patterns.) When there are multiple patterns of the same length, the first one listed gets precedence. This means in some cases the order of your rules will be important.\par{}\indent The transfer engine searches for matches in your words in sequential order starting with the first word. It will never find a match at some point in a text and then go back earlier in the text.\par{}\indent Another important concept is that the transfer engine processes words just once for whatever pattern is matched. After they are processed, the words are not examined again for any other matches. In other words, patterns cannot apply in an overlapping manner. For example, if you only have two rules, one that matches {\textit{determiner-noun}} and another one that matches {\textit{noun-adjective}}, when a phrase of the form {\textit{determiner-noun-adjective}} is processed, the engine uses the rule that is listed first and then discards the {\textit{adjective}} as there is no rule that matches this alone, and then the engine continues on to other words. The second rule does not get applied to that phrase. Another way to describe it is to say that the engine processes words in distinct chunks.\par{}\indent If the {\LangtToolFontFamily{\textbf{\textcolor[rgb]{0,0,0.5019607843137255}{Apertium}}}} transfer engine finds a match, it runs the action part of your rule. If the engine finds no match for a word, it does default translation of it according to what is in the bilingual dictionary. Cf. “\hyperlink{sRulesApplied}{How Rules are Applied}”.\par{}
\vspace{12pt}
\XLingPaperneedspace{5\baselineskip}

\penalty-3000{\noindent{\raisebox{\baselineskip}[0pt]{\protect\hypertarget{sDeleteWord}{}}\SectionLevelThreeFontFamily{\normalsize{\raisebox{\baselineskip}[0pt]{\pdfbookmark[3]{8.2.5 How do I delete a word from the target output? (How do I prevent a source word from being transferred to the target text?)}{sDeleteWord}}\textit{8.2.5 How do I delete a word from the target output? (How do I prevent a source word from being transferred to the target text?)}}}}
\markright{How do I delete a word from the target output? (How do I prevent a source word from being transferred to the target text?)}
\XLingPaperaddtocontents{sDeleteWord}}\par{}\penalty10000
\vspace{12pt}
\indent The short answer is don’t output it. In other words, don’t declare that it should be outputted into the target data stream.\par{}\indent When you match two or more words in the {\LangtRuleElemInXXEFontFamily{{\fontspec[Scale=0.8]{Arial}\textcolor[rgb]{0,0.4,0.2}{\textbf{pattern}}}}} section of a transfer rule, you can choose to only list some of the words in the {\LangtRuleElemInXXEFontFamily{{\fontspec[Scale=0.8]{Arial}\textcolor[rgb]{0,0.4,0.2}{\textbf{output}}}}} section. This is exactly what we did in “\hyperlink{step1}{Step 1}” of the tutorial. In \hyperlink{xOutput1}{(60)} we only declared that the word in {\LangtRuleElemInXXEFontFamily{{\fontspec[Scale=0.8]{Arial}\textcolor[rgb]{0,0.4,0.2}{\textbf{position}}}}} 1 should be outputted.\par{}
\vspace{12pt}
\XLingPaperneedspace{5\baselineskip}

\penalty-3000{\noindent{\raisebox{\baselineskip}[0pt]{\protect\hypertarget{sInsertWord}{}}\SectionLevelThreeFontFamily{\normalsize{\raisebox{\baselineskip}[0pt]{\pdfbookmark[3]{8.2.6 How do I insert a word into the target output?}{sInsertWord}}\textit{8.2.6 How do I insert a word into the target output?}}}}
\markright{How do I insert a word into the target output?}
\XLingPaperaddtocontents{sInsertWord}}\par{}\penalty10000
\vspace{12pt}
\indent Naturally to add something to the target output, we need to add something to the {\LangtRuleElemInXXEFontFamily{{\fontspec[Scale=0.8]{Arial}\textcolor[rgb]{0,0.4,0.2}{\textbf{output}}}}} part of our transfer rule. And what goes in {\LangtRuleElemInXXEFontFamily{{\fontspec[Scale=0.8]{Arial}\textcolor[rgb]{0,0.4,0.2}{\textbf{output}}}}} part the of our transfer rule? Lexical units. We need to insert a lexical unit block under the {\LangtRuleElemInXXEFontFamily{{\fontspec[Scale=0.8]{Arial}\textcolor[rgb]{0,0.4,0.2}{\textbf{output}}}}} element of our transfer rule. {\uline{The bare minimum we need for a lexical unit is the word-sense and the grammatical category.}} If we are manually outputting a word-sense, we need to use a {\LangtRuleElemInXXEFontFamily{{\fontspec[Scale=0.8]{Arial}\textcolor[rgb]{0,0.4,0.2}{\textbf{literal string}}}}} element and the text needs to be of the form word1.2. Where 1 would be the homograph number (if there isn’t one in {\LangtToolFontFamily{\textbf{\textcolor[rgb]{0,0,0.5019607843137255}{FLEx}}}}, use 1) and 2 would be the sense number - in this case the second sense of the word. \hyperlink{xInsertWord}{(96)} shows how it might look.\par{}{\vspace{12pt plus 2pt minus 1pt}\raggedright{}\XLingPaperexample{.125in}{0pt}{2.75em}{\raisebox{\baselineskip}[0pt]{\protect\hypertarget{xInsertWord}{}}(96)}{\parbox[t]{\textwidth - .125in - 0pt}{\vspace*{-\baselineskip}{\XeTeXpicfile "../Images/InsertWord.png" scaled 750}}}
\vspace{12pt plus 2pt minus 1pt}}\par\indent Of course you can output affixes or clitics as well. These would be additional {\LangtRuleElemInXXEFontFamily{{\fontspec[Scale=0.8]{Arial}\textcolor[rgb]{0,0.4,0.2}{\textbf{literal string}}}}} elements. Remember the first literal tag is the grammatical category, all subsequent tags are affixes and the like. If you have a valid word-sense and its corresponding grammatical category, {\LangtToolFontFamily{\textbf{\textcolor[rgb]{0,0,0.5019607843137255}{FLExTrans}}}} will look up this word-sense in the target {\LangtToolFontFamily{\textbf{\textcolor[rgb]{0,0,0.5019607843137255}{FLEx}}}} lexicon and use it in the final synthesis process.\par{}\indent If you have other lexical units that you will be outputting, remember to put a {\LangtRuleElemInXXEFontFamily{{\fontspec[Scale=0.8]{Arial}\textcolor[rgb]{0,0.4,0.2}{\textbf{blank space}}}}} element so that you get a space between your words. This is shown in \hyperlink{xLexUnitPron}{(75)}. “\hyperlink{step4}{Step 4}” of the transfer rules tutorial shows an example of inserting a word into the target output.\par{}
\vspace{12pt}
\XLingPaperneedspace{5\baselineskip}

\penalty-3000{\noindent{\raisebox{\baselineskip}[0pt]{\protect\hypertarget{sDeleteAffix}{}}\SectionLevelThreeFontFamily{\normalsize{\raisebox{\baselineskip}[0pt]{\pdfbookmark[3]{8.2.7 How do I delete an affix?}{sDeleteAffix}}\textit{8.2.7 How do I delete an affix?}}}}
\markright{How do I delete an affix?}
\XLingPaperaddtocontents{sDeleteAffix}}\par{}\penalty10000
\vspace{12pt}
\indent There are two methods for deleting an affix. You can either set the value for that affix (attribute) to blank or you can not declare that the affix (attribute) gets outputted.\par{}{\parskip .5pt plus 1pt minus 1pt
                    
\vspace{\baselineskip}

{\setlength{\XLingPapertempdim}{\XLingPapersingledigitlistitemwidth+\parindent{}}\leftskip\XLingPapertempdim\relax
\interlinepenalty10000
\XLingPaperlistitem{\parindent{}}{\XLingPapersingledigitlistitemwidth}{1.}{Method 1. Use the {\LangtRuleElemInXXEFontFamily{{\fontspec[Scale=0.8]{Arial}\textcolor[rgb]{0,0.4,0.2}{\textbf{let}}}}} element to set an attribute’s value to blank. See \hyperlink{xLetBlank}{(97)}.}\par{}{\vspace{12pt plus 2pt minus 1pt}\raggedright{}\XLingPaperexample{.125in}{0pt}{2.75em}{\raisebox{\baselineskip}[0pt]{\protect\hypertarget{xLetBlank}{}}(97)}{\parbox[t]{\textwidth - .125in - 0pt}{\vspace*{-\baselineskip}{\XeTeXpicfile "../Images/LetBlank.PNG" scaled 750}}}
\vspace{12pt plus 2pt minus 1pt}}{Here we are overriding whatever is in the tense attribute (a\_tense) with a blank string. Whenever the tense affix gets outputted, it will be outputted as nothing. As an example, in \hyperlink{xOutputWhole}{(98)} the whole word is outputted with all its attributes.}\par{}{\vspace{12pt plus 2pt minus 1pt}\raggedright{}\XLingPaperexample{.125in}{0pt}{2.75em}{\raisebox{\baselineskip}[0pt]{\protect\hypertarget{xOutputWhole}{}}(98)}{\parbox[t]{\textwidth - .125in - 0pt}{\vspace*{-\baselineskip}{\XeTeXpicfile "../Images/OutputWholeWord.PNG" scaled 750}}}
\vspace{12pt plus 2pt minus 1pt}}{}}
{\setlength{\XLingPapertempdim}{\XLingPapersingledigitlistitemwidth+\parindent{}}\leftskip\XLingPapertempdim\relax
\interlinepenalty10000
\XLingPaperlistitem{\parindent{}}{\XLingPapersingledigitlistitemwidth}{2.}{Method 2. In the {\LangtRuleElemInXXEFontFamily{{\fontspec[Scale=0.8]{Arial}\textcolor[rgb]{0,0.4,0.2}{\textbf{Output}}}}} element, explicitly declare each attribute of the word that you want to output. Don’t include the attribute that needs to be deleted.}\par{}{\vspace{12pt plus 2pt minus 1pt}\raggedright{}\XLingPaperexample{.125in}{0pt}{2.75em}{\raisebox{\baselineskip}[0pt]{\protect\hypertarget{xOutMultAffixes}{}}(99)}{\parbox[t]{\textwidth - .125in - 0pt}{\vspace*{-\baselineskip}{\XeTeXpicfile "../Images/OutputMultAffixes.PNG" scaled 750}}}
\vspace{12pt plus 2pt minus 1pt}}{In \hyperlink{xOutMultAffixes}{(99)} you can see that the lemma is outputted and then the grammatical category and then two attribute. The disadvantage of this method is that you have to explicitly declare every possible attribute that could occur. Where as in \hyperlink{xOutputWhole}{(98)} in Method 1, you just have to output the whole word.}}
\vspace{\baselineskip}
}
\vspace{12pt}
\XLingPaperneedspace{5\baselineskip}

\penalty-3000{\noindent{\raisebox{\baselineskip}[0pt]{\protect\hypertarget{sCondLogic}{}}\SectionLevelThreeFontFamily{\normalsize{\raisebox{\baselineskip}[0pt]{\pdfbookmark[3]{8.2.8 How do I use conditional logic in a transfer rule? (How do I say if this ... then that?) (video)}{sCondLogic}}\textit{8.2.8 How do I use conditional logic in a transfer rule? (How do I say if this ... then that?) (video)\vspace*{0pt}{\XeTeXpicfile "../Images/VideoIcon.PNG" scaled 750}}}}}
\markright{How do I use conditional logic in a transfer rule? (How do I say if this ... then that?) (video)\vspace*{0pt}{\XeTeXpicfile "../Images/VideoIcon.PNG" scaled 750}}
\XLingPaperaddtocontents{sCondLogic}}\par{}\penalty10000
\vspace{12pt}
\indent In the transfer rules you have the ability to use conditional logic. Please watch this \href{https://vimeo.com/934390874}{video} for an explanation. Some conditional logic can be automatically created for you using the {\LangtToolFontFamily{\textbf{\textcolor[rgb]{0,0,0.5019607843137255}{Rule Assistant}}}}. See Section \hyperlink{sRuleAssist}{4.7}. Also see examples of conditional logic in the sample rule called: {\textbf{Sample logic to copy and paste}} or in the file {\textit{transfer\_rules-Sample1.t1x}}.\par{}
\vspace{12pt}
\XLingPaperneedspace{5\baselineskip}

\penalty-3000{\noindent{\raisebox{\baselineskip}[0pt]{\protect\hypertarget{sMacro}{}}\SectionLevelThreeFontFamily{\normalsize{\raisebox{\baselineskip}[0pt]{\pdfbookmark[3]{8.2.9 How do I use a macro? (How can I repeat rule statements in multiple places?) (video)}{sMacro}}\textit{8.2.9 How do I use a macro? (How can I repeat rule statements in multiple places?) (video)\vspace*{0pt}{\XeTeXpicfile "../Images/VideoIcon.PNG" scaled 750}}}}}
\markright{How do I use a macro? (How can I repeat rule statements in multiple places?) (video)\vspace*{0pt}{\XeTeXpicfile "../Images/VideoIcon.PNG" scaled 750}}
\XLingPaperaddtocontents{sMacro}}\par{}\penalty10000
\vspace{12pt}
\indent In the transfer rules you have the ability to modularize your rules by putting repeated logic into a macro. Please watch this \href{https://vimeo.com/934391130}{video} for an explanation.\par{}
\vspace{12pt}
\XLingPaperneedspace{5\baselineskip}

\penalty-3000{\noindent{\raisebox{\baselineskip}[0pt]{\protect\hypertarget{sAgreement}{}}\SectionLevelThreeFontFamily{\normalsize{\raisebox{\baselineskip}[0pt]{\pdfbookmark[3]{8.2.10 How do I handle noun agreement? (How do I get a target word or words to agree with a noun?) (video)}{sAgreement}}\textit{8.2.10 How do I handle noun agreement? (How do I get a target word or words to agree with a noun?) (video)\vspace*{0pt}{\XeTeXpicfile "../Images/VideoIcon.PNG" scaled 750}}}}}
\markright{How do I handle noun agreement? (How do I get a target word or words to agree with a noun?) (video)\vspace*{0pt}{\XeTeXpicfile "../Images/VideoIcon.PNG" scaled 750}}
\XLingPaperaddtocontents{sAgreement}}\par{}\penalty10000
\vspace{12pt}
\indent One main strategy for getting words to agree with a noun, is to make sure features are assigned to the noun and then in the transfer rules make the needed adjustment to the words you output depending on what the feature values (or class values) are. You can use the {\LangtToolFontFamily{\textbf{\textcolor[rgb]{0,0,0.5019607843137255}{Rule Assistant}}}} to do this. See Section \hyperlink{sRuleAssist}{4.7}.\par{}\indent If you can't use the {\LangtToolFontFamily{\textbf{\textcolor[rgb]{0,0,0.5019607843137255}{Rule Assistant}}}} for some reason and you want to see how to do this ‘manually’ in a sample language pair, please watch this \href{https://vimeo.com/934391317}{video}.\par{}\indent If you would like to try creating the rules along with the video, please follow the instructions in the file: {\textit{Readme.txt}} in the {\LangtFoldernameFontFamily{{\fontspec[Scale=0.8]{Tahoma}\textup{\textmd{FLExTrans\textbackslash{}FLExTrans Documentation\textbackslash{}Agreement}}}}} folder. See the file {\textit{transfer\_rules - solution.t1x}} which shows the form of the rules shown in the video.\par{}
\vspace{12pt}
\XLingPaperneedspace{5\baselineskip}

\penalty-3000{\noindent{\raisebox{\baselineskip}[0pt]{\protect\hypertarget{sSampleLogic}{}}\SectionLevelThreeFontFamily{\normalsize{\raisebox{\baselineskip}[0pt]{\pdfbookmark[3]{8.2.11 How do I use the Sample logic that’s in the transfer rules file?}{sSampleLogic}}\textit{8.2.11 How do I use the Sample logic that’s in the transfer rules file?}}}}
\markright{How do I use the Sample logic that’s in the transfer rules file?}
\XLingPaperaddtocontents{sSampleLogic}}\par{}\penalty10000
\vspace{12pt}
\indent Each sample logic block starts with a comment shown in yellow. Read through each of the most outdented comments to see what kind of logic matches your need. When you find one, click on the first black word under the comment. That will select the block (outlined in red). Now you can copy that block and put it into your own rule. There's also a file called {\textit{transfer\_rules-Sample1.t1x}} you can copy and paste from that rule file as well.\par{}
\vspace{12pt}
\XLingPaperneedspace{5\baselineskip}

\penalty-3000{\noindent{\raisebox{\baselineskip}[0pt]{\protect\hypertarget{sSpecialEl}{}}\SectionLevelThreeFontFamily{\normalsize{\raisebox{\baselineskip}[0pt]{\pdfbookmark[3]{8.2.12 How do I use special elements in my rules?}{sSpecialEl}}\textit{8.2.12 How do I use special elements in my rules?}}}}
\markright{How do I use special elements in my rules?}
\XLingPaperaddtocontents{sSpecialEl}}\par{}\penalty10000
\vspace{12pt}

\vspace{12pt}
\XLingPaperneedspace{5\baselineskip}

\penalty-3000{\noindent{\raisebox{\baselineskip}[0pt]{\protect\hypertarget{sAnd}{}}\SectionLevelFourFontFamily{\normalsize{\raisebox{\baselineskip}[0pt]{\pdfbookmark[4]{8.2.12.1 and}{sAnd}}\textit{8.2.12.1 {\LangtRuleElemInXXEFontFamily{{\fontspec[Scale=0.8]{Arial}\textcolor[rgb]{0,0.4,0.2}{\textbf{and}}}}}}}}}
\markright{{\LangtRuleElemInXXEFontFamily{{\fontspec[Scale=0.8]{Arial}\textcolor[rgb]{0,0.4,0.2}{\textbf{and}}}}}}
\XLingPaperaddtocontents{sAnd}}\par{}\penalty10000
\vspace{12pt}
\indent Inside of a {\LangtRuleElemInXXEFontFamily{{\fontspec[Scale=0.8]{Arial}\textcolor[rgb]{0,0.4,0.2}{\textbf{test}}}}} element you normally have an {\LangtRuleElemInXXEFontFamily{{\fontspec[Scale=0.8]{Arial}\textcolor[rgb]{0,0.4,0.2}{\textbf{equal}}}}} element. This allows you to see if x is equal to y. But if you want to test to see if two things are true, you can use the {\LangtRuleElemInXXEFontFamily{{\fontspec[Scale=0.8]{Arial}\textcolor[rgb]{0,0.4,0.2}{\textbf{and}}}}} element and then list two {\LangtRuleElemInXXEFontFamily{{\fontspec[Scale=0.8]{Arial}\textcolor[rgb]{0,0.4,0.2}{\textbf{equal}}}}} elements. The easiest way to bring this element into your rule is to copy an {\LangtRuleElemInXXEFontFamily{{\fontspec[Scale=0.8]{Arial}\textcolor[rgb]{0,0.4,0.2}{\textbf{and}}}}} element from the {\textbf{Sample logic rule}} or from the {\textit{transfer\_rules-Sample1.t1x}} file.\par{}{\vspace{12pt plus 2pt minus 1pt}\raggedright{}\XLingPaperexample{.125in}{0pt}{2.75em}{\raisebox{\baselineskip}[0pt]{\protect\hypertarget{xAnd}{}}(100)}{\parbox[t]{\textwidth - .125in - 0pt}{\vspace*{-\baselineskip}{\XeTeXpicfile "../Images/ElementAnd.png" scaled 750}}}
}
\vspace{12pt}
\XLingPaperneedspace{5\baselineskip}

\penalty-3000{\noindent{\raisebox{\baselineskip}[0pt]{\protect\hypertarget{sApend}{}}\SectionLevelFourFontFamily{\normalsize{\raisebox{\baselineskip}[0pt]{\pdfbookmark[4]{8.2.12.2 append to}{sApend}}\textit{8.2.12.2 {\LangtRuleElemInXXEFontFamily{{\fontspec[Scale=0.8]{Arial}\textcolor[rgb]{0,0.4,0.2}{\textbf{append to}}}}}}}}}
\markright{{\LangtRuleElemInXXEFontFamily{{\fontspec[Scale=0.8]{Arial}\textcolor[rgb]{0,0.4,0.2}{\textbf{append to}}}}}}
\XLingPaperaddtocontents{sApend}}\par{}\penalty10000
\vspace{12pt}
\indent Inside of a {\LangtRuleElemInXXEFontFamily{{\fontspec[Scale=0.8]{Arial}\textcolor[rgb]{0,0.4,0.2}{\textbf{let}}}}} element you normally copy the value of the second element to the first element. See \hyperlink{sDeleteAffix}{8.2.7}, method 1 for an example. With this element you can add to the end of the first element whatever is in the second element. For lemmas and {\LangtRuleElemInXXEFontFamily{{\fontspec[Scale=0.8]{Arial}\textcolor[rgb]{0,0.4,0.2}{\textbf{literal strings}}}}} this is straightforward. For example, you could append {\LanglVernacularFontFamily{{\fontspec[Scale=0.9]{Courier New}\textup{\textbf{1.1}}}}} to the end of {\LanglVernacularFontFamily{{\fontspec[Scale=0.9]{Courier New}\textup{\textbf{praise}}}}} to end up with the value {\LanglVernacularFontFamily{{\fontspec[Scale=0.9]{Courier New}\textup{\textbf{praise1.1}}}}}. For tags this is little different. You can't use append to add on to a tag, you can only use it to add a second tag. You could add the tag {\LangtluAffixFontFamily{{\fontspec[Scale=0.9]{Courier New}\textcolor[rgb]{0,0.6901960784313725,0.3137254901960784}{pl}}}} after the tag {\LangtluAffixFontFamily{{\fontspec[Scale=0.9]{Courier New}\textcolor[rgb]{0,0.6901960784313725,0.3137254901960784}{3}}}} so that there is now two tags {\LangtluAffixFontFamily{{\fontspec[Scale=0.9]{Courier New}\textcolor[rgb]{0,0.6901960784313725,0.3137254901960784}{3}}}} and {\LangtluAffixFontFamily{{\fontspec[Scale=0.9]{Courier New}\textcolor[rgb]{0,0.6901960784313725,0.3137254901960784}{pl}}}}. You can't create a combined tag of {\LangtluAffixFontFamily{{\fontspec[Scale=0.9]{Courier New}\textcolor[rgb]{0,0.6901960784313725,0.3137254901960784}{3pl}}}} this way.\protect\footnote[13]{{\leftskip0pt\parindent1em\raisebox{\baselineskip}[0pt]{\protect\hypertarget{nappend}{}}You could achieve this by taking a literal string with a value of {\LanglVernacularFontFamily{{\fontspec[Scale=0.9]{Courier New}\textup{\textbf{3}}}}} and appending another literal string with a value of {\LanglVernacularFontFamily{{\fontspec[Scale=0.9]{Courier New}\textup{\textbf{pl}}}}}. But you couldn't start with a tag that has a value of {\LangtluAffixFontFamily{{\fontspec[Scale=0.9]{Courier New}\textcolor[rgb]{0,0.6901960784313725,0.3137254901960784}{3}}}}. This is because under the hood, the value of a tag includes the angle brackets. Tag {\LangtluAffixFontFamily{{\fontspec[Scale=0.9]{Courier New}\textcolor[rgb]{0,0.6901960784313725,0.3137254901960784}{3}}}} is actually {\LangtCourierFontFamily{{\textless{}3\textgreater{}}}}.}} The easiest way to produce this element is to simply use the {\LangtMenuFontFamily{\textbf{\textcolor[rgb]{0,0.6,0.2}{Insert}}}} command, to put it into your rule.\par{}\indent Note: you can append multiple things to your primary element. Just add more elements under {\LangtRuleElemInXXEFontFamily{{\fontspec[Scale=0.8]{Arial}\textcolor[rgb]{0,0.4,0.2}{\textbf{append to}}}}}.\par{}{\vspace{12pt plus 2pt minus 1pt}\raggedright{}\XLingPaperexample{.125in}{0pt}{2.75em}{\raisebox{\baselineskip}[0pt]{\protect\hypertarget{xAppend}{}}(101)}{\parbox[t]{\textwidth - .125in - 0pt}{\vspace*{-\baselineskip}{\XeTeXpicfile "../Images/ElementAppend.png" scaled 750}}}
}
\vspace{12pt}
\XLingPaperneedspace{5\baselineskip}

\penalty-3000{\noindent{\raisebox{\baselineskip}[0pt]{\protect\hypertarget{sBeginsWith}{}}\SectionLevelFourFontFamily{\normalsize{\raisebox{\baselineskip}[0pt]{\pdfbookmark[4]{8.2.12.3 begins with}{sBeginsWith}}\textit{8.2.12.3 {\LangtRuleElemInXXEFontFamily{{\fontspec[Scale=0.8]{Arial}\textcolor[rgb]{0,0.4,0.2}{\textbf{begins with}}}}}}}}}
\markright{{\LangtRuleElemInXXEFontFamily{{\fontspec[Scale=0.8]{Arial}\textcolor[rgb]{0,0.4,0.2}{\textbf{begins with}}}}}}
\XLingPaperaddtocontents{sBeginsWith}}\par{}\penalty10000
\vspace{12pt}
\indent Inside of a {\LangtRuleElemInXXEFontFamily{{\fontspec[Scale=0.8]{Arial}\textcolor[rgb]{0,0.4,0.2}{\textbf{test}}}}} element you normally have an {\LangtRuleElemInXXEFontFamily{{\fontspec[Scale=0.8]{Arial}\textcolor[rgb]{0,0.4,0.2}{\textbf{equal}}}}} element. This allows you to see if x is equal to y. But if you want to test to see if something begins with a certain value, you can use the {\LangtRuleElemInXXEFontFamily{{\fontspec[Scale=0.8]{Arial}\textcolor[rgb]{0,0.4,0.2}{\textbf{begins with}}}}} element and then list two elements. The first element is the thing you are checking. The second element contains what you want the first element to begin with. The easiest way to bring this element into your rule is to copy a {\LangtRuleElemInXXEFontFamily{{\fontspec[Scale=0.8]{Arial}\textcolor[rgb]{0,0.4,0.2}{\textbf{begins with}}}}} element from the {\textbf{Sample logic rule}} or from the {\textit{transfer\_rules-Sample1.t1x}} file.\par{}{\vspace{12pt plus 2pt minus 1pt}\raggedright{}\XLingPaperexample{.125in}{0pt}{2.75em}{\raisebox{\baselineskip}[0pt]{\protect\hypertarget{xBeginsWith}{}}(102)}{\parbox[t]{\textwidth - .125in - 0pt}{\vspace*{-\baselineskip}{\XeTeXpicfile "../Images/ElementBeginsWith.png" scaled 750}}}
}
\vspace{12pt}
\XLingPaperneedspace{5\baselineskip}

\penalty-3000{\noindent{\raisebox{\baselineskip}[0pt]{\protect\hypertarget{sBeginsWithList}{}}\SectionLevelFourFontFamily{\normalsize{\raisebox{\baselineskip}[0pt]{\pdfbookmark[4]{8.2.12.4 begins with something in list}{sBeginsWithList}}\textit{8.2.12.4 {\LangtRuleElemInXXEFontFamily{{\fontspec[Scale=0.8]{Arial}\textcolor[rgb]{0,0.4,0.2}{\textbf{begins with something in list}}}}}}}}}
\markright{{\LangtRuleElemInXXEFontFamily{{\fontspec[Scale=0.8]{Arial}\textcolor[rgb]{0,0.4,0.2}{\textbf{begins with something in list}}}}}}
\XLingPaperaddtocontents{sBeginsWithList}}\par{}\penalty10000
\vspace{12pt}
\indent Inside of a {\LangtRuleElemInXXEFontFamily{{\fontspec[Scale=0.8]{Arial}\textcolor[rgb]{0,0.4,0.2}{\textbf{test}}}}} element you normally have an {\LangtRuleElemInXXEFontFamily{{\fontspec[Scale=0.8]{Arial}\textcolor[rgb]{0,0.4,0.2}{\textbf{equal}}}}} element. This allows you to see if x is equal to y. But if you want to test to see if something begins with a certain value, you can use the {\LangtRuleElemInXXEFontFamily{{\fontspec[Scale=0.8]{Arial}\textcolor[rgb]{0,0.4,0.2}{\textbf{begins with something in list}}}}} element and then list two elements. The first element is the thing you are checking. The second element contains the name of a list. The text to match can be anything in the given list. Typically, the first element is "lem" and the things in the list are lemmas. Note: if you want to match an attribute, then the things in the list have to be surrounded by angle brackets. E.g. {\LangtluAffixFontFamily{{\fontspec[Scale=0.9]{Courier New}\textcolor[rgb]{0,0.6901960784313725,0.3137254901960784}{\textless{}PL\textgreater{}}}}}. The list has to be defined separately in the {\LangtRuleElemInXXEFontFamily{{\fontspec[Scale=0.8]{Arial}\textcolor[rgb]{0,0.4,0.2}{\textbf{Lists}}}}} section of the rules file. The easiest way to bring this element into your rule is to copy a {\LangtRuleElemInXXEFontFamily{{\fontspec[Scale=0.8]{Arial}\textcolor[rgb]{0,0.4,0.2}{\textbf{begins with something in list}}}}} element from the {\textbf{Sample logic rule}} or from the {\textit{transfer\_rules-Sample1.t1x}} file.\par{}{\vspace{12pt plus 2pt minus 1pt}\raggedright{}\XLingPaperexample{.125in}{0pt}{2.75em}{\raisebox{\baselineskip}[0pt]{\protect\hypertarget{xBeginsWithList}{}}(103)}{\parbox[t]{\textwidth - .125in - 0pt}{\vspace*{-\baselineskip}{\XeTeXpicfile "../Images/ElementBeginsWithList.png" scaled 750}}}
}
\vspace{12pt}
\XLingPaperneedspace{5\baselineskip}

\penalty-3000{\noindent{\raisebox{\baselineskip}[0pt]{\protect\hypertarget{sConcat}{}}\SectionLevelFourFontFamily{\normalsize{\raisebox{\baselineskip}[0pt]{\pdfbookmark[4]{8.2.12.5 concat}{sConcat}}\textit{8.2.12.5 {\LangtRuleElemInXXEFontFamily{{\fontspec[Scale=0.8]{Arial}\textcolor[rgb]{0,0.4,0.2}{\textbf{concat}}}}}}}}}
\markright{{\LangtRuleElemInXXEFontFamily{{\fontspec[Scale=0.8]{Arial}\textcolor[rgb]{0,0.4,0.2}{\textbf{concat}}}}}}
\XLingPaperaddtocontents{sConcat}}\par{}\penalty10000
\vspace{12pt}
\indent This element concatenates two or more things together, with the example below giving an end result of {\textit{v\_sample = ab}}.\par{}\indent The same thing regarding tags and literal strings as discussed in \hyperlink{sApend}{8.2.12.2} applies.\par{}{\vspace{12pt plus 2pt minus 1pt}\raggedright{}\XLingPaperexample{.125in}{0pt}{2.75em}{\raisebox{\baselineskip}[0pt]{\protect\hypertarget{xConcat}{}}(104)}{\parbox[t]{\textwidth - .125in - 0pt}{\vspace*{-\baselineskip}{\XeTeXpicfile "../Images/ElementConcat.png" scaled 750}}}
}
\vspace{12pt}
\XLingPaperneedspace{5\baselineskip}

\penalty-3000{\noindent{\raisebox{\baselineskip}[0pt]{\protect\hypertarget{sContainsSub}{}}\SectionLevelFourFontFamily{\normalsize{\raisebox{\baselineskip}[0pt]{\pdfbookmark[4]{8.2.12.6 contains substring}{sContainsSub}}\textit{8.2.12.6 {\LangtRuleElemInXXEFontFamily{{\fontspec[Scale=0.8]{Arial}\textcolor[rgb]{0,0.4,0.2}{\textbf{contains substring}}}}}}}}}
\markright{{\LangtRuleElemInXXEFontFamily{{\fontspec[Scale=0.8]{Arial}\textcolor[rgb]{0,0.4,0.2}{\textbf{contains substring}}}}}}
\XLingPaperaddtocontents{sContainsSub}}\par{}\penalty10000
\vspace{12pt}
\indent Inside of a {\LangtRuleElemInXXEFontFamily{{\fontspec[Scale=0.8]{Arial}\textcolor[rgb]{0,0.4,0.2}{\textbf{test}}}}} element you normally have an {\LangtRuleElemInXXEFontFamily{{\fontspec[Scale=0.8]{Arial}\textcolor[rgb]{0,0.4,0.2}{\textbf{equal}}}}} element. This allows you to see if x is equal to y. But if you want to test to see if something ends with a certain value, you can use the {\LangtRuleElemInXXEFontFamily{{\fontspec[Scale=0.8]{Arial}\textcolor[rgb]{0,0.4,0.2}{\textbf{contains substring}}}}} element and then list two elements. The first element is the thing you are checking. The second element contains what you want to find in the first element. The easiest way to bring this element into your rule is to copy a {\LangtRuleElemInXXEFontFamily{{\fontspec[Scale=0.8]{Arial}\textcolor[rgb]{0,0.4,0.2}{\textbf{contains substring}}}}} element from the {\textbf{Sample logic rule}} or from the {\textit{transfer\_rules-Sample1.t1x}} file.\par{}{\vspace{12pt plus 2pt minus 1pt}\raggedright{}\XLingPaperexample{.125in}{0pt}{2.75em}{\raisebox{\baselineskip}[0pt]{\protect\hypertarget{xContainsSub}{}}(105)}{\parbox[t]{\textwidth - .125in - 0pt}{\vspace*{-\baselineskip}{\XeTeXpicfile "../Images/ElementContainsSubstring.png" scaled 750}}}
}
\vspace{12pt}
\XLingPaperneedspace{5\baselineskip}

\penalty-3000{\noindent{\raisebox{\baselineskip}[0pt]{\protect\hypertarget{sCaseOf}{}}\SectionLevelFourFontFamily{\normalsize{\raisebox{\baselineskip}[0pt]{\pdfbookmark[4]{8.2.12.7 case of}{sCaseOf}}\textit{8.2.12.7 {\LangtRuleElemInXXEFontFamily{{\fontspec[Scale=0.8]{Arial}\textcolor[rgb]{0,0.4,0.2}{\textbf{case of}}}}}}}}}
\markright{{\LangtRuleElemInXXEFontFamily{{\fontspec[Scale=0.8]{Arial}\textcolor[rgb]{0,0.4,0.2}{\textbf{case of}}}}}}
\XLingPaperaddtocontents{sCaseOf}}\par{}\penalty10000
\vspace{12pt}
\indent Not needed. See \hyperlink{sGetCaseFrom}{8.2.12.10} or \hyperlink{sModifyCase}{8.2.12.12}.\par{}
\vspace{12pt}
\XLingPaperneedspace{5\baselineskip}

\penalty-3000{\noindent{\raisebox{\baselineskip}[0pt]{\protect\hypertarget{sEndsWith}{}}\SectionLevelFourFontFamily{\normalsize{\raisebox{\baselineskip}[0pt]{\pdfbookmark[4]{8.2.12.8 ends with}{sEndsWith}}\textit{8.2.12.8 {\LangtRuleElemInXXEFontFamily{{\fontspec[Scale=0.8]{Arial}\textcolor[rgb]{0,0.4,0.2}{\textbf{ends with}}}}}}}}}
\markright{{\LangtRuleElemInXXEFontFamily{{\fontspec[Scale=0.8]{Arial}\textcolor[rgb]{0,0.4,0.2}{\textbf{ends with}}}}}}
\XLingPaperaddtocontents{sEndsWith}}\par{}\penalty10000
\vspace{12pt}
\indent Inside of a {\LangtRuleElemInXXEFontFamily{{\fontspec[Scale=0.8]{Arial}\textcolor[rgb]{0,0.4,0.2}{\textbf{test}}}}} element you normally have an {\LangtRuleElemInXXEFontFamily{{\fontspec[Scale=0.8]{Arial}\textcolor[rgb]{0,0.4,0.2}{\textbf{equal}}}}} element. This allows you to see if x is equal to y. But if you want to test to see if something ends with a certain value, you can use the {\LangtRuleElemInXXEFontFamily{{\fontspec[Scale=0.8]{Arial}\textcolor[rgb]{0,0.4,0.2}{\textbf{begins with}}}}} element and then list two elements. The first element is the thing you are checking. The second element contains what you want the first element to end with. The easiest way to bring this element into your rule is to copy a {\LangtRuleElemInXXEFontFamily{{\fontspec[Scale=0.8]{Arial}\textcolor[rgb]{0,0.4,0.2}{\textbf{begins with}}}}} element from the {\textbf{Sample logic rule}} or from the {\textit{transfer\_rules-Sample1.t1x}} file.\par{}{\vspace{12pt plus 2pt minus 1pt}\raggedright{}\XLingPaperexample{.125in}{0pt}{2.75em}{\raisebox{\baselineskip}[0pt]{\protect\hypertarget{xEndsWith}{}}(106)}{\parbox[t]{\textwidth - .125in - 0pt}{\vspace*{-\baselineskip}{\XeTeXpicfile "../Images/ElementEndsWith.png" scaled 750}}}
}
\vspace{12pt}
\XLingPaperneedspace{5\baselineskip}

\penalty-3000{\noindent{\raisebox{\baselineskip}[0pt]{\protect\hypertarget{sEndsWithList}{}}\SectionLevelFourFontFamily{\normalsize{\raisebox{\baselineskip}[0pt]{\pdfbookmark[4]{8.2.12.9 ends with something in list}{sEndsWithList}}\textit{8.2.12.9 {\LangtRuleElemInXXEFontFamily{{\fontspec[Scale=0.8]{Arial}\textcolor[rgb]{0,0.4,0.2}{\textbf{ends with something in list}}}}}}}}}
\markright{{\LangtRuleElemInXXEFontFamily{{\fontspec[Scale=0.8]{Arial}\textcolor[rgb]{0,0.4,0.2}{\textbf{ends with something in list}}}}}}
\XLingPaperaddtocontents{sEndsWithList}}\par{}\penalty10000
\vspace{12pt}
\indent Inside of a {\LangtRuleElemInXXEFontFamily{{\fontspec[Scale=0.8]{Arial}\textcolor[rgb]{0,0.4,0.2}{\textbf{test}}}}} element you normally have an {\LangtRuleElemInXXEFontFamily{{\fontspec[Scale=0.8]{Arial}\textcolor[rgb]{0,0.4,0.2}{\textbf{equal}}}}} element. This allows you to see if x is equal to y. But if you want to test to see if something ends with a certain value, you can use the {\LangtRuleElemInXXEFontFamily{{\fontspec[Scale=0.8]{Arial}\textcolor[rgb]{0,0.4,0.2}{\textbf{ends with something in list}}}}} element and then list two elements. The first element is the thing you are checking. The second element contains the name of a list. The text to match can be anything in the given list. Typically, the first element is "lem" and the things in the list are lemmas. Note: if you want to match an attribute, then the things in the list have to be surrounded by angle brackets. E.g. {\LangtluAffixFontFamily{{\fontspec[Scale=0.9]{Courier New}\textcolor[rgb]{0,0.6901960784313725,0.3137254901960784}{\textless{}PL\textgreater{}}}}}. The list has to be defined separately in the {\LangtRuleElemInXXEFontFamily{{\fontspec[Scale=0.8]{Arial}\textcolor[rgb]{0,0.4,0.2}{\textbf{Lists}}}}} section of the rules file. The easiest way to bring this element into your rule is to copy a {\LangtRuleElemInXXEFontFamily{{\fontspec[Scale=0.8]{Arial}\textcolor[rgb]{0,0.4,0.2}{\textbf{ends with something in list}}}}} element from the {\textbf{Sample logic rule}} or from the {\textit{transfer\_rules-Sample1.t1x}} file.\par{}{\vspace{12pt plus 2pt minus 1pt}\raggedright{}\XLingPaperexample{.125in}{0pt}{2.75em}{\raisebox{\baselineskip}[0pt]{\protect\hypertarget{xEndsWithList}{}}(107)}{\parbox[t]{\textwidth - .125in - 0pt}{\vspace*{-\baselineskip}{\XeTeXpicfile "../Images/ElementEndsWithList.png" scaled 750}}}
}
\vspace{12pt}
\XLingPaperneedspace{5\baselineskip}

\penalty-3000{\noindent{\raisebox{\baselineskip}[0pt]{\protect\hypertarget{sGetCaseFrom}{}}\SectionLevelFourFontFamily{\normalsize{\raisebox{\baselineskip}[0pt]{\pdfbookmark[4]{8.2.12.10 get case from item}{sGetCaseFrom}}\textit{8.2.12.10 {\LangtRuleElemInXXEFontFamily{{\fontspec[Scale=0.8]{Arial}\textcolor[rgb]{0,0.4,0.2}{\textbf{get case from item}}}}}}}}}
\markright{{\LangtRuleElemInXXEFontFamily{{\fontspec[Scale=0.8]{Arial}\textcolor[rgb]{0,0.4,0.2}{\textbf{get case from item}}}}}}
\XLingPaperaddtocontents{sGetCaseFrom}}\par{}\penalty10000
\vspace{12pt}
\indent This element lets you change the case of something. It as the same function as the {\LangtRuleElemInXXEFontFamily{{\fontspec[Scale=0.8]{Arial}\textcolor[rgb]{0,0.4,0.2}{\textbf{modify case}}}}} element. But it helps you save the step of having to call {\LangtRuleElemInXXEFontFamily{{\fontspec[Scale=0.8]{Arial}\textcolor[rgb]{0,0.4,0.2}{\textbf{modify case}}}}} as a separate step. As you are using some piece of data, you can wrap this element around it to on-the-fly change its case. So instead of \hyperlink{xModifyCase2}{(111)} as a separate step, could you use this element as in \hyperlink{xGetCaseFrom}{(108)} to output the lemma for item 1, making it's case the same as item 2.\par{}\indent The easiest way to produce this element is to highlight an element such as {\LangtRuleElemInXXEFontFamily{{\fontspec[Scale=0.8]{Arial}\textcolor[rgb]{0,0.4,0.2}{\textbf{clip}}}}} or {\LangtRuleElemInXXEFontFamily{{\fontspec[Scale=0.8]{Arial}\textcolor[rgb]{0,0.4,0.2}{\textbf{variable}}}}} and then use the {\LangtMenuFontFamily{\textbf{\textcolor[rgb]{0,0.6,0.2}{Wrap}}}} command to wrap an element around it and choose {\LangtRuleElemInXXEFontFamily{{\fontspec[Scale=0.8]{Arial}\textcolor[rgb]{0,0.4,0.2}{\textbf{get-case-from}}}}}.\par{}{\vspace{12pt plus 2pt minus 1pt}\raggedright{}\XLingPaperexample{.125in}{0pt}{2.75em}{\raisebox{\baselineskip}[0pt]{\protect\hypertarget{xGetCaseFrom}{}}(108)}{\parbox[t]{\textwidth - .125in - 0pt}{\vspace*{-\baselineskip}{\XeTeXpicfile "../Images/ElementGetCaseFrom.png" scaled 750}}}
}
\vspace{12pt}
\XLingPaperneedspace{5\baselineskip}

\penalty-3000{\noindent{\raisebox{\baselineskip}[0pt]{\protect\hypertarget{sIn}{}}\SectionLevelFourFontFamily{\normalsize{\raisebox{\baselineskip}[0pt]{\pdfbookmark[4]{8.2.12.11 in list}{sIn}}\textit{8.2.12.11 {\LangtRuleElemInXXEFontFamily{{\fontspec[Scale=0.8]{Arial}\textcolor[rgb]{0,0.4,0.2}{\textbf{in list}}}}}}}}}
\markright{{\LangtRuleElemInXXEFontFamily{{\fontspec[Scale=0.8]{Arial}\textcolor[rgb]{0,0.4,0.2}{\textbf{in list}}}}}}
\XLingPaperaddtocontents{sIn}}\par{}\penalty10000
\vspace{12pt}
\indent Inside of a {\LangtRuleElemInXXEFontFamily{{\fontspec[Scale=0.8]{Arial}\textcolor[rgb]{0,0.4,0.2}{\textbf{test}}}}} element you normally have an {\LangtRuleElemInXXEFontFamily{{\fontspec[Scale=0.8]{Arial}\textcolor[rgb]{0,0.4,0.2}{\textbf{equal}}}}} element. This allows you to see if x is equal to y. But if you want to test to see if a value matches something in a list, you can use the {\LangtRuleElemInXXEFontFamily{{\fontspec[Scale=0.8]{Arial}\textcolor[rgb]{0,0.4,0.2}{\textbf{in list}}}}} element and then list two elements. The first element is the thing you are checking. The second element contains the name of a list. The logic will be true if the first element match anything in the given list. Typically, the first element is "lem" and the things in the list are lemmas. Note: if you want to match an attribute, then the things in the list have to be surrounded by angle brackets. E.g. {\LangtluAffixFontFamily{{\fontspec[Scale=0.9]{Courier New}\textcolor[rgb]{0,0.6901960784313725,0.3137254901960784}{\textless{}PL\textgreater{}}}}}. The list has to be defined separately in the {\LangtRuleElemInXXEFontFamily{{\fontspec[Scale=0.8]{Arial}\textcolor[rgb]{0,0.4,0.2}{\textbf{Lists}}}}} section of the rules file. The easiest way to bring this element into your rule is to copy an {\LangtRuleElemInXXEFontFamily{{\fontspec[Scale=0.8]{Arial}\textcolor[rgb]{0,0.4,0.2}{\textbf{in list}}}}} element from the {\textbf{Sample logic rule}} or from the {\textit{transfer\_rules-Sample1.t1x}} file.\par{}{\vspace{12pt plus 2pt minus 1pt}\raggedright{}\XLingPaperexample{.125in}{0pt}{2.75em}{\raisebox{\baselineskip}[0pt]{\protect\hypertarget{xIn}{}}(109)}{\parbox[t]{\textwidth - .125in - 0pt}{\vspace*{-\baselineskip}{\XeTeXpicfile "../Images/ElementIn.png" scaled 750}}}
}
\vspace{12pt}
\XLingPaperneedspace{5\baselineskip}

\penalty-3000{\noindent{\raisebox{\baselineskip}[0pt]{\protect\hypertarget{sModifyCase}{}}\SectionLevelFourFontFamily{\normalsize{\raisebox{\baselineskip}[0pt]{\pdfbookmark[4]{8.2.12.12 modify case}{sModifyCase}}\textit{8.2.12.12 {\LangtRuleElemInXXEFontFamily{{\fontspec[Scale=0.8]{Arial}\textcolor[rgb]{0,0.4,0.2}{\textbf{modify case}}}}}}}}}
\markright{{\LangtRuleElemInXXEFontFamily{{\fontspec[Scale=0.8]{Arial}\textcolor[rgb]{0,0.4,0.2}{\textbf{modify case}}}}}}
\XLingPaperaddtocontents{sModifyCase}}\par{}\penalty10000
\vspace{12pt}
\indent This element lets you change the case of something. Under {\LangtRuleElemInXXEFontFamily{{\fontspec[Scale=0.8]{Arial}\textcolor[rgb]{0,0.4,0.2}{\textbf{modify case}}}}} you have two elements. The first element is the thing that you want to change. The second element instructs the rule engine what kind of case to give it. You can use a literal string to intentionally change it to a certain case. See \hyperlink{xModifyCase}{(110)}. {\LanglVernacularFontFamily{{\fontspec[Scale=0.9]{Courier New}\textup{\textbf{Aa}}}}} will change it to have the first letter capitalized. {\LanglVernacularFontFamily{{\fontspec[Scale=0.9]{Courier New}\textup{\textbf{aa}}}}} will change it to lower case. {\LanglVernacularFontFamily{{\fontspec[Scale=0.9]{Courier New}\textup{\textbf{AA}}}}} will change it to upper case. Or you can have it copy the case of a value that already exists. In \hyperlink{xModifyCase2}{(111)} the case of the lemma of item 1 will be set to be the same case as the lemma of item 2.\par{}\indent The easiest way to produce this element is to simply use the {\LangtMenuFontFamily{\textbf{\textcolor[rgb]{0,0.6,0.2}{Insert}}}} command, to put it into your rule.\par{}{\vspace{12pt plus 2pt minus 1pt}\raggedright{}\XLingPaperexample{.125in}{0pt}{2.75em}{\raisebox{\baselineskip}[0pt]{\protect\hypertarget{xModifyCase}{}}(110)}{\parbox[t]{\textwidth - .125in - 0pt}{\vspace*{-\baselineskip}{\XeTeXpicfile "../Images/ElementModifyCase.png" scaled 750}}}
}{\vspace{12pt plus 2pt minus 1pt}\raggedright{}\XLingPaperexample{.125in}{0pt}{2.75em}{\raisebox{\baselineskip}[0pt]{\protect\hypertarget{xModifyCase2}{}}(111)}{\parbox[t]{\textwidth - .125in - 0pt}{\vspace*{-\baselineskip}{\XeTeXpicfile "../Images/ElementModifyCase2.png" scaled 750}}}
}
\vspace{12pt}
\XLingPaperneedspace{5\baselineskip}

\penalty-3000{\noindent{\raisebox{\baselineskip}[0pt]{\protect\hypertarget{sNot}{}}\SectionLevelFourFontFamily{\normalsize{\raisebox{\baselineskip}[0pt]{\pdfbookmark[4]{8.2.12.13 not}{sNot}}\textit{8.2.12.13 {\LangtRuleElemInXXEFontFamily{{\fontspec[Scale=0.8]{Arial}\textcolor[rgb]{0,0.4,0.2}{\textbf{not}}}}}}}}}
\markright{{\LangtRuleElemInXXEFontFamily{{\fontspec[Scale=0.8]{Arial}\textcolor[rgb]{0,0.4,0.2}{\textbf{not}}}}}}
\XLingPaperaddtocontents{sNot}}\par{}\penalty10000
\vspace{12pt}
\indent Inside of a {\LangtRuleElemInXXEFontFamily{{\fontspec[Scale=0.8]{Arial}\textcolor[rgb]{0,0.4,0.2}{\textbf{test}}}}} element you normally have an {\LangtRuleElemInXXEFontFamily{{\fontspec[Scale=0.8]{Arial}\textcolor[rgb]{0,0.4,0.2}{\textbf{equal}}}}} element. This allows you to see if x is equal to y. But if you want to test to see if x is not equal to y, you can use the {\LangtRuleElemInXXEFontFamily{{\fontspec[Scale=0.8]{Arial}\textcolor[rgb]{0,0.4,0.2}{\textbf{not}}}}} element and then list one {\LangtRuleElemInXXEFontFamily{{\fontspec[Scale=0.8]{Arial}\textcolor[rgb]{0,0.4,0.2}{\textbf{equal}}}}} element inside it. The easiest way to bring this element into your rule is to copy a {\LangtRuleElemInXXEFontFamily{{\fontspec[Scale=0.8]{Arial}\textcolor[rgb]{0,0.4,0.2}{\textbf{not}}}}} element from the {\textbf{Sample logic rule}} or from the {\textit{transfer\_rules-Sample1.t1x}} file.\par{}{\vspace{12pt plus 2pt minus 1pt}\raggedright{}\XLingPaperexample{.125in}{0pt}{2.75em}{\raisebox{\baselineskip}[0pt]{\protect\hypertarget{xNot}{}}(112)}{\parbox[t]{\textwidth - .125in - 0pt}{\vspace*{-\baselineskip}{\XeTeXpicfile "../Images/ElementNot.png" scaled 750}}}
}
\vspace{12pt}
\XLingPaperneedspace{5\baselineskip}

\penalty-3000{\noindent{\raisebox{\baselineskip}[0pt]{\protect\hypertarget{sOr}{}}\SectionLevelFourFontFamily{\normalsize{\raisebox{\baselineskip}[0pt]{\pdfbookmark[4]{8.2.12.14 or}{sOr}}\textit{8.2.12.14 {\LangtRuleElemInXXEFontFamily{{\fontspec[Scale=0.8]{Arial}\textcolor[rgb]{0,0.4,0.2}{\textbf{or}}}}}}}}}
\markright{{\LangtRuleElemInXXEFontFamily{{\fontspec[Scale=0.8]{Arial}\textcolor[rgb]{0,0.4,0.2}{\textbf{or}}}}}}
\XLingPaperaddtocontents{sOr}}\par{}\penalty10000
\vspace{12pt}
\indent Inside of a {\LangtRuleElemInXXEFontFamily{{\fontspec[Scale=0.8]{Arial}\textcolor[rgb]{0,0.4,0.2}{\textbf{test}}}}} element you normally have an {\LangtRuleElemInXXEFontFamily{{\fontspec[Scale=0.8]{Arial}\textcolor[rgb]{0,0.4,0.2}{\textbf{equal}}}}} element. This allows you to see if x is equal to y. But if you want to test to see if either one of two things is true, you can use the {\LangtRuleElemInXXEFontFamily{{\fontspec[Scale=0.8]{Arial}\textcolor[rgb]{0,0.4,0.2}{\textbf{or}}}}} element and then list two {\LangtRuleElemInXXEFontFamily{{\fontspec[Scale=0.8]{Arial}\textcolor[rgb]{0,0.4,0.2}{\textbf{equal}}}}} elements. The easiest way to bring this element into your rule is to copy an {\LangtRuleElemInXXEFontFamily{{\fontspec[Scale=0.8]{Arial}\textcolor[rgb]{0,0.4,0.2}{\textbf{or}}}}} element from the {\textbf{Sample logic rule}} or from the {\textit{transfer\_rules-Sample1.t1x}} file.\par{}{\vspace{12pt plus 2pt minus 1pt}\raggedright{}\XLingPaperexample{.125in}{0pt}{2.75em}{\raisebox{\baselineskip}[0pt]{\protect\hypertarget{xOr}{}}(113)}{\parbox[t]{\textwidth - .125in - 0pt}{\vspace*{-\baselineskip}{\XeTeXpicfile "../Images/ElementOr.png" scaled 750}}}
}
\vspace{12pt}
\XLingPaperneedspace{5\baselineskip}

\penalty-3000{\noindent{\raisebox{\baselineskip}[0pt]{\protect\hypertarget{sOther}{}}\SectionLevelTwoFontFamily{\normalsize{\raisebox{\baselineskip}[0pt]{\pdfbookmark[2]{8.3 Other How To’s}{sOther}}\textbf{8.3 Other How To’s}}}}
\markright{Other How To’s}
\XLingPaperaddtocontents{sOther}}\par{}\penalty10000
\vspace{12pt}

\vspace{12pt}
\XLingPaperneedspace{5\baselineskip}

\penalty-3000{\noindent{\raisebox{\baselineskip}[0pt]{\protect\hypertarget{sViewBiling}{}}\SectionLevelThreeFontFamily{\normalsize{\raisebox{\baselineskip}[0pt]{\pdfbookmark[3]{8.3.1 How do I view the bilingual lexicon?}{sViewBiling}}\textit{8.3.1 How do I view the bilingual lexicon?}}}}
\markright{How do I view the bilingual lexicon?}
\XLingPaperaddtocontents{sViewBiling}}\par{}\penalty10000
\vspace{12pt}
\indent You can view the bilingual lexicon by double-clicking on the file {\textit{bilingual.dix}} in the {\LangtFoldernameFontFamily{{\fontspec[Scale=0.8]{Tahoma}\textup{\textmd{Output}}}}} folder or click on the {\textup{\textbf{View Bilingual Lexicon}}} button in the {\LangtToolFontFamily{\textbf{\textcolor[rgb]{0,0,0.5019607843137255}{Live Rule Tester Tool}}}}. An example portion of a bilingual lexicon is shown in \hyperlink{xBilingEntries0}{(114)}.\par{}{\vspace{12pt plus 2pt minus 1pt}\raggedright{}\XLingPaperexample{.125in}{0pt}{2.75em}{\raisebox{\baselineskip}[0pt]{\protect\hypertarget{xBilingEntries0}{}}(114)}{\parbox[t]{\textwidth - .125in - 0pt}{\vspace*{-\baselineskip}{\XeTeXpicfile "../Images/BilingalEntries2.PNG" scaled 750}}}
}
\vspace{12pt}
\XLingPaperneedspace{5\baselineskip}

\penalty-3000{\noindent{\raisebox{\baselineskip}[0pt]{\protect\hypertarget{sSynthesisSelfTest}{}}\SectionLevelThreeFontFamily{\normalsize{\raisebox{\baselineskip}[0pt]{\pdfbookmark[3]{8.3.2 How do I run a synthesis self-test on a text in my Target Project?}{sSynthesisSelfTest}}\textit{8.3.2 How do I run a synthesis self-test on a text in my Target Project?}}}}
\markright{How do I run a synthesis self-test on a text in my Target Project?}
\XLingPaperaddtocontents{sSynthesisSelfTest}}\par{}\penalty10000
\vspace{12pt}
\indent To test whether a Target Project is set up so the words in a specific text will synthesize correctly (without worrying about the source project or transfer rules), do the following:\par{}{\parskip .5pt plus 1pt minus 1pt
                    
\vspace{\baselineskip}

{\setlength{\XLingPapertempdim}{\XLingPapersingledigitlistitemwidth+\parindent{}}\leftskip\XLingPapertempdim\relax
\interlinepenalty10000
\XLingPaperlistitem{\parindent{}}{\XLingPapersingledigitlistitemwidth}{1.}{In {\LangtToolFontFamily{\textbf{\textcolor[rgb]{0,0,0.5019607843137255}{FLEx}}}}, in your Target Project, create a text with the words you want to test for synthesis. For example, a paradigm, or some canonical sentence frames, or a particular text fragment. Analyze these words and approve all the analyses.}}
{\setlength{\XLingPapertempdim}{\XLingPapersingledigitlistitemwidth+\parindent{}}\leftskip\XLingPapertempdim\relax
\interlinepenalty10000
\XLingPaperlistitem{\parindent{}}{\XLingPapersingledigitlistitemwidth}{2.}{Create or edit a Transfer Rules file that simply deletes features from any category that has them. The file {\textit{identity\_rules.t1x}} in the {\LangtFoldernameFontFamily{{\fontspec[Scale=0.8]{Tahoma}\textup{\textmd{FLExTrans Documentation\textbackslash{}Synthesis Self-Test}}}}} folder is an example of that, set up for the {\textbf{Spanish}} sample project.}}
{\setlength{\XLingPapertempdim}{\XLingPapersingledigitlistitemwidth+\parindent{}}\leftskip\XLingPapertempdim\relax
\interlinepenalty10000
\XLingPaperlistitem{\parindent{}}{\XLingPapersingledigitlistitemwidth}{3.}{In {\LangtToolFontFamily{\textbf{\textcolor[rgb]{0,0,0.5019607843137255}{FLExTools}}}}, set the Source {\LangtToolFontFamily{\textbf{\textcolor[rgb]{0,0,0.5019607843137255}{FLEx}}}} Project to the Target Project. {\textup{\textup{\textmd{\textcolor[rgb]{1,0,0}{(Remember to change this back, when you go back to working with two projects.)}}}}}}}
{\setlength{\XLingPapertempdim}{\XLingPapersingledigitlistitemwidth+\parindent{}}\leftskip\XLingPapertempdim\relax
\interlinepenalty10000
\XLingPaperlistitem{\parindent{}}{\XLingPapersingledigitlistitemwidth}{4.}{In {\LangtToolFontFamily{\textbf{\textcolor[rgb]{0,0,0.5019607843137255}{FLExTrans}}}}, use the {\LangtToolFontFamily{\textbf{\textcolor[rgb]{0,0,0.5019607843137255}{FLExTrans Settings}}}} to do the following:}{\setlength{\XLingPaperlistitemindent}{\XLingPapersingledigitlistitemwidth + \parindent{}}
{\setlength{\XLingPapertempdim}{\XLingPapersingleletterlistitemwidth+\XLingPaperlistitemindent}\leftskip\XLingPapertempdim\relax
\interlinepenalty10000
\XLingPaperlistitem{\XLingPaperlistitemindent}{\XLingPapersingleletterlistitemwidth}{a.}{Verify the {\textbf{\textcolor[rgb]{0.5882352941176471,0.29411764705882354,0}{Target Project}}} is set to your Target Project.}}
{\setlength{\XLingPapertempdim}{\XLingPapersingleletterlistitemwidth+\XLingPaperlistitemindent}\leftskip\XLingPapertempdim\relax
\interlinepenalty10000
\XLingPaperlistitem{\XLingPaperlistitemindent}{\XLingPapersingleletterlistitemwidth}{b.}{Set the {\textbf{\textcolor[rgb]{0.5882352941176471,0.29411764705882354,0}{Transfer Rules File}}} to {\textit{identity\_rules.t1x}}. {\textup{\textup{\textmd{\textcolor[rgb]{1,0,0}{(Remember to change this when you go back to working with two projects.)}}}}}}}
{\setlength{\XLingPapertempdim}{\XLingPapersingleletterlistitemwidth+\XLingPaperlistitemindent}\leftskip\XLingPapertempdim\relax
\interlinepenalty10000
\XLingPaperlistitem{\XLingPaperlistitemindent}{\XLingPapersingleletterlistitemwidth}{c.}{Choose the text you want to work on.}}}}
{\setlength{\XLingPapertempdim}{\XLingPapersingledigitlistitemwidth+\parindent{}}\leftskip\XLingPapertempdim\relax
\interlinepenalty10000
\XLingPaperlistitem{\parindent{}}{\XLingPapersingledigitlistitemwidth}{5.}{In {\LangtToolFontFamily{\textbf{\textcolor[rgb]{0,0,0.5019607843137255}{FLExTrans}}}}, in the {\LangtCollectionFontFamily{{\fontspec[Scale=0.8]{Arial}\textup{\textbf{\textcolor[rgb]{0.4,0,0.4}{Drafting}}}}}} collection, use {\textup{\textbf{Run All}}} to process the text and insert the result into the Target {\LangtToolFontFamily{\textbf{\textcolor[rgb]{0,0,0.5019607843137255}{FLEx}}}} Project.}{\setlength{\XLingPaperlistitemindent}{\XLingPapersingledigitlistitemwidth + \parindent{}}
{\setlength{\XLingPapertempdim}{\XLingPapersingleletterlistitemwidth+\XLingPaperlistitemindent}\leftskip\XLingPapertempdim\relax
\interlinepenalty10000
\XLingPaperlistitem{\XLingPaperlistitemindent}{\XLingPapersingleletterlistitemwidth}{a.}{It will go into a text with the same name, with (Copy) appended.}}
{\setlength{\XLingPapertempdim}{\XLingPapersingleletterlistitemwidth+\XLingPaperlistitemindent}\leftskip\XLingPapertempdim\relax
\interlinepenalty10000
\XLingPaperlistitem{\XLingPaperlistitemindent}{\XLingPapersingleletterlistitemwidth}{b.}{Compare this output with your original text.}}
{\setlength{\XLingPapertempdim}{\XLingPapersingleletterlistitemwidth+\XLingPaperlistitemindent}\leftskip\XLingPapertempdim\relax
\interlinepenalty10000
\XLingPaperlistitem{\XLingPaperlistitemindent}{\XLingPapersingleletterlistitemwidth}{c.}{Any differences between them reflect what would happen if you used this text in a transfer process with a different source language, and they probably need to be fixed.}}}}
{\setlength{\XLingPapertempdim}{\XLingPapersingledigitlistitemwidth+\parindent{}}\leftskip\XLingPapertempdim\relax
\interlinepenalty10000
\XLingPaperlistitem{\parindent{}}{\XLingPapersingledigitlistitemwidth}{6.}{You can try this method with the {\textbf{Spanish}} sample project in {\LangtFoldernameFontFamily{{\fontspec[Scale=0.8]{Tahoma}\textup{\textmd{FLExTrans Documentation\textbackslash{}Synthesis Self-Test}}}}} to get an idea of how it works.}}
\vspace{\baselineskip}
}
\vspace{12pt}
\XLingPaperneedspace{5\baselineskip}

\penalty-3000{\noindent{\raisebox{\baselineskip}[0pt]{\protect\hypertarget{sRepl}{}}\SectionLevelThreeFontFamily{\normalsize{\raisebox{\baselineskip}[0pt]{\pdfbookmark[3]{8.3.3 I have a source word that maps to two different target words depending on the inflection of the source word. How do I handle that?}{sRepl}}\textit{8.3.3 I have a source word that maps to two different target words depending on the inflection of the source word. How do I handle that?}}}}
\markright{I have a source word that maps to two different target words depending on the inflection of the source word. How do I handle that?}
\XLingPaperaddtocontents{sRepl}}\par{}\penalty10000
\vspace{12pt}
\indent You can handle this by using the {\LangtToolFontFamily{\textbf{\textcolor[rgb]{0,0,0.5019607843137255}{Replacement Dictionary Editor}}}} to add additional mappings to the bilingual lexicon.\par{}\indent The bilingual lexicon is a mapping of source word-senses to target word-senses. There are no inflectional affixes present in the bilingual lexicon. That's the way {\LangtToolFontFamily{\textbf{\textcolor[rgb]{0,0,0.5019607843137255}{FLExTrans}}}} builds it. In theory, though, there could be inflectional features or classes present. An example portion of a bilingual lexicon is shown in \hyperlink{xBilingEntries}{(115)}. You can view the bilingual lexicon by double-clicking on the file {\textit{bilingual.dix}} in the {\LangtFoldernameFontFamily{{\fontspec[Scale=0.8]{Tahoma}\textup{\textmd{Output}}}}} folder or click on the {\textup{\textbf{View Bilingual Lexicon}}} button in the {\LangtToolFontFamily{\textbf{\textcolor[rgb]{0,0,0.5019607843137255}{Live Rule Tester Tool}}}}.\par{}{\vspace{12pt plus 2pt minus 1pt}\raggedright{}\XLingPaperexample{.125in}{0pt}{2.75em}{\raisebox{\baselineskip}[0pt]{\protect\hypertarget{xBilingEntries}{}}(115)}{\parbox[t]{\textwidth - .125in - 0pt}{\vspace*{-\baselineskip}{\XeTeXpicfile "../Images/BilingalEntries2.PNG" scaled 750}}}
\vspace{12pt plus 2pt minus 1pt}}\par\indent In order to have a word map differently than what is in the bilingual lexicon, you need to add a new line to the replacement dictionary. See “\hyperlink{sReplEditTool}{The {\LangtToolFontFamily{\textbf{\textcolor[rgb]{0,0,0.5019607843137255}{Replacement Dictionary Editor}}}}}”.\par{}
\vspace{12pt}
\XLingPaperneedspace{5\baselineskip}

\penalty-3000{\noindent{\raisebox{\baselineskip}[0pt]{\protect\hypertarget{sInflVariantTgt}{}}\SectionLevelThreeFontFamily{\normalsize{\raisebox{\baselineskip}[0pt]{\pdfbookmark[3]{8.3.4 How do I handle inflectional variants in the target language? (video)}{sInflVariantTgt}}\textit{8.3.4 How do I handle inflectional variants in the target language? (video)\vspace*{0pt}{\XeTeXpicfile "../Images/VideoIcon.PNG" scaled 750}}}}}
\markright{How do I handle inflectional variants in the target language? (video)\vspace*{0pt}{\XeTeXpicfile "../Images/VideoIcon.PNG" scaled 750}}
\XLingPaperaddtocontents{sInflVariantTgt}}\par{}\penalty10000
\vspace{12pt}
\indent To learn how {\LangtToolFontFamily{\textbf{\textcolor[rgb]{0,0,0.5019607843137255}{FLExTrans}}}} can handle these kind of variants, please watch this \href{https://vimeo.com/934391669}{video}.\par{}\indent The key thing in the techniques explained in the videos is to get inflectional features assigned to Irregularly Inflected Form variants that match exactly the suffixes that would normally be used.\par{}\indent If you would like to try doing the same thing while watching the video, please follow the instructions in the file: {\textit{Readme.txt}} in the {\LangtFoldernameFontFamily{{\fontspec[Scale=0.8]{Tahoma}\textup{\textmd{FLExTrans\textbackslash{}FLExTrans Documentation\textbackslash{}Irregular Form}}}}} folder.\par{}
\vspace{12pt}
\XLingPaperneedspace{5\baselineskip}

\penalty-3000{\noindent{\raisebox{\baselineskip}[0pt]{\protect\hypertarget{sInflVariantSrc}{}}\SectionLevelThreeFontFamily{\normalsize{\raisebox{\baselineskip}[0pt]{\pdfbookmark[3]{8.3.5 How do I handle inflectional variants in the source language? (video)}{sInflVariantSrc}}\textit{8.3.5 How do I handle inflectional variants in the source language? (video)\vspace*{0pt}{\XeTeXpicfile "../Images/VideoIcon.PNG" scaled 750}}}}}
\markright{How do I handle inflectional variants in the source language? (video)\vspace*{0pt}{\XeTeXpicfile "../Images/VideoIcon.PNG" scaled 750}}
\XLingPaperaddtocontents{sInflVariantSrc}}\par{}\penalty10000
\vspace{12pt}
\indent To learn how {\LangtToolFontFamily{\textbf{\textcolor[rgb]{0,0,0.5019607843137255}{FLExTrans}}}} can handle these kind of variants, please watch this \href{https://vimeo.com/934391509}{video}.\par{}\indent The key thing in the techniques explained in the videos is to get inflectional features assigned to Irregularly Inflected Form variants that match exactly the suffixes that would normally be used.\par{}\indent If you would like to try doing the same thing while watching the video, please follow the instructions in the file: {\textit{Readme.txt}} in the {\LangtFoldernameFontFamily{{\fontspec[Scale=0.8]{Tahoma}\textup{\textmd{FLExTrans\textbackslash{}FLExTrans Documentation\textbackslash{}Irregular Form}}}}} folder. Note: be sure to set the {\textbf{\textcolor[rgb]{0.5882352941176471,0.29411764705882354,0}{Source Text Name}}} property in the {\LangtToolFontFamily{\textbf{\textcolor[rgb]{0,0,0.5019607843137255}{FLExTrans Settings}}}} to {\textit{Text2}}.\par{}
\vspace{12pt}
\XLingPaperneedspace{5\baselineskip}

\penalty-3000{\noindent{\raisebox{\baselineskip}[0pt]{\protect\hypertarget{sPhrasalVerbs}{}}\SectionLevelThreeFontFamily{\normalsize{\raisebox{\baselineskip}[0pt]{\pdfbookmark[3]{8.3.6 How do I deal with phrasal verbs using FLExTrans?}{sPhrasalVerbs}}\textit{8.3.6 How do I deal with phrasal verbs using {\LangtToolFontFamily{\textbf{\textcolor[rgb]{0,0,0.5019607843137255}{FLExTrans}}}}?}}}}
\markright{How do I deal with phrasal verbs using {\LangtToolFontFamily{\textbf{\textcolor[rgb]{0,0,0.5019607843137255}{FLExTrans}}}}?}
\XLingPaperaddtocontents{sPhrasalVerbs}}\par{}\penalty10000
\vspace{12pt}
\indent A phrasal verb is a verb that consists of two or more words. Often there is a verbal component and a non-verbal component. This discussion assumes that a verbal component is present.\par{}\indent If the source and target language both use a phrasal verb for a verbal concept and the non-verbal component(s) and verbal component individually match between the languages, you have nothing to worry about. Just link each component sense in the source to appropriate target sense.\par{}\indent If you have a situation where there isn't a match, there are steps you can take to have {\LangtToolFontFamily{\textbf{\textcolor[rgb]{0,0,0.5019607843137255}{FLExTrans}}}} translate correctly. Let's take each situation in turn.\par{}
\vspace{12pt}
\XLingPaperneedspace{5\baselineskip}

\penalty-3000{\noindent{\raisebox{\baselineskip}[0pt]{\protect\hypertarget{sPVSrcIsPhrasal}{}}\SectionLevelFourFontFamily{\normalsize{\raisebox{\baselineskip}[0pt]{\pdfbookmark[4]{8.3.6.1 The source uses a phrasal verb, but the target uses a normal verb.}{sPVSrcIsPhrasal}}\textit{8.3.6.1 The source uses a phrasal verb, but the target uses a normal verb.}}}}
\markright{The source uses a phrasal verb, but the target uses a normal verb.}
\XLingPaperaddtocontents{sPVSrcIsPhrasal}}\par{}\penalty10000
\vspace{12pt}
\indent In this case you want to link the entire source phrasal verb to the one-word target verb. Here are the steps:\par{}{\parskip .5pt plus 1pt minus 1pt
                    
\vspace{\baselineskip}

{\setlength{\XLingPapertempdim}{\XLingPapersingledigitlistitemwidth+\parindent{}}\leftskip\XLingPapertempdim\relax
\interlinepenalty10000
\XLingPaperlistitem{\parindent{}}{\XLingPapersingledigitlistitemwidth}{1.}{Create the phrasal verb as its own entry in the source {\LangtToolFontFamily{\textbf{\textcolor[rgb]{0,0,0.5019607843137255}{FLEx}}}} project.}{\setlength{\XLingPaperlistitemindent}{\XLingPapersingledigitlistitemwidth + \parindent{}}
{\setlength{\XLingPapertempdim}{\XLingPapersingleletterlistitemwidth+\XLingPaperlistitemindent}\leftskip\XLingPapertempdim\relax
\interlinepenalty10000
\XLingPaperlistitem{\XLingPaperlistitemindent}{\XLingPapersingleletterlistitemwidth}{a.}{In the source {\LangtToolFontFamily{\textbf{\textcolor[rgb]{0,0,0.5019607843137255}{FLEx}}}} project create a custom category {\textcolor[rgb]{0,0.4392156862745098,0.7529411764705882}{Verb Phrasal}}. (This is recommended, but not required.)}}
{\setlength{\XLingPapertempdim}{\XLingPapersingleletterlistitemwidth+\XLingPaperlistitemindent}\leftskip\XLingPapertempdim\relax
\interlinepenalty10000
\XLingPaperlistitem{\XLingPaperlistitemindent}{\XLingPapersingleletterlistitemwidth}{b.}{Put the whole phrasal verb as the {\textcolor[rgb]{0.8,0.6,0}{Lexeme Form}}.}}
{\setlength{\XLingPapertempdim}{\XLingPapersingleletterlistitemwidth+\XLingPaperlistitemindent}\leftskip\XLingPapertempdim\relax
\interlinepenalty10000
\XLingPaperlistitem{\XLingPaperlistitemindent}{\XLingPapersingleletterlistitemwidth}{c.}{Set the {\textcolor[rgb]{0.8,0.6,0}{Morph Type}} to phrase}}
{\setlength{\XLingPapertempdim}{\XLingPapersingleletterlistitemwidth+\XLingPaperlistitemindent}\leftskip\XLingPapertempdim\relax
\interlinepenalty10000
\XLingPaperlistitem{\XLingPaperlistitemindent}{\XLingPapersingleletterlistitemwidth}{d.}{Set the {\textcolor[rgb]{0.8,0.6,0}{Grammatical Info.}} to {\textcolor[rgb]{0,0.4392156862745098,0.7529411764705882}{Verb Phrasal}}}}
{\setlength{\XLingPapertempdim}{\XLingPapersingleletterlistitemwidth+\XLingPaperlistitemindent}\leftskip\XLingPapertempdim\relax
\interlinepenalty10000
\XLingPaperlistitem{\XLingPaperlistitemindent}{\XLingPapersingleletterlistitemwidth}{e.}{Set {\textcolor[rgb]{0.8,0.6,0}{Components}} to the two words that form the phrasal verb. Note: the components should be the order they occur in the phrasal verb.\protect\footnote[14]{{\leftskip0pt\parindent1em\raisebox{\baselineskip}[0pt]{\protect\hypertarget{nMoveComponent}{}}You can right-click on a component word and select Move Left or Move Right.}}}}
{\setlength{\XLingPapertempdim}{\XLingPapersingleletterlistitemwidth+\XLingPaperlistitemindent}\leftskip\XLingPapertempdim\relax
\interlinepenalty10000
\XLingPaperlistitem{\XLingPaperlistitemindent}{\XLingPapersingleletterlistitemwidth}{f.}{Set the {\textcolor[rgb]{0.8,0.6,0}{Complex Form Type}} to {\textcolor[rgb]{0.6,0,0.6}{Phrasal Verb}}}}}}
{\setlength{\XLingPapertempdim}{\XLingPapersingledigitlistitemwidth+\parindent{}}\leftskip\XLingPapertempdim\relax
\interlinepenalty10000
\XLingPaperlistitem{\parindent{}}{\XLingPapersingledigitlistitemwidth}{2.}{Open the {\LangtToolFontFamily{\textbf{\textcolor[rgb]{0,0,0.5019607843137255}{FLExTrans Settings}}}}.}}
{\setlength{\XLingPapertempdim}{\XLingPapersingledigitlistitemwidth+\parindent{}}\leftskip\XLingPapertempdim\relax
\interlinepenalty10000
\XLingPaperlistitem{\parindent{}}{\XLingPapersingledigitlistitemwidth}{3.}{Change {\textbf{\textcolor[rgb]{0.5882352941176471,0.29411764705882354,0}{Source Complex Form Types}}} to {\textit{Phrasal Verb}}.}}
{\setlength{\XLingPapertempdim}{\XLingPapersingledigitlistitemwidth+\parindent{}}\leftskip\XLingPapertempdim\relax
\interlinepenalty10000
\XLingPaperlistitem{\parindent{}}{\XLingPapersingledigitlistitemwidth}{4.}{Set the {\textbf{\textcolor[rgb]{0.5882352941176471,0.29411764705882354,0}{Source Text}}} to a text that contains the phrasal verb.}}
{\setlength{\XLingPapertempdim}{\XLingPapersingledigitlistitemwidth+\parindent{}}\leftskip\XLingPapertempdim\relax
\interlinepenalty10000
\XLingPaperlistitem{\parindent{}}{\XLingPapersingledigitlistitemwidth}{5.}{Run the {\LangtModuleFontFamily{\textbf{\textcolor[rgb]{0.4,0,0.4}{Sense Linker Tool}}}} and link the phrasal verb sense to the appropriate target verb sense.}}
\vspace{\baselineskip}
}\indent Now the source phrasal verb will correctly translate to the target verb.\par{}
\vspace{12pt}
\XLingPaperneedspace{5\baselineskip}

\penalty-3000{\noindent{\raisebox{\baselineskip}[0pt]{\protect\hypertarget{sPVTgtIsPhrasal}{}}\SectionLevelFourFontFamily{\normalsize{\raisebox{\baselineskip}[0pt]{\pdfbookmark[4]{8.3.6.2 The source uses a normal verb, but the target uses a phrasal verb.}{sPVTgtIsPhrasal}}\textit{8.3.6.2 The source uses a normal verb, but the target uses a phrasal verb.}}}}
\markright{The source uses a normal verb, but the target uses a phrasal verb.}
\XLingPaperaddtocontents{sPVTgtIsPhrasal}}\par{}\penalty10000
\vspace{12pt}
\indent In this case you want to link the one-word source verb to the entire target phrasal verb. Here are the steps:\par{}{\parskip .5pt plus 1pt minus 1pt
                    
\vspace{\baselineskip}

{\setlength{\XLingPapertempdim}{\XLingPapersingledigitlistitemwidth+\parindent{}}\leftskip\XLingPapertempdim\relax
\interlinepenalty10000
\XLingPaperlistitem{\parindent{}}{\XLingPapersingledigitlistitemwidth}{1.}{Create the phrasal verb as its own entry in the target {\LangtToolFontFamily{\textbf{\textcolor[rgb]{0,0,0.5019607843137255}{FLEx}}}} project.}{\setlength{\XLingPaperlistitemindent}{\XLingPapersingledigitlistitemwidth + \parindent{}}
{\setlength{\XLingPapertempdim}{\XLingPapersingleletterlistitemwidth+\XLingPaperlistitemindent}\leftskip\XLingPapertempdim\relax
\interlinepenalty10000
\XLingPaperlistitem{\XLingPaperlistitemindent}{\XLingPapersingleletterlistitemwidth}{a.}{In the target {\LangtToolFontFamily{\textbf{\textcolor[rgb]{0,0,0.5019607843137255}{FLEx}}}} project create a custom category {\textcolor[rgb]{0,0.4392156862745098,0.7529411764705882}{Verb Phrasal}}. (This is recommended, but not required.)}}
{\setlength{\XLingPapertempdim}{\XLingPapersingleletterlistitemwidth+\XLingPaperlistitemindent}\leftskip\XLingPapertempdim\relax
\interlinepenalty10000
\XLingPaperlistitem{\XLingPaperlistitemindent}{\XLingPapersingleletterlistitemwidth}{b.}{Put the whole phrasal verb as the {\textcolor[rgb]{0.8,0.6,0}{Lexeme Form}}.}}
{\setlength{\XLingPapertempdim}{\XLingPapersingleletterlistitemwidth+\XLingPaperlistitemindent}\leftskip\XLingPapertempdim\relax
\interlinepenalty10000
\XLingPaperlistitem{\XLingPaperlistitemindent}{\XLingPapersingleletterlistitemwidth}{c.}{Set the {\textcolor[rgb]{0.8,0.6,0}{Morph Type}} to phrase}}
{\setlength{\XLingPapertempdim}{\XLingPapersingleletterlistitemwidth+\XLingPaperlistitemindent}\leftskip\XLingPapertempdim\relax
\interlinepenalty10000
\XLingPaperlistitem{\XLingPaperlistitemindent}{\XLingPapersingleletterlistitemwidth}{d.}{Set the {\textcolor[rgb]{0.8,0.6,0}{Grammatical Info.}} to {\textcolor[rgb]{0,0.4392156862745098,0.7529411764705882}{Verb Phrasal}}}}
{\setlength{\XLingPapertempdim}{\XLingPapersingleletterlistitemwidth+\XLingPaperlistitemindent}\leftskip\XLingPapertempdim\relax
\interlinepenalty10000
\XLingPaperlistitem{\XLingPaperlistitemindent}{\XLingPapersingleletterlistitemwidth}{e.}{Set {\textcolor[rgb]{0.8,0.6,0}{Components}} to the two words that form the phrasal verb. Note: the components should be the order they occur in the phrasal verb.\footnote[15]{See footnote \hyperlink{nMoveComponent}{14}.}}}
{\setlength{\XLingPapertempdim}{\XLingPapersingleletterlistitemwidth+\XLingPaperlistitemindent}\leftskip\XLingPapertempdim\relax
\interlinepenalty10000
\XLingPaperlistitem{\XLingPaperlistitemindent}{\XLingPapersingleletterlistitemwidth}{f.}{Set the {\textcolor[rgb]{0.8,0.6,0}{Complex Form Type}} to {\textcolor[rgb]{0.6,0,0.6}{Phrasal Verb}}}}}}
{\setlength{\XLingPapertempdim}{\XLingPapersingledigitlistitemwidth+\parindent{}}\leftskip\XLingPapertempdim\relax
\interlinepenalty10000
\XLingPaperlistitem{\parindent{}}{\XLingPapersingledigitlistitemwidth}{2.}{Open the {\LangtToolFontFamily{\textbf{\textcolor[rgb]{0,0,0.5019607843137255}{FLExTrans Settings}}}}.}}
{\setlength{\XLingPapertempdim}{\XLingPapersingledigitlistitemwidth+\parindent{}}\leftskip\XLingPapertempdim\relax
\interlinepenalty10000
\XLingPaperlistitem{\parindent{}}{\XLingPapersingledigitlistitemwidth}{3.}{Depending on whether the verb comes first or second in the two-word phrasal verb, change either {\textbf{\textcolor[rgb]{0.5882352941176471,0.29411764705882354,0}{Target Complex Form Types with Inflection on 1st Element}}} or {\textbf{\textcolor[rgb]{0.5882352941176471,0.29411764705882354,0}{Target Complex Form Types with Inflection on 2nd Element}}} to {\textit{Phrasal Verb}}.}}
{\setlength{\XLingPapertempdim}{\XLingPapersingledigitlistitemwidth+\parindent{}}\leftskip\XLingPapertempdim\relax
\interlinepenalty10000
\XLingPaperlistitem{\parindent{}}{\XLingPapersingledigitlistitemwidth}{4.}{Set the {\textbf{\textcolor[rgb]{0.5882352941176471,0.29411764705882354,0}{Source Text}}} to a text that contains the source verb.}}
{\setlength{\XLingPapertempdim}{\XLingPapersingledigitlistitemwidth+\parindent{}}\leftskip\XLingPapertempdim\relax
\interlinepenalty10000
\XLingPaperlistitem{\parindent{}}{\XLingPapersingledigitlistitemwidth}{5.}{Run the {\LangtModuleFontFamily{\textbf{\textcolor[rgb]{0.4,0,0.4}{Sense Linker Tool}}}} and link the source verb sense to the appropriate target phrasal verb sense.}}
\vspace{\baselineskip}
}\indent Now the source verb will correctly translate to the target phrasal verb.\par{}
\vspace{12pt}
\XLingPaperneedspace{5\baselineskip}

\penalty-3000{\noindent{\raisebox{\baselineskip}[0pt]{\protect\hypertarget{sPVBothPhrasal}{}}\SectionLevelFourFontFamily{\normalsize{\raisebox{\baselineskip}[0pt]{\pdfbookmark[4]{8.3.6.3 The source uses a phrasal verb and the target also uses a phrasal verb (and the components don't match).}{sPVBothPhrasal}}\textit{8.3.6.3 The source uses a phrasal verb and the target also uses a phrasal verb (and the components don't match).}}}}
\markright{The source uses a phrasal verb and the target also uses a phrasal verb (and the components don't match).}
\XLingPaperaddtocontents{sPVBothPhrasal}}\par{}\penalty10000
\vspace{12pt}
\indent In this case you want to link the source phrasal verb to the target phrasal verb. Follow the first 3 steps in \hyperlink{sPVSrcIsPhrasal}{8.3.6.1} and \hyperlink{sPVTgtIsPhrasal}{8.3.6.2}, link the verbs and the source phrasal verb will now translate to the target phrasal verb.\par{}
\vspace{12pt}
\XLingPaperneedspace{5\baselineskip}

\penalty-3000{\noindent{\raisebox{\baselineskip}[0pt]{\protect\hypertarget{sLinkDup}{}}\SectionLevelThreeFontFamily{\normalsize{\raisebox{\baselineskip}[0pt]{\pdfbookmark[3]{8.3.7 I just copied my source FLEx project to become my target FLEx project. Is there a way to link all the senses?}{sLinkDup}}\textit{8.3.7 I just copied my source {\LangtToolFontFamily{\textbf{\textcolor[rgb]{0,0,0.5019607843137255}{FLEx}}}} project to become my target {\LangtToolFontFamily{\textbf{\textcolor[rgb]{0,0,0.5019607843137255}{FLEx}}}} project. Is there a way to link all the senses?}}}}
\markright{I just copied my source {\LangtToolFontFamily{\textbf{\textcolor[rgb]{0,0,0.5019607843137255}{FLEx}}}} project to become my target {\LangtToolFontFamily{\textbf{\textcolor[rgb]{0,0,0.5019607843137255}{FLEx}}}} project. Is there a way to link all the senses?}
\XLingPaperaddtocontents{sLinkDup}}\par{}\penalty10000
\vspace{12pt}
\indent There is a {\LangtToolFontFamily{\textbf{\textcolor[rgb]{0,0,0.5019607843137255}{FLEx}}}} module called {\LangtModuleFontFamily{\textbf{\textcolor[rgb]{0.4,0,0.4}{Link All Senses As Duplicate}}}}. This module is ideal when you duplicate one {\LangtToolFontFamily{\textbf{\textcolor[rgb]{0,0,0.5019607843137255}{FLEx}}}} project to be another one. In such a situation all the sense ids are exactly the same so it can be a big head start to link all source senses to their matches in the target project. The end result is that you will have a link for every source sense to every corresponding target sense. To add this module to your {\LangtCollectionFontFamily{{\fontspec[Scale=0.8]{Arial}\textup{\textbf{\textcolor[rgb]{0.4,0,0.4}{Tools}}}}}} collection in the {\LangtToolFontFamily{\textbf{\textcolor[rgb]{0,0,0.5019607843137255}{FLExTools}}}} app, see Section \hyperlink{sCollections}{3.3}.\par{}
\vspace{12pt}
\XLingPaperneedspace{5\baselineskip}

\penalty-3000{{\centering\raisebox{\baselineskip}[0pt]{\protect\hypertarget{sTroubleshooting}{}}\SectionLevelOneFontFamily{\large{\raisebox{\baselineskip}[0pt]{\pdfbookmark[1]{9 Troubleshooting}{sTroubleshooting}}\textbf{9 Troubleshooting}}}\\{}}\markright{Troubleshooting}
\XLingPaperaddtocontents{sTroubleshooting}}\par{}\penalty10000
\vspace{12pt}

\vspace{12pt}
\XLingPaperneedspace{5\baselineskip}

\penalty-3000{\noindent{\raisebox{\baselineskip}[0pt]{\protect\hypertarget{sSynFail}{}}\SectionLevelTwoFontFamily{\normalsize{\raisebox{\baselineskip}[0pt]{\pdfbookmark[2]{9.1 Synthesis Troubleshooting}{sSynFail}}\textbf{9.1 Synthesis Troubleshooting}}}}
\markright{Synthesis Troubleshooting}
\XLingPaperaddtocontents{sSynFail}}\par{}\penalty10000
\vspace{12pt}
\indent One of the first things to do if you are getting synthesis errors is to put a word that is failing into the {\textbf{Try A Word}} window in the target {\LangtToolFontFamily{\textbf{\textcolor[rgb]{0,0,0.5019607843137255}{FLEx}}}} project and see if it parses. If it doesn’t parse in {\LangtToolFontFamily{\textbf{\textcolor[rgb]{0,0,0.5019607843137255}{FLEx}}}}, then don’t expect it to synthesize in {\LangtToolFontFamily{\textbf{\textcolor[rgb]{0,0,0.5019607843137255}{FLExTrans}}}}. Figure out why it is not parsing in {\LangtToolFontFamily{\textbf{\textcolor[rgb]{0,0,0.5019607843137255}{FLEx}}}} and then try synthesis again in {\LangtToolFontFamily{\textbf{\textcolor[rgb]{0,0,0.5019607843137255}{FLExTrans}}}}.\par{}\indent There’s a check box in the {\LangtToolFontFamily{\textbf{\textcolor[rgb]{0,0,0.5019607843137255}{Live Rule Tester Tool}}}} labelled {\textup{\textbf{Trace HermitCrab synthesis}}}. When this is checked and you click the {\textup{\textbf{Synthesize}}} button, the {\LangtToolFontFamily{\textbf{\textcolor[rgb]{0,0,0.5019607843137255}{Live Rule Tester Tool}}}} will open a browser window showing a trace of what happened in the synthesis. This is similar to the output you see when tracing with {\textbf{Try A Word}} in {\LangtToolFontFamily{\textbf{\textcolor[rgb]{0,0,0.5019607843137255}{FLEx}}}}.\par{}
\vspace{12pt}
\XLingPaperneedspace{5\baselineskip}

\penalty-3000{\noindent{\raisebox{\baselineskip}[0pt]{\protect\hypertarget{sErrorCond}{}}\SectionLevelTwoFontFamily{\normalsize{\raisebox{\baselineskip}[0pt]{\pdfbookmark[2]{9.2 Error Conditions}{sErrorCond}}\textbf{9.2 Error Conditions}}}}
\markright{Error Conditions}
\XLingPaperaddtocontents{sErrorCond}}\par{}\penalty10000
\vspace{12pt}
\indent The following is a list of various errors you may see in the Live Rule Tester tool or in the final synthesized text.\par{}{\vspace{12pt plus 2pt minus 1pt}\raggedright{}\XLingPaperneedspace{10\baselineskip}\XLingPaperexample{.125in}{0pt}{2.75em}{\raisebox{\baselineskip}[0pt]{\protect\hypertarget{xErrorCond}{}}(116)}{{
\XLingPaperminmaxcellincolumn{Display}{\XLingPapermincola}{\textbf{Live Rule Tester Display}}{\XLingPapermaxcola}{+0\tabcolsep}
\XLingPaperminmaxcellincolumn{Drafted}{\XLingPapermincolb}{\textbf{Drafted Text Display}}{\XLingPapermaxcolb}{+0\tabcolsep}
\XLingPaperminmaxcellincolumn{Meaning}{\XLingPapermincolc}{\textbf{Meaning}}{\XLingPapermaxcolc}{+0\tabcolsep}
\XLingPaperminmaxcellincolumn{Possible}{\XLingPapermincold}{\textbf{Possible Solution}}{\XLingPapermaxcold}{+0\tabcolsep}
\XLingPaperminmaxcellincolumn{Zug1.1}{\XLingPapermincola}{{\textup{\textup{\textmd{\textcolor[rgb]{1,0,0}{Zug{{\fontspec[Scale=0.65]{Times New Roman}\textsubscript{1.1}}}}}}}}}{\XLingPapermaxcola}{+0\tabcolsep}
\XLingPaperminmaxcellincolumn{@zug1.1}{\XLingPapermincolb}{@zug1.1}{\XLingPapermaxcolb}{+0\tabcolsep}
\XLingPaperminmaxcellincolumn{bilingual}{\XLingPapermincolc}{The lemma could not be found in the bilingual lexicon.}{\XLingPapermaxcolc}{+0\tabcolsep}
\XLingPaperminmaxcellincolumn{Bilingual}{\XLingPapermincold}{Rebuild the Bilingual Lexicon.}{\XLingPapermaxcold}{+0\tabcolsep}
\XLingPaperminmaxcellincolumn{älska1.1}{\XLingPapermincola}{älska1.1}{\XLingPapermaxcola}{+0\tabcolsep}
\XLingPaperminmaxcellincolumn{älska1.1}{\XLingPapermincolb}{älska1.1 i.e. a {\textbf{target}} lemma with numbers instead of a correctly formed target word.}{\XLingPapermaxcolb}{+0\tabcolsep}
\XLingPaperminmaxcellincolumn{lexicon.}{\XLingPapermincolc}{The target word could not be found in the target lexicon.}{\XLingPapermaxcolc}{+0\tabcolsep}
\XLingPaperminmaxcellincolumn{lexicon.}{\XLingPapermincold}{First ensure the word is in the target lexicon. In the LRT, Refresh Target Lexicon. For a text, ensure {\LangtToolFontFamily{\textbf{\textcolor[rgb]{0,0,0.5019607843137255}{FLEx}}}} has saved everything.}{\XLingPapermaxcold}{+0\tabcolsep}
\XLingPaperminmaxcellincolumn{lieben1.1}{\XLingPapermincola}{lieben1.1}{\XLingPapermaxcola}{+0\tabcolsep}
\XLingPaperminmaxcellincolumn{lieben1.1}{\XLingPapermincolb}{lieben1.1 i.e. a {\textbf{source}} lemma with numbers instead of a correctly formed target word.}{\XLingPapermaxcolb}{+0\tabcolsep}
\XLingPaperminmaxcellincolumn{target}{\XLingPapermincolc}{There was no link made to a target word.}{\XLingPapermaxcolc}{+0\tabcolsep}
\XLingPaperminmaxcellincolumn{lexicon}{\XLingPapermincold}{Check the target lexicon to see if the sense exists. Run the SL to check the link exists.}{\XLingPapermaxcold}{+0\tabcolsep}
\XLingPaperminmaxcellincolumn{Zug}{\XLingPapermincola}{{\textup{\textup{\textmd{\textcolor[rgb]{1,0,0}{Zug}}}}} {\textup{\textup{\textmd{\textcolor[rgb]{0.8,0,0.4}{UNK}}}}}}{\XLingPapermaxcola}{+0\tabcolsep}
\XLingPaperminmaxcellincolumn{@Zug}{\XLingPapermincolb}{@Zug}{\XLingPapermaxcolb}{+0\tabcolsep}
\XLingPaperminmaxcellincolumn{source}{\XLingPapermincolc}{The word was left unanalyzed in the source interlinear text.}{\XLingPapermaxcolc}{+0\tabcolsep}
\XLingPaperminmaxcellincolumn{Analyze}{\XLingPapermincold}{Analyze and approve the word in the source text. Ensure {\LangtToolFontFamily{\textbf{\textcolor[rgb]{0,0,0.5019607843137255}{FLEx}}}} has saved everything. If you intentionally don't want to analyze the word and want to keep its form, check {\textit{Yes}} for the option {\textbf{\textcolor[rgb]{0.5882352941176471,0.29411764705882354,0}{Clean Up Unknown Target Words?}}}}{\XLingPapermaxcold}{+0\tabcolsep}
\XLingPaperminmaxcellincolumn{bilPL}{\XLingPapermincola}{bilPL}{\XLingPapermaxcola}{+0\tabcolsep}
\XLingPaperminmaxcellincolumn{surface}{\XLingPapermincolb}{bilPL i.e. an affix that didn't get turned into its surface form.}{\XLingPapermaxcolb}{+0\tabcolsep}
\XLingPaperminmaxcellincolumn{lexicon.}{\XLingPapermincolc}{The affix could not be found in the target lexicon.}{\XLingPapermaxcolc}{+0\tabcolsep}
\XLingPaperminmaxcellincolumn{lexicon}{\XLingPapermincold}{Check the target lexicon for the affix with this gloss.}{\XLingPapermaxcold}{+0\tabcolsep}
\XLingPaperminmaxcellincolumn{\%0\%bilar\%}{\XLingPapermincola}{\%0\%bilar\%}{\XLingPapermaxcola}{+0\tabcolsep}
\XLingPaperminmaxcellincolumn{\%0\%bilar\%}{\XLingPapermincolb}{\%0\%bilar\%}{\XLingPapermaxcolb}{+0\tabcolsep}
\XLingPaperminmaxcellincolumn{morphemes}{\XLingPapermincolc}{The given morphemes could not be synthesized into a word because of an incompatibility.}{\XLingPapermaxcolc}{+0\tabcolsep}
\XLingPaperminmaxcellincolumn{analyzes}{\XLingPapermincold}{Check that the word analyzes properly in target {\LangtToolFontFamily{\textbf{\textcolor[rgb]{0,0,0.5019607843137255}{FLEx}}}} project.}{\XLingPapermaxcold}{+0\tabcolsep}
\XLingPaperminmaxcellincolumn{\%0\%\^{}kon1.1\textless{}v\textgreater{}\textdollar{}\%}{\XLingPapermincola}{\%0\%\^{}kon1.1\textless{}v\textgreater{}\textdollar{}\%}{\XLingPapermaxcola}{+0\tabcolsep}
\XLingPaperminmaxcellincolumn{warning}{\XLingPapermincolb}{\vbox{\hbox{\strut{}\%0\%\^{}kon1.1\textless{}v\textgreater{}\textdollar{}\%}\hbox{\strut{}(you should see a warning in the {\LangtToolFontFamily{\textbf{\textcolor[rgb]{0,0,0.5019607843137255}{FLExTools}}}} output saying: One or more glosses not found: 'kon1.1';)}}}{\XLingPapermaxcolb}{+0\tabcolsep}
\XLingPaperminmaxcellincolumn{indicate}{\XLingPapermincolc}{When using {\LangtToolFontFamily{\textbf{\textcolor[rgb]{0,0,0.5019607843137255}{HermitCrab}}}}, this output could indicate that the root was not found in the target lexicon.}{\XLingPapermaxcolc}{+0\tabcolsep}
\XLingPaperminmaxcellincolumn{lexicon.}{\XLingPapermincold}{Check the target lexicon.}{\XLingPapermaxcold}{+0\tabcolsep}
\XLingPaperminmaxcellincolumn{siata1.1}{\XLingPapermincola}{siata{{\fontspec[Scale=0.65]{Times New Roman}\textsubscript{1.1}}} {\LangtluGrammCatFontFamily{{\fontspec[Scale=0.9]{Courier New}\textcolor[rgb]{0,0.4392156862745098,0.7529411764705882}{prep}}}} {\LangtluAffixFontFamily{{\fontspec[Scale=0.9]{Courier New}\textcolor[rgb]{0,0.6901960784313725,0.3137254901960784}{mana1.1 prep}}}}}{\XLingPapermaxcola}{+0\tabcolsep}
\XLingPaperminmaxcellincolumn{\%0\%\^{}siata1.1\textless{}prep\textgreater{}\textless{}mana1.1\textgreater{}\textless{}prep\textgreater{}\textdollar{}\%}{\XLingPapermincolb}{\%0\%\^{}siata1.1\textless{}prep\textgreater{}\textless{}mana1.1\textgreater{}\textless{}prep\textgreater{}\textdollar{}\%}{\XLingPapermaxcolb}{+0\tabcolsep}
\XLingPaperminmaxcellincolumn{outputted}{\XLingPapermincolc}{{\LangtToolFontFamily{\textbf{\textcolor[rgb]{0,0,0.5019607843137255}{FLExTrans}}}} is interpreting mana1.1 and the 2nd prep as affixes, likely because they were outputted in the same lexical unit as the first word.}{\XLingPapermaxcolc}{+0\tabcolsep}
\XLingPaperminmaxcellincolumn{transfer}{\XLingPapermincold}{Correct the transfer rule that produced this to give a lexical unit block for each word.}{\XLingPapermaxcold}{+0\tabcolsep}
\XLingPaperminmaxcellincolumn{happened}{\XLingPapermincola}{An error happened when running the Apertium tools.}{\XLingPapermaxcola}{+0\tabcolsep}
\XLingPaperminmaxcellincolumn{}{\XLingPapermincolb}{}{\XLingPapermaxcolb}{+0\tabcolsep}
\XLingPaperminmaxcellincolumn{applying}{\XLingPapermincolc}{The rule engine failed when applying the rules.}{\XLingPapermaxcolc}{+0\tabcolsep}
\XLingPaperminmaxcellincolumn{bilingual}{\XLingPapermincold}{Close the {\LangtToolFontFamily{\textbf{\textcolor[rgb]{0,0,0.5019607843137255}{Live Rule Tester Tool}}}}. Run the {\LangtToolFontFamily{\textbf{\textcolor[rgb]{0,0,0.5019607843137255}{Clean Files}}}} module. Rebuild the bilingual lexicon. Re-open the {\LangtToolFontFamily{\textbf{\textcolor[rgb]{0,0,0.5019607843137255}{Live Rule Tester Tool}}}}.}{\XLingPapermaxcold}{+0\tabcolsep}
\setlength{\XLingPaperavailabletablewidth}{433.62pt - .125in - 0pt - 2.75em}
\setlength{\XLingPapertableminwidth}{\XLingPapermincola+\XLingPapermincolb+\XLingPapermincolc+\XLingPapermincold}
\setlength{\XLingPapertablemaxwidth}{\XLingPapermaxcola+\XLingPapermaxcolb+\XLingPapermaxcolc+\XLingPapermaxcold}
\XLingPapercalculatetablewidthratio{}
\XLingPapersetcolumnwidth{\XLingPapercolawidth}{\XLingPapermincola}{\XLingPapermaxcola}{-0\tabcolsep}
\XLingPapersetcolumnwidth{\XLingPapercolbwidth}{\XLingPapermincolb}{\XLingPapermaxcolb}{-2\tabcolsep}
\XLingPapersetcolumnwidth{\XLingPapercolcwidth}{\XLingPapermincolc}{\XLingPapermaxcolc}{-2\tabcolsep}
\XLingPapersetcolumnwidth{\XLingPapercoldwidth}{\XLingPapermincold}{\XLingPapermaxcold}{-2\tabcolsep}\setlength{\LTpre}{-.5\baselineskip}\setlength{\LTleft}{.125in + 2.75em}\setlength{\LTpost}{0pt}
\begin{longtable}
[t]{@{}>{\raggedright}p{\XLingPapercolawidth}>{\raggedright}p{\XLingPapercolbwidth}>{\raggedright}p{\XLingPapercolcwidth}>{\raggedright}p{\XLingPapercoldwidth}@{}}\specialrule{\heavyrulewidth}{-4\aboverulesep}{\belowrulesep}\multicolumn{1}{@{}>{\raggedright}p{\XLingPapercolawidth}}{\textbf{Live Rule Tester Display}}&\multicolumn{1}{>{\raggedright}p{\XLingPapercolbwidth}}{\textbf{Drafted Text Display}}&\multicolumn{1}{>{\raggedright}p{\XLingPapercolcwidth}}{\textbf{Meaning}}&\multicolumn{1}{>{\raggedright}p{\XLingPapercoldwidth}@{}}{\textbf{Possible Solution}}\\%
\midrule\multicolumn{1}{@{}>{\raggedright}p{\XLingPapercolawidth}}{{\textup{\textup{\textmd{\textcolor[rgb]{1,0,0}{Zug{{\fontspec[Scale=0.65]{Times New Roman}\textsubscript{1.1}}}}}}}}}&\multicolumn{1}{>{\raggedright}p{\XLingPapercolbwidth}}{@zug1.1}&\multicolumn{1}{>{\raggedright}p{\XLingPapercolcwidth}}{The lemma could not be found in the bilingual lexicon.}&\multicolumn{1}{>{\raggedright}p{\XLingPapercoldwidth}@{}}{Rebuild the Bilingual Lexicon.}\\%
\multicolumn{1}{@{}>{\raggedright}p{\XLingPapercolawidth}}{älska1.1}&\multicolumn{1}{>{\raggedright}p{\XLingPapercolbwidth}}{älska1.1 i.e. a {\textbf{target}} lemma with numbers instead of a correctly formed target word.}&\multicolumn{1}{>{\raggedright}p{\XLingPapercolcwidth}}{The target word could not be found in the target lexicon.}&\multicolumn{1}{>{\raggedright}p{\XLingPapercoldwidth}@{}}{First ensure the word is in the target lexicon. In the LRT, Refresh Target Lexicon. For a text, ensure {\LangtToolFontFamily{\textbf{\textcolor[rgb]{0,0,0.5019607843137255}{FLEx}}}} has saved everything.}\\%
\multicolumn{1}{@{}>{\raggedright}p{\XLingPapercolawidth}}{lieben1.1}&\multicolumn{1}{>{\raggedright}p{\XLingPapercolbwidth}}{lieben1.1 i.e. a {\textbf{source}} lemma with numbers instead of a correctly formed target word.}&\multicolumn{1}{>{\raggedright}p{\XLingPapercolcwidth}}{There was no link made to a target word.}&\multicolumn{1}{>{\raggedright}p{\XLingPapercoldwidth}@{}}{Check the target lexicon to see if the sense exists. Run the SL to check the link exists.}\\%
\multicolumn{1}{@{}>{\raggedright}p{\XLingPapercolawidth}}{{\textup{\textup{\textmd{\textcolor[rgb]{1,0,0}{Zug}}}}} {\textup{\textup{\textmd{\textcolor[rgb]{0.8,0,0.4}{UNK}}}}}}&\multicolumn{1}{>{\raggedright}p{\XLingPapercolbwidth}}{@Zug}&\multicolumn{1}{>{\raggedright}p{\XLingPapercolcwidth}}{The word was left unanalyzed in the source interlinear text.}&\multicolumn{1}{>{\raggedright}p{\XLingPapercoldwidth}@{}}{Analyze and approve the word in the source text. Ensure {\LangtToolFontFamily{\textbf{\textcolor[rgb]{0,0,0.5019607843137255}{FLEx}}}} has saved everything. If you intentionally don't want to analyze the word and want to keep its form, check {\textit{Yes}} for the option {\textbf{\textcolor[rgb]{0.5882352941176471,0.29411764705882354,0}{Clean Up Unknown Target Words?}}}}\\%
\multicolumn{1}{@{}>{\raggedright}p{\XLingPapercolawidth}}{bilPL}&\multicolumn{1}{>{\raggedright}p{\XLingPapercolbwidth}}{bilPL i.e. an affix that didn't get turned into its surface form.}&\multicolumn{1}{>{\raggedright}p{\XLingPapercolcwidth}}{The affix could not be found in the target lexicon.}&\multicolumn{1}{>{\raggedright}p{\XLingPapercoldwidth}@{}}{Check the target lexicon for the affix with this gloss.}\\%
\multicolumn{1}{@{}>{\raggedright}p{\XLingPapercolawidth}}{\%0\%bilar\%}&\multicolumn{1}{>{\raggedright}p{\XLingPapercolbwidth}}{\%0\%bilar\%}&\multicolumn{1}{>{\raggedright}p{\XLingPapercolcwidth}}{The given morphemes could not be synthesized into a word because of an incompatibility.}&\multicolumn{1}{>{\raggedright}p{\XLingPapercoldwidth}@{}}{Check that the word analyzes properly in target {\LangtToolFontFamily{\textbf{\textcolor[rgb]{0,0,0.5019607843137255}{FLEx}}}} project.}\\%
\multicolumn{1}{@{}>{\raggedright}p{\XLingPapercolawidth}}{\%0\%\^{}kon1.1\textless{}v\textgreater{}\textdollar{}\%}&\multicolumn{1}{>{\raggedright}p{\XLingPapercolbwidth}}{\vbox{\hbox{\strut{}\%0\%\^{}kon1.1\textless{}v\textgreater{}\textdollar{}\%}\hbox{\strut{}(you should see a warning in the {\LangtToolFontFamily{\textbf{\textcolor[rgb]{0,0,0.5019607843137255}{FLExTools}}}} output saying: One or more glosses not found: 'kon1.1';)}}}&\multicolumn{1}{>{\raggedright}p{\XLingPapercolcwidth}}{When using {\LangtToolFontFamily{\textbf{\textcolor[rgb]{0,0,0.5019607843137255}{HermitCrab}}}}, this output could indicate that the root was not found in the target lexicon.}&\multicolumn{1}{>{\raggedright}p{\XLingPapercoldwidth}@{}}{Check the target lexicon.}\\%
\multicolumn{1}{@{}>{\raggedright}p{\XLingPapercolawidth}}{siata{{\fontspec[Scale=0.65]{Times New Roman}\textsubscript{1.1}}} {\LangtluGrammCatFontFamily{{\fontspec[Scale=0.9]{Courier New}\textcolor[rgb]{0,0.4392156862745098,0.7529411764705882}{prep}}}} {\LangtluAffixFontFamily{{\fontspec[Scale=0.9]{Courier New}\textcolor[rgb]{0,0.6901960784313725,0.3137254901960784}{mana1.1 prep}}}}}&\multicolumn{1}{>{\raggedright}p{\XLingPapercolbwidth}}{\%0\%\^{}siata1.1\textless{}prep\textgreater{}\textless{}mana1.1\textgreater{}\textless{}prep\textgreater{}\textdollar{}\%}&\multicolumn{1}{>{\raggedright}p{\XLingPapercolcwidth}}{{\LangtToolFontFamily{\textbf{\textcolor[rgb]{0,0,0.5019607843137255}{FLExTrans}}}} is interpreting mana1.1 and the 2nd prep as affixes, likely because they were outputted in the same lexical unit as the first word.}&\multicolumn{1}{>{\raggedright}p{\XLingPapercoldwidth}@{}}{Correct the transfer rule that produced this to give a lexical unit block for each word.}\\%
\multicolumn{1}{@{}>{\raggedright}p{\XLingPapercolawidth}}{An error happened when running the Apertium tools.}&\multicolumn{1}{>{\raggedright}p{\XLingPapercolbwidth}}{}&\multicolumn{1}{>{\raggedright}p{\XLingPapercolcwidth}}{The rule engine failed when applying the rules.}&\multicolumn{1}{>{\raggedright}p{\XLingPapercoldwidth}@{}}{Close the {\LangtToolFontFamily{\textbf{\textcolor[rgb]{0,0,0.5019607843137255}{Live Rule Tester Tool}}}}. Run the {\LangtToolFontFamily{\textbf{\textcolor[rgb]{0,0,0.5019607843137255}{Clean Files}}}} module. Rebuild the bilingual lexicon. Re-open the {\LangtToolFontFamily{\textbf{\textcolor[rgb]{0,0,0.5019607843137255}{Live Rule Tester Tool}}}}.}\\\bottomrule%
\end{longtable}
}}
}
\vspace{12pt}
\XLingPaperneedspace{5\baselineskip}

\penalty-3000{\noindent{\raisebox{\baselineskip}[0pt]{\protect\hypertarget{sOtherTips}{}}\SectionLevelTwoFontFamily{\normalsize{\raisebox{\baselineskip}[0pt]{\pdfbookmark[2]{9.3 Other Troubleshooting Hints}{sOtherTips}}\textbf{9.3 Other Troubleshooting Hints}}}}
\markright{Other Troubleshooting Hints}
\XLingPaperaddtocontents{sOtherTips}}\par{}\penalty10000
\vspace{12pt}
\indent If a rule doesn't appear to be working, first check to see that the rule is listed in the {\textit{Rule Execution Information}} area. If not, there is a problem with the category definition. Check your category definition to make sure it correctly matches the incoming data.\par{}
\vspace{12pt}
\XLingPaperneedspace{5\baselineskip}

\penalty-3000{{\centering\raisebox{\baselineskip}[0pt]{\protect\hypertarget{sLocalization}{}}\SectionLevelOneFontFamily{\large{\raisebox{\baselineskip}[0pt]{\pdfbookmark[1]{10 User Interface Languages}{sLocalization}}\textbf{10 User Interface Languages}}}\\{}}\markright{User Interface Languages}
\XLingPaperaddtocontents{sLocalization}}\par{}\penalty10000
\vspace{12pt}
\indent From version 3.14 of {\LangtToolFontFamily{\textbf{\textcolor[rgb]{0,0,0.5019607843137255}{FLExTrans}}}}, it has been possible to install {\LangtToolFontFamily{\textbf{\textcolor[rgb]{0,0,0.5019607843137255}{FLExTrans}}}} with a choice of user interface language. The following components will be localized:\par{}\indent button labels, module descriptions, collection names, error messages, labels for elements in the transfer rules, sample transfer rule comments, the FLExTools interface.\par{}
\vspace{12pt}
\XLingPaperneedspace{5\baselineskip}

\penalty-3000{\noindent{\raisebox{\baselineskip}[0pt]{\protect\hypertarget{sChangingLanguages}{}}\SectionLevelTwoFontFamily{\normalsize{\raisebox{\baselineskip}[0pt]{\pdfbookmark[2]{10.1 Changing The FLExTrans User Interface Language}{sChangingLanguages}}\textbf{10.1 Changing The {\LangtToolFontFamily{\textbf{\textcolor[rgb]{0,0,0.5019607843137255}{FLExTrans}}}} User Interface Language}}}}
\markright{Changing The {\LangtToolFontFamily{\textbf{\textcolor[rgb]{0,0,0.5019607843137255}{FLExTrans}}}} User Interface Language}
\XLingPaperaddtocontents{sChangingLanguages}}\par{}\penalty10000
\vspace{12pt}
\indent If {\LangtToolFontFamily{\textbf{\textcolor[rgb]{0,0,0.5019607843137255}{FLExTrans}}}} has been installed in one language, and it is now required to change the user interface language to a different language, there are the following possibilities:\par{}
\vspace{12pt}
\XLingPaperneedspace{5\baselineskip}

\penalty-3000{\noindent{\raisebox{\baselineskip}[0pt]{\protect\hypertarget{sChangeFromEnglish}{}}\SectionLevelThreeFontFamily{\normalsize{\raisebox{\baselineskip}[0pt]{\pdfbookmark[3]{10.1.1 Changing From English}{sChangeFromEnglish}}\textit{10.1.1 Changing From English}}}}
\markright{Changing From English}
\XLingPaperaddtocontents{sChangeFromEnglish}}\par{}\penalty10000
\vspace{12pt}
\indent If the initially chosen installation language was English, and it is now required to change from English to a different language, the software should just be reinstalled, choosing the required language.\par{}
\vspace{12pt}
\XLingPaperneedspace{5\baselineskip}

\penalty-3000{\noindent{\raisebox{\baselineskip}[0pt]{\protect\hypertarget{sChangeOtherLanguages}{}}\SectionLevelThreeFontFamily{\normalsize{\raisebox{\baselineskip}[0pt]{\pdfbookmark[3]{10.1.2 Changing From Other Language}{sChangeOtherLanguages}}\textit{10.1.2 Changing From Other Language}}}}
\markright{Changing From Other Language}
\XLingPaperaddtocontents{sChangeOtherLanguages}}\par{}\penalty10000
\vspace{12pt}
\indent If the initially chosen installation language was not English, and another language is now required, the {\LangtToolFontFamily{\textbf{\textcolor[rgb]{0,0,0.5019607843137255}{Collections}}}} files should first be manually deleted. The {\LangtToolFontFamily{\textbf{\textcolor[rgb]{0,0,0.5019607843137255}{Collections}}}} ini files are found in the WorkProjects\textbackslash{}\{ProjectName\}\textbackslash{}Config\textbackslash{}Collections folder, and all of the files in that folder should be deleted. After that, the software should be reinstalled, choosing the required language.\par{}
\vspace{12pt}
\XLingPaperneedspace{5\baselineskip}

\penalty-3000{\noindent{\raisebox{\baselineskip}[0pt]{\protect\hypertarget{sChangingXXELanguage}{}}\SectionLevelTwoFontFamily{\normalsize{\raisebox{\baselineskip}[0pt]{\pdfbookmark[2]{10.2 Change User Interface Language of XMLmind XML Editor}{sChangingXXELanguage}}\textbf{10.2 Change User Interface Language of XMLmind XML Editor}}}}
\markright{Change User Interface Language of XMLmind XML Editor}
\XLingPaperaddtocontents{sChangingXXELanguage}}\par{}\penalty10000
\vspace{12pt}
\indent It is possible to change the user interface language of {\LangtToolFontFamily{\textbf{\textcolor[rgb]{0,0,0.5019607843137255}{XMLmind XML Editor}}}} from English to French or Spanish. These are the instructions from the {\LangtToolFontFamily{\textbf{\textcolor[rgb]{0,0,0.5019607843137255}{XMLmind XML Editor}}}} \href{https://software.sil.org/downloads/r/xlingpaper/resources/documentation/xxe7/UserDocXMLmind.htm\#sFAQLocalization}{webpage}:\par{}\indent The default user interface for both the XMLmind XML Editor and XLingPaper is in English. There also are versions in French and Spanish. (If you notice any error in these versions, please let us know.)\par{}\indent To use one of these non-English versions of the user interface, do the following steps:\par{}{\parskip .5pt plus 1pt minus 1pt
                    
\vspace{\baselineskip}

{\setlength{\XLingPapertempdim}{\XLingPaperdoubledigitlistitemwidth+\parindent{}}\leftskip\XLingPapertempdim\relax
\interlinepenalty10000
\XLingPaperlistitem{\parindent{}}{\XLingPaperdoubledigitlistitemwidth}{1.}{Use Options menu item / Install Add-ons}}
{\setlength{\XLingPapertempdim}{\XLingPaperdoubledigitlistitemwidth+\parindent{}}\leftskip\XLingPapertempdim\relax
\interlinepenalty10000
\XLingPaperlistitem{\parindent{}}{\XLingPaperdoubledigitlistitemwidth}{2.}{Look for the Translation you want and check the box before it.}}
{\setlength{\XLingPapertempdim}{\XLingPaperdoubledigitlistitemwidth+\parindent{}}\leftskip\XLingPapertempdim\relax
\interlinepenalty10000
\XLingPaperlistitem{\parindent{}}{\XLingPaperdoubledigitlistitemwidth}{3.}{Click OK.}}
{\setlength{\XLingPapertempdim}{\XLingPaperdoubledigitlistitemwidth+\parindent{}}\leftskip\XLingPapertempdim\relax
\interlinepenalty10000
\XLingPaperlistitem{\parindent{}}{\XLingPaperdoubledigitlistitemwidth}{4.}{Restart the XMLmind XML Editor.}}
{\setlength{\XLingPapertempdim}{\XLingPaperdoubledigitlistitemwidth+\parindent{}}\leftskip\XLingPapertempdim\relax
\interlinepenalty10000
\XLingPaperlistitem{\parindent{}}{\XLingPaperdoubledigitlistitemwidth}{5.}{Use the Options / Preferences... menu item.}}
{\setlength{\XLingPapertempdim}{\XLingPaperdoubledigitlistitemwidth+\parindent{}}\leftskip\XLingPapertempdim\relax
\interlinepenalty10000
\XLingPaperlistitem{\parindent{}}{\XLingPaperdoubledigitlistitemwidth}{6.}{In the ensuing dialog box, click on the word “General” in the left hand pane.}}
{\setlength{\XLingPapertempdim}{\XLingPaperdoubledigitlistitemwidth+\parindent{}}\leftskip\XLingPapertempdim\relax
\interlinepenalty10000
\XLingPaperlistitem{\parindent{}}{\XLingPaperdoubledigitlistitemwidth}{7.}{In the right hand pane, find the word “Locale” and click on the drop-down button.}}
{\setlength{\XLingPapertempdim}{\XLingPaperdoubledigitlistitemwidth+\parindent{}}\leftskip\XLingPapertempdim\relax
\interlinepenalty10000
\XLingPaperlistitem{\parindent{}}{\XLingPaperdoubledigitlistitemwidth}{8.}{Look for the language you want. Please note that both French (français) and Spanish (español) come after all the uppercase languages in this list.}}
{\setlength{\XLingPapertempdim}{\XLingPaperdoubledigitlistitemwidth+\parindent{}}\leftskip\XLingPapertempdim\relax
\interlinepenalty10000
\XLingPaperlistitem{\parindent{}}{\XLingPaperdoubledigitlistitemwidth}{9.}{Click on the OK button.}}
{\setlength{\XLingPapertempdim}{\XLingPaperdoubledigitlistitemwidth+\parindent{}}\leftskip\XLingPapertempdim\relax
\interlinepenalty10000
\XLingPaperlistitem{\parindent{}}{\XLingPaperdoubledigitlistitemwidth}{10.}{Restart the XMLmind XML Editor.}}
\vspace{\baselineskip}
}\indent You should now see the menu items using the language you chose.\par{}\indent Please note that these non-English versions are for the user interface only. The element names, attribute names, and some of the validity messages will still be in English. Also the user documentation files are only in English.\par{}
\vspace{12pt}
\XLingPaperneedspace{5\baselineskip}

\penalty-3000{{\centering\raisebox{\baselineskip}[0pt]{\protect\hypertarget{sReferenceDocs}{}}\SectionLevelOneFontFamily{\large{\raisebox{\baselineskip}[0pt]{\pdfbookmark[1]{11 Reference documents}{sReferenceDocs}}\textbf{11 Reference documents}}}\\{}}\markright{Reference documents}
\XLingPaperaddtocontents{sReferenceDocs}}\par{}\penalty10000
\vspace{12pt}
\indent The first version of {\LangtToolFontFamily{\textbf{\textcolor[rgb]{0,0,0.5019607843137255}{FLExTrans}}}} was produced as part of a Master’s Thesis see \hyperlink{rLockwood15}{Lockwood (2015)}. In 2018 and 2021 Ron presented papers about {\LangtToolFontFamily{\textbf{\textcolor[rgb]{0,0,0.5019607843137255}{FLExTrans}}}} at the respective BT Conferences. See \hyperlink{rLockwood17}{Lockwood (2018)} and \hyperlink{rLockwood21a}{Lockwood (2021a)}. Also in 2021 Ron presented on Syntactic Parsing as a front-end to {\LangtToolFontFamily{\textbf{\textcolor[rgb]{0,0,0.5019607843137255}{FLExTrans}}}}. See \hyperlink{rLockwood21b}{Lockwood (2021b)}.\par{}
\vspace{12pt plus 2pt minus 1pt}
\XLingPaperneedspace{5\baselineskip}

\penalty-3000{{\centering{\raisebox{\baselineskip}[0pt]{\protect\hypertarget{aAppend}{}}\large{\raisebox{\baselineskip}[0pt]{\pdfbookmark[1]{A. Appendix}{aAppend}}\textbf{A. Appendix\\}}}
}\markright{Appendix}
\XLingPaperaddtocontents{aAppend}}\par{}\penalty10000
\vspace{12pt plus 2pt minus 1pt}

\vspace{12pt}
\XLingPaperneedspace{5\baselineskip}

\penalty-3000{{\centering\raisebox{\baselineskip}[0pt]{\protect\hypertarget{sStartFlextools}{}}\SectionLevelOneFontFamily{\large{\raisebox{\baselineskip}[0pt]{\pdfbookmark[2]{A.1 Starting FLExTools}{sStartFlextools}}\textbf{A.1 Starting {\LangtToolFontFamily{\textbf{\textcolor[rgb]{0,0,0.5019607843137255}{FLExTools}}}}}}}\\{}}\markright{Starting {\LangtToolFontFamily{\textbf{\textcolor[rgb]{0,0,0.5019607843137255}{FLExTools}}}}}
\XLingPaperaddtocontents{sStartFlextools}}\par{}\penalty10000
\vspace{12pt}
\indent Here are the steps:\par{}{\parskip .5pt plus 1pt minus 1pt
                    
\vspace{\baselineskip}

{\setlength{\XLingPapertempdim}{\XLingPapersingledigitlistitemwidth+\parindent{}}\leftskip\XLingPapertempdim\relax
\interlinepenalty10000
\XLingPaperlistitem{\parindent{}}{\XLingPapersingledigitlistitemwidth}{1.}{Navigate to the installation folder (typically {\textit{Documents\textbackslash{}FLExTrans\textbackslash{}WorkProjects\textbackslash{}{[}YourProject{]}}}).}}
{\setlength{\XLingPapertempdim}{\XLingPapersingledigitlistitemwidth+\parindent{}}\leftskip\XLingPapertempdim\relax
\interlinepenalty10000
\XLingPaperlistitem{\parindent{}}{\XLingPapersingledigitlistitemwidth}{2.}{Double-click on the file {\textbf{FlexTools.vbs}} (a Visual Basic script file).}}
{\setlength{\XLingPapertempdim}{\XLingPapersingledigitlistitemwidth+\parindent{}}\leftskip\XLingPapertempdim\relax
\interlinepenalty10000
\XLingPaperlistitem{\parindent{}}{\XLingPapersingledigitlistitemwidth}{3.}{The {\LangtToolFontFamily{\textbf{\textcolor[rgb]{0,0,0.5019607843137255}{FLExTools}}}} window should appear which will look something like \hyperlink{xFirstStart}{(1)}.}}
\vspace{\baselineskip}
}
\vspace{12pt}
\XLingPaperneedspace{5\baselineskip}

\penalty-3000{{\centering\raisebox{\baselineskip}[0pt]{\protect\hypertarget{sDataStreamFormat}{}}\SectionLevelOneFontFamily{\large{\raisebox{\baselineskip}[0pt]{\pdfbookmark[2]{A.2 Data Stream Format}{sDataStreamFormat}}\textbf{A.2 Data Stream Format}}}\\{}}\markright{Data Stream Format}
\XLingPaperaddtocontents{sDataStreamFormat}}\par{}\penalty10000
\vspace{12pt}
\indent The \raisebox{\baselineskip}[0pt]{\protect\hypertarget{gDataStream}{}}{\LangtToolFontFamily{\textbf{\textcolor[rgb]{0,0,0.5019607843137255}{Apertium}}}} data stream format consists of one or more lexical units. In {\LangtToolFontFamily{\textbf{\textcolor[rgb]{0,0,0.5019607843137255}{FLExTrans}}}}, a lexical unit looks something like this:\par{}\indent {\LanglVernacularFontFamily{{\fontspec[Scale=0.9]{Courier New}\textup{\textbf{take{{\fontspec[Scale=0.65]{Times New Roman}\textsubscript{1.1}}} {\LangtluGrammCatFontFamily{{\fontspec[Scale=0.9]{Courier New}\textcolor[rgb]{0,0.4392156862745098,0.7529411764705882}{v}}}} {\LangtluAffixFontFamily{{\fontspec[Scale=0.9]{Courier New}\textcolor[rgb]{0,0.6901960784313725,0.3137254901960784}{3sg}}}}}}}}\protect\footnote[16]{{\leftskip0pt\parindent1em\raisebox{\baselineskip}[0pt]{\protect\hypertarget{nToolToView}{}}You can use the {\LangtToolFontFamily{\textbf{\textcolor[rgb]{0,0,0.5019607843137255}{View Source/Target Apertium File Tool}}}} for a more friendly view of the data stream format found in the files {\textit{source\_text.txt}} and {\textit{target\_text.txt}}.}}\LanglVernacularFontFamily{{\fontspec[Scale=0.9]{Courier New}\textup{\textbf{}}}}\textsuperscript{, }\protect\footnote[17]{{\leftskip0pt\parindent1em\raisebox{\baselineskip}[0pt]{\protect\hypertarget{nRawDataStream}{}}In the plain format the stream format can be explained as follows:
\vspace{10pt plus 2pt minus 1pt}\hbox{\vspace*{0pt}{\XeTeXpicfile "../Images/LexicalUnit.PNG" scaled 300}}}}\LanglVernacularFontFamily{{\fontspec[Scale=0.9]{Courier New}\textup{\textbf{}}}}}\par{}\indent Where take{{\fontspec[Scale=0.65]{Times New Roman}\textsubscript{1}}}\protect\footnote[18]{{\leftskip0pt\parindent1em\raisebox{\baselineskip}[0pt]{\protect\hypertarget{nNoOneHomograph}{}}Technically {\LangtToolFontFamily{\textbf{\textcolor[rgb]{0,0,0.5019607843137255}{FLEx}}}} only displays a subscript 1 if there is a homograph that exists with subscript 2. Without any homograph the {\LangtToolFontFamily{\textbf{\textcolor[rgb]{0,0,0.5019607843137255}{FLEx}}}} headword is just plain -- without any subscript.}} is the headword from {\LangtToolFontFamily{\textbf{\textcolor[rgb]{0,0,0.5019607843137255}{FLEx}}}}. The 1.1 subscript means homograph one, sense one. The grammatical category is colored in blue and all other tags are colored in green.\par{}
\vspace{12pt plus 2pt minus 1pt}
\XLingPaperneedspace{5\baselineskip}

\penalty-3000{{\centering{\raisebox{\baselineskip}[0pt]{\protect\hypertarget{rXLingPapReferences}{}}\large{\raisebox{\baselineskip}[0pt]{\pdfbookmark[1]{References}{rXLingPapReferences}}\textbf{References\\}}}
}\markright{References}
\XLingPaperaddtocontents{rXLingPapReferences}}\par{}\penalty10000
\vspace{12pt plus 2pt minus 1pt}
\raggedright
\hangindent.25in\relax
\hangafter1\relax
\fontsize{10}{12}\selectfont
\raisebox{\baselineskip}[0pt]{\protect\hypertarget{rApertiumSite}{}}Apertium.   2021. Apertium, A free/open-source machine translation platform.  (\href{http://apertium.org}{http://apertium.org}).  (accessed 12/20/2021).\par
\hangindent.25in\relax
\hangafter1\relax
\fontsize{10}{12}\selectfont
\raisebox{\baselineskip}[0pt]{\protect\hypertarget{rBlack}{}}Black, H. Andrew.   2014. A Conceptual Introduction to Morphological Parsing for Stage 1 of the Fieldworks Language Explorer. SIL International Manuscript.\par
\hangindent.25in\relax
\hangafter1\relax
\fontsize{10}{12}\selectfont
\raisebox{\baselineskip}[0pt]{\protect\hypertarget{rApertium}{}}Forcada, Mikel L., Bonev, Boyan Ivanov, Rojas, Sergio Ortiz, Ortiz, Juan Antonio Perez, Sanchez, Gema Ramirez, Martinez, Felipe Sanchez, Armentano-Oller, Carme, Montava, Marco A., Tyers, Francis M.   2010. Documentation of the Open-Source Shallow-Transfer Machine Translation Platform Apertium. Departament de Llenguatges i Sistemes Informàtics Universitat d’Alacant Manuscript. (\href{https://wiki.apertium.org/wiki/File:Apertium2-documentation.pdf}{https://wiki.apertium.org/wiki/File:Apertium2-documentation.pdf}).\par
\hangindent.25in\relax
\hangafter1\relax
\fontsize{10}{12}\selectfont
\raisebox{\baselineskip}[0pt]{\protect\hypertarget{rLockwood15}{}}Lockwood, Ronald Milton.   2015. A Linguist-Friendly Machine Translation System for Low-Resource Languages. University of Washington Manuscript. (\href{https://digital.lib.washington.edu/researchworks/handle/1773/33999}{https://digital.lib.washington.edu/researchworks/handle/1773/33999}).\par
\hangindent.25in\relax
\hangafter1\relax
\fontsize{10}{12}\selectfont
\raisebox{\baselineskip}[0pt]{\protect\hypertarget{rLockwood17}{}}Lockwood, Ron.   2018. Linguistically-Based Machine Translation with the New Tool FLExTrans.  In  .  \textit{Proceedings of Bible Translation 2017 Conference}.  pages unknown. Dallas.\par
\hangindent.25in\relax
\hangafter1\relax
\fontsize{10}{12}\selectfont
\raisebox{\baselineskip}[0pt]{\protect\hypertarget{rLockwood21a}{}}Lockwood, Ron.   2021a. Results of Machine Translation with FLExTrans in Production-Mode.  In \hyperlink{rBTConference21}{BTConf2021, ,  pages unknown.}\par
\hangindent.25in\relax
\hangafter1\relax
\fontsize{10}{12}\selectfont
\raisebox{\baselineskip}[0pt]{\protect\hypertarget{rLockwood21b}{}}Lockwood, Ron.   2021b. Using Syntactic Parsing to Enable Machine Translation for Language Pairs with Thorny Differences.  In \hyperlink{rBTConference21}{BTConf2021, ,  pages unknown.}\par
\hangindent.25in\relax
\hangafter1\relax
\fontsize{10}{12}\selectfont
\raisebox{\baselineskip}[0pt]{\protect\hypertarget{rBTConference21}{}}Bible Translation Conference 2021.   2021. \textit{Proceedings of Bible Translation 2021 Conference}.  Online.\par
\end{MainFont}
\clearpage\XLingPaperendtableofcontents
\end{document}
